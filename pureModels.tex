\section{An alternative model}\label{pureModels}

In Section \ref{coreEL} we used node-labelled transition systems as
models for eremic logic. The purpose of the node labels was to state
the constrains, if any, on outgoing actions. This concern is reflected
in the semantics of $!A$.

\[
\begin{array}{lclcl}
  ((S, \rightarrow, \lambda), s) & \models & !A  &\mbox{\quad iff\quad } & \lambda(s) \subseteq A
\end{array}
\]

\NI There is an alternative, and in some sense even simpler approach
to giving semantics to $!A$ which does not require a node-labelling:
we simply check if all actions of all outgoing transitions are in $A$.
As no other formula requires the labelling function $\lambda$ in the
definition of it's satisfaction condition, this approach means we can
use plain labelled transition systems (together with a current state)
as models. This gives rise to a subtly different theory that we now
explore, albeit not in depth.

\begin{definition}
By a \emph{pure eremic model}, ranged over by $\PPP$, we mean a pair
$(\LLL, s)$ where $\LLL = (S, \rightarrow)$ is a labelled transition
system and $s \in S$ a state.
\end{definition}

\NI Adapting the satisfaction relation to pure eremic models is
straightfoward.

\begin{definition}
Using pure eremic models, the  \emph{satisfaction relation} is defined 
inductively by the following clauses, where we assume that $\MMM =
(\LLL, s)$ and $\LLL = (S, \rightarrow)$.

\[
\begin{array}{lclcl}
  \MMM & \models & \top   \\
  \MMM & \models & \phi \AND \psi &\ \mbox{ iff } \ & \MMM  \models \phi \mbox { and } \MMM \models \psi  \\
  \MMM & \models & \langle a \rangle \phi & \mbox{ iff } & \text{there is a } s \xrightarrow{a} t \mbox { such that } (\LLL, t) \models \phi  \\
  \MMM & \models & A &\mbox{ iff } & \{a\ |\ \exists t.s \TRANS{a} t \} \subseteq A
\end{array}
\]

\NI Note that all but the last clause are unchanged from Definition
\ref{ELsatisfaction}.
\end{definition}

In this interpretation, $!A$ can easily be seen to restrict the
out-degree of the current state $s$, i.e.~it constraints the 'width'
of the graph.

It is easy to see that all rules in Figure \ref{figure:elAndBangRules}
are sound with respect to the new semantics too\martin{what about
  completeness?}.  The key advantage pure eremic models have is their
simplicity: they are unadorned labelled transition systems, the key
model of concurrency theory \cite{SassoneV:modcontac}. The connection
with concurrency theory is even stronger than that, because, as we
show below (Theorem \ref{hennessymilnertheorem}), the elementary
equivalence on (finitely branching) pure eremic models is
bisimilarity, one of the more widely used notions of process
equivalence.

This then leads to the question why we most of our development is
about eremic models. The reason is one of philosophy: on pure eremic
models, eremic logic can distinguish models based on branching
structure, for example those in Figure
\ref{figure:counterexample}. However, eremic logic speaks about
exclusion. Prima facie, branching structure has nothing to do with
exclusion. Therefore eremic logic should not distinguish models such
as those just mentioned.

Nevertheless, both kinds of models are compelling, and it is
interesting to see if they can be related in a systematic way.

\subsection{Relationship between pure and eremic models}

The obvious way of converting an eremic model into a pure eremic model
is by forgetting about the node-labelling:
\[
   ((S, \rightarrow, \lambda), s ) \qquad\mapsto\qquad ((S, \rightarrow), s ) 
\]
Let this function be $\FORGET{\cdot}$. For going the other way, we
have two obvious choices:

\begin{itemize}

\item $((S, \rightarrow), s ) \mapsto ((S, \rightarrow, \lambda), s )$
  where $\lambda(t) = \Sigma$ for all states $t$. Call this map $\MAX{\cdot}$.

\item $((S, \rightarrow), s ) \mapsto ((S, \rightarrow, \lambda), s )$
  where $\lambda(t) = \{a \ |\ \exists t'. t \TRANS{a} t'\}$ for all
  states $t$. Call this map $\MIN{\cdot}$.

\end{itemize}

\begin{lemma}\label{modelRelationships}
Let $\MMM$ be an eremic model, and $\PPP$ a pure eremic model.
\begin{enumerate}

\item\label{modelRelationships:1}  $\MMM \models \phi$ implies
  $\FORGET{\MMM} \models \phi$. The reverse implication does not hold.

\item\label{modelRelationships:2}  $ \MAX{\PPP} \models \phi$ implies
  $\PPP \models \phi$. The reverse implication does not hold.

\item\label{modelRelationships:3} $\MIN{\PPP} \models \phi$ if and only if
  $\PPP \models \phi$. 

\end{enumerate}
\end{lemma}

\begin{proof}
The implication in (\ref{modelRelationships:1}) is immediate by
induction on $\phi$. A counterexample for the reverse implication is
given by the formula $\phi = !\{a\}$ and the eremic model $\MMM = ( \{s,
t\}, s \TRANS{a} t, \lambda), s)$ where $\lambda (s) = \{a, b, c\}$:
clearly $\FORGET{\MMM} \models \phi$, but $\MMM \not\models
\phi$.

The implication in (\ref{modelRelationships:2}) is immediate by
induction on $\phi$. To construct a counterexample for the reverse
implication, assume that $\Sigma$ is a strict superset of $\{a\}$
$a$. The formula $\phi = !\{a\}$ and the pure eremic model $\PPP = (
\{s, t\}, s \TRANS{a} t ), s)$ satisfy $\PPP \models \phi$, but clealy
$\MAX{\PPP} \not\models \phi$.

Finally, (\ref{modelRelationships:3}) is also straightforward by
induction on $\phi$.

\end{proof}


