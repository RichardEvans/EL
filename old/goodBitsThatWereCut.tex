\subsection{WHy not pure models -- works best with non-determinism}

This then leads to the question why we most of our development is
about cathoristic models. The reason is one of philosophy: on pure
cathoristic models, \cathoristic{} can distinguish models based on
branching structure, for example those in Figure
\ref{figure:counterexample}. However, \cathoristic{} speaks about
exclusion. Prima facie, branching structure has nothing to do with
exclusion. Therefore \cathoristic{} should not distinguish models such
as those just mentioned.

Nevertheless, both kinds of models are compelling, and it is
interesting to see if they can be related in a systematic way.



\subsubsection{Incompatibility as a Fundamental Semantic Relation}
According to Brandom's, the notion of incompatibility is more than just a pre-logical version of negation. 
It also has a \emph{modal} component:
\begin{quote}
The notion of incompatibility can be thought of as a sort of conceptual vector-product of a \emph{negative} component and a \emph{modal} component. It is \emph{non}-com\emph{possibility}\footnote{\cite{brandom} p.126.}.
\end{quote}
Incompatibility is the fundamental master concept that Brandom uses to make sense of the notion of \emph{object} and \emph{agent}.

First, \emph{object-hood}.
Treating ``$A$ is $\phi$'' and ``$B$ is $\psi$'' as incompatible (where $\phi$ and $\psi$ are incompatible predicates) \emph{just is} what it is to treat $A$ and $B$ as the same object:
\begin{quote}
It is not impossible for two different objects to have incompatible properties - say, being copper and electrically insulating. What is impossible is for \emph{one and the same object} to do so. 
Objects play the conceptual functional role of \emph{units of account for alethic modal incompatibilities}. 
A single object just is what cannot have incompatible properties (at the same time).
\end{quote}
It is only because we are continually resolving incompatibilities that we are able to represent an external (mind-independent) object at all:
\begin{quote}
Shouldering the responsibility of repair and rectification of incompatible commitments is what one has to \emph{do} in order to be taking one's claims to be \emph{about} an objective world
\end{quote}
Second, \emph{agent-hood}.
Just as an object \emph{cannot} have incompatible properties, just so a subject \emph{should not} ascribe incompatible properties to something.
To say that two sentences are incompatible is to say that an agent who holds one \emph{should not} hold the other.
\begin{quote}
Objects play the conceptual functional role of \emph{units of account for alethic modal incompatibilities}. 
It is an essential individuating feature of the metaphysical categorical sortal metaconcept \emph{object} that objects have the metaproperty of \emph{modally} repelling incompatibilities...
And, in a parallel fashion, subjects too are individuated by the way they normatively `repel' incompatible commitments.
It is not impermissible for two different subjects to have incompatible commitments - say, for me to take the coin to be copper and you take it to be an electrical insulator. What is impermissible is for \emph{one and the same subject} to do so. Subjects play the conceptual functional role of \emph{units of account for deontic normative incompatibilities}. 
That is, it is an essential individuating feature of the metaphysical categorical sortal metaconcept \emph{subject} that subjects have the metaproperty of \emph{normatively} repelling incompatibilities. A single subject just is what ought not to have incompatible commitments (at the same time)\footnote{\cite{brandom} p.192.}.
\end{quote}

\NI Sellars and Brandom follow Hegel's \emph{Wissenschaft der Logic}
\cite{HegelGWF:wisdlog} in seeing material incompatibility as a
foundational concept from which even the supposedly primitive ideas of
object and agent can be explicated.  If incompatibility is to fill
this fundamental load-bearing role, we should explore all possible
ways of formalising it.  The tantum operator is the simplest
formalisation we could find in which incompatible propositions can be
expressed.







\subsubsection{\Cathoristic{} as a fragment of first- and second-order logic }

A question that has been fruitful in modal logic has been which
fragment of first-order logic corresponds to modal formulae?  Van
Benthem established that modal logic is the fragment of first-order
logic invariant under bisimulation \cite{BlackburnP:modlog}. What is
the the fragment corresponding to \cathoristic{}? Our Theorem
\ref{theorem:completeLattice} suggests to look at closure under mutual
simulation.  

Van Benthem's result was generalised to monadic second
order by Janin and Walukiewicz \cite{JaninD:expcomotpmcwrtmsol} who
showed that the bisimulation invariant fragment of monadic
second-order logic is the modal $\mu$-calculus, an extension of modal
logic with fixpoint operators \cite{KozenD:respromc}.  It would be
interesting to study the \emph{cathoristic $\mu$-calculus}, which adds
fixpoints to \cathoristic{}:
\begin{GRAMMAR}
  \phi
     &::=&
  \TRUE
     \VERTICAL
  \phi \AND \psi
     \VERTICAL
  \MAY{a}\phi
     \VERTICAL
  !A
     \VERTICAL
  \mu \PVAR{x}.\phi
     \VERTICAL
  \nu \PVAR{x}.\phi
     \VERTICAL
  \PVAR{x}
\end{GRAMMAR}

\NI Here $\PVAR{x}$ ranges over propositional variables.  Such a logic
would be useful for talking about infinite transition systems and the
structures of exclusion they give rise to.  For example the phases of
a traffic light can be described as follows.
\[
   \MAY{tl}{\MAY{colour}{\mu \PVAR{x}.
       (!\{green\} \AND \MAY{green}{
         (!\{amber\} \AND \MAY{amber}{
           (!\{red\} \AND \MAY{red}{\PVAR{x}})})})}}
\]

 What properties does the
cathoristic $\mu$-calculus have? In the absence of negation, a lot of proof
techniques taken for granted in the meta-theory of the modal
$\mu$-calculus, such as the relationship between least and greatest
fixpoints, is no longer available.


\subsubsection{Compactness, ultraproducts and \cathoristic{}}

Mathematicals knows three substantially different approaches for
proving compactness: G\"odel's original approach, refined by Morley
and many others, uses proof rules and the completeness theorem.  Our
proof in Section \ref{compactnessProof} works by translation and
piggy-backs on the compactness of first-order logic. Finally, the
model-theoretic approach uses ultra-products as a tool for 'gluing'
together the models that guarantee finite
satisfiability. Ultra-products are quotients of reduced products by an
ultra-filter. The conditions defining ultra-filters are carefully
chosen to ensure that the quotient is indeed an appropriate model with
nice properties such as \L{}o\'{s}' theorem. In particular, for $U$ to
be an ultra-filter over some set $S$, for each $A \subseteq S$, either
$A$ or its complemnet $S\setminus A$ must be in $U$. This axiom
clearly reflects the semantics of negation. But \cathoristic{} doesn't
have negation. So what would an appropriate equivalent of
ultra-filters for \cathoristic{} look like that can be used to prove
compactness directly on models?
\richard{Do we need this?}

\subsubsection{Extending \cathoristic{} with dualities}
In any modal logic with negation, the may modality and the must modality are interdefinable:
\[
\MUST{a}{\phi} = \neg
\MAY{a}{\neg \phi}
\]
In \cathoristic{}, the situation is different: since
negation is not available, adding $\MUST{a}{\phi}$ is a substantial
change to the logic.
For example, the processes in Figure
\ref{figure:counterexample}\richard{Which figure is this?}, indistinguishable by cathoristic formulae, are
separated by
  \[
     \MUST{a}{\MAY{b}{}},
  \]
  assuming the usual semantics: $(\LLL, s) \models \MUST{a}{\phi}$ if
  $s \TRANS{a} t$ implies that $(\LLL, t) \models \phi$. This means
  Theorem \ref{theorem:completeLattice} fails. What notion of model
  equivalence does elementary equivalence become in this scenario?
  What do complete proof rules look like?

  Similar questions can be asked for other additional primitives. If negation is 
  available, the formula
  \[
     \MAY{a}{\neg \MAY{b}{} }
  \]
  separates the afromentioned examples. If we add implication instead,
  negation becomes definable: $\neg \phi = \phi \rightarrow \bot$.
\richard{In one sense, adding the must modality is not further work. If we add negation, as in section 10, then we can define must modality in the obvious way. It is only further work if we add $\Box$ as a primitive without adding negation.}

