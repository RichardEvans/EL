The relation $\MODELLEQ$ is not a partial order, only a pre-order. For example
with 
\begin{itemize}

\item $\MMM_1 = ( (\{w\}, \{\}, \{w \mapsto \Sigma\}), w)$ and
\item $\MMM_2 = ( (\{v\}, \{\}, \{v \mapsto \Sigma\}), v)$ 

\end{itemize}

\NI we have models where $\MMM_1 \MODELLEQ \MMM_2$ and $\MMM_2
\MODELLEQ \MMM_1$, but the two models are not equal. The difference
between the two models is trival and not relevant for the formulae
they make true. Indeed $\THEORY{\MMM_1} = \THEORY{\MMM_2}$. 
As briefly mentioned in the mathematical preliminaries (Section \ref{preliminaries}),
we obtain a
proper partial-order we simply quotient the set of models:

\[
   \MMM \MODELEQ \MMM'
      \qquad\text{iff}\qquad
   \MMM \MODELLEQ \MMM' \ \text{and}\ \MMM' \MODELLEQ \MMM.
\]

\NI and the ordering the $\MODELEQ$-equivalence classes as follows:
\[
    [\MMM]_{\MODELEQ} \MODELLEQ [\MMM']_{\MODELEQ}
      \qquad\text{iff}\qquad
    \MMM \MODELLEQ \MMM'.
\]

\NI Since this process is independent for the chosen representatives,
we obtain a partial order. Greatest lower and least upper bounds can also
be computed on representatives:
\[
   \BIGLUB \{[\MMM]_{\MODELEQ} \ |\ \MMM \in S\ \} = [\BIGLUB S]_{\MODELEQ}
\]
and likewise for the greated lower bound.

In the rest of this text, we will usually be sloppy and work with
concrete models instead of equivalence classes of models because the
quotienting process is straightfoward and not especially
interesting. We can do this because all relevant constructions in this
text are independent from the specific choice of representative.  Our
Haskell implementation, described in Section \ref{hahahaskell} uses a
slightly different approach: instead of equivalence classes, it uses
canonical representatives.  \martin{add more explanation, and
  integrate better with premiminaries}.

\begin{itemize}

\item $\bot = [\bot]_{\MODELEQ}$.

\end{itemize}


