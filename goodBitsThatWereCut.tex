

\subsubsection{Incompatibility as a Fundamental Semantic Relation}
According to Brandom's, the notion of incompatibility is more than just a pre-logical version of negation. 
It also has a \emph{modal} component:
\begin{quote}
The notion of incompatibility can be thought of as a sort of conceptual vector-product of a \emph{negative} component and a \emph{modal} component. It is \emph{non}-com\emph{possibility}\footnote{\cite{brandom} p.126.}.
\end{quote}
Incompatibility is the fundamental master concept that Brandom uses to make sense of the notion of \emph{object} and \emph{agent}.

First, \emph{object-hood}.
Treating ``$A$ is $\phi$'' and ``$B$ is $\psi$'' as incompatible (where $\phi$ and $\psi$ are incompatible predicates) \emph{just is} what it is to treat $A$ and $B$ as the same object:
\begin{quote}
It is not impossible for two different objects to have incompatible properties - say, being copper and electrically insulating. What is impossible is for \emph{one and the same object} to do so. 
Objects play the conceptual functional role of \emph{units of account for alethic modal incompatibilities}. 
A single object just is what cannot have incompatible properties (at the same time).
\end{quote}
It is only because we are continually resolving incompatibilities that we are able to represent an external (mind-independent) object at all:
\begin{quote}
Shouldering the responsibility of repair and rectification of incompatible commitments is what one has to \emph{do} in order to be taking one's claims to be \emph{about} an objective world
\end{quote}
Second, \emph{agent-hood}.
Just as an object \emph{cannot} have incompatible properties, just so a subject \emph{should not} ascribe incompatible properties to something.
To say that two sentences are incompatible is to say that an agent who holds one \emph{should not} hold the other.
\begin{quote}
Objects play the conceptual functional role of \emph{units of account for alethic modal incompatibilities}. 
It is an essential individuating feature of the metaphysical categorical sortal metaconcept \emph{object} that objects have the metaproperty of \emph{modally} repelling incompatibilities...
And, in a parallel fashion, subjects too are individuated by the way they normatively `repel' incompatible commitments.
It is not impermissible for two different subjects to have incompatible commitments - say, for me to take the coin to be copper and you take it to be an electrical insulator. What is impermissible is for \emph{one and the same subject} to do so. Subjects play the conceptual functional role of \emph{units of account for deontic normative incompatibilities}. 
That is, it is an essential individuating feature of the metaphysical categorical sortal metaconcept \emph{subject} that subjects have the metaproperty of \emph{normatively} repelling incompatibilities. A single subject just is what ought not to have incompatible commitments (at the same time)\footnote{\cite{brandom} p.192.}.
\end{quote}

\NI Sellars and Brandom follow Hegel's \emph{Wissenschaft der Logic}
\cite{HegelGWF:wisdlog} in seeing material incompatibility as a
foundational concept from which even the supposedly primitive ideas of
object and agent can be explicated.  If incompatibility is to fill
this fundamental load-bearing role, we should explore all possible
ways of formalising it.  The tantum operator is the simplest
formalisation we could find in which incompatible propositions can be
expressed.





