\section{Introduction}\label{introduction}

Natural language is full of mutually-exclusive alternatives.
If Pierre is the current king of France, then nobody else can concurrently fill that role.
A traffic light can be green, amber or red - but it cannot be more than one at a time.
Mutual exclusion is a natural and ubiquitous concept.

Predicate logic \emph{can} represent mutually exclusive alternatives, of course.
To say that Pierre is the \emph{only} king of France, we can write, following Russell:
\[
king(france, pierre) \land \forall X . king(france, X) \rightarrow X = pierre
\]
To say that a particular traffic light, $t$, is red - and red is its \emph{only} colour - we could write:
\[
colour(t, red) \land \forall X . colour(t, X) \rightarrow X = red
\]
In predicate logic, exclusion is a \emph{derived} concept, reduced to a combination of universal quantification and identity.
Predicate logic, in other words, uses relatively complex machinery to express a simple concept:
\begin{itemize}
\item Quantification is an expensive concept that comes with
      many rules governing its behaviour - such as the distinction
      between free and bound variables\footnote{As of 2014 there is a substantial amount
      of ongoing research regarding good formalisations of handling
      free/bound variables, see e.g.~nominal approaches to
      logic \cite{PittsAM:newaas,PittsAM:nomsetnasics}. The problem
      was put in focus in recent years with the rise in interest in the
      computational cost of syntax manipulation in languages with
      binders.}. The costs of quantification have to be
      borne every time one uses exclusion, \emph{even though exclusion does
      not, prima facie, appear to have anything to do with the
      free/bound variable distinction}.
\item The identity relation is also an expensive piece of machinery. In first order predicate logic, identity is a special-case relation which requires an infinite axiom schema (the Indiscernibility of Identicals) to capture its unique properties.
\end{itemize}

This paper introduces an alternative logic, \ELFULL{}, in which exclusion is expressed \emph{directly}, as a first-class concept - rather than defined in terms of relatively complex machinery.
\ELFULL{} is the simplest logic we could find that expresses exclusion directly. 
Exclusion and conjunction are the \emph{only} logical operators. 
There is no negation, disjunction or implication.
Because of its extreme simplicity, it has a linear-time decision procedure.

\ELFULL{} is a multi-modal logic, a variant of Hennessy-Milner Logic, that replaces negation with a different logical connective:
\[
   !A
\]
called \emph{just} $A$, or \emph{tantum} $A$.
Here, $A$ is a finite set of alternatives that exhaust all
possibilities. 
For example:
\begin{eqnarray*}
\fBang \{green, amber, red\}
\end{eqnarray*}
Any statement of a fact that exceeds what tantum $A$ allows is necessarily false:
\[
   !\{green, amber, red\} \AND \MAY{blue}
\]
is necessarily false.
If the only available options are green, amber, or red, then blue is not an available option.

Now to say that Pierre is the only king of France, we write:
\[
\MAY{king}\MAY{france}(\MAY{pierre} \land \fBang \{pierre\})
\]
Note that \ELFULL{}'s representation of exclusion involves no universal quantifier and no identity relation.
It is a purely propositional representation of exclusion.

To say that the traffic light is currently red, and \emph{red is its only colour}, we write:
\[
\MAY{t} \MAY{colour} (\MAY{red} \land !\{red\})
\]
This is markedly simpler (both in terms of representation length and computational complexity) than the representation in predicate logic:
\[
colour(t, red) \land \forall X . colour(t, X) \rightarrow X = red
\]
To summarise the argument we shall be presenting:
\begin{itemize}
\item
Every fundamental logical concept needs a logic in which it can be expressed naturally (ergonomically)
\item
Exclusion is a fundamental logical concept
\item
First order logic does not express exclusion naturally\footnote{We will precisify this claim in later sections by showing that (i) first order logic's representation of exclusion is significantly longer in terms of formula length than \ELFULL{}'s (see Section \ref{incompatiblepredicatesinfol}); and (ii) logic programs in \ELFULL{} can be optimised to run significantly faster than their equivalent in FOL (see Section \ref{optimizingpreconditions}).}
\item
\ELFULL{} expresses exclusion directly
\item
Because it expresses exclusion directly, it has various unusual features. 
Because it eschews the complexifying logical operators $\neg, \lor$ and $\Rightarrow$, it has a \emph{linear-time} decision procedure. 
Nevertheless, despite its simplicity, it is sufficiently expressive to satisfy both:
\begin{itemize}
\item
The Hennessy-Milner theorem (Theorem \ref{theorem:completeLattice})
\item
Brandom's Incompatibility Semantics property (Theorem \ref{incompatibilitytheorem})
\end{itemize}
\end{itemize}
In the next subsections, we shall:
\begin{itemize}
\item
Strengthen the claim that exclusion is a fundamental logical concept, by arguing that exclusion is conceptually prior to negation
\item
Broaden the claim that exclusion is a fundamental logical concept, by adding other sorts of intra-sentential logical concepts that predicate logic cannot handle (naturally, or at all). These other forms of intra-sentential inferential relations can all be handled naturally in \ELFULL{}.
\end{itemize}

\subsection{Exclusion as a Fundamental Logical Concept, Conceptually Prior to Negation}

Consider the sentences:
\begin{enumerate}

\item ``Jack is male'' is incompatible with ``Jack is female''
\item For every person $x$, if $x$ is male, then it is not the case that $x$ is female

\end{enumerate}

\NI The orthodox position is that (1) is true because of (2).  There
are \emph{general universally-quantified rules} describing which
predicates are incompatible, and these general rules explain why
particular sentence instances are incompatible.  We know (1) because
we know (2).

In \emph{Making It Explicit}\cite{brandom2}, Brandom proposes a reversal of this
orthodox direction of explanation.  He claims that we can understand
incompatible sentence pairs, like (1), even if we do not understand
universal quantification or negation.  \emph{Material incompatibility is
conceptually prior to logical negation.}

Imagine, to make this vivid, a primitive people speaking a
primordial language.  This language contains atomic sentences, but has
no ``complex'' logical connectives.  By ``complex'', I mean logical
connectives that, like disjunction, generate sentences which can be
satisfied in many different distinct ways.  Negation, disjunction,
implication, existential quantification all create formulae that can
be satisfied in many different ways (but conjunction does not). This condition will be
precisified in Section 2: a logic is ``simple'' if the set of
satisfying models of every sentence has a unique least upper
bound. Disjunction fails to be simple because there are two equally
simple models of $\phi \lor \psi$. Similarly, negation fails to be simple
because there are two equally simple models of $\neg (\phi \land
\psi)$. Conjunction is the \emph{only} connective of propositional logic that is simple,
according to this criterion.

Imagine, then, a primitive people speaking a simple language of
atomic sentences - a language containing no operators for generating
sentences that can be satisfied in different ways.  These people
recognise when atomic sentences are incompatible, and can see that one
sentence entails another - but they have no way of saying explicitly
that these sentences have these properties.  Their behaviour
outreaches their ability to articulate it.  Over time, these people
may advance to a more sophisticated language in which
incompatibilities and entailments can be made explicit (using the
negation and entailment operators respectively) - but this is a later
(and optional) development. The speakers of the core primordial
language understand incompatibility relations between atomic
sentences, but use no complex logical connectives.

If this picture is coherent, then material incompatibility - exclusion - is conceptually independent of logical negation.
We do not need negation to make sense of exclusion.

Now imagine a variation of our primitive language in which no sentences are ever treated as incompatible.
The native speakers never disagree, back down, retract their claims, or justify them. They just make assertions.
Without an understanding of incompatibility, and the variety of behaviour that it engenders, we submit (following Brandom) that there is insufficient richness in the linguistic practice for their sounds to count as assertions.
Without exclusion, their sounds are just \emph{barks}.
If this further claim is also accepted, then material incompatibility - exclusion - is not just conceptually \emph{independent} of logical negation, but is conceptually \emph{prior} to it.


\subsection{Intra-Sentential Logical Operators}

The core claim of this paper is that \ELFULL{} is an ergonomic logic for describing the exclusion relation.
Now exclusion is an inferential relation between \emph{atomic sentences}. 
In this subsection, we shall describe \emph{other} inferential relations between atomic sentences - inferential relations that predicate logic cannot articulate, but that \ELFULL{} is able to handle.

The \emph{atomic sentences} of a natural language can be
characterised as the sentences which do not contain any other
sentences as constituent parts\footnote{Compare Russell \cite{russell}
  p.117: ``A sentence is of atomic form when it contains no logical
  words and no subordinate sentence''. We use a broader notion of
  atomicity by focusing solely on whether or not it contains a
  subordinate sentence, allowing logical words such as ``and'' as long
  as they are conjoining noun-phrases and not sentences.}.  According
to this criterion, the following are atomic:

\begin{itemize}

\item Jack is male
\item Jack loves Jill
\end{itemize}

\NI The following is not atomic:

\begin{quote}
  Jack is male and Jill is female
\end{quote}

\NI because it contains the complete sentence ``Jack is male'' as a
syntactic constituent.  Note that, according to this criterion, the
following \emph{is} atomic, despite using ``and'' :

\begin{quote}
  Jack loves Jill and Joan
\end{quote}

\NI Here, ``Jack loves Jill'' is not a syntactic constituent since
``and'' is used to conjoin \emph{noun-phrases}, not sentences.

There are many types of inferential relations between atomic
sentences of a natural language.  For example:

\begin{itemize}

\item ``Jack is male'' is incompatible with ``Jack is female''
\item ``Jack loves Jill'' implies ``Jack loves''
\item ``Jack walks slowly'' implies ``Jack walks''
\item ``Jack loves Jill and Joan'' implies ``Jack loves Jill''
\item ``Jack is wet and cold'' implies ``Jack is cold''

\end{itemize}

\NI Some of these inferential relations are \emph{exclusion}
relations (two sentences cannot both be true) while others are
\emph{entailment} relations (if one sentence is true, the other must
also be true).  The main question this paper seeks to answer is:
\emph{what is the simplest logic that can capture these inferential
  relations?}

\subsection{Wittgenstein's vision of a logic of elementary propositions}

\NI In the \emph{Tractatus} \cite{wittgenstein-tractatus}, Wittgenstein
claimed that the world is a set of atomic sentences in an idealized
logical language.  Each atomic sentence was supposed to be
\emph{logically independent} of every other, so that they could be
combined together in every possible permutation, without worrying
about their mutual compatibility.

But already there were doubts and problem cases.  He was aware that
certain sorts of statements seemed atomic, but did not seem logically
independent:

\begin{quote}
  For two colours, e.g., to be at one place in the visual field is
  impossible, and indeed logically impossible, for it is excluded by
  the logical structure of colour. (6.3751)
\end{quote}

\NI At the time he was writing the Tractatus, he hoped that further
analysis would reveal that these statements were not really atomic.

But in the \emph{Philosophical Remarks} \cite{wittgenstein-remarks}, he
renounced the thesis of the logical independence of atomic
propositions.  In \S 76, talking about incompatible colour predicates,
he writes:

\begin{quote}
  That makes it look as if \emph{a construction might be possible
    within the elementary proposition}. That is to say, as if there
  were a construction in logic which didn't work by means of truth
  functions.  What's more, it also seems that these constructions have
  an effect on one proposition's following logically from another.
  For, if different degrees exclude one another it follows from the
  presence of one that the other is not present.  In that case,
  \emph{two elementary propositions can contradict one another}.
\end{quote}

\NI Here, he is clearly imagining a logical language in which there
are incompatibilities between atomic propositions. In \S 82:

\begin{quote}
  This is how it is, what I said in the Tractatus doesn't exhaust the
  grammatical rules for 'and', 'not', 'or', etc; \emph{there are rules
    for the truth functions which also deal with the elementary part
    of the proposition}.  The fact that one measurement is right
  \emph{automatically} excludes all others.
\end{quote}

\NI But Wittgenstein does not, unfortunately, show us what this
language would look like.  In this paper, a new modal logic is
introduced to articulate Wittgenstein's vision.  \ELFULL{} is the simplest logic we could find that handles the various types of inferences between elementary propositions.

\subsection{Outline}
The rest of this paper is organised as follows:
\begin{itemize}
\item
We describe the syntax and semantics of \ELFULL{}
\item
We provide inference rules, and prove soundness, completeness and compactness
\item
We describe a \emph{linear-time} decision procedure
\item
We prove a version of the Hennessy-Milner theorem
\item
We show that \ELFULL{} satisfies Brandom's Incompatibility Semantics requirement
\item
We show how \ELFULL{} has been used as an ergonomic knowledge-representation language in an industrial application
\end{itemize}

\subsubsection{Discussion: Precisifying the Negation. }

\NI If someone asserts that, say, Jack does not support Manchester United,
we have so far said very little.  Until we know the \emph{range of
  football teams} he could support, we don't know what the negation
amounts to.  Until we have some more determinate information about the
range of possible choices, there are an \emph{indefinite} number of
ways in which this could be true.  But if we knew that the only
possible teams Jack could support are Manchester United, Arsenal or
Chelsea, then suddenly our negation has some determinate content.
SNEL captures this intuition.  Once a negated formula has been
precisified by specifying the range of allowable options (via the $!$
operator), the negated claim can be made precise, and the law of
excluded middle can be proven.

\subsubsection{Discussion: Negation as Infinitary Disjunction. }
As is well known the existential quantifier of Predicate Logic can be translated into an infinitary disjunction of propositions in Propositional Logic.

Analogously, given an infinite set $S$ of symbols, the
negation of propositional logic can be translated into an infinitary
disjunction of formulae in SNEL.\martin{doesn't this argument also work for a finite
set $S$?}


\subsection{Introduction TODOs}

\begin{itemize}
\item Consider expressing ``Pierre is either male or female'' in
  propositional logic, without quantifiers: $sex(pierre, male) \lor
  sex(pierre, female) \land \neg (sex(pierre, male) \land sex(pierre,
  female))$.
\item
Consider placing the point (currently in Related Work, Brandom's
Incompatibility Semantics) - that $\neg p$ is the least information
claim that is incompatible with $p$ - into the section on Exclusion
being Conceptually prior to Negation.

\item Integrate the two section above on negation
\end{itemize}
