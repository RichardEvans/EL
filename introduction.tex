\section{Introduction}\label{introduction}

\begin{FIGURE}
\begin{tikzpicture}[node distance=3.3cm,>=stealth',bend angle=45,auto]
  \tikzstyle{place}=[circle,thick,draw=blue!75,fill=blue!20,minimum size=6mm]
  \tikzstyle{red place}=[place,draw=red!75,fill=red!20]
  \tikzstyle{transition}=[rectangle,thick,draw=black!75,
  			  fill=black!20,minimum size=4mm]
  \tikzstyle{every label}=[red]
  \begin{scope}  
    \node [place] (w1) {PL[$\land$]};
    \node [place] (e1) [below of=w1] {PL [$\land, \neg$] };
  \end{scope}
  \begin{scope}[xshift=4cm]
    \node [place] (w1) {HML[$\land$]};
    \node [place] (e1) [below of=w1] {HML [$\land, \neg$] };
  \end{scope} 
  \begin{scope}[xshift=8cm]
    \node [place] (w1) {EL[$\land, !$]};
    \node [place] (e1) [below of=w1] {EL [$\land, !, \neg$] };
  \end{scope}
  \draw (2,0) node {$\subseteq $};
  \draw (6,0) node {$\subseteq$};
  \draw (2,-3) node {$\subseteq $};
  \draw (6,-3) node {$\subseteq$};
  \draw (0,-1.5) node {$\subseteq $};
  \draw (4,-1.5) node {$\subseteq $};
  \draw (8,-1.5) node {$\subseteq $};
\end{tikzpicture}
\caption{Relationships between logics established in this paper. Here $L_1 \subseteq L_2$
means that the logic $L_2$ is a conservative extension of $L_1$. NB: we need to
explain the concept of conservative extension.}\label{figure:relationships}
\end{FIGURE}


\subsection{A logic of atomic sentences}

\NI The \emph{atomic sentences} of a natural language can be
characterised as the sentences which do not contain any other
sentences as constituent parts\footnote{Compare Russell \cite{russell}
  p.117: ``A sentence is of atomic form when it contains no logical
  words and no subordinate sentence''. We use a broader notion of
  atomicity by focusing solely on whether or not it contains a
  subordinate sentence, allowing logical words such as ``and'' as long
  as they are conjoining noun-phrases and not sentences.}.  According
to this criterion, the following are atomic:

\begin{itemize}

\item Jack is male
\item Jack loves Jill
\end{itemize}

\NI The following is not atomic:

\begin{quote}
  Jack is male and Jill is female
\end{quote}

\NI because it contains the complete sentence ``Jack is male'' as a
syntactic constituent.  Note that, according to this criterion, the
following \emph{is} atomic, despite using ``and'' :

\begin{quote}
  Jack loves Jill and Joan
\end{quote}

\NI Here, ``Jack loves Jill'' is not a syntactic constituent since
``and'' is used to conjoin \emph{noun-phrases}, not sentences.

\subsection{Inferences between atomic sentences}

\NI There are many types of inferential relations between atomic
sentences of a natural language.  For example:

\begin{itemize}

\item ``Jack is male'' is incompatible with ``Jack is female''
\item ``Jack loves Jill'' implies ``Jack loves''
\item ``Jack walks slowly'' implies ``Jack walks''
\item ``Jack loves Jill and Joan'' implies ``Jack loves Jill''
\item ``Jack is wet and cold'' implies ``Jack is cold''

\end{itemize}

\NI Some of these inferential relations are \emph{incompatibility}
relations (two sentences cannot both be true) while others are
\emph{entailment} relations (if one sentence is true, the other must
also be true).  The main question this paper seeks to answer is:
\emph{what is the simplest logic that can capture these inferential
  relations?}

\subsection{Wittgenstein's vision of a logic of elementary propositions}

\NI In the \emph{Tractatus} \cite{wittgenstein-tractatus}, Wittgenstein
claimed that the world is a set of atomic sentences in an idealized
logical language.  Each atomic sentence was supposed to be
\emph{logically independent} of every other, so that they could be
combined together in every possible permutation, without worrying
about their mutual compatibility.

But already there were doubts and problem cases.  He was aware that
certain sorts of statements seemed atomic, but did not seem logically
independent:

\begin{quote}
  For two colours, e.g., to be at one place in the visual field is
  impossible, and indeed logically impossible, for it is excluded by
  the logical structure of colour. (6.3751)
\end{quote}

\NI At the time he was writing the Tractatus, he hoped that further
analysis would reveal that these statements were not really atomic.

But in the \emph{Philosophical Remarks} \cite{wittgenstein-remarks}, he
renounced the thesis of the logical independence of atomic
propositions.  In \S 76, talking about incompatible colour predicates,
he writes:

\begin{quote}
  That makes it look as if \emph{a construction might be possible
    within the elementary proposition}. That is to say, as if there
  were a construction in logic which didn't work by means of truth
  functions.  What's more, it also seems that these constructions have
  an effect on one proposition's following logically from another.
  For, if different degrees exclude one another it follows from the
  presence of one that the other is not present.  In that case,
  \emph{two elementary propositions can contradict one another}.
\end{quote}

\NI Here, he is clearly imagining a logical language in which there
are incompatibilities between atomic propositions. In \S 82:

\begin{quote}
  This is how it is, what I said in the Tractatus doesn't exhaust the
  grammatical rules for 'and', 'not', 'or', etc; \emph{there are rules
    for the truth functions which also deal with the elementary part
    of the proposition}.  The fact that one measurement is right
  \emph{automatically} excludes all others.
\end{quote}

\NI But Wittgenstein does not, unfortunately, show us what this
language would look like.  In this paper, a new modal logic is
introduced to articulate Wittgenstein's vision.  Eremic Logic was
designed to handle the various types of inferences between elementary
propositions.

\subsection{An autonomous language game based on incompatibility between atomic sentences}

\NI Another motivation for Eremic Logic starts with Robert Brandom's
arguments that material incompatibility is conceptually prior to
logical negation.  Consider the sentences:
\begin{enumerate}

\item ``Jack is male'' is incompatible with ``Jack is female''
\item For every person $x$, if $x$ is male, then it is not the case that $x$ is female

\end{enumerate}

\NI The orthodox position is that (1) is true because of (2).  There
are \emph{general universally-quantified rules} describing which
predicates are incompatible, and these general rules explain why
particular sentence instances are incompatible.  We know (1) because
we know (2).

\NI In \emph{Making It Explicit}\cite{brandom2}, Brandom proposes a reversal of this
orthodox direction of explanation.  He claims that we can understand
incompatible sentence pairs, like (1), even if we do not understand
universal quantification or negation.  Material incompatibility is
conceptually prior to logical negation.

\NI Imagine, to make this vivid, a primitive people speaking a
primordial language.  This language contains atomic sentences, but has
no ``complex'' logical connectives.  By ``complex'', I mean logical
connectives that, like disjunction, generate sentences which can be
satisfied in many different distinct ways.  Negation, disjunction,
implication, existential quantification all create formulae that can
be satisfied in many different ways. This condition will be
precisified in Section 2: a logic is ``simple'' if the set of
satisfying models of every sentence has a unique least upper
bound. Disjunction fails to be simple because there are two equally
simple models of $\phi \lor \psi$. Similarly, negation fails to be simple
because there are two equally simple models of $\neg (\phi \land
\psi)$. Conjunction is the \emph{only} logical connective that is simple,
according to this criterion.

\NI Imagine, then, a primitive people speaking a simple language of
atomic sentences, a language containing no operators for generating
sentences that can be satisfied in different ways.  These people
recognise when atomic sentences are incompatible, and can see that one
sentence entails another - but they have no way of saying explicitly
that these sentences have these properties.  Their behaviour
outreaches their ability to articulate it.  Over time, these people
may advance to a more sophisticated language in which
incompatibilities and entailments can be made explicit (using the
negation and entailment operators respectively) - but this is a later
(and optional) development. The speakers of the core primordial
language understand incompatibility relations between atomic
sentences, but use no complex logical connectives.

Brandom treats incompatibility relations \emph{extensionally}:
incompatibility is introduced as a set of sets of sentences which are
jointly incompatible.  Once the speakers of the primordial language
learn this incompatibility relation (extensionally defined, as a set
of sets), they can use it to make entailment inferences.  Given a
sentence $\phi$ in a language $L$, define $\INC{\phi}$ as the set of sets of
sentences in $L$ that are incompatible with $\phi$.  The speakers of the
primordial language can use $\INC{\cdot}$ to make inferences, using the
equivalence\footnote{This equivalence is described in \cite{brandom2}:
  ``A relation of entailment, required for an inferential semantics,
  can be derived according to the principle that p entails q just in
  case everything incompatible with q is incompatible with p.'' For a
  further elaboration, see \cite{brandom}, especially Chapter 5 and
  Appendices.}:

\[
   \phi \mbox{ entails } \psi \quad \mbox{ iff }\quad \INC{\psi} \subseteq \INC{\phi}
\]

\NI This is a highly appealing picture of language development.  It
suggests a way language could have got off the ground\footnote{The
  primordial language described is an ``autonomous'' language game - a
  game you could play although you played no other. See
  \cite{brandom3}, Chapter 12.}, without requiring early language
users to make complex inferences with logical symbols.  But, as
described, the combinatorics of the situation make the incompatibility
set so large that the language is not learnable.

The trouble with characterising the incompatibility relation
\emph{extensionally} is that it quickly grows too large to be
learnable. We need to know that

\begin{itemize}

\item ``Jack is male'' is incompatible with ``Jack is female''
\item ``Jill is male'' is incompatible with ``Jill is female''
\item ``Joan is male'' is incompatible with ``Joan is female''
\item ``Jack is 39 years old'' is incompatible with ``Jack is 40 years old''
\item ``Jill is 39 years old'' is incompatible with ``Jill is 40 years old''
\item ``Joan is 39 years old'' is incompatible with ``Joan is 40 years old''
\item etc...

\end{itemize}

\NI Even if we restrict our attention to one-place predicates, if
there are $n$ objects and $p$ incompatible predicates, then there are
$n\cdot C^p_2$ facts to learn\martin{what is $C_2$?} \richard{$C_2$ is the choose operator as in 5 choose 2 = 5!/(3!2!)}.  This quickly becomes impractical as $n$ and
$p$ increases.  As Davidson has emphasized \cite{davidson}, for a
language to be learnable, we must discern some sort of recursive
structure so that the combinatorics can be tamed.

What is wanted, then, is a simple language (which does not contain
complex logical connectives) that has recursive structure which can
generate the multiplicity of incompatibility sets.  But \emph{how can
  we introduce recursive structure without introducing complex logical
  connectives?} EL$[\AND, !]$, to be introduced in Section Two, is
just such a language: it characterises the multiplicity of
incompatible sentence sets recursively, but without using complex
truth-functional connectives.

