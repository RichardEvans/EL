\section{Introduction}\label{introduction}

Natural language is full of incompatible alternatives.
If Pierre is the current king of France, then nobody else can simultaneously fill that role.
A traffic light can be green, amber or red - but it cannot be more than one colour at a time.
Mutual exclusion is a natural and ubiquitous concept.

Predicate logic \emph{can} represent mutually exclusive alternatives, of course.
To say that Pierre is the \emph{only} king of France, we can write, following Russell:
\[
king(france, pierre) \land \forall x . king(france, x) \rightarrow x = pierre
\]
To say that a particular traffic light, $tl$, is red - and red is its \emph{only} colour - we could write:
\[
colour(tl, red) \land \forall x . colour(tl, x) \rightarrow x = red
\]
In \fol{}, incompatibility is a \emph{derived} concept, reduced to 
a combination of universal quantification and identity.  \FOL{}, in other words, uses relatively complex machinery to express a
simple concept:
\begin{itemize}

\item Quantification is an expensive operation that comes with many
  rules governing its behaviour - such as the distinction between free
  and bound variables\footnote{Efficient handling free/bound variables
    is an active field of research, e.g.~nominal approaches to logic
    \cite{PittsAM:nomsetnasics}. The problem was put in focus in
    recent years with the rising interest in the computational cost of
    syntax manipulation in languages with binders.}.

\item The identity relation is also relatively expensive, as it
  requires an infinite collection of axioms to formalise the
  indiscernibility of identicals.

\end{itemize}

\NI The costs of quantification and identity, such as a larger proof
search space, have to be borne every time one uses incompatibility - even
though incompatibility does not, prima facie, appear to have anything to do
with the free/bound variable distinction, or require the full power of 
the identity relation.

This paper introduces an alternative, \cathoristic{}, where
incompatibility is expressed directly, as a first-class concept -
rather than being relegated to second-class status as a defined
concept. \Cathoristic{}\footnote{``Cathoristic'' comes from the Greek
  $\kappa \alpha \theta o \rho$ $\iota \zeta \omega$: to narrow the
  boundaries. We are grateful to Tim Whitmarsh for suggesting this
  word.} is the simplest logic we could find in which incompatible
statements can be expressed.  To express facts, \cathoristic{} uses
the has may-modalities of Hennessy-Milner logic.  But in
\cathoristic{}, there is no negation, disjunction or
implication. Because of this simplicity, it has a quadratic-time
decision procedure.

\Cathoristic{} is a multi-modal logic, a variant of Hennessy-Milner Logic,
that replaces negation with a different logical connective:
\[
   !A
\]
called \emph{tantum} $A$. Here, $A$ is a finite set of alternatives
that exhaust all possibilities.  For example:
\begin{eqnarray*}
\fBang \{green, amber, red\}
\end{eqnarray*}
We use modalities to state  facts, for example
\[
   \MAY{amber}{}\ \AND\ !\{green, amber, red\} 
\]
expresses that $amber$ is currently the case and $red$ as well as
$green$ are the only two possible alternatives to $amber$.  Any
statement that exceeds what tantum $A$ allows, like
\[
   !\{green, amber, red\}\ \AND\ \MAY{blue},
\]
is necessarily false.  When the only options are green, amber, or red,
then blue is not available.  Now to say that Pierre is the only king
of France, we write:
\[
\MAY{king}\MAY{france}(\MAY{pierre} \land \fBang \{pierre\})
\]
Crucially, \cathoristic{}'s representation involves no
universal quantifier and no identity relation.  It is a purely
propositional representation.  To say that the traffic
light is currently red, and red is its only colour, we write:
\[
\MAY{tl} \MAY{colour} (\MAY{red} \land !\{red\})
\]
This is simpler (both in terms of representation length and
computational complexity) than the formulation in \fol{}:
\[
colour(tl, red) \land \forall x . colour(tl, x) \rightarrow x = red
\]
If we want to be able to describe properties changing over time, we
can add an extra transition representing the time-stamp.  To say that
that the traffic light was red at time $t_1$ and amber at time $t_2$,
we can write:
\[
   \MAY{tl} \MAY{colour} (\MAY{t1} (\MAY{red} \land !\{red\}) \land \MAY{t2} (\MAY{amber} \land !\{amber\}))
\]
Change over time can be expressed in first-order logic with bounded
quantification - but modalities do this more succinctly, without introducing bound variables.

Having claimed that incompatibility is a natural logical concept, not
easily expressed in first-order logic\footnote{We will precisify this
  claim in later sections; first order logic's representation of
  incompatibility is longer in terms of formula length than \cathoristic{}'s (see
  Section \ref{incompatiblepredicatesinfol}); and (ii) logic programs
  in \cathoristic{} can be optimised to run significantly faster than their
  equivalent in \fol{} (see Section \ref{optimizingpreconditions}).}, we
will now argue the following:

\begin{itemize}

\item Incompatibility is conceptually prior to negation.

\item In particular, negation arises as the weakest form of incompatibility.

\end{itemize}

\subsection{Material incompatibility is conceptually prior to negation}

\NI Every English speaker knows that
\begin{quote}
``Jack is male'' is incompatible with ``Jack is female''
\end{quote}

\NI But \emph{why} are these sentences incompatible? The orthodox
position is that these sentences are incompatible because of the
following general law:
\begin{quote}
If someone is male, then it is not the case that they are female
\end{quote}
Recast in first order logic:
\[
\forall x. male(x) \Rightarrow \neg female(x)
\]
According to the orthodox position, the incompatibility between the
two particular sentences depends on a general law involving universal
quantification, implication and negation.

Brandom \cite{brandom2} follows Sellars in proposing an alternative explanation.
Brandom contrasts a formalist explanation (attempting to explain incompatibility in terms of negation) with a pragmatist explanation (explaining negation as a degenerate form of incompatibility):
\begin{quote}
The question is how one ought to construe the relation between what is explicit in the form of a rule or principle and what is implicit in proprieties of practice. 
The formalist line of thought begins with explicit propositional licenses that license inferences in virtue of their logical form. Material inferences are understood privatively: as enthymemes resulting from the suppression or hiding of one of the premises required for a proper warrant. 
Opposed to this might be a pragmatist line of thought, beginning with material inferences\footnote{\cite{brandom2}, p.101.}.
\end{quote}
According to the alternative, pragmatist strategy of Sellars and Brandom, ``Jack
is male'' is incompatible with ``Jack is female'' because ``is male''
and ``is female'' are \emph{materially incompatible} predicates.  They
claim we can understand incompatible predicates even if we do
not understand universal quantification or negation.  
Material incompatibility is conceptually prior to logical negation.

Imagine, to make this vivid, a primitive people speaking a primordial
language of atomic sentences\footnote{In this paper, we define a sentence as \emph{atomic} if it does
  not contain another sentence as a syntactic constituent.}. These people can express sentences
that \emph{are} incompatible.  But they cannot express \emph{that}
they are incompatible.  They recognise when atomic sentences are
incompatible, and see that one sentence entails another - but their
behaviour outreaches their ability to articulate it.

Over time, these people \emph{may} advance to a more
sophisticated language where incompatibilities are made explicit', using some sort of negation operator - but this is a later (and optional) development:
\begin{quote}
[If negation is added to the language], it lets one say that two
claims are materially incompatible:``If a monochromatic patch is red,
then it is not blue.'' That is, negation lets one make explicit in the
form of claims - something that can be said and (so) thought - a
relation that otherwise remained implicit in what one practically did,
namely treat two claims as materially
incompatible\footnote{\cite{brandom} pp.47-48}.
\end{quote}

\NI But before making this optional
explicitating step, our primitive people understand incompatibility
without understanding negation.  If this picture of our primordial
language is coherent, then material incompatibility is conceptually
independent of logical negation.

Now imagine a modification of our primitive linguistic practice in
which no sentences are ever treated as incompatible.  If one person
says ``Jack is male'' and another says ``Jack is female'', nobody
counts these claims as \emph{conflicting}.  The native speakers never
disagree, back down, retract their claims, or justify them. They just
say things.  Without an understanding of incompatibility, and the
variety of behaviour that it engenders, we submit (following Brandom)
that there is insufficient richness in the linguistic practice for
their sounds to count as assertions.  Without material
incompatibility, their sounds are just \emph{barks}.
\begin{quote}

  Suppose the reporter's differential responsive dispositions to call
  things red are matched by those of a parrot trained to utter the
  same noises under the same stimulation. What practical capacities of
  the human distinguish the reporter from the parrot? What, besides
  the exercise of regular differential responsive dispositions, must
  one be able to \emph{do}, in order to count as having or grasping
  \emph{concepts}? ... To grasp or understand a concept is, according
  to Sellars, to have practical mastery over the inferences it is
  involved in... The parrot does not treat ``That's red'' as
  incompatible with ``That's green''\footnote{\cite{brandom2}
    pp.88-89, our emphasis.}.
\end{quote}

\NI If this claim is also accepted, then material incompatibility is
not just conceptually \emph{independent} of logical negation, but is
conceptually \emph{prior}.  

\subsection{Negation as logically weakest form of material incompatibility}

In \cite{brandom2} and \cite{brandom}, Brandom describes 
logical negation as a limiting form of material incompatibility:
\begin{quote}
Incompatible sentences are Aristotelian \emph{contraries}. A sentence
and its negation are \emph{contradictories}. What is the relation
between these? Well, the contradictory is a contrary: any sentence is
incompatible with its negation. What distinguishes the contradictory
of a sentence from all the rest of its contraries? The contradictory
is the \emph{minimal} contrary: the one that is entailed by all the
rest. Thus every contrary of ``Plane figure $f$ is a circle'' - for
instance ``$f$ is a triangle'', ``$f$ is an octagon'', and so on -
entails ``$f$ is \emph{not} a circle''.
\end{quote}

\NI If someone asserts, say, that Jack does not support Manchester United, we have so far said very little.  
There are so many different ways in which it could be true that Jack does not support Manchester United\footnote{We are assuming, in this example, that ``supports'' is a many-one relation - that a person can only authentically support a single football team.}:
\begin{itemize}
\item
He might support Arsenal
\item
He might support Chelsea
\item
...
\item
He might not support any football team at all
\item
There may be nobody answering to the name ``Jack''
\end{itemize}
These possibilities are concrete propositions which are incompatible with Jack supporting Manchester United.
To say ``Jack does not support Manchester United'' is just to claim that one of these indefinitely many concrete possibilities is true.
Negation is just the logically weakest form of incompatibility.

In the rest of this paper, we assume - without further argument - that material incompatibility is conceptually prior to logical negation.
We develop a simple
 modal logic to articulate Brandom's intuition formally: a language, without negation, in which we can nevertheless make incompatible claims\footnote{Although this logic is inspired by Brandom's thoughts, this is not to say that he himself would endorse the approach described in this paper. As a pragmatist, Brandom stresses that dispositions or practices are prior to logical codification - so he might not see the need for, or wisdom of, formalising pre-logical practices.}.

\subsection{Intra-sentential logical concepts}
\label{intrasentential}
So far, we have justified the claim that incompatibility is a fundamental logical concept by arguing (following Brandom) that incompatibility is conceptually prior to negation. 

Now incompatibility is an inferential relation between \emph{atomic sentences}. 
In this subsection, we shall describe \emph{other} inferential relations between atomic sentences - inferential relations that \fol{} cannot articulate, but that \cathoristic{} handles naturally.

The \emph{atomic sentences} of a natural language can be
characterised as the sentences which do not contain any other
sentences as constituent parts\footnote{Compare Russell \cite{russell}
  p.117: ``A sentence is of atomic form when it contains no logical
  words and no subordinate sentence''. We use a broader notion of
  atomicity by focusing solely on whether or not it contains a
  subordinate sentence, allowing logical words such as ``and'' \emph{as long
  as they are conjoining noun-phrases} and not sentences.}.  According
to this criterion, the following are atomic:

\begin{itemize}

\item Jack is male
\item Jack loves Jill
\end{itemize}

\NI The following is not atomic:

\begin{quote}
  Jack is male and Jill is female
\end{quote}

\NI because it contains the complete sentence ``Jack is male'' as a
syntactic constituent.  Note that, according to this criterion, the
following \emph{is} atomic, despite using ``and'' :

\begin{quote}
  Jack loves Jill and Joan
\end{quote}

\NI Here, ``Jack loves Jill'' is not a syntactic constituent\footnote{To see that ``Jack loves Jill'' is not a constituent of ``Jack loves Jill and Joan'', observe that ``and'' conjoins constituents of the \emph{same syntactic type}. But ``Jack loves Jill'' is a sentence, while ``Joan'' is a noun. Hence the correct parsing is ``Jack (loves (Jill and Joan))'', rather than ``(Jack loves Jill) and Joan''.}.

There are many types of inferential relations between atomic
sentences of a natural language.  For example:

\begin{itemize}

\item ``Jack is male'' is incompatible with ``Jack is female''
\item ``Jack loves Jill'' implies ``Jack loves''
\item ``Jack walks slowly'' implies ``Jack walks''
\item ``Jack loves Jill and Joan'' implies ``Jack loves Jill''
\item ``Jack is wet and cold'' implies ``Jack is cold''

\end{itemize}

\NI Some of these inferential relations are \emph{incompatibility}
relations (two sentences cannot both be true) while others are
\emph{entailment} relations (if one sentence is true, the other must
also be true).  The main question this paper seeks to answer is:
\emph{what is the simplest logic that can capture these inferential
  relations?}

\subsection{Wittgenstein's vision of a logic of elementary propositions}

\NI In the \emph{Tractatus} \cite{wittgenstein-tractatus}, Wittgenstein
claimed that the world is a set of atomic sentences in an idealized
logical language.  Each atomic sentence was supposed to be
\emph{logically independent} of every other, so that they could be
combined together in every possible permutation, without worrying
about their mutual compatibility.
But already there were doubts and problem cases.  He was aware that
certain sorts of statements seemed atomic, but did not seem logically
independent:

\begin{quote}
  For two colours, e.g., to be at one place in the visual field is
  impossible, and indeed logically impossible, for it is excluded by
  the logical structure of colour. (6.3751)
\end{quote}

\NI At the time he was writing the Tractatus, he hoped that further
analysis would reveal that these statements were not really atomic.

In the \emph{Philosophical Remarks} \cite{wittgenstein-remarks}, he
renounced the thesis of the logical independence of atomic
propositions.  In \S 76, talking about incompatible colour predicates,
he writes:

\begin{quote}
  That makes it look as if \emph{a construction might be possible
    within the elementary proposition}. That is to say, as if there
  were a construction in logic which didn't work by means of truth
  functions.  What's more, it also seems that these constructions have
  an effect on one proposition's following logically from another.
  For, if different degrees exclude one another it follows from the
  presence of one that the other is not present.  In that case,
  \emph{two elementary propositions can contradict one another}.
\end{quote}

\NI Here, he is clearly imagining a logical language in which there
are incompatibilities between atomic propositions. In \S 82:

\begin{quote}
  This is how it is, what I said in the Tractatus doesn't exhaust the
  grammatical rules for 'and', 'not', 'or', etc; \emph{there are rules
    for the truth functions which also deal with the elementary part
    of the proposition}.  The fact that one measurement is right
  \emph{automatically} excludes all others.
\end{quote}

\NI Wittgenstein does not, unfortunately, show us what this
language would look like.  \Cathoristic{} is one way of formalising these sorts of inferences
between atomic sentences.

\subsection{Outline}

\NI The rest of this paper is organised as follows: Section
\ref{coreEL} introduces the syntax and semantics of \cathoristic{} with
examples. Section \ref{naturalLanguageInference} discusses how
\cathoristic{} can be used to model adverbial inferences and inferences
from dyadic to monadic s.  Section \ref{kr} describes
informally how \cathoristic{} is useful as a knowledge representation
language.  Section \ref{elAndBangCore} presents core results of the
paper, in particular a semantic characterisation of elementary
equivalence and a decision procedure with quadratic time-complexity. The
decision procedure has been implemented in Haskell and is available
for public use \cite{HaskellImplementation} under a liberal
open-source license.\martin{add a license blurb on Github}.  This
section also shows that Brandom's incompatibility semantics condition
holds for \cathoristic{}.  Section \ref{elAndBangMore} presents the
proof rules for \cathoristic{} and proves completeness. Section
\ref{compactness} provides two translations from \cathoristic{} into
\fol{}, and proves compactness using one of them.  Section
\ref{ELAndNegation} investigates the role of negation in \cathoristic{}
in two ways: by introducing a variant of \cathoristic{} with negation,
and with a decision procedure for this fragment that has an
exponential time-complexity.  Section \ref{relatingELToOtherLogics}
relates \cathoristic{} and Hennessy-Milner Logic.  Section
\ref{quantifiedEL} extends \cathoristic{} with first-order quantifiers
and sketches the translation of first-order formulae into first-order
\cathoristic{}. The conclusion surveys related work and lists open
problems.  Appendix \ref{pureModels} outlines a different approach to
giving the semantics of \cathoristic{}, including a characterisation of
the corresponding elementary equivalence. The appendix also discusses
the question of non-deterministic models. The remaining appendices
present routine proof of facts used in the main section.
