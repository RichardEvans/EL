\section{Using \Cathoristic{} as a knowledge representation language}\label{kr}


\Cathoristic{} has been used as the representation language for a
large, complex, dynamic multi-agent simulation \cite{evans-and-short}.
This is an industrial-sized application, involving tens of thousands
of rules and facts.\martin{Maybe also mention that it's a commercial
  service with a large numbert of paying users, maybe also mention it
  was part of Second Life?}  In this program, the entire world state
is stored as a set of \cathoristic{} formulae - all other aspects of
world state are transient and computed as needed.  The current true
sentences are represented in a deterministic labeled transition
system.\martin{Can we say ``... are represented in a cathoristic model''?}
\martin{What is the relation of the CL formulae and the LTS?}

We shall first sketch how facts are represented, before describing some
advantages of \cathoristic{} over a more traditional \fol{}
representation.

\subsection{Representing facts  in \cathoristic{}}

We used the following strategy for representing facts in \cathoristic{}.  
(The translation is detailed later in Section
\ref{translationFOLtoFOEL}).  
A sentence involving a one-place predicate of the form $p(a)$ is
expressed in \cathoristic{} as
\begin{eqnarray*}
   \MAY{a} \MAY{p}
\end{eqnarray*}

\NI A sentence involving a many-to-many two-place relation of the form
$r(a,b)$ is expressed in \cathoristic{} as
\begin{eqnarray*}
\MAY{a} \MAY{r} \MAY{b}
\end{eqnarray*}
But a sentence involving a many-to-one two-place relation of the form $r(a,b)$ is expressed in \cathoristic{}:
\begin{eqnarray*}
\MAY{a} \MAY{r} (\MAY{b} \land \fBang \{b\})
\end{eqnarray*}
So, for example, to say that ``Jack likes Jill'' (where ``likes'' is, of course, a many-many relation), we would write:
\begin{eqnarray*}
\MAY{jack} \MAY{likes} \MAY{jill}
\end{eqnarray*}
But to say that ``Jack is married to Joan'' (where``is-married-to'' is a many-one relation), we would write:
\begin{eqnarray*}
\MAY{jack} \MAY{married} (\MAY{joan} \land \fBang \{joan\})
\end{eqnarray*}

\NI Colloquially, we might say that ``Jack is
married to Joan - and only Joan''.  Note that the  relations are
placed in infix position,  so that the facts about an object are
``contained'' within the object.   But we \emph{could}
have chosen an alternative strategy, in which the predicate/relation
symbols are to the left of the first argument.
 
Consider the following facts about a gentleman named Brown:

\[
   \MAY{brown} 
   \left(
   \begin{array}{l}
     \MAY{sex} (\MAY{male} \land \fBang \{male\}) \\
        \qquad \AND \\
     \MAY{friends} (\MAY{lucy} \land \MAY{elizabeth}) 
   \end{array}
   \right)
\]

\NI All the facts which start with the prefix $\MAY{brown}$ form a
sub-tree of the entire database.  And all the facts which start with
the prefix $\MAY{brown} \MAY{friends}$ form a sub-tree of that tree.
A sub-tree can be treated as an individual via its prefix.  
A sub-tree of formulae is the \cathoristic{} equivalent of an
\emph{object} in an object-oriented programming language.

To model change over time, we assert and retract statements from the database, using a non-monotonic update mechanism.
If a fact is inserted into the database that involves a node-labelling restricting the available transitions emanating from that node, then all transitions out of that node that are incompatible with the restriction are removed.
So, for example, if the database currently contains the fact that the traffic light is amber, and then we update the database to assert the traffic light is red:
\[
\MAY{tl} \MAY{colour} (\MAY{red} \land !\{red\})
\]
Now the restriction on the node (that red is the only transition) means that the previous transition from that node (the transition labelled with amber) is automatically removed.

The tree-structure of formulae allows us to express the \emph{life-time of data} in a natural way. 
If we wish a piece of data $d$ to exist for just the duration of a proposition $t$, then we make $t$ be a sub-expression of $d$. 
For example, if we want the friendships of an agent to exist just as long as the agent, then we place the relationships inside the agent: 
\[
\MAY{brown} \MAY{friends}
\]
Now, when we remove $\MAY{brown}$ all the sub-trees, including the data about who he is friends with, will be automatically deleted as well.

Another advantage of our representation is that we get a form of \emph{automatic currying} which simplifies queries.
So if, for example, Brown is married to Elizabeth, then the database would contain 
\begin{eqnarray*}
\MAY{brown} \MAY{married} (\MAY{elizabeth} \land \fBang \{elizabeth\})
\end{eqnarray*}
In \cathoristic{}, if we want to find out whether Brown is married, we can query the sub-formula directly -  we just ask if 
\begin{eqnarray*}
\MAY{brown} \MAY{married}
\end{eqnarray*}
In \fol, if $married$ is a two-place predicate, then we need to fill in the extra argument place with a free variable - we would need to find out if there exists an $X$ such that $married(brown, X)$ - this is slower to compute and more cumbersome to type. 

\subsection{Simpler postconditions}

In this section, we contrast the representation in action languages based on \fol{}\footnote{E.g. STRIPS \cite{strips}}, with our \cathoristic{}-based representation.
Action definitions are rendered in typewriter font.

When expressing the pre- and post-conditions of an action, planners
based on \fol{} have explicitly to describe the propositions that
are removed when an action is performed:
\begin{verbatim}
   action move(A, X, Y)
       preconditions
           at(A, X)
       postconditions
           add: at(A, Y) 
           remove: at(A, X)
\end{verbatim}
Here, we need to explicitly state that when $A$ moves from $X$ to $Y$, $A$ is no longer at $X$. It might seem obvious to us that if $A$ is now at $Y$, he is no longer at $X$ - but we need to explicitly tell the system this. This is unnecessary, cumbersome and error-prone. In \cathoristic{}, by contrast, the exclusion operator means we do not need to specify the facts that are no longer true:
\begin{verbatim}
   action move (A, X, Y)
       preconditions
           <A><at>(<X> /\ !{X})
       postconditions
           add: <A><at>(<Y> /\ !{Y})
\end{verbatim}
The ``!" operator makes it clear that something can only be at one
place at a time, and the non-monotonic update rule described above
\emph{automatically} removes the old invalid location data.

\subsection{Using tantum $!$  to optimise preconditions}
\label{optimizingpreconditions}
Suppose, for example, we want to find all married couples who are both Welsh.
In Prolog, we might write something like:
\begin{verbatim}
   married_couple_at_home(X, Y) :-
       welsh(X),
       welsh(Y),
       spouse(X,Y).
\end{verbatim}	
Rules like this create a large search-space because we need to find all instances of $welsh(X)$ and all instances of  $welsh(Y)$ and take the cross-product \cite{smith-and-genesereth}. If there are $n$ Welsh people, then we will be searching $n^2$ instances of $(X,Y)$ substitutions.

If we express the rule in \cathoristic{}, the compiler is able to use the extra information expressed in the $!$ operator to reorder the literals to find the result significantly faster.
Assuming someone can only have a single spouse at any moment, the rule is expressed in \cathoristic{} as:
\begin{verbatim}
   married_couple_at_home(X, Y) :-
       <welsh> <X>,
       <welsh> <Y>,
       <spouse> <X> (<Y> /\ !{Y}).
\end{verbatim}	
Now the compiler is able to reorder these literals to minimize the search-space. 
It can see that, once $X$ is instantiated, the following literal can be instantiated without increasing the search-space:
\begin{verbatim}
   <spouse> <X> (<Y> /\ !{Y})
\end{verbatim}
The \emph{tantum} operator can be used by the compiler to see that there is at most one $Y$ who is the spouse of $X$.
So the compiler reorders the clauses to produce:
\begin{verbatim}
   married_couple_at_home (X, Y) :-
       <welsh> <X>,
       <spouse> <X> (<Y> /\ !{Y}),
       <welsh> <Y>.
\end{verbatim}	
Now it is able to find all results by just searching $n$ instances - a significant optimization.
In our application, this optimisation has made a significant difference to the run-time cost of query evaluation.






