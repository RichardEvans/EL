\section{Using Eremic Logic as a Knowledge Representation Language}
Eremic Logic has been found to be a highly-ergonomic language for the representation of complex simulational state.
It has been used as the representation language for a large, complex, highly-dynamic multi-agent simulation \cite{evans-and-short}.
The entire world state is stored as a set of formulae - all other aspects of world state are transient and computed as needed.
The entire world state is stored as a Labelled Rooted Tree, using the non-monotonic update mechanism described above.
The application in which Eremic Logic has been proven is not a toy academic problem, but an industrial-sized application involving tens of thousands of rules and tens of thousands of atomic facts.

When applying it in a large knowledge base, Eremic Logic has been found to have a number of advantages over traditional predicate logic as a language to represent simulational state:
\begin{itemize}
\item
The notion of a sub-tree in Eremic Logic gives us a natural way to represent \emph{objects}
\item
The update rule (described above) means that \emph{garbage collection} of invalid data happens automatically.
\item
Eremic logic allows a simpler less error-prone way of specifying \emph{postconditions} of actions
\item
The $!$ operator gives additional information to the type-checker, allowing an author to specify her intent more precisely
\item
The $!$ operator provides additional information to the compiler, allowing significantly more efficient clause ordering
\end{itemize}
I shall go through each of these in turn.

\subsection{Using Sub-trees of Expressions to Represent Objects}
Consider the following facts about Brown\footnote{The database is composed of Y-formulae. For readability, the trailing $\star \top$ has been omitted. In Praxis, constants are identifiers starting with a lower-case letter while variables start with an upper-case letter.}:
\begin{verbatim}
brown *
    (
        sex!male /\
        class!upper
        in!dining_room /\
        friends *
        (
            lucy /\
            elizabeth
        )
)
\end{verbatim}
All the facts which start with the prefix ``brown'' form a sub-tree of the entire database.
And all the facts which start with the prefix ``brown * friends'' form a sub-tree of that tree.
A sub-tree can be treated as an individual via its prefix.
The whole sub-tree can be removed in one fell swoop by deleting its associated prefix.
So we can e.g. remove Brown's friends just by deleting the prefix ``brown * friends\'.
(Compare this with Prolog - where it is much harder to remove all formulae containing a particular symbol).
A sub-tree of formulae is the Eremic Logic equivalent of an \emph{object} in an object-oriented programming language.

The tree-structure of formulae also allows us to express the \emph{life-time of data} in a natural way. 
If we wish a piece of data $d$ to exist for just the duration of a proposition $t$, then we make $t$ be a sub-expression of $d$. 
For example, if we want the friendships of an agent to exist just as long as the agent, then we place the relationships inside the agent: 
\begin{verbatim}
brown * friends
\end{verbatim}
Or if we want the score of a game to exist just as long as the game itself, then we place the score data inside the social practice for the game: 
\begin{verbatim}
process * whist * data * whose_move ! brown
\end{verbatim}

Another advantage of using a tree structure is that we get a form of \emph{automatic currying} which simplifies queries.
So if, for example, Brown is married to Elizabeth, then the database would contain 
\begin{verbatim}
brown * married ! elizabeth
\end{verbatim}
In Eremic Logic, if we want to find out whether Brown is married, we can query the sub-formula directly -  we just ask if 
\begin{verbatim}
brown * married
\end{verbatim}
In traditional predicate logic, if married is a two-place predicate, then we need to fill in the extra argument place with a free variable - we would need to find out if there exists an $X$ such that ``married(brown, X)'' - this is slower to compute and more cumbersome to type. 

\subsection{Automatic Garbage Collection}
The $!$ operator supports automatic clean up of dead data.
For example, an object $p$ might have two states: $a$ and $b$. State $a$ might have two pieces of information, $x1$ and $x2$. State $b$ might have three pieces of information $y1$, $y2$ and $y3$. Being in state $a$ would be represented as:
\[p \; ! \; a \star  (x1 \land x2)\]
State $b$ would be represented as:
\[p \; ! \; b \star (y1 \land y2 \land y3)\]
Now that if we are in state $a$, and we switch to state $b$ by inserting $p \fBang b$ into the database, all the local data from  state $a$ is \emph{automatically} removed from the database according to the update rules for Eremic Logic. Everything is cleaned up \emph{automatically}.

\subsection{Simpler Postconditions}
When expressing the pre- and post-conditions of an action, planners based on predicate logic\footnote{An early example is STRIPS \cite{strips}.} have to explicitly describe the propositions that are removed when an action is performed:
\begin{verbatim}
action move(A, X, Y)
    preconditions
        at(A, X)
    postconditions
        add: at(A, Y) 
        remove: at(A, X)
\end{verbatim}
Here, we need to explicitly state that when $A$ moves from $X$ to $Y$, $A$ is no longer at $X$. It might seem obvious to us that if $A$ is now at $Y$, he is no longer at $X$ - but we need to explicitly tell the system this. This is unnecessary, cumbersome and error-prone. In Eremic Logic, by contrast, the exclusion operator means we do not need to specify the facts that are no longer true\footnote{Note that the rule is a horn-clause of Y-formulae, not X-formulae. Note also that the trailing $\star \top$ has been suppressed for readability.}:
\begin{verbatim}
action move * A * X * Y
    preconditions
        A * at ! X
    postconditions
        add: A * at ! Y
\end{verbatim}
The ``!" operator makes it clear that something can only be at one place at a time, and the non-monotonic update rule \emph{automatically} removes the old invalid location data.

\subsection{The $!$ Operator Helps Remove Errors During Type-Checking}
The $!$ operator allows the simulation author to specify her intent more precisely.
When we specify, for example, the type-signature for an agent's gender:
\begin{verbatim}
A(agent) * sex ! G(gender)
\end{verbatim}
We are saying that an agent has \empty{exactly one} gender.
This exclusion-information is available to the type-checker, which can rule out at initialisation-time any piece of code which assumes that an agent could have multiple genders. 

Some modern logic-programming languages are starting to add the ability to specify uniqueness properties of predicates \cite{mercury}. 
But they treat uniqueness properties as meta-linguistic predicates. 
Eremic Logic is the first language to treat uniqueness as a first-class element of the language.

\subsection{Using the $!$ Operator to Optimize Logic Programs}
Suppose, for example, we want to find all married couples who are currently at home.
In Prolog, we might write something like:
\begin{verbatim}
married_couple_at_home(X, Y) :-
    at_home(X),
    at_home(Y),
    spouse(X,Y).
\end{verbatim}	
Rules like this create a large search-space because we need to find all instances of $at\_home(X)$ and all instances of  $at\_home(Y)$ and take the cross-product \cite{smith-and-genesereth}. If there are $n$ instances of $at\_home$, then we will be searching $n^2$ instances of $(x,y)$ substitutions.

If we express the rule in Eremic Logic, the compiler is able to use the extra information expressed in the $!$ operator to reorder the clauses to find the result significantly faster.
Since someone can only have a single spouse at any moment, the rule is expressed in Eremic Logic as\footnote{Note that the rule is a horn-clause of Y-formulae, not X-formulae. Note also that the trailing $\star \top$ has been suppressed for readability.}:
\begin{verbatim}
married_couple_at_home * X * Y :-
    at_home * X,
    at_home* Y,
    spouse * X ! Y.
\end{verbatim}	
Now the compiler is able to reorder these clauses to minimize the search-space. 
It can see that, once $X$ is instantiated, evaluating $spouse * X \; ! \; Y$ involves no increase to the search-space.
So it reorders the clauses to produce:
\begin{verbatim}
married_couple_at_home * X * Y :-
    at_home * X,
    spouse * X ! Y,
    at_home* Y.
\end{verbatim}	
Now it is able to find all results by just searching $n$ instances - a significant optimization.

