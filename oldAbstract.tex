\begin{abstract}
\ELFULL{} is a modal logic with a novel operator expressing \emph{exclusion} between sentences. 

Instead of the traditional negation operator, \ELFULL{} uses a simple non-compositional exclusion operator. 
At a high level, the exclusion operator and the negation operator play the \emph{same role}: they are both used to make incompatible claims.

Because it does not have complexifying logical operators (~, \/, =>), \ELFULL{} has a \emph{linear-time decision procedure}.
Nevertheless, despite its simplicity, it is expressive enough to satisfy the Hennessy-Milner Theorem and Brandom's Incompatibility Semantics property.

Our interest in \ELFULL{} derives from our belief that it occupies a sweet spot in logical space: it is expressive enough to satisfy key theorems (such as Hennessy-Milner) but simple enough for validity to be decidable in linear-time.

In the introductory section, we motivate the preference for exclusion over negation, by appealing to the arguments of Robert Brandom.

Then we present the syntax, semantics, and proof rules for \ELFULL{}, and establish key meta-theorems such as completeness and compactness.  
We provide a decision procedure, and show that it is linear-time.
We show that the syntactic notion
of elementary equivalence induced by the the formulae of \ELFULL{}
coincides with the semantic notion of mutual simulation on models.  

\ELFULL{} was designed to express logical inferences between \emph{atomic sentences}. 
We show how ELFULL{} can capture these inferences, and how it has been used as the representational core of a large industrial knowledge-based system.




\martin{here is an alternative abstract. What do you think?}

\ELFULL{} is a modal logic with a novel operator expressing
\emph{exclusion} between sentences. Exclusion can be used to express
the concept of negation.  We present the syntax, proof rules, and
semantics of eremic logic, and establish key meta-theorems such as
completeness and compactness.  We also show that the syntactic notion
of elementary equivalence induced by the the formulae of eremic logic
coincides with the semantic notion of bisimilarity on models.  Then we
give a linear-time decision procedure for an important fragment of of
eremic logic. This fragment has been used in an industrial
application.  We argue that this fragment is a suitable query language
for knowledge representation and logic programming. We also further
elusidate eremic logic by translation to first-order logic as well as
Hennessy-Milner logic. \martin{do we go the other way round?} We
conclude with a discussion of using \ELABR{} to model features of
natural language, such as inferences between ``atomic'' sentences,
refining Brandom's incompatibility semantics.

\end{abstract}
