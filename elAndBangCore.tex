\section{Semantics and Decision Procedure}\label{elAndBangCore}

In this section we provide our key semantic results. 
We define a partial ordering $\MODELLEQ$ on models, and show how the partial ordering can be extended into a bounded lattice.
We use the bounded lattice to provide a decision procedure.

\subsection{A semantic characterisation of elementary equivalence}\label{elementaryEquivalence}

Elementary equivalence induces a notion of model equivalence: two
models are elementarily equivalent exactly when they make the same
formulae true. Elementary equivalent as a concept thus relies on
\cathoristic{} even for its definition. We now present an alternative
characterisation that is purely semantic, using the concept of
(mutual) simulation from process theory. Apart from its intrinsic
interest, this characterisation will also be crucial for proving
completeness of the proof rules.

We first define a pre-order $\MODELLEQ$ on models by extending the
notion of simulation on labeled transition systems to cathoristic
models. Then we prove an alternative characterisation of $\MODELLEQ$
in terms of set-inclusion of the theories induced by models. We then
show that two models are elementarily equivalent exactly when they are
related by $\MODELLEQ$ and by $\MODELLEQ^{-1}$.

\begin{definition}
Let $\LLL_i = (S_i, \rightarrow_i, \lambda_i)$ be cathoristic transition
systems for $i = 1, 2$.  A relation $\RRR \subseteq S_1 \times S_2$ is
a \emph{simulation from $\LLL_1$ to $\LLL_2$}, provided:
\begin{itemize} 

\item $\RRR$ is a simulation on the underlying transition systems. 

\item Whenever $(x, y) \in \RRR$ then also $\lambda_1(x) \supseteq
  \lambda_2(y)$.

\end{itemize}

\NI If $\MMM_i = (\LLL_i, x_i)$ are models, we say $\RRR$ is a
\emph{simulation from $\MMM_1$ to $\MMM_2$}, provided the following hold.

\begin{itemize}

\item $\RRR$ is a simulation from $\LLL_1$ to $\LLL_2$ as cathoristic transition systems.

\item  $(x_1, x_2) \in \RRR$. 

\end{itemize}

\end{definition}

\NI Note that the only difference from the usual definition of
simulation is the additional requirement on the node labelling
functions $\lambda_1$ and $\lambda_2$.

\begin{definition}
The largest simulation from $\MMM_1$ to $\MMM_2$ is denoted $\MMM_1
\SIM \MMM_2$.  It is easy to see that $\SIM$ is itself a
simulation from $\MMM_1$ to $\MMM_2$, and the union of all such
simulations.  If $\MMM_1 \SIM \MMM_2$ we say $\MMM_2$
\emph{simulates} $\MMM_1$.

We write $\MODELEQ$ for $\SIM \cap \SIM^{-1}$. We call $\MODELEQ$ the
\emph{multual simulation} relation.
\end{definition}

\begin{definition}
Let $\THEORY{\MMM}$ be the \emph{theory} of $\MMM$, i.e.~the formulae
made true by $\MMM$, i.e.~$\THEORY{\MMM} = \{\phi\ |\ \MMM \models
\phi \}$.
\end{definition}

\NI We give an alternative characterisation on $\SIM^{-1}$ in terms of
theories of models. In what follows, we will mostly be interested in
$\SIM^{-1}$, so we give it its own symbol.

\begin{definition}
Let $\MODELLEQ$  be short for $\SIM^{-1}$. 
\end{definition}

\NI Figure \ref{figure:leq} gives some examples of models and how they
are related by $\MODELLEQ$.

\begin{FIGURE}
\centering
\begin{tikzpicture}[node distance=1.3cm,>=stealth',bend angle=45,auto]
  \tikzstyle{place}=[circle,thick,draw=blue!75,fill=blue!20,minimum size=6mm]
  \tikzstyle{red place}=[place,draw=red!75,fill=red!20]
  \tikzstyle{transition}=[rectangle,thick,draw=black!75,
  			  fill=black!20,minimum size=4mm]
  \tikzstyle{every label}=[red]
  \begin{scope}[xshift=0cm]
    \node [place] (w1) {$\Sigma$};
    \node [place] (e1) [below left of=w1] {$\{b\}$}
      edge [pre]  node[swap] {a}                 (w1);      
    \node [place] (c) [below of=e1] {$\Sigma $}
      edge [pre]  node[swap] {b}                 (e1);      
    \node [place] (e2) [below right of=w1] {$\Sigma $}
      edge [pre]  node[swap] {c}                 (w1);      
  \end{scope}  
  
  \begin{scope}[xshift=3cm]
    \node [place] (w1) {$\Sigma $};
    \node [place] (e1) [below of=w1] {$\{b,c\}$}
      edge [pre]  node[swap] {a}                 (w1);      
    \node [place] (e2) [below of=e1] {$\Sigma $}
      edge [pre]  node[swap] {b}                 (e1);      
  \end{scope}  
  
  \begin{scope}[xshift=6cm]
    \node [place] (w1) {$\Sigma $};
    \node [place] (e1) [below of=w1] {$\Sigma $}
      edge [pre]  node[swap] {a}                 (w1);      
    \node [place] (e2) [below of=e1] {$\Sigma $}
      edge [pre]  node[swap] {b}                 (e1);      
  \end{scope}  
  
  \begin{scope}[xshift=9cm]
    \node [place] (w1) {$\Sigma $};
    \node [place] (e1) [below of=w1] {$\Sigma $}
      edge [pre]  node[swap] {a}                 (w1);      
  \end{scope}  
  
  \draw (2,0) node {$\MODELLEQ $};
  \draw (4.5,0) node {$\MODELLEQ $};
  \draw (7.5,0) node {$\MODELLEQ $};
  
\end{tikzpicture}
\caption{Examples of $\MODELLEQ $}\label{figure:leq}
\end{FIGURE}



\begin{theorem}[Characterisation of elementary equivalence]\label{theorem:completeLattice}
\begin{enumerate}

\item\label{theorem:completeLattice:1} $\MMM' \MODELLEQ \MMM$ if and
  only if $\THEORY{\MMM} \subseteq \THEORY{\MMM'}$.

\item\label{theorem:completeLattice:2} $\MMM' \MODELEQ \MMM$ if and
  only if $\THEORY{\MMM} = \THEORY{\MMM'}$.

\end{enumerate}
\end{theorem}

\begin{proof}
For (\ref{theorem:completeLattice:1}) assume $\MMM' \MODELLEQ \MMM$
and $\MMM \models \phi$.  We must show $\MMM' \models \phi$.  Let
$\MMM = (\LLL, w)$ and $\MMM' = (\LLL', w')$.  The proof proceeds by
induction on $\phi$.  The cases for $\top$ and $\land$ are trivial.
Assume $\phi = \MAY{a}\psi$ and assume $(\LLL, w) \models
\MAY{a}\psi$.  Then $w \xrightarrow{a} x$ and $(\LLL, x) \models
\psi$.  As $\MMM'$ simulates $\MMM$, there is an $x'$ such that
$(x,x') \in R$ and $w' \xrightarrow{a} x'$.  By the induction
hypothesis, $(\LLL', x') \models \psi$.  Therefore, by the semantic
clause for $\MAY{}$, $(\LLL', w') \models \MAY{a}\psi$.  Assume now
that $\phi = \; ! \; A$, for some finite $A \subseteq \Sigma$, and
that $(\LLL, w) \models \; ! \; A$.  By the semantic clause for $!$,
$\lambda(w) \subseteq A$.  Since $(\LLL', w') \MODELLEQ (\LLL, w)$, by
the definition of simulation of cathoristic transition systems, $\lambda(w)
\supseteq \lambda'(w')$.  Therefore, $\lambda'(w') \subseteq
\lambda(w) \subseteq A$.  Therefore, by the semantic clause for $!$,
$(\LLL', w') \models \; ! \; A$.

For the other direction, let $\MMM = (\LLL, w)$ and $\MMM' = (\LLL',
w')$.  Assume $\THEORY{\MMM} \subseteq \THEORY{\MMM'} $. We need to
show that $\MMM'$ simulates $\MMM$.  In other words, we need to
produce a relation $R \subseteq S \times S'$ where $S$ is the state
set of $\LLL$, $S'$ is the state set for $\LLL'$ and $(w,w') \in R$
and $R$ is a simulation from $(\LLL, w)$ to $ (\LLL', w')$.  Define $R
= \{(x,x') \; | \; \THEORY{ (\LLL, x)} \subseteq \THEORY{ (\LLL',
  x')}\}$.  Clearly, $(w,w') \in R$, as $\THEORY{(\LLL, w)} \subseteq
\THEORY{(\LLL', w')} $.  To show that $R$ is a simulation, assume $x
\xrightarrow{a} y$ in $\LLL$ and $(x,x') \in R$. 
We need to provide a
$y'$ such that $x' \xrightarrow{a} y'$ in $\LLL'$ and $(y,y') \in R$.  
Consider the formula $\MAY{a}\CHAR{(\LLL, y)}$. 
Now $x \models \MAY{a}\CHAR{(\LLL, y)}$, and since $(x,x') \in R$, $x' \models \MAY{a}\CHAR{(\LLL, y)}$.
By the semantic clause for $\MAY{a}$, if $x' \models \MAY{a}\CHAR{(\LLL, y)}$ then there is a $y'$ such that 
$y' \models \CHAR{(\LLL, y)}$.
We need to show $(y,y') \in R$, i.e. that $y \models \phi$ implies $y' \models \phi$ for all $\phi$.
Assume $y \models \phi$. 
Then by the definition of $\CHAR$, $\CHAR{(\LLL, y)} \models \phi$.
Since $y' \models \CHAR{(\LLL, y)}$, $y' \models \phi$. 
So $(y,y') \in R$, as required.

Finally,we need to show that whenever $(x,x') \in R$, then $\lambda(x)
\supseteq \lambda'(x')$.  Assume, first, that $\lambda(x)$ is finite.
Then $(\LLL, x) \models \; ! \; \lambda(x)$.  But as $(x,x') \in R$,
$\THEORY{(\LLL, x)} \subseteq \THEORY{(\LLL', x')} $, so $(\LLL', x')
\models \; ! \; \lambda(x)$.  But, by the semantic clause for $!$,
$(\LLL', x') \models \; ! \; \lambda(x)$ iff $\lambda'(x') \subseteq
\lambda(x)$.  Therefore $\lambda(x) \supseteq \lambda'(x')$.  If, on
the other hand, $\lambda(x)$ is infinite, then $\lambda(x) = \Sigma$
(because the only infinite node labelling that we allow is
$\Sigma$). Every node labelling is a subset of $\Sigma$, so here too,
$\lambda(x) = \Sigma \supseteq \lambda'(x')$.  

This establishes (\ref{theorem:completeLattice:1}), and
(\ref{theorem:completeLattice:2}) is immediate from the definitions.


\end{proof}

\NI Theorem
\ref{theorem:completeLattice}.\ref{theorem:completeLattice:1}
illustrates from a model-theoretic point of view one dimension of how
classical and \cathoristic{} differ. In classical logic the theory of
each model is complete, and $\THEORY{\CAL{M}} \subseteq
\THEORY{\CAL{N}}$ already implies that $\THEORY{\CAL{M}} =
\THEORY{\CAL{N}}$, i.e.~$\CAL{M}$ and $\CAL{N}$ are elementarily
equivalent. \Cathoristic{}'s lack of negation changes this drastically, and
gives $\MODELLEQ$ the structure of a non-trivial bounded lattice as we
shall demonstrate below.

But first we show that $\MODELEQ$ is a strictly coarser relation than 
bisimilarity.

\begin{definition}
We say $\RRR$ is a \emph{bisimulation} if $\RRR$ is a simulation from
$\MMM_1$ to $\MMM_2$ and $\RRR^{-1}$ is a simulation from $\MMM_2$ to
$\MMM_1$. By $\BISIM$ we denote the largest bisimulation, and we say
that $\MMM_1$ and $\MMM_2$ are \emph{bisimilar} whenever $\MMM_1
\BISIM \MMM_2$.
\end{definition}

Clearly $\BISIM \subseteq \MODELEQ$, but the two relations do not
coincide: the former is strictly more discriminating than the latter,
i.e.~$\BISIM \subsetneq \MODELEQ$. Figure \ref{figure:counterexample}
shows two models that are $\MODELEQ$-equal but not bisimilar. Note
that this counterexample is \emph{non-deterministic}. That is not
coincidence, because on deterministic cathoristic models, the two relations
coincide.

\begin{FIGURE}
\centering
\begin{tikzpicture}[node distance=1.3cm,>=stealth',bend angle=45,auto]
  \tikzstyle{place}=[circle,thick,draw=blue!75,fill=blue!20,minimum size=6mm]
  \tikzstyle{red place}=[place,draw=red!75,fill=red!20]
  \tikzstyle{transition}=[rectangle,thick,draw=black!75,
  			  fill=black!20,minimum size=4mm]
  \tikzstyle{every label}=[red]

  \begin{scope}[xshift=6cm]
    \node [place] (w1) {$\Sigma $};
    \node [place] (e1) [below of=w1] {$\Sigma $}
      edge [pre]  node[swap] {a}                 (w1);      
    \node [place] (e2) [below of=e1] {$\Sigma $}
      edge [pre]  node[swap] {b}                 (e1);      
  \end{scope}  

  \begin{scope}[xshift=10=cm]
    \node [place] (ww1) {$\Sigma $};
    \node [place] (ee1) [below left of=ww1] {$\Sigma $}
      edge [pre]  node {a}                 (ww1);      
    \node [place] (ee2) [below of=ee1] {$\Sigma $}
      edge [pre]  node[swap] {b}                 (ee1);      
    \node [place] (ee3) [below right of=ww1] {$\Sigma $}
      edge [pre]  node[swap] {a}                 (ww1);      
  \end{scope}  


\end{tikzpicture}
\caption{The models are equated by  $\MODELEQ$ but are not bisimilar}\label{figure:counterexample}
\end{FIGURE}



Theorem \ref{theorem:completeLattice} has various interesting
consequences.

\begin{corollary}
\begin{enumerate}

\item If $\phi$ has a model then it has a model who's underlying
  transition system is a tree, i.e.~all states except for the start state
  have exactly one predecessor, and the start state has no predecessors.

\item If $\phi$ has a model then it has a model where every state is
  reachable from the start state.

\end{enumerate}
\end{corollary}
\begin{proof}
Both are straightforward because $\MODELEQ$ (and bisimilarity) is
closed under tree-unfoldings as well as under removal of states not
reachable from the start state.
\end{proof}


\subsection{Quotienting Models}

\NI The relation $\MODELLEQ$ is not a partial order, only a
pre-order. For example with
\begin{itemize}

\item $\MMM_1 = ( (\{w\}, \{\}, \{w \mapsto \Sigma\}), w)$ and
\item $\MMM_2 = ( (\{v\}, \{\}, \{v \mapsto \Sigma\}), v)$ 

\end{itemize}

\NI we have models where $\MMM_1 \MODELLEQ \MMM_2$ and $\MMM_2
\MODELLEQ \MMM_1$, but the two models are not equal. The difference
between the two models, the name of the unique state, is trival and
not relevant for the formulae they make true. Indeed $\THEORY{\MMM_1}
= \THEORY{\MMM_2}$.  As briefly mentioned in the mathematical
preliminaries (Section \ref{preliminaries}), we obtain a proper
partial-order by simply quotienting models:

\[
   \MMM \MODELEQ \MMM'
      \qquad\text{iff}\qquad
   \MMM \MODELLEQ \MMM' \ \text{and}\ \MMM' \MODELLEQ \MMM.
\]

\NI and the ordering the $\MODELEQ$-equivalence classes as follows:
\[
    [\MMM]_{\MODELEQ} \MODELLEQ [\MMM']_{\MODELEQ}
      \qquad\text{iff}\qquad
    \MMM \MODELLEQ \MMM'.
\]

\NI Since this process is independent for the chosen representatives,
we obtain a partial order. Greatest lower and least upper bounds can also
be computed on representatives:
\[
   \BIGLUB \{[\MMM]_{\MODELEQ} \ |\ \MMM \in S\ \} = [\BIGLUB S]_{\MODELEQ}
\]
whenever $\BIGLUB S$ exists, and likewise for the greated lower bound.
We also define 
\[
   [\MMM]_{\MODELEQ} \models \phi 
      \qquad\text{iff}\qquad
   \MMM \models \phi.
\]

\NI Theorem \ref{theorem:completeLattice} guarantees that the choice
of representative is irrelevant.

In the rest of this text, we will usually be sloppy and work with
concrete models instead of $\MODELEQ$-equivalence classes of models
because the quotienting process is straightfoward and not especially
interesting. We can do this because all relevant constructions in this
text are independent from the specific choice of representative.  Our
Haskell implementation, described in Section \ref{hahahaskell} uses a
slightly different approach: instead of equivalence classes, it uses
canonical representatives.  \martin{add more explanation, and
  integrate better with premiminaries}.

\subsection{Extending $\MODELLEQ$ to a bounded lattice}
\label{boundedlattice}
It turns out that $\MODELLEQ $ on ($\MODELEQ$-equivalence classes of)
models is not just a partial order, but a bounded lattice, except
that a bottom element is missing.

\begin{definition}
We extend the collection of models with a single \emph{bottom} element
$\bot$, where $\bot \models \phi$ for all $\phi$. We also write $\bot$
for $[\bot]_{\MODELEQ}$.  We extend the relation $\MODELLEQ $ and
stipulate that $\bot \MODELLEQ \MMM$ for all models $\MMM$.
\end{definition}

\begin{theorem}
The collection of (equivalence classes of) models together with
$\bot$, and ordered by $\MODELLEQ$ is a bounded lattice.
\end{theorem}
\begin{proof}
The topmost element in the
lattice is the model $( (\{w\}, \{\}, \{w \mapsto \Sigma\}), w)$ (for
some state $w$): this is the model with no transitions and no
transition restrictions.
The bottom element is $\bot$. 
$\sqcap$ and $\sqcup$ are defined below.
\end{proof}

\Cathoristic{} has the unusual property that every formula has a unique (up to isomorphism) simplest  model satisfying a formula. 
Any logic with disjunction, negation or implication does not have this property.
For example, in propositional logic:
\begin{itemize}
\item
There are two equally-simple models satisfying $p \lor q$
\item
There are two equally-simple  models satisfying $\neg (\neg p \land \neg q)$
\item
There are two equally-simple models satisfying $p \Rightarrow q$
\end{itemize}
But as \cathoristic{} does not have disjunction, negation or implication, it does satisfy this unusual property.
In \cathoristic{}, both the quadratic-time decision procedure and the completeness proof rely on computing the simplest model that satisfies a formula. 

\subsection{Computing the simplest model satisfying a formula}
\label{simpl}

Given $\MODELLEQ $, we can define $\SIMPL{\phi}$, the simplest model w.r.t.~$\MODELLEQ $ that
satisfies $\phi$.

\begin{eqnarray*}
  \SIMPL{\top} &\ = \ & ( (\{v\}, \{\}, \{v \mapsto \Sigma\}), v)  \\
  \SIMPL{\fBang A} & = & ( (\{v\}, \{\}, \{v \mapsto A\}), v)  \\
  \SIMPL{\phi_1 \AND \phi_2} & = & \SIMPL{\phi_1} \sqcap \SIMPL{\phi_2}  \\
  \SIMPL{\langle a \rangle \phi} 
     & = & ( (S \cup \{w'\}, \rightarrow \cup (w' \xrightarrow{a} w), \lambda \cup \{w' \mapsto \Sigma\}]), w')  \\
		& & \mbox{where }\SIMPL{\phi} = ( (S, \rightarrow, \lambda), w) \mbox{and } w' \mbox{ is a new state} \\
                &&  \mbox{not appearing in }S 
\end{eqnarray*}

\begin{figure}[H]
\centering
\begin{tikzpicture}[node distance=1.3cm,>=stealth',bend angle=45,auto]
  \tikzstyle{place}=[circle,thick,draw=blue!75,fill=blue!20,minimum size=6mm]
  \tikzstyle{red place}=[place,draw=red!75,fill=red!20]
  \tikzstyle{transition}=[rectangle,thick,draw=black!75,
  			  fill=black!20,minimum size=4mm]
  \tikzstyle{every label}=[red]
  \begin{scope}
    \node [place] (w1) {$\Sigma$};
    \node [place] (e1) [below of=w1] {$\Sigma$}
      edge [pre]  node[swap] {a}                 (w1);      
  \end{scope}
  \begin{scope}[xshift=4cm]
    \node [place] (w1) {$\Sigma$};
    \node [place] (e1) [below of=w1] {$\Sigma$}
      edge [pre]  node[swap] {b}                 (w1);      
  \end{scope} 
  \begin{scope}[xshift=8cm]
    \node [place] (w1) {$\Sigma$};
    \node [place] (e1) [below left of=w1] {$\Sigma$}
      edge [pre]  node[swap] {a}                 (w1);      
    \node [place] (e1) [below right of=w1] {$\Sigma$}
      edge [pre]  node[swap] {b}                 (w1);      
  \end{scope}
  \draw (2,0) node {$\sqcap$};
  \draw (6,0) node {$=$};
\end{tikzpicture}
\caption{Example of $\sqcap$.}
\end{figure}

\begin{figure}[H]
\centering
\begin{tikzpicture}[node distance=1.3cm,>=stealth',bend angle=45,auto]
  \tikzstyle{place}=[circle,thick,draw=blue!75,fill=blue!20,minimum size=6mm]
  \tikzstyle{red place}=[place,draw=red!75,fill=red!20]
  \tikzstyle{transition}=[rectangle,thick,draw=black!75,
  			  fill=black!20,minimum size=4mm]
  \tikzstyle{every label}=[red]
  \begin{scope}
    \node [place] (w1) {$\Sigma$};
    \node [place] (e1) [below of=w1] {$\{b\}$}
      edge [pre]  node[swap] {a}                 (w1);      
    \node [place] (e2) [below of=e1] {$\Sigma$}
      edge [pre]  node[swap] {b}                 (e1);      
  \end{scope}
  \begin{scope}[xshift=4cm]
    \node [place] (w1) {$\Sigma$};
    \node [place] (e1) [below of=w1] {$\Sigma$}
      edge [pre]  node[swap] {a}                 (w1);      
    \node [place] (e2) [below right of=e1] {$\Sigma$}
      edge [pre]  node[swap] {c}                 (e1);      
    \node [place] (e3) [below left of=e1] {$\Sigma$}
      edge [pre]  node[swap] {b}                 (e1);      
  \end{scope} 
  \begin{scope}[xshift=8cm]
    \node (w1) {$\bot$};
  \end{scope}
  \draw (2,0) node {$\sqcap$};
  \draw (6,0) node {$=$};
\end{tikzpicture}
\caption{Example of $\sqcap$.}
\end{figure}

\begin{figure}[H]
\centering
\begin{tikzpicture}[node distance=1.3cm,>=stealth',bend angle=45,auto]
  \tikzstyle{place}=[circle,thick,draw=blue!75,fill=blue!20,minimum size=6mm]
  \tikzstyle{red place}=[place,draw=red!75,fill=red!20]
  \tikzstyle{transition}=[rectangle,thick,draw=black!75,
  			  fill=black!20,minimum size=4mm]
  \tikzstyle{every label}=[red]
  \begin{scope}
    \node [place] (w1) {$\{a,b\}$};
    \node [place] (e1) [below of=w1] {$\Sigma$}
      edge [pre]  node[swap] {a}                 (w1);      
    \node [place] (e2) [below of=e1] {$\Sigma $}
      edge [pre]  node[swap] {b}                 (e1);      
  \end{scope}
  \begin{scope}[xshift=4cm]
    \node [place] (w1) {$\{a,c\}$};
    \node [place] (e1) [below of=w1] {$\{b,c\}$}
      edge [pre]  node[swap] {a}                 (w1);      
    \node [place] (e2) [below of=e1] {$\Sigma $}
      edge [pre]  node[swap] {c}                 (e1);      
    \node [place] (e3) [below of=e2] {$\Sigma $}
      edge [pre]  node[swap] {d}                 (e2);      
  \end{scope} 
  \begin{scope}[xshift=8cm]
    \node [place] (w1) {$\{a\}$};
    \node [place] (e1) [below of=w1] {$\{b, c\} $}
      edge [pre]  node[swap] {a}                 (w1);      
    \node [place] (e2) [below left of=e1] {$\Sigma $}
      edge [pre]  node[swap] {b}                 (e1);      
    \node [place] (e3) [below right of=e1] {$\Sigma $}
      edge [pre]  node[swap] {c}                 (e1);      
    \node [place] (e4) [below of=e3] {$\Sigma $}
      edge [pre]  node[swap] {d}                 (e3);      
  \end{scope}
  \draw (2,0) node {$\sqcap$};
  \draw (6,0) node {$=$};
\end{tikzpicture}
\caption{Example of $\sqcap$. }
\end{figure}

\begin{figure}[H]
\centering
\begin{tikzpicture}[node distance=1.3cm,>=stealth',bend angle=45,auto]
  \tikzstyle{place}=[circle,thick,draw=blue!75,fill=blue!20,minimum size=6mm]
  \tikzstyle{red place}=[place,draw=red!75,fill=red!20]
  \tikzstyle{transition}=[rectangle,thick,draw=black!75,
  			  fill=black!20,minimum size=4mm]
  \tikzstyle{every label}=[red]
  
  \begin{scope}
    \node [place] (w1) {$\Sigma$};
    \node [place] (e1) [below left of=w1] {$\{c\}$}
      edge [pre]  node[swap] {a}                 (w1);      
    \node [place] (e2) [below right of=w1] {$\Sigma$}
      edge [pre]  node[swap] {b}                 (w1);      
    \node [place] (e3) [below of=e2] {$\Sigma$}
      edge [pre]  node[swap] {d}                 (e2);      
  \end{scope}
  
  \begin{scope}[xshift=4cm]
    \node [place] (w1) {$\Sigma$};
    \node [place] (e1) [below left of=w1] {$\Sigma$}
      edge [pre]  node[swap] {a}                 (w1);      
    \node [place] (e2) [below right of=w1] {$\{d\}$}
      edge [pre]  node[swap] {b}                 (w1);      
    \node [place] (e3) [below of=e1] {$\Sigma$}
      edge [pre]  node[swap] {c}                 (e1);      
  \end{scope}
  
  
  \begin{scope}[xshift=8cm]
    \node [place] (w1) {$\Sigma$};
    \node [place] (e1) [below left of=w1] {$\{c\}$}
      edge [pre]  node[swap] {a}                 (w1);      
    \node [place] (e2) [below right of=w1] {$\{d\}$}
      edge [pre]  node[swap] {b}                 (w1);      
    \node [place] (e3) [below of=e1] {$\Sigma$}
      edge [pre]  node[swap] {c}                 (e1);      
    \node [place] (e4) [below of=e2] {$\Sigma$}
      edge [pre]  node[swap] {d}                 (e2);      
  \end{scope}
  
  \draw (2,0) node {$\sqcap$};
  \draw (6,0) node {$=$};
\end{tikzpicture}
\caption{Example of $\sqcap$.}
\end{figure}





\NI Note that by our conventions, $\SIMPL{\phi}$ really returns a
$\MODELEQ$-equivalence class of models.

The only complex case is the clause for $\SIMPL{\phi_1 \AND \phi_2}$,
which uses the $\sqcap$ function, defined as follows, where we assume
that the sets of states in the two models are disjoint.

\begin{eqnarray*}
  \bot \sqcap \MMM  &\ =\ &  \bot  \\
  \MMM \sqcap \bot      & = &  \bot  
     \\
  \MMM \sqcap \MMM'
     & = & 
     \mathsf{merge}(\mathcal{L}, \mathcal{L}', \{(w,w')\}) 
     \\
     & & \text{where } \MMM = (\mathcal{L}, w) \text{ and } \MMM' = (\mathcal{L'}, w')
\end{eqnarray*}

\NI The $\mathsf{merge}$ function attempts to merge two LTSs together, given a set of state-identification pairs (a set of pairs of states from the two LTSs that need to be identified).
The state-identification pairs are used to make sure that the resulting model is deterministic.

\begin{eqnarray*}
  \mathsf{merge}(\mathcal{L}, \mathcal{L}', ids) 
     & = & 
  \begin{cases}
    \bot & \text{if } \mathsf{inconsistent}(\mathcal{L}, \mathcal{L}', ids)  \\
    \mathsf{join}(\mathcal{L}, \mathcal{L}') & \text{if } ids = \emptyset  \\
    \mathsf{merge}(\mathcal{L}, \mathcal{L}'', ids')  & \text{else, where }
          \mathcal{L}'' = \mathsf{applyIds}(ids, \mathcal{L}') \\
          & \text{and } ids' = \mathsf{getIds}(\mathcal{L}, \mathcal{L}', ids)
  \end{cases}
\end{eqnarray*}

\NI The $\mathsf{inconsistent}$ predicate is true if there is pair of
states in the state-identification set such that the out-transitions
of one state is incompatible with the node-labelling on the other
state:
\begin{eqnarray*}
  \lefteqn{\mathsf{inconsistent}(\mathcal{L}, \mathcal{L}', ids)}\qquad
     \\
     &\text{ iff }& \exists (w,w') \in ids \; \text{with} \; \mathsf{out}(\mathcal{L},w) \nsubseteq \lambda'(w') \; \text{or} \; \mathsf{out}(\mathcal{L}',w') \nsubseteq \lambda(w).
\end{eqnarray*}
 
\NI Here the $\mathsf{out}$ function returns all the actions
immediately available from the given state $w$.
\[
  \mathsf{out}(((S,\rightarrow,\lambda),w)) 
     \ =\  \{ a \fOr \exists w' . w \xrightarrow{a} w'\} 
\]

\NI The $\mathsf{join}$ function takes the union of the two LTSs.
\[
   \mathsf{join}((S, \rightarrow,\lambda), (S', \rightarrow', \lambda')) 
      \quad=\quad
   (S \cup S', \rightarrow \cup \rightarrow', \lambda'')
\]

\NI Here $\lambda''$ takes the constrains arising from both, $\lambda$ and
$\lambda'$ into account: 
\[
   \lambda''(s) 
      \quad = \quad
   \begin{array}{l}
      \{\lambda(s) \cap \lambda'(s) \; | \; s \in S \cup S'\} \\ \cup \ 
      \{\lambda(s)\; |\; s\in S \setminus S' \} \\ \cup \ 
      \{\lambda(s)\; |\; s\in S' \setminus S \}. 
   \end{array}
\]

\NI The $\mathsf{applyIds}$ function applies all the
state-identification pairs as substitutions to the Labelled Transition
System:
\[
   \mathsf{applyIds}(ids, (S, \rightarrow, \lambda)) 
      \quad=\quad 
   (S', \rightarrow', \lambda')
\]

\NI where
\begin{eqnarray*}
  S' &\quad =\quad & S \; [ w / w' \; | \; (w,w') \in ids] \\
  \rightarrow' & = & \rightarrow \; [ w / w' \; | \; (w,w') \in ids] \\
  \lambda' & = & \lambda \; [ w / w' \; | \; (w,w') \in ids]
\end{eqnarray*}

\NI Here $[ w / w' \; | \; (w,w') \in ids]$ means the simultaneous
substitution of $w$ for $w'$ for all pairs $(w, w')$ in $ids$.  The
$\mathsf{getIds}$ function returns the set of extra
state-identification pairs that need to be added to respect
determinism:
\[
   \mathsf{getIds}(\mathcal{L}, \mathcal{L}', ids) 
      \quad=\quad 
   \{(x,x') \; | \; (w,w') \in ids, \exists a \; . \; w \xrightarrow{a} x, w' \xrightarrow{a} x'\}
\]

\NI The function $\SIMPL{\cdot}$ has the expected properties, as the next
lemma shows.  It has also been implemented in Haskell
\cite{HaskellImplementation}.

\begin{lemma}
$\SIMPL{\phi} \models \phi.$
\end{lemma}
\begin{proof}
By straightforward, yet laborious induction on $\phi$.
\end{proof}

\subsubsection{Showing that $\sqcap$ as defined is the greatest lower bound}
We will show that:
\begin{itemize}
\item
$\MMM \sqcap \MMM' \MODELLEQ \MMM$ and $\MMM \sqcap \MMM' \MODELLEQ \MMM'$
\item
If $\NNN \MODELLEQ \MMM$ and $\NNN \MODELLEQ \MMM'$, then $\NNN \MODELLEQ \MMM \sqcap \MMM'$
\end{itemize}
If $\MMM$, $ \MMM'$ or $\MMM \sqcap \MMM'$ are equal to $\bot$, then we just apply the rule that $\bot \MODELLEQ m$ for all models $m$. 
So let us assume that $\mathsf{consistent}(\MMM, \MMM')$ and that $\MMM \sqcap \MMM'  \neq \bot$.

\begin{proof}
To show $\MMM \sqcap \MMM' \MODELLEQ \MMM$, we need to provide a simulation $\mathcal{R}$ from $\MMM$ to  $\MMM \sqcap \MMM'$.
If $\MMM = ((S,\rightarrow,\lambda),w)$, then define $\mathcal{R}$ as the identity relation on the states of $S$:
\[
\mathcal{R} = \{(x,x) \; | \; x \in S\}
\]
It is straightforward to show that $\mathcal{R}$ as defined is a simulation from $\MMM$ to  $\MMM \sqcap \MMM'$.
If there is a transition $x \xrightarrow{a} y$ in $\MMM$, then by the construction of $\mathsf{merge}$, there is also a transition $x \xrightarrow{a} y$ in $\MMM \sqcap \MMM'$.
We also need to show that $\lambda_{\MMM}(x) \supseteq \lambda_{\MMM \sqcap \MMM'}(x)$ for all states $x$ in $\MMM$. This is immediate from the construction of $\mathsf{merge}$.

\end{proof}

\begin{proof}
To show that $\NNN \MODELLEQ \MMM$ and $\NNN \MODELLEQ \MMM'$ imply $\NNN \MODELLEQ \MMM \sqcap \MMM'$, assume there is a simulation $\mathcal{R}$ from $\MMM$ to $\NNN$ and there is a simulation $\mathcal{R}'$ from $\MMM'$ to $\NNN$.
We need to provide a simulation $\mathcal{R}*$ from $\MMM \sqcap \MMM'$ to $\NNN$.

Assume the states of $\MMM$ and $\MMM'$ are disjoint.
Define:
\[
\mathcal{R}* = \mathcal{R} \cup \mathcal{R}'
\]
We need to show that $\mathcal{R}*$ as defined is a simulation from $\MMM \sqcap \MMM'$ to $\NNN$.

Suppose $x \xrightarrow{a} y$ in $\MMM \sqcap \MMM'$ and that $(x,x_2) \in \mathcal{R} \cup \mathcal{R}'$.
We need to provide a $y_2$ such that $x_2 \xrightarrow{a} y_2$ in  $\NNN$ and $(y,y_2) \in \mathcal{R} \cup \mathcal{R}'$.
If  $x \xrightarrow{a} y$ in $\MMM \sqcap \MMM'$, then, from the definition of $\mathsf{merge}$, either $x \xrightarrow{a} y$ in $\MMM$ or $x \xrightarrow{a} y$ in $\MMM'$. If the former, and given that $\mathcal{R}$ is a simulation from $\MMM$ to $\NNN$, then there is a $y_2$ such that $(y,y_2) \in \mathcal{R}$ and $x_2 \xrightarrow{a} y_2$ in $\NNN$. But, if $(y,y_2) \in \mathcal{R}$, then also $(y,y_2) \in \mathcal{R} \cup \mathcal{R}'$.

Finally, we need to show that if $(x,y) \in \mathcal{R} \cup \mathcal{R}'$ then
\[
\lambda_{\MMM \sqcap \MMM'}(x) \supseteq \lambda_{\NNN}(y)
\]
If $(x,y) \in \mathcal{R} \cup \mathcal{R}'$ then either $(x,y) \in \mathcal{R}$ or $(x,y) \in \mathcal{R}'$.
Assume the former.
Given that $\mathcal{R}$ is a simulation from $\MMM$ to $\NNN$, we know that if $(x,y) \in \mathcal{R}$, then 
\[
\lambda_{\MMM}(x) \supseteq \lambda_{\NNN}(y)
\]
Let $\MMM = ((S,\rightarrow,\lambda),w)$.
If $x \neq w$ (i.e. $x$ is some node other than the start state), then, from the definition of $\mathsf{merge}$, $\lambda_{\MMM \sqcap \MMM'}(x) = \lambda_{\MMM}(x)$.
So, given $\lambda_{\MMM} \supseteq \lambda_{\NNN}(y)$, $\lambda_{\MMM \sqcap \MMM'}(x) \supseteq \lambda_{\NNN}(y)$.
If, on the other hand, $x = w$ (i.e. $x$ is the start state  of our pointed model $\MMM$), then, from the definition of $\mathsf{merge}$:
\[
\lambda_{\MMM \sqcap \MMM'}(w) = \lambda_{\MMM}(w) \cap \lambda_{\MMM'}(w')
\]
where $w'$ is the start state  of $\MMM'$.
In this case, given $\lambda_{\MMM}(w) \supseteq \lambda_{\NNN}(y)$ and $\lambda_{\MMM'}(w') \supseteq \lambda_{\NNN}(y)$, it follows that $\lambda_{\MMM}(w) \cap \lambda_{\MMM'}(w') \supseteq \lambda_{\NNN}(y)$ and hence $\lambda_{\MMM \sqcap \MMM'}(w) \supseteq \lambda_{\NNN}(y)$.

\end{proof}

\subsection{Computing the least upper bound ($\sqcup$)}

\NI Intuitively, the least upper bound of two models $\MMM$ and
$\MMM'$ contains the intersection of the transitions of $\MMM$ and
$\MMM'$ for every associated pair of states in $\MMM$ and $\MMM'$, and
the union of the node-labellings.

Formally, define the least upper bound ($\sqcup$) of two models as:
\begin{eqnarray*}
\MMM \sqcup \bot & = & \MMM \\
\bot \sqcup \MMM & = & \MMM \\
(\CAL{L},w) \sqcup (\CAL{L}',w') & = & \mathsf{lub}(\CAL{L}, \CAL{L}', (\MMM_\top, z), \{(w, w', z)\})
\end{eqnarray*}

\NI where $\MMM_\top$ is the topmost model $(\mathcal{W}=\{z\},
\rightarrow=\{\}, \lambda=\{z \mapsto \Sigma\})$ for some state $z$.
$\mathsf{lub}$ takes four parameters: the two cathoristic transition
systems $\CAL{L}$ and $\CAL{L}'$, an accumulator representing the
constructed result so far, and a list of state triples (each triple
contains one state from each of the two input models plus the state of
the accumulated result) to consider next.  It is defined as:
\begin{eqnarray*}
  \mathsf{lub}(\CAL{L}, \CAL{L}', \MMM, \{\}) 
     &\quad =\quad & 
  \MMM \\
  \mathsf{lub}(\CAL{L}, \CAL{L}', ((\mathcal{W}, \rightarrow, \lambda), y), \{(w,w',x)\} \cup R) 
     & = & 
  \mathsf{lub}(\CAL{L}, \CAL{L}', ((\mathcal{W} \cup \mathcal{W}', \rightarrow \cup \rightarrow', \lambda'), y), R' \cup R\}
\end{eqnarray*}
where:
\begin{eqnarray*}
  \{(a_i, w_i, w'_i) \;|\; i = 1 ... n\} 
     &\quad =\quad & 
  \mathsf{sharedT}((\CAL{L},w), (\CAL{L}',w')) \\
  \mathcal{W}' 
     & = & 
  \{x_i \;|\; i = 1 ... n\} \\
  \rightarrow' 
     & = & 
  \{(x, a_i, x_i) \;|\; i = 1 ... n\} \\
  \lambda' 
     & = & 
  \lambda [x \mapsto \lambda(w) \cup \lambda(w)'] \\
  R' 
     & = & 
  \{(w_i, w'_i, x_i) \;|\; i = 1 ... n\}
\end{eqnarray*}
Here, $\mathsf{sharedT}$ returns the shared transitions between two models, and is defined as:
\[
\mathsf{sharedT}(((\mathcal{W}, \rightarrow, \lambda),w) ((\mathcal{W}', \rightarrow', \lambda'),w')) =  \{(a, x, x') \;|\; w \xrightarrow{a} x \land w' \xrightarrow{a}' x'\}
\]
If $((S*,\rightarrow*,\lambda*),w*) = ((S,\rightarrow,\lambda),w) \sqcup ((S',\rightarrow',\lambda'),w')$ then define
the set $\mathsf{triples}_\mathsf{lub}$ as the set of triples $(x,x',x*) \; | \; x \in S, x' \in S', x* \in S*$ that were used during the construction of $\mathsf{lub}$ above. So $\mathsf{triples}_\mathsf{lub}$ stores the associations between states in $\MMM$, $\MMM'$ and $\MMM \sqcup \MMM'$. 

\begin{figure}[H]
\centering
\begin{tikzpicture}[node distance=1.3cm,>=stealth',bend angle=45,auto]
  \tikzstyle{place}=[circle,thick,draw=blue!75,fill=blue!20,minimum size=6mm]
  \tikzstyle{red place}=[place,draw=red!75,fill=red!20]
  \tikzstyle{transition}=[rectangle,thick,draw=black!75,
  			  fill=black!20,minimum size=4mm]
  \tikzstyle{every label}=[red]
  \begin{scope}
    \node [place] (w1) {$\{a\}$};
    \node [place] (e1) [below of=w1] {$\Sigma$}
      edge [pre]  node[swap] {a}                 (w1);      
  \end{scope}
  \begin{scope}[xshift=4cm]
    \node [place] (w1) {$\{b\}$};
    \node [place] (e1) [below of=w1] {$\Sigma$}
      edge [pre]  node[swap] {b}                 (w1);      
  \end{scope}
  \begin{scope}[xshift=8cm]
    \node [place] (w1) {$\{a,b\}$};
  \end{scope}
  \draw (2,0) node {$\sqcup$};
  \draw (6,0) node {$=$};
\end{tikzpicture}
\caption{Example of $\sqcup$}
\end{figure}


\begin{figure}[H]
\centering
\begin{tikzpicture}[node distance=1.3cm,>=stealth',bend angle=45,auto]
  \tikzstyle{place}=[circle,thick,draw=blue!75,fill=blue!20,minimum size=6mm]
  \tikzstyle{red place}=[place,draw=red!75,fill=red!20]
  \tikzstyle{transition}=[rectangle,thick,draw=black!75,
  			  fill=black!20,minimum size=4mm]
  \tikzstyle{every label}=[red]
  \begin{scope}
    \node [place] (w1) {$\{a\}$};
    \node [place] (e1) [below of=w1] {$\Sigma$}
      edge [pre]  node[swap] {a}                 (w1);      
    \node [place] (e2) [below of=e1] {$\{c\}$}
      edge [pre]  node[swap] {b}                 (e1);      
  \end{scope}
  \begin{scope}[xshift=4cm]
    \node [place] (w1) {$\{a,b\}$};
    \node [place] (e1) [below of=w1] {$\{b,c\}$}
      edge [pre]  node[swap] {a}                 (w1);      
    \node [place] (e2) [below left of=e1] {$\{d\}$}
      edge [pre]  node[swap] {b}                 (e1);      
    \node [place] (e2) [below right of=e1] {$\Sigma$}
      edge [pre]  node[swap] {c}                 (e1);      
  \end{scope} 
  \begin{scope}[xshift=8cm]
    \node [place] (w1) {$\{a,b\}$};
    \node [place] (e1) [below of=w1] {$\Sigma$}
      edge [pre]  node[swap] {a}                 (w1);      
    \node [place] (e2) [below of=e1] {$\{c,d\}$}
      edge [pre]  node[swap] {b}                 (e1);      
  \end{scope}
  \draw (2,0) node {$\sqcup$};
  \draw (6,0) node {$=$};
\end{tikzpicture}
\caption{Example of $\sqcup$}
\end{figure}


\subsubsection{Showing that $\sqcup$ as defined is the least upper bound}
We will show that:
\begin{itemize}
\item
$\MMM \MODELLEQ \MMM \sqcup \MMM'$ and $\MMM' \MODELLEQ \MMM \sqcup \MMM'$
\item
If $\MMM \MODELLEQ \NNN $ and $\MMM' \MODELLEQ \NNN $, then $\MMM \sqcup \MMM' \MODELLEQ \NNN$
\end{itemize}
If $\MMM$ or $ \MMM'$ are equal to $\bot$, then we just apply the rule that $\bot \MODELLEQ m$ for all models $m$. 
So let us assume that neither $\MMM$ not $\MMM'$ are $bot$.

\begin{proof}
To see that $\MMM \MODELLEQ \MMM \sqcup \MMM'$, observe that, by construction of $\sqcup$ above, every transition in $\MMM \sqcup \MMM'$ has a matching transition in $\MMM$, and every node label in  $\MMM \sqcup \MMM'$ is a superset of the corresponding node label in $\MMM$.

To show that $\MMM \MODELLEQ \NNN $ and $\MMM' \MODELLEQ \NNN $ together imply $\MMM \sqcup \MMM' \MODELLEQ \NNN$, assume a simulation $\mathcal{R}$ from $\NNN$ to $\MMM$ and a simulation $\mathcal{R}'$ from $\NNN$ to $\MMM'$.
We need to produce a simulation relation $\mathcal{R}*$ from $\NNN$ to $\MMM \sqcup \MMM' $.
Define
\[
\mathcal{R}* =   \{(x, y*) \; | \; \exists y_1 . \exists y_2 . (x,y_1) \in \mathcal{R}, (x,y_2) \in \mathcal{R}', (y_1,y_2,y*) \in \mathsf{triples}_\mathsf{lub} \}
\]
In other words, $\mathsf{R}*$ contains the pairs corresponding to the pairs in both $\mathsf{R}$ and $\mathsf{R}'$.
We just need to show that $\mathsf{R}*$ as defined is a simulation from $\NNN$ to $\MMM \sqcup \MMM' $.
Assume $(x,x*) \in \mathsf{R}*$ and $x \xrightarrow{a} y$ in $\NNN$. 
We need to produce a $y*$ such that $(x*,y*) \in \mathsf{R}*$ and $x* \xrightarrow{a} y*$ in $\MMM \sqcup \MMM' $.
Given that $\mathcal{R}$ is a simulation from $\NNN$ to $\MMM$, and that  $\mathcal{R}'$ is a simulation from $\NNN$ to $\MMM'$, we know that there is a pair of states $x_1, y_1$ in $\MMM$ and a pair of states $x_2, y_2$ in $\MMM'$ such that $(x,x_1) \in \mathsf{R}$ and $(x,x_2) \in \mathsf{R}'$ and $x_1 \xrightarrow{a} y_1$ in $\MMM$ and $x_2 \xrightarrow{a} y_2$ in $\MMM'$.
Now, from the construction of $\mathsf{lub}$ above, there is a triple $(y_1, y_2, y*) \in \mathsf{triples}_\mathsf{lub}$.
Now, from the construction of $\mathsf{R}*$ above, $(x*,y*) \in \mathsf{R}*$.

Finally, we need to show that for all states $x$ and $y$, if $(x,y) \in \mathsf{R}*, \lambda_{\NNN}(x) \supseteq \lambda_{\MMM \sqcup \MMM'}(y)$.
Given that $\mathcal{R}$ is a simulation from $\NNN$ to $\MMM$, and that  $\mathcal{R}'$ is a simulation from $\NNN$ to $\MMM'$, we know that if $(x,y_1) \in \mathsf{R}$, then $\lambda_\NNN(x) \supseteq \lambda_\MMM(y_1)$.
Similarly, if  $(x,y_2) \in \mathsf{R}$, then $\lambda_\NNN(x) \supseteq \lambda_\MMM'(y_2)$.
Now, from the construction of $\mathsf{lub}$, $\lambda_{\MMM \sqcup \MMM'}(y*) = \lambda_{\MMM}(y_1) \cup \lambda_{\MMM}(y_2)$ for all triples $(y_1, y_2, y*) \in \mathsf{triples}_\mathsf{lub}$. 
So $\lambda_{\NNN}(x) \supseteq \lambda_{\MMM \sqcup \MMM'}(y)$, as required.
\end{proof}

\subsection{A decision procedure for \cathoristic{}}\label{decisionprocedure}

We use the semantic constructions above to provide a quadratic-time
decision procedure.  The complexity of the decision procedure is an
indication that \cathoristic{} is useful as a query language in knowledge
representation.

\Cathoristic{}'s lack of connectives for negation, disjunction or
implication is the key reason for the efficiency of the decision
procedure.  Although any satisfiable formula has an infinite number of
models, we have shown that the satisfying models form a bounded
lattice with a least upper bound.  The $\SIMPL$ function defined above
gives us the minimal model satisfying an expression.  Using this least
upper bound, we can calculate entailment by checking a \emph{single
  model}.  To decide whether $\phi \models \psi$, we use the following
algorithm.

\begin{enumerate}

\item Compute $\SIMPL{\phi}$.

\item Check if $\SIMPL{\phi} \models \psi$.

\end{enumerate}

\NI The correctness of this algorithm is given by the follow theorm.

\begin{theorem}\label{theorem:decision}
  The following are equivalent:
  \begin{enumerate}
    \item\label{theorem:decision:1} For all cathoristic models $\MMM$,
      $\MMM \models \phi$ implies $\MMM \models \psi$.
    \item\label{theorem:decision:2} $\SIMPL{\phi} \models \psi$.
  \end{enumerate}
\end{theorem}

\begin{proof}
The implication from  (\ref{theorem:decision:1}) to
(\ref{theorem:decision:2}) is trivial because $\SIMPL{\phi} \models \phi$ by construction.

For the reverse direction, we make use of the following lemma (proved in the Appendix):
\begin{lemma}
\label{lemmasimpl}
If $\MMM \models \phi$ then $\MMM \MODELLEQ \SIMPL{\phi}$.
\end{lemma}

With Lemma \ref{lemmasimpl} in hand, the proof of Theorem \ref{theorem:decision} is straightforward.
Assume $\MMM \models \phi$. We need to show
$\MMM \models \psi$.  Now if $\MMM \models \phi$ then $\MMM \MODELLEQ
\SIMPL{\phi}$ (by Lemma \ref{lemmasimpl}).  Further, if $\MMM' \models \xi $
and $\MMM \MODELLEQ \MMM'$ then $\MMM \models \xi $ by Theorem
\ref{theorem:completeLattice}. So, substituting $\psi$ for $\xi $ and
$\SIMPL{\phi}$ for $\MMM'$, it follows that $\MMM \models \psi$.
\end{proof}

Given this theorem, the decision procedure is straightforward: to
test if $\phi \models \psi$, we construct $\SIMPL{\phi}$, and then inspect whether
$\SIMPL{\phi} \models \psi$.  Construction of $\SIMPL{\phi}$ is quadratic in the size of
$\phi$, and computing whether a model satisfies $\psi$ is of order $|\psi| \times |\phi|$, so computing whether $\phi \models \psi$ is quadratic time.

There is a Haskell implementation of the decision procedure
\cite{HaskellImplementation}.  The reader is invited to inspect the Haskell code to verify that the algorithm runs in quadratic time.

\subsection{Incompatibility semantics}\label{incompatibility}

\NI One of \cathoristic{}'s key features is that it satisfies Brandom's
incompatibility semantics condition. In this section we formalise what
this means and prove it.

Define the set of formulae\footnote{Brandom \cite{brandom} defines
  incompatibility slightly differently: he defines the set of
  \emph{sets} of formulae which are incompatible with a \emph{set} of
  formulae.  But in \cathoristic{}, if a set of formulae is incompatible,
  then there is an incompatible subset of that set with exactly two
  members.  So we can work with the simpler definition in the text
  above.}  incompatible with $\phi$ as:
\[
\mathcal{I}(\phi) = \{
  \psi \; | \; \forall \MMM. \MMM \not \models \phi \AND \psi\}
\]
The reason why Brandom introduces the incompatibility set is that he wants to use it define \emph{semantic content}:
\begin{quote}
Here is a semantic suggestion: represent the propositional content expressed by a sentence with the set of sentences that express propositions incompatible with it\footnote{\cite{brandom} p.123.}.
\end{quote}
Now if the propositional content of a claim determines its logical consequences, and the propositional content is identified with the incompatibility set, then the incompatibility set must determine the logical consequences.
A logic satisfies Brandom's \textbf{incompatibility semantics condition} if
\[
\phi \models \psi \; \mbox{ iff } \; \mathcal{I}(\psi) \subseteq \mathcal{I}(\phi)
\]

Not all logics satisfy this
property.  Brandom has shown that First Order Logic and S5 satisfy the
incompatibility semantics property, but it is an open question which
other logics satisfy it.  HML satisfies it, but HML without negation
does not.  \cathoristic{} is the \emph{simplest logic we have found} that
satisfies the property.

To prove that \cathoristic{} satisfies the incompatibility semantics condition, we need to first define a related incompatibility function on  models.
$\mathcal{J}(\MMM)$ is the set of models that are incompatible with $\MMM$:
\[
\mathcal{J}(\MMM) = \{ \MMM_2 \; | \; \MMM \sqcap \MMM_2 = \bot \}
\]
We shall make use of two lemmas (proved in the Appendix):
\begin{lemma}
\label{inc1}
$\mbox{if }\phi \models \psi \mbox{ then } \SIMPL{\phi} \MODELLEQ \SIMPL{\psi}$
\end{lemma}
\begin{lemma}
\label{inc3}
$\mbox{if }\mathcal{I}(\psi) \subseteq \mathcal{I}(\phi) \mbox{ then } \mathcal{J}(\SIMPL{\psi}) \subseteq \mathcal{J}(\SIMPL{\phi})$
\end{lemma}

\begin{theorem}
\label{incompatibilitytheorem}
$\phi \models \psi \; \mbox{ iff } \; \mathcal{I}(\psi) \subseteq \mathcal{I}(\phi)$
\end{theorem}

\begin{proof}

Left to right: Assume $\phi \models \psi$ and $\xi \in \mathcal{I}(\psi)$.  
We need to show $\xi \in \mathcal{I}(\phi)$.
By the definition of $\mathcal{I}$, if $\xi \in \mathcal{I}(\psi)$ then
$\SIMPL{\xi} \sqcap \SIMPL{\psi} = \bot$.
If $\SIMPL{\xi} \sqcap \SIMPL{\psi} = \bot$, then either
\begin{itemize}
\item $\SIMPL{\xi} = \bot$
\item $\SIMPL{\psi} = \bot$
\item Neither $\SIMPL{\xi}$ nor $\SIMPL{\psi}$ are $\bot$, but $\SIMPL{\xi} \sqcap \SIMPL{\psi} = \bot$.
\end{itemize}
If $\SIMPL{\xi} = \bot$, then $\SIMPL{\xi} \sqcap \SIMPL{\phi} = \bot$ and we are done.
If $\SIMPL{\psi} = \bot$, then as $\phi \models \psi$, by Lemma \ref{inc1}, $\SIMPL{\phi} \MODELLEQ \SIMPL{\psi}$.
Now the only model that is $\MODELLEQ \bot$ is $\bot$ itself, so $\SIMPL{\phi} = \bot$. Hence $\SIMPL{\xi} \sqcap \SIMPL{\phi} = \bot$, and we are done.
The interesting case is when neither $\SIMPL{\xi}$ nor $\SIMPL{\psi}$ are $\bot$, but $\SIMPL{\xi} \sqcap \SIMPL{\psi} = \bot$.
Then (by the definition of $\mathsf{consistent}$ in Section \ref{simpl}), either $\mathsf{out}(\SIMPL{\xi}) \nsubseteq \lambda(\SIMPL{\psi})$ or $\mathsf{out}(\SIMPL{\psi}) \nsubseteq \lambda(\SIMPL{\xi})$.
In the first sub-case, if  $\mathsf{out}(\SIMPL{\xi}) \nsubseteq \lambda(\SIMPL{\psi})$, then there is some action $a$ such that $\xi \models \MAY{a} \top$ and $a \notin \lambda(\SIMPL{\psi})$.
If $a \notin \lambda(\SIMPL{\psi})$ then $\psi \models ! A$ where $a \notin A$.
Now $\phi \models \psi$, so $\phi \models ! A$.
In other words, $\phi$ also entails the $A$-restriction that rules out the $a$ transition.
So $\SIMPL{\xi} \sqcap \SIMPL{\phi} = \bot$ and $\xi \in \mathcal{I}(\phi)$.
In the second sub-case, $\mathsf{out}(\SIMPL{\psi}) \nsubseteq \lambda(\SIMPL{\xi})$.
Then there is some action $a$ such that $\psi \models \MAY{a} \top$ and $a \notin \lambda(\SIMPL{\xi})$.
If $a \notin \lambda(\SIMPL{\xi})$ then $\xi \models ! A$ where $a \notin A$.
But if $\psi \models \MAY{a} \top$ and $\phi \models \psi$, then $\phi \models \MAY{a} \top$ and $\phi$ is also incompatible with $\xi$'s $A$-restriction.
So $\SIMPL{\xi} \sqcap \SIMPL{\phi} = \bot$ and $\xi \in \mathcal{I}(\phi)$.

Right to left: assume, for reductio, that $\MMM \models \phi$ and $\MMM \nvDash
\psi$. we will show that $\mathcal{I}(\psi) \nsubseteq \mathcal{I}(\phi)$.
Assume $\MMM \models \phi \mbox{ and } \MMM \nvDash \psi$. We will construct
another model $\MMM_2$ such that $\MMM_2 \in \mathcal{J}(\SIMPL{\psi})$ but $\MMM_2
\notin \mathcal{J}(\SIMPL{\phi})$.  This will entail, via Lemma \ref{inc3}, that
$\mathcal{I}(\psi) \nsubseteq \mathcal{I}(\phi)$.

If $\MMM \nvDash \psi$, then there is a formula $\psi'$ that does not contain
$\AND$ such that $\psi \models \psi'$ and $\MMM \nvDash \psi'$. $\psi'$ must be
either of the form (i) $\langle a_1 \rangle ... \langle a_n \rangle
\top$ (for $n > 0$) or (ii) of the form $\langle a_1 \rangle
... \langle a_n \rangle \; !\{A\}$ where $A \subseteq \mathcal{S}
\mbox{ and } n >= 0$.

In case (i), there must be an $i$ between $0$ and $n$ such that $\MMM
\models \langle a_1 \rangle ... \langle a_i \rangle \top$ but $\MMM
\nvDash \langle a_1 \rangle ... \langle a_{i+1} \rangle \top$. We need
to construct another model $\MMM_2$ such that $\MMM_2 \sqcap \SIMPL{\psi} = \bot$,
but $\MMM_2 \sqcap \SIMPL{\phi} \neq \bot$. Letting $\MMM =
((\mathcal{W},\rightarrow,\lambda),w)$, then $\MMM \models \langle a_1
\rangle ... \langle a_i \rangle \top$ implies that there is at least
one sequence of states of the form $w, w_1, ..., w_i$ such that $w
\xrightarrow{a_1} w_1 \rightarrow ... \xrightarrow{a_i} w_i$.  Now let
$\MMM_2$ be just like $\MMM$ but with additional transition-restrictions on
each $w_i$ that it not include $a_{i+1}$.  In other words,
$\lambda_{\MMM_2}(w_i) = \lambda_\MMM(w_i) - \{a_{i+1}\}$ for all $w_i$ in
sequences of the form $w \xrightarrow{a_1} w_1 \rightarrow
... \xrightarrow{a_i} w_i$. Now $\MMM_2 \sqcap \SIMPL{\psi} = \bot$ because of
the additional transition restriction we added to $\MMM_2$, which rules out
$\langle a_1 \rangle ... \langle a_{i+1} \rangle \top$, and
a-forteriori $\psi$. But $\MMM_2 \sqcap \SIMPL{\phi} \neq \bot$, because $\MMM
\models \phi$ and $\MMM_2 \MODELLEQ \MMM$ together imply $\MMM_2 \models \phi$. So $\MMM_2$ is
indeed the model we were looking for, that is incompatible with
$\SIMPL{\psi}$ while being compatible with $\SIMPL{\phi}$.

In case (ii), $\MMM \models \langle a_1 \rangle ... \langle a_n \rangle
\top$ but $\MMM \nvDash \langle a_1 \rangle ... \langle a_n \rangle !A$
for some $A \subset \mathcal{S}$. We need to produce a model $\MMM_2$ that
is incompatible with $\SIMPL{\psi}$ but not with $\SIMPL{\phi}$. Given that
$\MMM \models \langle a_1 \rangle ... \langle a_n \rangle \top$, there is
a sequence of states $w, w_1, ..., w_n$ such that $w \xrightarrow{a_1}
w_1 \rightarrow ... \xrightarrow{a_i} w_n$. Let $\MMM_2$ be the model just
like $\MMM$ except it has an additional transition from each such $w_n$
with a symbol $a \notin A$. 
Clearly, $\MMM_2 \sqcap \SIMPL{\psi'} = \bot$
because of the additional $a$-transition, and given that $\psi \models
\psi'$, it follows that $\MMM_2 \sqcap \SIMPL{\psi} = \bot$. Also, $\MMM_2 \sqcap
\SIMPL{\phi} \neq \bot$, because $\MMM_2 \MODELLEQ \MMM$ and $\MMM \models \phi$.


\end{proof}

