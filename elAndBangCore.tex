\section{Core eremic logic: the EL$[\AND, !]$ fragment}

\subsection{Syntax}

\begin{definition} Given a set $\mathcal{S}$ of symbols, with $a$ ranging over
$\mathcal{S}$, and $A$ ranging over finite subsets of $\mathcal{S}$,
the formulae of EL$[\AND, !]$ are given by:

\begin{GRAMMAR}
  \phi 
     &\quad ::= \quad & 
  \top 
     \VERTICAL 
  \phi_1 \AND \phi_2  
     \VERTICAL 
  \MAY{a}{\phi}
     \VERTICAL 
  \fBang A 
\end{GRAMMAR}

\NI The $!$ operator is used to restrict the allowable transitions
coming out of a state.  Intuitively, $\fBang A$ means that the
\emph{only} transitions coming out of the current state are those
specified in $A$.
\end{definition}

\subsection{Semantics}

\begin{definition}
A {\bf model} is a triple $(\mathcal{W}, \rightarrow, \lambda)$,
containing a Labeled Transition System (a set of states $\mathcal{W}$,
and a transition relation $\rightarrow \; \subseteq \; \mathcal{W}
\times \mathcal{S} \times \mathcal{W}$), together with a
node-labelling $\lambda$ that maps each element of $\mathcal{W}$ to a
subset of $\mathcal{S}$.
\end{definition}
The intended interpretation is that $\lambda(w)$ is the set of allowed transition symbols emanating from $w$.
The $\lambda$ function is the semantic counterpart of the $!$ operator.

Now, for a model to be valid, we insist that the transitions coming out of a node $w$ are a subset of the allowed transitions in $\lambda(w)$:
\begin{definition}
A model $(\mathcal{W}, \rightarrow, \lambda)$ is {\bf valid} iff for all $w \in \mathcal{W}$, $ \{s \fOr \exists w' \; w \xrightarrow{s} w'\} \subseteq \lambda(w)$.
\end{definition}

\begin{definition}
A {\bf pointed model} is a pair $(l,w)$, where $m$ is a \emph{valid} model of the form $(\mathcal{W}, \rightarrow, \lambda)$, and $w$ is a distinguished state in $\mathcal{W}$.
\end{definition}
Formulae are interpreted in a pointed model $(l,w)$:
\begin{eqnarray}
(l,w) & \models & \top  \mbox{ always } \nonumber \\
(l,w) & \models & \phi_1 \AND \phi_2 \mbox{ iff } (l,w)  \models \phi_1 \mbox { and } (l,w) \models \phi_2 \nonumber \\
(l,w) & \models & \langle a \rangle \phi \mbox{ iff there is a } w \xrightarrow{a} w' \mbox { such that } (l,w') \models \phi \nonumber \\
(l,w) & \models & \fBang A \mbox{ iff } \lambda(w) \subseteq A\nonumber
\end{eqnarray}
Note that if a model is valid then $(l,w) \models \fBang A$ implies $\{s \fOr \exists w' \; w \xrightarrow{s} w'\} \subseteq A$.
From now on, we will restrict ourselves to valid models.

\begin{FIGURE}
\centering
\begin{tikzpicture}[node distance=1.3cm,>=stealth',bend angle=45,auto]
  \tikzstyle{place}=[circle,thick,draw=blue!75,fill=blue!20,minimum size=6mm]
  \tikzstyle{red place}=[place,draw=red!75,fill=red!20]
  \tikzstyle{transition}=[rectangle,thick,draw=black!75,
  			  fill=black!20,minimum size=4mm]
  \tikzstyle{every label}=[red]
  \begin{scope}[xshift=0cm]
    \node [place] (w1) {$S$};
    \node [place] (e1) [below of=w1] {$\mathcal{S}$}
      edge [pre]  node[swap] {a}                 (w1);
    \node [place] (e2) [below of=e1] {$\mathcal{S}$}
      edge [pre]  node[swap] {b}                 (e1);
  \end{scope}   
  \begin{scope}[xshift=4cm]
    \node [place] (w1) {$\mathcal{S}$};
    \node [place] (e1) [below of=w1] {$\{a,b,c\}$}
      edge [pre]  node[swap] {a}                 (w1);
    \node [place] (e2) [below of=e1] {$\mathcal{S}$}
      edge [pre]  node[swap] {b}                 (e1);
  \end{scope}   
  \begin{scope}[xshift=8cm]
    \node [place] (w1) {$\{a\}$};
    \node [place] (e1) [below of=w1] {$\{b\}$}
      edge [pre]  node[swap] {a}                 (w1);
    \node [place] (e2) [below of=e1] {$\{\}$}
      edge [pre]  node[swap] {b}                 (e1);
  \end{scope}   
\end{tikzpicture}
\caption{Various models of $\langle a \rangle \langle b \rangle \top$}
\end{FIGURE}



\begin{FIGURE}
\begin{RULES}

  \ZEROPREMISERULENAMEDRIGHT
  {
    \phi \judge \phi
  }{Identity}
    \quad
  \ZEROPREMISERULENAMEDRIGHT
  {
    \phi \judge \top
  }{$\top$-Right}
    \quad
  \ZEROPREMISERULENAMEDRIGHT
  {
    \bot \judge \phi
  }{$\bot$-Left}
    \quad
  \TWOPREMISERULENAMEDRIGHT
  {
    \phi \judge \psi
  }
  {
    \psi \judge \xi
  }
  {
    \phi \judge \xi
  }{Transitivity}
    \\\\
  \ONEPREMISERULENAMEDRIGHT
  {
    \phi \judge \psi
  }
  {
    \phi \AND \xi \judge \psi
  }{$\AND$-Left 1}
     \quad
  \ONEPREMISERULENAMEDRIGHT
  {
    \phi \judge \psi
  }
  {
    \xi \AND \phi  \judge \psi
  }{$\AND$-Left 2}
     \quad
  \TWOPREMISERULENAMEDRIGHT
  {
    \phi \judge \psi
  }
  {
    \phi \judge \xi
  }
  {
    \phi \judge \psi \AND \xi
  }{$\AND$-Right}
     \\\\
     \ONEPREMISERULENAMEDRIGHT
     {
       a \notin A
     }
     {
       !A \AND \MAY{a}{\phi} \judge \bot
     }{$\bot$-Right 1}
        \quad
     \ZEROPREMISERULENAMEDRIGHT
     {
       \MAY{a}{\bot} \judge \bot
     }{$\bot$-Right 2}
        \quad
     \TWOPREMISERULENAMEDRIGHT
     {
       \phi \AND \, !A \judge \psi
     }
     {
       A' \subseteq A
     }
     {
       \phi \AND\, !A' \judge \psi
     }{!-Left}
     \\\\
     \TWOPREMISERULENAMEDRIGHT
     {
       \phi \judge !A
     }
     {
       A \subseteq A'
     }
     {
       \phi \judge!A'
     }{!-Right 1}
     \quad
     \TWOPREMISERULENAMEDRIGHT
     {
       \phi \judge !A
     }
     {
       \phi \judge !B
     }
     {
       \phi \judge !(A \cap B)
     }{!-Right 2}
     \quad
     \ONEPREMISERULENAMEDRIGHT
     {
       \phi \judge \psi
     }
     {
       \MAY{a}{\phi} \judge \MAY{a}{\psi}
     }{Transition Normal}
\end{RULES}
\caption{Proof rules.}\label{figure:elAndBangRules}
\end{FIGURE}


\subsection{Inference Rules}

This section presend the inference rules for EL.
In EL, a judgement is of the form:

\[
  A \judge B
\]

\NI Here, $A$ and $B$ are \emph{single formulae}, not sequents.  To
avoid the need for structural inference rules, we restrict sequents to
single formulae on the left and right hand side.
EL proof rules can be grouped in two parts: standard rules and rules unique to EL.
Standard rules are [\RULENAME{Identity}

$\top$-Right]



\subsection{Standard Rules}
EL has the following standard axioms and inference-rules:
Our first axiom is {\bf Identity}:
\[
\boxed
{
X \judge X
}
\]
{\bf $\top$ Right} states that $\top$ is always provable:
\[
\boxed
{
X \judge \top
}
\]
{\bf $\bot$ Left} states that anything is provable from $\bot$:
\[
\boxed
{
\bot \judge X
}
\]
Note that $\bot$ is not a primitive symbol of EL - it is defined as one of the unsatisfiable formulae, for example $\langle a \rangle \top \; \AND \; \fBang \{\}$. 

EL has the standard transitivity inference rule:
\begin{center}
\fbox{
\AxiomC{$X \judge Y$}
\AxiomC{$Y \judge Z$}
\LeftLabel{{\bf Transitivity: \quad}}
\BinaryInfC{$X \judge Z$}
\DisplayProof
}
\end{center}
It has the standard three rules for handling conjunction:
\begin{center}
\fbox{
\AxiomC{$X \judge Y$}
\LeftLabel{{\bf $\AND$ Left 1: \quad}}
\UnaryInfC{$X \AND Z \judge Y$}
\DisplayProof
}
\end{center}

\begin{center}
\fbox{
\AxiomC{$X \judge Y$}
\LeftLabel{{\bf $\AND$ Left 2: \quad}}
\UnaryInfC{$Z \AND X \judge Y$}
\DisplayProof
}
\end{center}

\begin{center}
\fbox{
\AxiomC{$X \judge Y$}
\AxiomC{$X \judge Z$}
\LeftLabel{{\bf $\AND$ Right: \quad}}
\BinaryInfC{$X \judge Y \AND Z$}
\DisplayProof
}
\end{center}

\subsection{Rules Unique to EL}
EL has six rules for capturing its distinctive properties: the relations betwen $\langle \rangle$, $!$ and $\bot$.
The {\bf Bottom Right 1} axiom captures how $\langle \rangle$ and $!$ interact:
\[
\boxed
{
\langle a \rangle X \AND \fBang A \judge \bot \mbox{ if } a \notin A
}
\]
{\bf Bottom Right 2} is another axiom for deriving $\bot$ from inside $\langle \rangle$:
\[
\boxed
{
\langle a \rangle \bot \judge \bot
}
\]
There is a rule for strengthening $\fBang$ on the left hand side: 
\begin{center}
\fbox{
\AxiomC{$X \AND \fBang A \judge Y$}
\LeftLabel{{\bf ! Left: \quad}}
\RightLabel{where $A' \subseteq A $}
\UnaryInfC{$X \AND \fBang A' \judge Y$}
\DisplayProof
}
\end{center}
There is a rule for weakening $\fBang$ on the right hand side: 
\begin{center}
\fbox{
\AxiomC{$X \judge \fBang A$}
\LeftLabel{{\bf ! Right 1: \quad}}
\RightLabel{where $A \subseteq A'$}
\UnaryInfC{$X \judge \fBang A'$}
\DisplayProof
}
\end{center}
There is a rule for taking the intersection of two $!$s on the right hand side:
\begin{center}
\fbox{
\AxiomC{$X \judge \fBang A_1$}
\AxiomC{$X \judge \fBang A_2$}
\LeftLabel{{\bf ! Right 2: \quad}}
\BinaryInfC{$X \judge \fBang (A_1 \cap A_2)$}
\DisplayProof
}
\end{center}
Finally, there is a rule for adding transitions to judgements:
\begin{center}
\fbox{
\AxiomC{$X \judge Y$}
\LeftLabel{{\bf Transition Normal: \quad}}
\UnaryInfC{$\langle a \rangle X \judge \langle a \rangle Y$}
\DisplayProof
}
\end{center}

\subsection{Soundness and completeness}

We will show that $p \models q$ implies there is a derivation of $p \judge q$.
Our proof will make use of two lemmas:
\begin{itemize}
\item
Lemma 4: if $m \models p$ then $\theta(m) \judge p$.
\item
Lemma 5: for all formulae $p$, $p \judge \theta(\mu(p))$.
\end{itemize}
With these two lemmas in hand, the proof is straightforward.
\begin{theorem}
If $p \models q$ then $p \judge q$
\end{theorem}
\begin{proof}
Assume $p \models q$. 
Then all models which satisfy $p$ also satisfy $q$.
In particular, $\mu(p) \models q$.
Then $\theta(\mu(p)) \judge q$ by Lemma 1.
But we also have, by Lemma 2, $p \judge \theta(\mu(p)) $.
So by transitivity, we have $p \judge q$.
\qed
\end{proof}
Next we will prove Lemma 4.
\begin{lemma}
If $m \models p$ then $\theta(m) \judge p$.
\end{lemma}
\begin{proof}
Induction on $p$.
\setcounter{mycase}{0}

\begin{mycase}
$p$ is $\top$
\end{mycase}
Then we can prove  $\theta(m) \judge p$ immediately using axiom {\bf $\top$ Right}.

\begin{mycase}
$p$ is $q \AND q'$
\end{mycase}
By the induction hypothesis, $\theta(m) \judge q$ and $\theta(m) \judge q'$.
The proof of $\theta(m) \judge q \AND q'$ follows immediately using {\bf $\AND$ Right}.

\begin{mycase}
$p$ is $\langle a \rangle q$
\end{mycase}
If $m \models \langle a \rangle q$, then either $m = \bot$ or $m$ is a pointed model of the form $(l,w)$.
\begin{subcase}
$m = \bot$
\end{subcase}
In this case, $\theta(m) = \theta(\bot) = \bot$. (Recall, that we are overloading $\bot$ to mean both the pointed model at the bottom of our lattice and a formula (such as $\langle s \rangle \top \AND !\{\}$) which is always false).
In this case, $ \theta(\bot) \judge  \langle a \rangle q$ using {\bf $\bot$ Left}.

\begin{subcase}
 $m$ is a pointed model of the form $(l,w)$
 \end{subcase}
Given $m \models \langle a \rangle q$, and that $m$ is a pointed model of the form $(l,w)$, we know that:
\[
(l,w) \models \langle a \rangle q
\]
From the satisfaction clause for $\langle a \rangle$, it follows that:
\[
\exists w' \mbox{ such that } w \xrightarrow{a} w' \mbox { and } (l,w') \models q
\]
By the induction hypothesis:
\[
\theta( (l,w') ) \judge q
\]
Now by {\bf Transition Normal}:
\[
\langle a \rangle \theta( (l,w') ) \judge \langle a \rangle q
\]
Using repeated application of {\bf $\AND$ Left}, we can show:
\[
\theta((l,w)) \judge \langle a \rangle \theta((l,w'))
\]
Finally, using {\bf Transitivity}, we derive:
\[
\theta((l,w)) \judge  \langle a \rangle q
\]
\begin{mycase}
$p$ is $\fBang q$
\end{mycase}
If $(l,w) \models \fBang A$, then $\lambda(w) \subseteq A$.
Then $\theta(l,w) = ! \; \lambda(w) \AND \phi$.
Now we can prove $! \; \lambda(w) \AND \phi \judge \fBang A$ using  {\bf $!$ Right 1} and repeated applications of {\bf $\AND$ Left}.
\qed
\end{proof}

\begin{lemma}
For all formulae $p$, we can derive $p \judge \theta(\mu(p))$.
\end{lemma}
Explanation: $\mu(p)$ is the simplest model satisfying $p$, and $\theta(m)$ is the simplest formula describing $m$, so $\theta(\mu(p))$ is a simplified form of $p$. This lemma states that EL has the inferential capacity to transform any proposition into its simplified form.
\begin{proof}
Induction on $p$.

\setcounter{mycase}{0}

\begin{mycase}
$p$ is $\top$
\end{mycase}
Then we can prove  $\top \judge \top$ using either {\bf $\top$ Right} or {\bf Identity}.

\begin{mycase}
$p$ is $q \AND q'$
\end{mycase}
By the induction hypothesis, $q \judge \theta(\mu(q))$ and $q' \judge \theta(\mu(q'))$.
Using {\bf $\AND$ Left} and {\bf $\AND$ Right}, we can show:
\[
q \AND q' \judge \theta(\mu(q)) \AND \theta(\mu(q'))
\]
Lemma 6, proven below, states that, for all models $m$ and $n$:
\[
\theta(m) \AND \theta(n) \judge \theta (m \sqcap n)
\]
From Lemma 6 (substituting $\mu(q)$ for $m$ and $\mu(q')$ for $n$), it follows that:
\[
\theta(\mu(q)) \AND \theta(\mu(q')) \judge \theta(\mu(q \AND q'))
\]
Our desired result follows using {\bf Transitivity}.

\begin{mycase}
$p$ is $\langle a \rangle q$
\end{mycase}
By the induction hypothesis, $q \judge \theta(\mu(q))$.
Now there are two sub-cases to consider, depending on whether or not $\theta(\mu(q)) = \bot$.
\begin{subcase}
$\theta(\mu(q)) = \bot$
\end{subcase}
In this case, $\theta(\mu(\langle a \rangle q))$ also equals $\bot$. 
By the induction hypothesis:
\[
q \judge \bot
\]
By {\bf Transition Normal}:
\[
\langle a \rangle q \judge \langle a \rangle \bot
\]
By {\bf Bottom Right 2}:
\[
\langle a \rangle \bot \judge \bot
\]
The desired proof that:
\[
\langle a \rangle q \judge \bot
\]
follows by {\bf Transitivity}.
\begin{subcase}
$\theta(\mu(q)) \neq \bot$
\end{subcase}
By the induction hypothesis, $q \judge \theta(\mu(q))$.
So, by {\bf Transition Normal}:
\[
\langle a \rangle q \judge \langle a \rangle \theta(\mu(q))
\]
The desired conclusion follows from noting that:
\[
 \langle a \rangle \theta(\mu(q)) = \theta(\mu(\langle a \rangle q))
 \]
 \begin{mycase}
$p$ is $\fBang A$
\end{mycase}
If $p$ is $\fBang A$, then $ \theta(\mu(p))$ is $\fBang A \AND \top$.
We can prove $\fBang A \judge \fBang A \AND \top$ using {\bf $\AND$ Right}, {\bf $\top$ Right} and {\bf Identity}.
\qed
\end{proof}
Finally, to fill the hole in Case 2 above, we need to show that:
\begin{lemma}
For all models $m$ and $n$, $\theta(m) \AND \theta(n) \judge \theta (m \sqcap n)$.
\end{lemma}

\begin{proof}

There are two cases to consider, depending on whether or not $(m \sqcap n) = \bot$.

\setcounter{mycase}{0}

\begin{mycase}
$(m \sqcap n) = \bot$
\end{mycase}
If $(m \sqcap n) = \bot$, there are three possibilities:
\begin{itemize}
\item
$m = \bot$
\item
$n = \bot$
\item
Neither $m$ nor $n$ are $\bot$, but together they are incompatible. 
\end{itemize}
If either $m$ or $n$ is $\bot$, then the proof is a simple application of {\bf Identity} followed by {\bf $\AND$ Left}.

Next, let us consider the case where neither $m$ nor $n$ are $\bot$, but together they are incompatible.
Let $m$ be the pointed model $(l, w)$ and let $n$ be $(l', w')$.
If $(l, w) \sqcap (l', w') = \bot$, then\footnote{The alternative, in which the $s$-transition is in $n$ and the transition-restriction is in $m$, is identical, swapping $m$ with $n$.} there exists a symbol $s$ and a state $w_2$ such that $w \xrightarrow{s} w_2$ but $s \notin \lambda'(w')$.

In this case, by the definition of $\theta$, $\theta(m) \judge \langle s \rangle \top$, using  {\bf Identity} and repeated applications of {\bf $\AND$ Left}.
Further, by the definition of $\theta$, $\theta(n) \judge \; ! \; \lambda'(w')$, using  {\bf Identity} and repeated applications of by {\bf $\AND$ Left}. Again, $s \notin  \lambda'(w')$.

Therefore,  $\theta(m) \AND \theta(n) \judge \bot$, using  {\bf $\AND$ Right} and  {\bf Bottom Right 1}.
\begin{mycase}
$(m \sqcap n) \neq \bot$
\end{mycase}
In this case, let $(l,w) = m$ and let $(l',w')=n$.
Then, from the definition of $\mathsf{merge}$ above:
\[
\theta((l,w) \sqcap (l',w')) = \; ! \; (\lambda(w) \cap \lambda'(w')) \AND \bigwedge_{w \xrightarrow{s} w_2} \langle s \rangle \theta((l, w_2)) \AND \bigwedge_{w' \xrightarrow{s} w_3} \langle s \rangle \theta((l', w_3))
\]
We need to show that $\theta((l,w)) \AND \theta((l',w')) \judge \theta((l,w) \sqcap (l',w'))$ - or in other words, that:
\begin{itemize}
\item
$\theta((l,w)) \AND \theta((l',w')) \judge \; ! \; (\lambda(w) \cap \lambda'(w'))$
\item
$\theta((l,w)) \AND \theta((l',w')) \judge \langle s \rangle \theta((l, w_2))$ for all $s,w_2$ such that $w \xrightarrow{s} w_2$
\item
$\theta((l,w)) \AND \theta((l',w')) \judge \langle s \rangle \theta((l, w_3))$ for all $s,w_3$ such that $w' \xrightarrow{s} w_3$
\end{itemize}
To show $\theta((l,w)) \AND \theta((l',w')) \judge \; ! \; (\lambda(w) \cap \lambda'(w'))$, note that $\theta((l,w))  \judge \; ! \; \lambda(w)$ and $\theta((l',w')) \judge \; ! \;  \lambda'(w'))$.
We can derive $\theta((l,w)) \AND \theta((l',w')) \judge \; ! \; (\lambda(w) \cap \lambda'(w'))$ using {\bf $\AND$ Right} and {\bf $!$ Right 2}. 

To show $\theta((l,w)) \AND \theta((l',w')) \judge \langle s \rangle \theta((l, w_2))$, observe from the definition of $\theta$ that $\theta((l,w))$ contains a conjunct $\langle s \rangle \theta((l, w_2))$, so the proof follows from  {\bf $\AND$ Right} and repeated applications of  {\bf $\AND$ Left}. The same procedure applies to show $\theta((l,w)) \AND \theta((l',w')) \judge \langle s \rangle \theta((l, w_3))$ for all $s,w_3$ such that $w' \xrightarrow{s} w_3$.

This completes Lemma 6 and hence the Completeness Proof.
\qed

\end{proof}

\subsection{The standard translation from  EL into FOL}

We will translate EL into a restricted fragment of FOL.
There are three types of predicate:
\begin{itemize}
\item
A 0-place predicate $\top$, which is true in all models.
\item
A set of two-place predicates $Arr_s(x, y)$, one for each $s \in \mathcal{S}$, where $x$ and $y$ are of type $State$. $Arr_s(x, y)$ is true if $x \xrightarrow{s} y$.
\item
A set of one-place predicates $Restrict_A$, one for each finite subset $A \subseteq \mathcal{S}$. 
$Restrict_{A}(x)$ is true if $\lambda(x) = A$.
\end{itemize}
With $x_1, x_2$ being state variables, $s$ a symbol in $\mathcal{S}$, and $A$ ranging over subsets of $\mathcal{S}$, our restricted fragment of FOL has formulae of the form:
\begin{GRAMMAR}
  \phi 
     &\quad ::= \quad&
  \top \fOr Arr_{s}(x_1, x_2)\fOr Restrict_A(x_1) \fOr \phi_1 \AND \phi_2 \fOr \exists x_1 . \phi 
\end{GRAMMAR}
Notice that this fragment of FOL has no negation, disjunction, implication, or universal quantification.

The translation of an EL formula is relative to a variable $x$ (which will be instantiated to the particular state at which we are evaluating the formula):
\begin{eqnarray}
T_x(\top) & = & \top \nonumber \\
T_x(\phi_1 \AND \phi_2) & = & T_x(\phi_1) \AND T_x(\phi_2) \nonumber \\
T_x(\langle s \rangle \phi) & = & \exists y \; . \; Arr_s(x,y) \AND T_y(\phi) \nonumber \\
T_x(\fBang A) & = & Restrict_A(x) \nonumber
\end{eqnarray}
So, for example:
\[
T_x(\langle a \rangle \top \AND \fBang \{a\}) = \exists y \; . \; Arr_a(x,y) \AND \top \AND Restrict_{\{a\}}(x)
\]


\subsection{Compactness}

