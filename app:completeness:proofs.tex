\section{Proofs for Lemmas \ref{lemma:completeness:4} and \ref{lemma:completeness:5}}\label{app:completeness:proofs}

\subsection{Proof for Lemmas \ref{lemma:completeness:4}}
If $\MMM\models \phi$ then $\CHAR{\MMM} \judge \phi$.

\NI We proceed by induction on $\phi$.

\martin{no need to number cases if numbers are not referred to later.}
\setcounter{mycase}{0}

\begin{mycase}
$\phi$ is $\top$
\end{mycase}
Then we can prove  $ \CHAR{\MMM} \judge \phi$ immediately using axiom {\bf $\top$ Right}.

\begin{mycase}
$\phi$ is $\psi \AND \psi'$
\end{mycase}
By the induction hypothesis, $  \CHAR{\MMM} \judge \psi$ and $  \CHAR{\MMM} \judge \psi'$.
The proof of $  \CHAR{\MMM} \judge \psi \AND \psi'$ follows immediately using {\bf $\AND$ Right}.

\begin{mycase}
$\phi$ is $\langle a \rangle \psi$
\end{mycase}
If $\MMM \models \langle a \rangle \psi$, then either $\MMM = \bot$ or $\MMM$ is a  model of the form $(\CAL{L},w)$.
\begin{subcase}
$\MMM = \bot$
\end{subcase}
In this case, $  \CHAR{\MMM} =  \CHAR{\bot} = \bot$. (Recall, that we are overloading $\bot$ to mean both the  model at the bottom of our lattice and a formula (such as $\langle s \rangle \top \AND !\{\}$) which is always false).
In this case, $  \CHAR{\bot} \judge  \langle a \rangle \psi$ using {\bf $\bot$ Left}.

\begin{subcase}
 $m$ is a  model of the form $(\CAL{L},w)$
 \end{subcase}
Given $\MMM \models \langle a \rangle \psi$, and that $\MMM$ is a  model of the form $(\CAL{L},w)$, we know that:
\[
(\CAL{L},w) \models \langle a \rangle \psi
\]
From the satisfaction clause for $\langle a \rangle$, it follows that:
\[
\exists w' \mbox{ such that } w \xrightarrow{a} w' \mbox { and } (\CAL{L},w') \models \psi
\]
By the induction hypothesis:
\[
 \CHAR{(\CAL{L},w')} \judge \psi
\]
Now by {\bf Transition Normal}:
\[
\langle a \rangle  \CHAR{(\CAL{L},w')} \judge \langle a \rangle \psi
\]
Using repeated application of {\bf $\AND$ Left}, we can show:
\[
 \CHAR{(\CAL{L},w)} \judge \langle a \rangle  \CHAR{(\CAL{L},w')}
\]
Finally, using {\bf Transitivity}, we derive:
\[
 \CHAR{(\CAL{L},w)} \judge  \langle a \rangle \psi
\]
\begin{mycase}
$\phi$ is $\fBang \psi$
\end{mycase}
If $(\CAL{L},w) \models \fBang A$, then $\lambda(w) \subseteq A$.
Then $ \CHAR{(\CAL{L},w)} = ! \; \lambda(w) \AND \phi$.
Now we can prove $! \; \lambda(w) \AND \phi \judge \fBang A$ using  {\bf $!$ Right 1} and repeated applications of {\bf $\AND$ Left}.


\subsection{Proof for Lemmas \ref{lemma:completeness:5}}

Now we prove Lemma \ref{lemma:completeness:5}: 
For all formulae $\phi$, we can derive $\phi \judge \CHAR{\SIMPL{\phi}}$.

\begin{proof}
Induction on $\phi$.

\setcounter{mycase}{0}

\begin{mycase}
$\phi$ is $\top$
\end{mycase}
Then we can prove  $\top \judge \top$ using either {\bf $\top$ Right} or {\bf Identity}.

\begin{mycase}
$\phi$ is $\psi \AND \psi'$
\end{mycase}
By the induction hypothesis, $\psi \judge  \CHAR{\SIMPL{\psi}}$ and $\psi' \judge  \CHAR{\SIMPL{\psi'}}$.
Using {\bf $\AND$ Left} and {\bf $\AND$ Right}, we can show:
\[
\psi \AND \psi' \judge  \CHAR{\SIMPL{\psi}} \AND  \CHAR{\SIMPL{\psi'}}
\]
Lemma \ref{final_completeness_lemma}, proven below, states that, for all models $\MMM$ and $\MMM_2$:
\[
  \CHAR{\MMM} \AND  \CHAR{\MMM_2} \judge  \CHAR{\MMM \sqcap \MMM_2}
\]
From Lemma \ref{final_completeness_lemma} (substituting $\SIMPL{\psi}$ for $\MMM$ and $\SIMPL{\psi'}$ for $\MMM_2$), it follows that:
\[
 \CHAR{\SIMPL{\psi}} \AND  \CHAR{\SIMPL{\psi'}} \judge  \CHAR{\SIMPL{\psi \AND \psi'}}
\]
Our desired result follows using {\bf Transitivity}.

\begin{mycase}
$\phi$ is $\langle a \rangle \psi$
\end{mycase}
By the induction hypothesis, $\psi \judge  \CHAR{\SIMPL{\psi}}$.
Now there are two sub-cases to consider, depending on whether or not $ \CHAR{\SIMPL{\psi}} = \bot$.
\begin{subcase}
$ \CHAR{\SIMPL{\psi}} = \bot$
\end{subcase}
In this case, $ \CHAR{\SIMPL{\langle a \rangle \psi}}$ also equals $\bot$. 
By the induction hypothesis:
\[
\psi \judge \bot
\]
By {\bf Transition Normal}:
\[
\langle a \rangle \psi \judge \langle a \rangle \bot
\]
By {\bf Bottom Right 2}:
\[
\langle a \rangle \bot \judge \bot
\]
The desired proof that:
\[
\langle a \rangle \psi \judge \bot
\]
follows by {\bf Transitivity}.
\begin{subcase}
$ \CHAR{\SIMPL{\psi}} \neq \bot$
\end{subcase}
By the induction hypothesis, $\psi \judge  \CHAR{\SIMPL{\psi}}$.
So, by {\bf Transition Normal}:
\[
\langle a \rangle \psi \judge \langle a \rangle  \CHAR{\SIMPL{\psi}}
\]
The desired conclusion follows from noting that:
\[
 \langle a \rangle  \CHAR{\SIMPL{\psi}} =  \CHAR{\SIMPL{\langle a \rangle \psi}}
 \]
 \begin{mycase}
$\phi$ is $\fBang A$
\end{mycase}
If $\phi$ is $\fBang A$, then $  \CHAR{\SIMPL{\phi}}$ is $\fBang A \AND \top$.
We can prove $\fBang A \judge \fBang A \AND \top$ using {\bf $\AND$ Right}, {\bf $\top$ Right} and {\bf Identity}.

\end{proof}
Finally, to fill the hole in Case 2 above, we need to show that:
\begin{lemma}
\label{final_completeness_lemma}
For all models $\MMM$ and $\MMM_2$, $  \CHAR{\MMM} \AND  \CHAR{\MMM_2} \judge  \CHAR{\MMM \sqcap \MMM_2}$.
\end{lemma}

\begin{proof}

There are two cases to consider, depending on whether or not $(\MMM \sqcap \MMM_2) = \bot$.

\setcounter{mycase}{0}

\begin{mycase}
$(\MMM \sqcap \MMM_2) = \bot$
\end{mycase}
If $(\MMM \sqcap \MMM_2) = \bot$, there are three possibilities:
\begin{itemize}
\item
$\MMM = \bot$
\item
$\MMM_2 = \bot$
\item
Neither $\MMM$ nor $\MMM_2$ are $\bot$, but together they are incompatible. 
\end{itemize}
If either $\MMM$ or $\MMM_2$ is $\bot$, then the proof is a simple application of {\bf Identity} followed by {\bf $\AND$ Left}.

Next, let us consider the case where neither $\MMM$ nor $\MMM_2$ are $\bot$, but together they are incompatible.
Let $\MMM$ be the  model $(\CAL{L}, w)$ and let $\MMM_2$ be $(\CAL{L}', w')$.
If $(\CAL{L}, w) \sqcap (\CAL{L}', w') = \bot$, then\footnote{The alternative, in which the $a$-transition is in $n$ and the transition-restriction is in $\MMM$, is identical, swapping $\MMM$ with $\MMM_2$.} there exists a symbol $a$ and a state $w_2$ such that $w \xrightarrow{a} w_2$ but $a \notin \lambda'(w')$.

In this case, by the definition of $ \CHAR$, $  \CHAR{\MMM} \judge \langle a \rangle \top$, using  {\bf Identity} and repeated applications of {\bf $\AND$ Left}.
Further, by the definition of $ \CHAR$, $ \CHAR{\MMM_2} \judge \; ! \; \lambda'(w')$, using  {\bf Identity} and repeated applications of by {\bf $\AND$ Left}. Again, $a \notin  \lambda'(w')$.

Therefore,  $  \CHAR{\MMM} \AND  \CHAR{\MMM_2} \judge \bot$, using  {\bf $\AND$ Right} and  {\bf Bottom Right 1}.
\begin{mycase}
$(\MMM \sqcap \MMM_2) \neq \bot$
\end{mycase}
In this case, let $(\CAL{L},w) = \MMM$ and let $(\CAL{L}',w')=\MMM_2$.
Then, from the definition of $\mathsf{merge}$ above:
\[
 \CHAR{(\CAL{L},w) \sqcap (\CAL{L}',w')} = \; ! \; (\lambda(w) \cap \lambda'(w')) \AND \bigwedge_{w \xrightarrow{a} w_2} \langle a \rangle  \CHAR{(\CAL{L}, w_2)} \AND \bigwedge_{w' \xrightarrow{a} w_3} \langle a \rangle  \CHAR{(\CAL{L}', w_3)}
\]
We need to show that $ \CHAR{(\CAL{L},w)} \AND  \CHAR{(\CAL{L}',w')} \judge  \CHAR{(\CAL{L},w) \sqcap (\CAL{L}',w')}$ - or in other words, that:
\begin{itemize}
\item
$ \CHAR{(\CAL{L},w)} \AND  \CHAR{(\CAL{L}',w')} \judge \; ! \; (\lambda(w) \cap \lambda'(w'))$
\item
$ \CHAR{(\CAL{L},w)} \AND  \CHAR{(\CAL{L}',w')} \judge \langle a \rangle  \CHAR{(\CAL{L}, w_2)}$ for all $a,w_2$ such that $w \xrightarrow{a} w_2$
\item
$ \CHAR{(\CAL{L},w)} \AND  \CHAR{(\CAL{L}',w')} \judge \langle a \rangle  \CHAR{(\CAL{L}, w_3)}$ for all $a,w_3$ such that $w' \xrightarrow{a} w_3$
\end{itemize}
To show $ \CHAR{(\CAL{L},w)} \AND  \CHAR{(\CAL{L}',w')} \judge \; ! \; (\lambda(w) \cap \lambda'(w'))$, note that $ \CHAR{(\CAL{L},w)}  \judge \; ! \; \lambda(w)$ and $ \CHAR{(\CAL{L}',w')} \judge \; ! \;  \lambda'(w'))$.
We can derive $ \CHAR{(\CAL{L},w)} \AND  \CHAR{(\CAL{L}',w')} \judge \; ! \; (\lambda(w) \cap \lambda'(w'))$ using {\bf $\AND$ Right} and {\bf $!$ Right 2}. 

To show $ \CHAR{(\CAL{L},w)} \AND  \CHAR{(\CAL{L}',w')} \judge \langle a \rangle  \CHAR{(\CAL{L}, w_2)}$, observe from the definition of $ \CHAR$ that $ \CHAR{(\CAL{L},w)}$ contains a conjunct $\langle a \rangle  \CHAR{(\CAL{L}, w_2)}$, so the proof follows from  {\bf $\AND$ Right} and repeated applications of  {\bf $\AND$ Left}. The same procedure applies to show $ \CHAR{(\CAL{L},w)} \AND  \CHAR{(\CAL{L}',w')} \judge \langle a \rangle  \CHAR{(\CAL{L}, w_3)}$ for all $a,w_3$ such that $w' \xrightarrow{a} w_3$.

This completes Lemma 6 and hence the Completeness Proof.


\end{proof}
