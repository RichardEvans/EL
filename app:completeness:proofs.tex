\section{Proofs for Lemmas \ref{lemma:completeness:4} and \ref{lemma:completeness:5}}\label{app:completeness:proofs}

\subsection{Proofs for Lemmas \ref{lemma:completeness:4}}

\NI We proceed by induction on $p$.

\martin{no need to number cases if numbers are not referred to later.}
\setcounter{mycase}{0}

\begin{mycase}
$p$ is $\top$
\end{mycase}
Then we can prove  $\theta(m) \judge p$ immediately using axiom {\bf $\top$ Right}.

\begin{mycase}
$p$ is $q \AND q'$
\end{mycase}
By the induction hypothesis, $\theta(m) \judge q$ and $\theta(m) \judge q'$.
The proof of $\theta(m) \judge q \AND q'$ follows immediately using {\bf $\AND$ Right}.

\begin{mycase}
$p$ is $\langle a \rangle q$
\end{mycase}
If $m \models \langle a \rangle q$, then either $m = \bot$ or $m$ is a  model of the form $(l,w)$.
\begin{subcase}
$m = \bot$
\end{subcase}
In this case, $\theta(m) = \theta(\bot) = \bot$. (Recall, that we are overloading $\bot$ to mean both the  model at the bottom of our lattice and a formula (such as $\langle s \rangle \top \AND !\{\}$) which is always false).
In this case, $ \theta(\bot) \judge  \langle a \rangle q$ using {\bf $\bot$ Left}.

\begin{subcase}
 $m$ is a  model of the form $(l,w)$
 \end{subcase}
Given $m \models \langle a \rangle q$, and that $m$ is a  model of the form $(l,w)$, we know that:
\[
(l,w) \models \langle a \rangle q
\]
From the satisfaction clause for $\langle a \rangle$, it follows that:
\[
\exists w' \mbox{ such that } w \xrightarrow{a} w' \mbox { and } (l,w') \models q
\]
By the induction hypothesis:
\[
\theta( (l,w') ) \judge q
\]
Now by {\bf Transition Normal}:
\[
\langle a \rangle \theta( (l,w') ) \judge \langle a \rangle q
\]
Using repeated application of {\bf $\AND$ Left}, we can show:
\[
\theta((l,w)) \judge \langle a \rangle \theta((l,w'))
\]
Finally, using {\bf Transitivity}, we derive:
\[
\theta((l,w)) \judge  \langle a \rangle q
\]
\begin{mycase}
$p$ is $\fBang q$
\end{mycase}
If $(l,w) \models \fBang A$, then $\lambda(w) \subseteq A$.
Then $\theta(l,w) = ! \; \lambda(w) \AND \phi$.
Now we can prove $! \; \lambda(w) \AND \phi \judge \fBang A$ using  {\bf $!$ Right 1} and repeated applications of {\bf $\AND$ Left}.
\qed

\subsection{Proofs for Lemmas \ref{lemma:completeness:5}}

Now we prove Lemma \ref{lemma:completeness:5}

\begin{proof}
Induction on $p$.

\setcounter{mycase}{0}

\begin{mycase}
$p$ is $\top$
\end{mycase}
Then we can prove  $\top \judge \top$ using either {\bf $\top$ Right} or {\bf Identity}.

\begin{mycase}
$p$ is $q \AND q'$
\end{mycase}
By the induction hypothesis, $q \judge \theta(\SIMPL{q})$ and $q' \judge \theta(\SIMPL{q'})$.
Using {\bf $\AND$ Left} and {\bf $\AND$ Right}, we can show:
\[
q \AND q' \judge \theta(\SIMPL{q}) \AND \theta(\SIMPL{q'})
\]
Lemma 6, proven below, states that, for all models $m$ and $n$:
\[
\theta(m) \AND \theta(n) \judge \theta (m \sqcap n)
\]
From Lemma 6 (substituting $\SIMPL{q}$ for $m$ and $\SIMPL{q'}$ for $n$), it follows that:
\[
\theta(\SIMPL{q}) \AND \theta(\SIMPL{q'}) \judge \theta(\SIMPL{q \AND q'})
\]
Our desired result follows using {\bf Transitivity}.

\begin{mycase}
$p$ is $\langle a \rangle q$
\end{mycase}
By the induction hypothesis, $q \judge \theta(\SIMPL{q})$.
Now there are two sub-cases to consider, depending on whether or not $\theta(\SIMPL{q}) = \bot$.
\begin{subcase}
$\theta(\SIMPL{q}) = \bot$
\end{subcase}
In this case, $\theta(\SIMPL{\langle a \rangle q})$ also equals $\bot$. 
By the induction hypothesis:
\[
q \judge \bot
\]
By {\bf Transition Normal}:
\[
\langle a \rangle q \judge \langle a \rangle \bot
\]
By {\bf Bottom Right 2}:
\[
\langle a \rangle \bot \judge \bot
\]
The desired proof that:
\[
\langle a \rangle q \judge \bot
\]
follows by {\bf Transitivity}.
\begin{subcase}
$\theta(\SIMPL{q}) \neq \bot$
\end{subcase}
By the induction hypothesis, $q \judge \theta(\SIMPL{q})$.
So, by {\bf Transition Normal}:
\[
\langle a \rangle q \judge \langle a \rangle \theta(\SIMPL{q})
\]
The desired conclusion follows from noting that:
\[
 \langle a \rangle \theta(\SIMPL{q}) = \theta(\SIMPL{\langle a \rangle q})
 \]
 \begin{mycase}
$p$ is $\fBang A$
\end{mycase}
If $p$ is $\fBang A$, then $ \theta(\SIMPL{p})$ is $\fBang A \AND \top$.
We can prove $\fBang A \judge \fBang A \AND \top$ using {\bf $\AND$ Right}, {\bf $\top$ Right} and {\bf Identity}.
\qed
\end{proof}
Finally, to fill the hole in Case 2 above, we need to show that:
\begin{lemma}
For all models $m$ and $n$, $\theta(m) \AND \theta(n) \judge \theta (m \sqcap n)$.
\end{lemma}

\begin{proof}

There are two cases to consider, depending on whether or not $(m \sqcap n) = \bot$.

\setcounter{mycase}{0}

\begin{mycase}
$(m \sqcap n) = \bot$
\end{mycase}
If $(m \sqcap n) = \bot$, there are three possibilities:
\begin{itemize}
\item
$m = \bot$
\item
$n = \bot$
\item
Neither $m$ nor $n$ are $\bot$, but together they are incompatible. 
\end{itemize}
If either $m$ or $n$ is $\bot$, then the proof is a simple application of {\bf Identity} followed by {\bf $\AND$ Left}.

Next, let us consider the case where neither $m$ nor $n$ are $\bot$, but together they are incompatible.
Let $m$ be the  model $(l, w)$ and let $n$ be $(l', w')$.
If $(l, w) \sqcap (l', w') = \bot$, then\footnote{The alternative, in which the $s$-transition is in $n$ and the transition-restriction is in $m$, is identical, swapping $m$ with $n$.} there exists a symbol $s$ and a state $w_2$ such that $w \xrightarrow{s} w_2$ but $s \notin \lambda'(w')$.

In this case, by the definition of $\theta$, $\theta(m) \judge \langle s \rangle \top$, using  {\bf Identity} and repeated applications of {\bf $\AND$ Left}.
Further, by the definition of $\theta$, $\theta(n) \judge \; ! \; \lambda'(w')$, using  {\bf Identity} and repeated applications of by {\bf $\AND$ Left}. Again, $s \notin  \lambda'(w')$.

Therefore,  $\theta(m) \AND \theta(n) \judge \bot$, using  {\bf $\AND$ Right} and  {\bf Bottom Right 1}.
\begin{mycase}
$(m \sqcap n) \neq \bot$
\end{mycase}
In this case, let $(l,w) = m$ and let $(l',w')=n$.
Then, from the definition of $\mathsf{merge}$ above:
\[
\theta((l,w) \sqcap (l',w')) = \; ! \; (\lambda(w) \cap \lambda'(w')) \AND \bigwedge_{w \xrightarrow{s} w_2} \langle s \rangle \theta((l, w_2)) \AND \bigwedge_{w' \xrightarrow{s} w_3} \langle s \rangle \theta((l', w_3))
\]
We need to show that $\theta((l,w)) \AND \theta((l',w')) \judge \theta((l,w) \sqcap (l',w'))$ - or in other words, that:
\begin{itemize}
\item
$\theta((l,w)) \AND \theta((l',w')) \judge \; ! \; (\lambda(w) \cap \lambda'(w'))$
\item
$\theta((l,w)) \AND \theta((l',w')) \judge \langle s \rangle \theta((l, w_2))$ for all $s,w_2$ such that $w \xrightarrow{s} w_2$
\item
$\theta((l,w)) \AND \theta((l',w')) \judge \langle s \rangle \theta((l, w_3))$ for all $s,w_3$ such that $w' \xrightarrow{s} w_3$
\end{itemize}
To show $\theta((l,w)) \AND \theta((l',w')) \judge \; ! \; (\lambda(w) \cap \lambda'(w'))$, note that $\theta((l,w))  \judge \; ! \; \lambda(w)$ and $\theta((l',w')) \judge \; ! \;  \lambda'(w'))$.
We can derive $\theta((l,w)) \AND \theta((l',w')) \judge \; ! \; (\lambda(w) \cap \lambda'(w'))$ using {\bf $\AND$ Right} and {\bf $!$ Right 2}. 

To show $\theta((l,w)) \AND \theta((l',w')) \judge \langle s \rangle \theta((l, w_2))$, observe from the definition of $\theta$ that $\theta((l,w))$ contains a conjunct $\langle s \rangle \theta((l, w_2))$, so the proof follows from  {\bf $\AND$ Right} and repeated applications of  {\bf $\AND$ Left}. The same procedure applies to show $\theta((l,w)) \AND \theta((l',w')) \judge \langle s \rangle \theta((l, w_3))$ for all $s,w_3$ such that $w' \xrightarrow{s} w_3$.

This completes Lemma 6 and hence the Completeness Proof.
\qed

\end{proof}
