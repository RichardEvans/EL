\section{Mathematical preliminaries}\label{preliminaries}

\NI This section briefly surveys the mathematical background of our
paper.

\begin{definition}
Let $\Sigma$ be a set of \emph{actions}.  A \emph{labelled transition
  system over $\Sigma$} is a pair $(S, \rightarrow)$ where $S$ is a
set of \emph{states} and $\rightarrow \subseteq S \times \Sigma \times
S$ is the \emph{transition relation}.  We write $x \xrightarrow{s} y$
to abbreviate $(x,s,y) \in \rightarrow$. We let $s, s', t, ...$ range
over states, $a, a', b, ...$ range over actions and $\LLL, \LLL', ...$
range over labelled transition systems. We usually speak of labelled
transition systems (LTS) when the set of actions is clear from the
context. If we want to emphasise that the set of actions is
\emph{finite}, we use $A$ instead of $\Sigma$.
\end{definition}

\begin{definition}
Given two labelled transition systems $\LLL_i = (S_i, \rightarrow_i)$
over $\Sigma$ for $i = 1, 2$, a \emph{simulation from $\LLL_1$ to $\LLL_2$}
is a relation $\RRR \subseteq S_1 \times S_2$ such that whenever $(s,
s') \in \RRR$: if $s \TRANS{a} s'$ then there exists a transition $t
\TRANS{a} t'$ with $(t, t') \in \RRR$.  We say $\RRR$ is a
\emph{bisimulation} if both, $\RRR$ and $\RRR^{-1}$ are
simulations. Here $\RRR^{-1} = \{(y, x)\ |\ (x, y) \in \RRR\}$.
We say two states $s, s'$ are \emph{bisimilar}, written $s \BISIM s'$
if there is a bisimulation $\RRR$ with $(s, s') \in \RRR$.
\end{definition}

\NI For a fuller account on labelled transition systems, see
e.g.~\cite{SassoneV:modcontac,HennessyM:Algtheop}. Bisimulations are
discussed in great detail in \cite{SangiorgiD:intbisac}.
