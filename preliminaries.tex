\section{Mathematical preliminaries}\label{preliminaries}

\NI This section briefly surveys the mathematical background of our
paper.

\PARAGRAPH{Order-theory}
A \emph{preorder} is a pair $(S, \sqsubseteq)$ where $S$ is a set, and
$\sqsubseteq$ is a binary relation on $S$ that is reflexive and
transitive. Let $T \subseteq S$ and $x \in S$. We say $x$ is an
\emph{upper bound} of $T$ provided $t \sqsubseteq x$ for all $t \in
T$. If in addition $x \sqsubseteq y$ for all upper bounds $y$ of $T$,
we say that $x$ is the \emph{least} upper bound of $T$.  The set of
all least upper bounds of $T$ is denoted $\BIGLUB T$.  \emph{Lower
  bounds}, \emph{greatest lower bounds} and $\BIGGLB T$ are defined
mutatis mutandis.  A \emph{partial order} is a preorder $\sqsubseteq$
that is also anti-symmetric.  A partial order $(S, \sqsubseteq)$ is a
\emph{lattice} if every pair of elements in $S$ has a least upper and a
greatest lower bound.
A lattice is a \emph{bounded lattice} if it has top and bottom elements $\top$ and $\bot$ such that for all $x \in S$
\begin{eqnarray}
x \sqcap \bot & = & \bot \nonumber \\
x \sqcup \bot & = & x \nonumber \\
x \sqcap \top & = & x \nonumber \\
x \sqcup \top & = & \top \nonumber
\end{eqnarray}

If $(S, \sqsubseteq)$ is a preorder, we can turn it into a
partial-order by quotienting: let $a \simeq b$ iff $a \sqsubseteq b$
as well as $b \sqsubseteq a$. Clearly $\simeq$ is an equivalence. Let
$E$ be the set of all $\simeq$-equivalence classes of $S$. We get a
canonical partial order, denoted $\sqsubseteq_E$, on $E$ by setting:
$[a]_{\simeq} \sqsubseteq_E [b]_{\simeq}$ whenever $a \sqsubseteq
b$. If all relevant upper and lower bounds exist in $(S,
\sqsubseteq)$, then $E, \sqsubseteq_E$ becomes a complete lattice by
setting
\[
   \BIGLUB \{[s]_{\simeq} \ |\ s \in S\ \} = [\BIGLUB S]_{\simeq}
\]
and likewise for the greated lower bound. 

\PARAGRAPH{Transition systems}
Let $\Sigma$ be a set of \emph{actions}.  A \emph{labelled transition
  system over $\Sigma$} is a pair $(\mathcal{S}, \rightarrow)$ where
$\mathcal{S}$ is a set of \emph{states} and $\rightarrow \subseteq
\mathcal{S} \times \Sigma \times \mathcal{S}$ is the \emph{transition
  relation}.  We write $x \xrightarrow{a} y$ to abbreviate $(x,a,y)
\in \rightarrow$. We let $w, w', x, y, z, ...$\martin{we also use $s,
  t,...$! make consistent!} range over states, $a, a', b, ...$ range
over actions and $\LLL, \LLL', ...$ range over labelled transition
systems. We usually speak of labelled transition systems (LTS) when
the set of actions is clear from the context. If we want to emphasise
that the set of actions is \emph{finite}, we use $A$ instead of
$\Sigma$.

We say $\LLL$ is \emph{deterministic} if $x \TRANS{a} y$ and $x \TRANS{a} z$ imply that $y = z$. Otherwise $\LLL$ is
\emph{non-deterministic}.

A labelled transition system is \emph{finitely branching} if for each state $s$,
the set $\{t\ |\ s \TRANS{a} t\}$ is finite.

\PARAGRAPH{Simulations and bisimulations}
Given two labelled transition systems $\LLL_i = (S_i, \rightarrow_i)$
over $\Sigma$ for $i = 1, 2$, a \emph{simulation from $\LLL_1$ to
  $\LLL_2$} is a relation $\RRR \subseteq S_1 \times S_2$ such that
whenever $(s, s') \in \RRR$: if $s \TRANS{a} s'$ then there exists a
transition $t \TRANS{a} t'$ with $(t, t') \in \RRR$.  We write $s \SIM
t$ whenever $(s, t) \in \RRR$ for some simulation $\RRR$.  We say
$\RRR$ is a \emph{bisimulation between $\LLL_1$ and $\LLL_2$} if both,
$\RRR$ and $\RRR^{-1}$ are simulations. Here $\RRR^{-1} = \{(y,
x)\ |\ (x, y) \in \RRR\}$.  We say two states $s, s'$ are
\emph{bisimilar}, written $s \BISIM s'$ if there is a bisimulation
$\RRR$ with $(s, s') \in \RRR$.

\PARAGRAPH{First-order logic}
A \emph{first-order signature} is specified by the following data.
 A set of \emph{function symbols} with associated \emph{arities},
i.e.~a positive integer $\#(f)$ for each function symbol $f$;  a
set of \emph{relation symbols} with associated \emph{arities}, i.e.~a
positive integer $\#(R)$ for each relation symbol $R$;  a set of
\emph{constant symbols}.  A signature $\SSS$ with no function symbols
is called a \emph{relational signature}.

Let $\SSS$ be a signature. An \emph{$\SSS$-model} $\CAL{M}$ is an
object with the following components.  A set $U$ called
\emph{universe}.  The members of $U$ are called \emph{elements} of
$\CAL{M}$; an element $c^\CAL{M}$ for each constant $c$; a function
$f^\CAL{M} : U^{\#f} \rightarrow U$ for each function symbol $f$; a
relation $R^\CAL{M} \subseteq U^{\#R}$ for each relation symbol $R$.

The \emph{terms} and \emph{first-order formulae} for $\SSS$ are given
by the following grammar
\begin{GRAMMAR}
  t &\ ::=\ & x \VERTICAL c \VERTICAL f(t_1, ..., t_n) \\[1mm]
  \phi &::=& t = t' \VERTICAL R(t_1, ..., t_n) \VERTICAL \neg \phi \VERTICAL \phi \AND \psi \VERTICAL \forall x.A
\end{GRAMMAR}

\NI Here $x$ ranges over an infinite set of \emph{variables}, $c$ over
constants, $R$ over $n$-ary relational symbols and $f$ over $n$-ary
function symbols from $\SSS$.  Other logical constructs such as
disjunction or existential quantification are given by de Morgan
duality, and truth $\top$ is an abbreviation for $x = x$.

Given an $\SSS$-model $\CAL{M}$, an \emph{environment}, ranged over by
$\sigma$, is a partial function from variables to $\CAL{M}$'s
universe.  We write $x \mapsto u$ for the environment that maps $x$ to
$u$ and is undefined for all other variables. Moreover, if $\sigma, x
\mapsto u$ is the environment that is exactly like $\sigma$, except
that it also maps $x$ to $u$, assuming that $x$ is not in the domain
of $\sigma$.  The \emph{interpretation} $\SEMB{t}_{\CAL{M}, \sigma}$
of a term $t$ w.r.t. $\CAL{M}$ and $\sigma$ is given by the following
clauses, assuming that the domain of $\sigma$ contains all free
variables of $t$:
\begin{itemize}

\item $\SEMB{x}_{\CAL{M}, \sigma} = \sigma(x)$.
\item $\SEMB{c}_{\CAL{M}, \sigma} = c^{\CAL{M}}$.
\item $\SEMB{f(t_1, ..., t_n)}_{\CAL{M}, \sigma} =
  f^{\CAL{M}}(\SEMB{t_1}_{\CAL{M}, \sigma}, ..., \SEMB{t_n}_{\CAL{M},
    \sigma})$.

\end{itemize}

\NI The \emph{satisfaction relation} $\CAL{M} \models_{\sigma} \phi$
is given by the following clauses, this time assuming that the domain
of $\sigma$ contains all free variables of $\phi$:
\begin{itemize}

\item $\CAL{M} \models_{\sigma} t = t'$ iff $\SEMB{t}_{\CAL{M}, \sigma} = \SEMB{t'}_{\CAL{M}, \sigma}$.
\item $\CAL{M} \models_{\sigma} R(t_1, ..., t_n)$ iff
  $R^{\CAL{M}}(\SEMB{t_1}_{\CAL{M}, \sigma}, ..., \SEMB{t_n}_{\CAL{M},
  \sigma})$.
\item $\CAL{M} \models_{\sigma} \neg \phi$ iff $\CAL{M} \not\models_{\sigma} \phi$.
\item $\CAL{M} \models_{\sigma} \phi \AND \psi$ iff $\CAL{M} \models_{\sigma} \phi$ and $\CAL{M} \models_{\sigma} \psi$.
\item $\CAL{M} \models_{\sigma} \forall x.\phi$ iff for all $u$ in the
  universe of $\CAL{M}$ we have $\CAL{M} \models_{\sigma, x \mapsto v} \phi$.

\end{itemize}

\NI Note that if $\sigma$ and $\sigma'$ agree on the free variables of
$t$, then $\SEMB{t}_{\CAL{M}, \sigma} =\SEMB{t}_{\CAL{M},
  \sigma'}$. Likewise $\CAL{M} \models_{\sigma} \phi$ if and only iff
$\CAL{M} \models_{\sigma'} \phi$, provided $\sigma$ and $\sigma'$agree
on the free variables of $\phi$.

The \emph{theory} of a model $\CAL{M}$, written $\THEORY{\CAL{M}}$, is
the set of all formulae made true by $\CAL{M}$, i.e.~$\THEORY{\CAL{M}}
= \{\phi\ |\ \CAL{M}\models \phi\}$. We say two models $\CAL{M}$ and
$\CAL{N}$ are \emph{elementarily equivalent} if $\THEORY{\CAL{M}} =
\THEORY{\CAL{N}}$. In first-order logic $\THEORY{\CAL{M}} \subseteq
\THEORY{\CAL{N}}$ already implies that $\CAL{M}$ and $\CAL{N}$ are
elementarily equivalent.

\PARAGRAPH{Further reading} Partial orders are explored in \cite{DaveyBA:intlatao}. For a
fuller account on labelled transition systems, see
e.g.~\cite{SassoneV:modcontac,HennessyM:Algtheop}. Bisimulations are
discussed in great detail in \cite{SangiorgiD:intbisac} and
\cite{EndertonHB:matinttl} is one of many good books on first-order
logic.

