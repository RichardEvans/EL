\section{\ELFULL}\label{coreEL}

In this section we introduce the syntax and semantics of \ELFULL{}
(hereafter \ELABR{}).  

\subsection{Syntax}
\label{elsyntax}
\NI Syntactically, eremic logic is a multi-modal logic with one new
operator.

\begin{definition} Let $\Sigma$ be a non-empty set of \emph{actions}.
Actions are ranged over by $a, a', a_1, b, ...$, and $A$ ranges over
finite subsets of $\Sigma$. The \emph{formulae}, ranged over by $\phi,
\psi, \xi ...$, of \ELABR{} are given by the
following grammar.

\begin{GRAMMAR}
  \phi 
     &\quad ::= \quad & 
  \TRUE 
     \VERTICAL 
  \phi \AND \psi
     \VERTICAL 
  \MAY{a}{\phi}
     \VERTICAL 
  \fBang A 
\end{GRAMMAR}
\end{definition}

\NI Before presenting models and the satisfaction relation, we sketch
the meaning of formulae informally. $\TRUE$ is truth and $\phi \AND
\psi$ is the conjunction of formulae $\phi$ and $\psi$. Clearly
$\MAY{a}{\phi}$ is the may-modality which asserts that the current
state may do action $a$, and in doing $a$ transitions evolves into a
now state at which $\phi$ holds. Tantum $A$, from Latin ``tantum''
meaning ``only'' and written $!A$, is the key novelty of eremic logic.
$!A$ means that if $!A$ holds at the current state, then the only
modalities $\MAY{a}{}$ available at the corrent state are those with
$a \in A$.

The set $\Sigma$ of actions must be non-empty to avoid that the logic
is trivial. The restriction to finite subsets of action substantially
simplifies the meta-theory of the logic.

We assume that $\MAY{a}{\phi}$ binds more tightly than conjunction, so
$\MAY{a}{\phi} \AND \psi$ is short for $(\MAY{a}{\phi}) \AND \psi$.
We often abbreviate $\MAY{a}{\TRUE}$ to $\MAY{a}{}$. We define falsity
$\FALSE$ as $!\emptyset \AND \MAY{a}{}$ where $a$ is an arbitrary
action. Note that in the absence of conventional negation, we cannot
readily define disjunction, implication, or must-modalities by de
Morgan duality. We discuss later \martin{Where?} how some of the usual
propositional connectives can be (partially) recovered. \martin{What
  about must modalities?}

\begin{convention}
From now on we assume a fixed set $\Sigma$ of actions, except where
stated otherwise.
\end{convention}

\subsection{Semantics}

\NI The semantics of eremic logic is close to Hennessy-Milner logic
\cite{HennessyM:alglawfndac} but uses \emph{deterministic} transition systems, as
described in Section \ref{preliminaries}, augmented with labels on
states.

%% \begin{definition}

%% \NI We say $\LLL$ is \emph{deterministic} if $x \TRANS{a} y$ and $x \TRANS{a} z$ imply that $y = z$. Otherwise $\LLL$ is
%% \emph{non-deterministic}.
%% \end{definition}

\begin{definition}
An \emph{eremic transition system} is a triple $\LLL = (S,
\rightarrow, \lambda)$, where $(S, \rightarrow)$ is a deterministic
labelled transition system over $\Sigma$, and $\lambda$ is a function
from states to sets of actions (not necessarily finite), subject to
the following constraints:
\begin{itemize}

\item For all states $s \in S$ it is the case that $ \{a \fOr \exists
  t \; s \xrightarrow{a} t\} \subseteq \lambda(s)$. We call this
  condition \emph{admissibility}.

\item For all states $s \in S$, $\lambda (s)$ is either finite or
  $\Sigma$. We call this condition \emph{well-sizedness}.

\end{itemize}
\end{definition}

\NI The intended interpretation is that $\lambda(w)$ is the set of
allowed transition symbols emanating from $w$.  The $\lambda$ function
is the semantic counterpart of the $!$ operator.  The admissibility
restriction is in place because transitions $s \TRANS{a} t$ where $a
\notin \lambda(s)$ would be saying that an $a$ action is possible at
$s$ but at the same time prohibit $a$ actions at that state.
Well-sizedness is not a fundamental restriction but rather a
convenient trick. Essentially eremic transition systems have two kinds
of nodes:

\begin{itemize}

\item Nodes $s$ without restrictions on outgoing transitions. Those are
  labelled with $\lambda ( s) = \Sigma$.

\item Nodes $s$ with restriction on outgoing transitions. Those are
  labelled by a finite set $\lambda ( s)$ of actions.

\end{itemize}

\NI Defining $\lambda$ on all nodes and not just on those with
restrictions makes some definitions and proofs slightly easier.

As with other modal logics, satisfaction of formulae is defined
relative to a state in the ambient eremic transition system, giving
rise to the following definition.

\begin{definition}
A \emph{eremic model}, ranged over by $\MMM, \MMM', ...$, is a pair
$(\LLL, s)$, where $\LLL$ is an eremic transition system $(S,
\rightarrow, \lambda)$, and $s$ is a state from $S$. We call $s$ the
\emph{root} of the model.
\end{definition}

\NI With eremic models at hand, we can finally formalise the
satisfaction relation for eremic formulae.

\begin{definition}\label{ELsatisfaction}
The \emph{satisfaction relation} $\MMM \models \phi$ is defined
inductively by the following clauses, where we assume that $\MMM =
(\LLL, s)$ and $\LLL = (S, \rightarrow, \lambda)$.
\[
\begin{array}{lclcl}
  \MMM & \models & \top   \\
  \MMM & \models & \phi \AND \psi &\ \mbox{ iff } \ & \MMM  \models \phi \mbox { and } \MMM \models \psi  \\
  \MMM & \models & \langle a \rangle \phi & \mbox{ iff } & \text{there is transition } s \xrightarrow{a} t \mbox { such that } (\LLL, t) \models \phi  \\
  \MMM & \models & \fBang A &\mbox{ iff } & \lambda(s) \subseteq A
\end{array}
\]
\end{definition}

\NI The first three clauses are standard. The last clause enforces the
intended meaning of $!A$: the available modalities in the model are
\emph{at least as constrained} as required by $!A$. They may even be
more constrained if the inclusion $\lambda(s) \subseteq A$ is
proper. We note in the case where the set $\Sigma$ of actions is
infinite, allowing $\lambda(s)$ to return arbitrary inifinite sets in
addition to $\sigma$ does not make a difference because $A$ is finite
by construction, so $\lambda(s) \subseteq A$ can never hold when
$\lambda(s)$ is infinite. 

\begin{FIGURE}
\centering
\begin{tikzpicture}[node distance=1.3cm,>=stealth',bend angle=45,auto]
  \tikzstyle{place}=[circle,thick,draw=blue!75,fill=blue!20,minimum size=6mm]
  \tikzstyle{red place}=[place,draw=red!75,fill=red!20]
  \tikzstyle{transition}=[rectangle,thick,draw=black!75,
  			  fill=black!20,minimum size=4mm]
  \tikzstyle{every label}=[red]
  
  \begin{scope}
    \node [place] (w1) {$\Sigma$};
    \node [place] (e1) [below left of=w1] {$\{b,c\}$}
      edge [pre]  node[swap] {a}                 (w1);      
    \node [place] (e2) [below right of=w1] {$\emptyset$}
      edge [pre]  node[swap] {c}                 (w1);      
    \node [place] (e3) [below of=e1] {$\Sigma$}
      edge [pre]  node[swap] {b}                 (e1);      
  \end{scope}
    
\end{tikzpicture}
\caption{Example model.}\label{figure:elSmall}
\end{FIGURE}



We continue with concrete examples.  The model in Figure
\ref{figure:elSmall} satisfies all the following formulae, amongst
others:
\[
\begin{array}{lclclclcl}
\MAY{a} &\qquad&
\MAY{a} \MAY{b} &\qquad&
\MAY{a} \fBang \{b,c\} &\qquad&
\MAY{a} \fBang \{b,c,d\} &\qquad&
\MAY{c} \\[1mm]
\MAY{c} \fBang \{\} &&
\MAY{c} \fBang \{a\} &&
\MAY{c} \fBang \{a,b\} &&
\MAY{a} \land \MAY{c} &&
\MAY{a} (\MAY{b} \land \fBang \{b,c\}
\end{array}
\]
The same model does \emph{not} satisfy any of the following formulae
\[
\MAY{b} \qquad
\fBang \{a\} \qquad
\fBang \{a, c\} \qquad
\MAY{a} \fBang \{b\} \qquad
\MAY{a} \MAY{c} \qquad
\MAY{a} \MAY{b} \fBang \{c\} 
\]

\NI Figure \ref{threemodels} shows various models of $\MAY{a} \MAY{b}$. 

Because \ELABR{} does not include the operators $\neg, \lor, $ or
$\Rightarrow$, \ELABR{} has the unusual property that every
satisfiable formula has a unique (up to isomorphism) simplest model
that satisfies it.  In Figure \ref{threemodels}, the left model is the
unique simplest model satisfying$\MAY{a} \MAY{b}$.  This property will
be used in Section \ref{decisionprocedure} to construct a
\emph{linear-time} decision procedure.
\begin{FIGURE}
\centering
\begin{tikzpicture}[node distance=1.3cm,>=stealth',bend angle=45,auto]
  \tikzstyle{place}=[circle,thick,draw=blue!75,fill=blue!20,minimum size=6mm]
  \tikzstyle{red place}=[place,draw=red!75,fill=red!20]
  \tikzstyle{transition}=[rectangle,thick,draw=black!75,
  			  fill=black!20,minimum size=4mm]
  \tikzstyle{every label}=[red]
  \begin{scope}[xshift=0cm]
    \node [place] (w1) {$S$};
    \node [place] (e1) [below of=w1] {$\mathcal{S}$}
      edge [pre]  node[swap] {a}                 (w1);
    \node [place] (e2) [below of=e1] {$\mathcal{S}$}
      edge [pre]  node[swap] {b}                 (e1);
  \end{scope}   
  \begin{scope}[xshift=4cm]
    \node [place] (w1) {$\mathcal{S}$};
    \node [place] (e1) [below of=w1] {$\{a,b,c\}$}
      edge [pre]  node[swap] {a}                 (w1);
    \node [place] (e2) [below of=e1] {$\mathcal{S}$}
      edge [pre]  node[swap] {b}                 (e1);
  \end{scope}   
  \begin{scope}[xshift=8cm]
    \node [place] (w1) {$\{a\}$};
    \node [place] (e1) [below of=w1] {$\{b\}$}
      edge [pre]  node[swap] {a}                 (w1);
    \node [place] (e2) [below of=e1] {$\{\}$}
      edge [pre]  node[swap] {b}                 (e1);
  \end{scope}   
\end{tikzpicture}
\caption{Various models of $\langle a \rangle \langle b \rangle \top$}
\end{FIGURE}


Figure \ref{more models} shows one model that does, and one that does
not, satisfy the formula $\fBang \{a,b\}$.  Both models validate
$!\{a, b, c\}$.

\begin{FIGURE}
\centering
\begin{tikzpicture}[node distance=1.3cm,>=stealth',bend angle=45,auto]
  \tikzstyle{place}=[circle,thick,draw=blue!75,fill=blue!20,minimum size=6mm]
  \tikzstyle{red place}=[place,draw=red!75,fill=red!20]
  \tikzstyle{transition}=[rectangle,thick,draw=black!75,
  			  fill=black!20,minimum size=4mm]
  \tikzstyle{every label}=[red]
  \begin{scope}[xshift=0cm]
    \node [place] (w1) {$\{a, b\}$};
    \node [place] (e1) [below left of=w1] {$\Sigma$}
      edge [pre]  node {a}                 (w1);
    \node [place] (e2) [below right of=w1] {$\{c\}$}
      edge [pre]  node[swap] {b}                 (w1);
  \end{scope}   
  \begin{scope}[xshift=4cm]
    \node [place] (w1) {$\{a, b, c, d\}$};
    \node [place] (e1) [below of=w1] {$\{a\}$}
      edge [pre]  node[swap] {c}                 (w1);
  \end{scope}   
\end{tikzpicture}
\caption{The model on the left validates $!\{a, b, c\}$
while the model on the right does not.}\label{figure:elAndBang:moreMdels}
\end{FIGURE}


Because \ELABR{} does not include any of the ``complexifying''
operators $\neg, \lor, $ or $\Rightarrow$, the only tautological
formulae that are true in all models are $\top$ and conjunctions of
$\top$.  But there are an infinite number of distinct contradictory
formulae that are false in all models.  For example:
\[
   \MAY{a} \land \fBang \{\} \qquad
   \MAY{a} \land \fBang \{b\} \qquad
   \MAY{a} \land \fBang \{b, c\} \qquad
   \MAY{b} \land \fBang \{\} \qquad
\]

\begin{definition}
Let $\Gamma$ be an arbitrary set of formulae. We say \emph{$\Gamma$
  semantically implies $\phi$}, written $\Gamma \models \phi$,
provided for all eremic models $\MMM$ if it is the case that $\MMM
\models \Gamma$ implies $\MMM \models \phi$. 
\richard{We do not define $\models \Gamma$ for a set $\Gamma$, only for individual formulae - do we want to change this definition to be of the form $\phi \models \psi$?}
\end{definition}

ARSE
\begin{example}
\ELABR{} shares with HML the following implications:
\begin{eqnarray*}
\MAY{a} \MAY{b} \models \MAY{a} \\
\MAY{a} (\MAY{b} \land \MAY{c}) \models \MAY{a} \MAY{b}
\end{eqnarray*}
But because \ELABR{} is restricted to deterministic models, it also validates the following formula (which HML does not):
\begin{eqnarray*}
\MAY{a} \MAY{b} \land \MAY{a} \MAY{b}  \models \MAY{a} (\MAY{b} \land \MAY{c})
\end{eqnarray*}
\ELABR{} also validates all implications in which the set of constraints is relaxed from left to right. For example:
\begin{eqnarray*}
\fBang \{\} \models \fBang \{a\} \\
\fBang \{\} \models \fBang \{a, b\} \\
...
\end{eqnarray*}
\end{example}
