\section{\Cathoristic{}}\label{coreEL}

In this section we introduce the syntax and semantics of \cathoristic{}.

\subsection{Syntax}
\label{elsyntax}
\NI Syntactically, \cathoristic{} is a multi-modal logic with one new
operator.

\begin{definition} Let $\Sigma$ be a non-empty set of \emph{actions}.
Actions are ranged over by $a, a', a_1, b, ...$, and $A$ ranges over
finite subsets of $\Sigma$. The \emph{formulae} of \cathoristic{}, ranged over by $\phi,
\psi, \xi ...$, are given by the
following grammar.

\begin{GRAMMAR}
  \phi 
     &\quad ::= \quad & 
  \TRUE 
     \VERTICAL 
  \phi \AND \psi
     \VERTICAL 
  \MAY{a}{\phi}
     \VERTICAL 
  \fBang A 
\end{GRAMMAR}
\end{definition}

\NI The first three forms of $\phi$ are standard from Hennessy-Milner
logic \cite{HennessyM:alglawfndac}: $\TRUE$ is logical truth, $\AND$
is conjunction, and $\MAY{a}{\phi}$ means that the current state can
transition via action $a$ to a new state at which $\phi$ holds. Tantum
$A$, written $!A$, is the key novelty of \cathoristic{}.  Asserting
$!A$ means: in the current state at most the modalities $\MAY{a}{}$
that satisfy $a \in A$ are permissible.

We assume that $\MAY{a}{\phi}$ binds more tightly than conjunction, so
$\MAY{a}{\phi} \AND \psi$ is short for $(\MAY{a}{\phi}) \AND \psi$.
We often abbreviate $\MAY{a}{\TRUE}$ to $\MAY{a}{}$. We define falsity
$\FALSE$ as $!\emptyset \AND \MAY{a}{}$ where $a$ is an arbitrary
action in $\Sigma$. 
Hence, $\Sigma$ must be
non-empty. 
Note that, in the absence of negation, we cannot
readily define disjunction, implication, or $[a]$ modalities by de
Morgan duality. 

\begin{convention}
From now on we assume a fixed set $\Sigma$ of actions, except where
stated otherwise.
\end{convention}

\subsection{Semantics}

\NI The semantics of \cathoristic{} is close to Hennessy-Milner logic,
but uses deterministic transition systems augmented with labels on
states.

%% \begin{definition}

%% \NI We say $\LLL$ is \emph{deterministic} if $x \TRANS{a} y$ and $x \TRANS{a} z$ imply that $y = z$. Otherwise $\LLL$ is
%% \emph{non-deterministic}.
%% \end{definition}

\begin{definition}\label{cathoristicTS}
A \emph{cathoristic transition system} is a triple $\LLL = (S,
\rightarrow, \lambda)$, where $(S, \rightarrow)$ is a deterministic
labelled transition system over $\Sigma$, and $\lambda$ is a function
from states to sets of actions (not necessarily finite), subject to
the following constraints:
\begin{itemize}

\item For all states $s \in S$ it is the case that $ \{a \fOr \exists
  t \; s \xrightarrow{a} t\} \subseteq \lambda(s)$. We call this
  condition \emph{admissibility}.

\item For all states $s \in S$, $\lambda (s)$ is either finite or
  $\Sigma$. We call this condition \emph{well-sizedness}.

\end{itemize}
\end{definition}

\NI The intended interpretation is that $\lambda(w)$ is the set of
allowed actions emanating from $w$.  The $\lambda$ function
is the semantic counterpart of the $!$ operator.  The admissibility
restriction is in place because transitions $s \TRANS{a} t$ where $a
\notin \lambda(s)$ would be saying that an $a$ action is possible at
$s$ but at the same time prohibited at $s$.
Well-sizedness is not a fundamental restriction but rather a
convenient trick. Cathoristic transition systems have two kinds
of states:

\begin{itemize}

\item States $s$ without restrictions on outgoing transitions. Those are
  labelled with $\lambda ( s) = \Sigma$.

\item States $s$ with restriction on outgoing transitions. Those are
  labelled by a finite set $\lambda ( s)$ of actions.

\end{itemize}

\NI Defining $\lambda$ on all states and not just on those with
restrictions makes some definitions and proofs slightly easier.

As with other modal logics, satisfaction of formulae is defined
relative to a particular state in the transition system, giving
rise to the following definition.

\begin{definition}
A \emph{cathoristic model}, ranged over by $\MMM, \MMM', ...$, is a
pair $(\LLL, s)$, where $\LLL$ is a cathoristic transition system $(S,
\rightarrow, \lambda)$, and $s$ is a state from $S$. We call $s$ the
\emph{start state} of the model.  An cathoristic model 
 is a \emph{tree} if the underlying transition system is a tree
whose root is the start state.
\end{definition}

\NI Satisfaction of a formula is defined relative to a cathoristic model.

\begin{definition}\label{ELsatisfaction}
The \emph{satisfaction relation} $\MMM \models \phi$ is defined
inductively by the following clauses, where we assume that $\MMM =
(\LLL, s)$ and $\LLL = (S, \rightarrow, \lambda)$.
\[
\begin{array}{lclcl}
  \MMM & \models & \top   \\
  \MMM & \models & \phi \AND \psi &\ \mbox{ iff } \ & \MMM  \models \phi \mbox { and } \MMM \models \psi  \\
  \MMM & \models & \langle a \rangle \phi & \mbox{ iff } & \text{there is a transition } s \xrightarrow{a} t \mbox { such that } (\LLL, t) \models \phi  \\
  \MMM & \models & \fBang A &\mbox{ iff } & \lambda(s) \subseteq A
\end{array}
\]
\end{definition}

\NI The first three clauses are standard. The last clause enforces the
intended meaning of $!A$: the permissible modalities in the model are
\emph{at least as constrained} as required by $!A$. They may even be
more constrained if the inclusion $\lambda(s) \subseteq A$ is
proper. For infinite sets $\Sigma$ of actions, allowing $\lambda(s)$
to return arbitrary infinite sets in addition to $\Sigma$ does not
make a difference because $A$ is finite by construction, so
$\lambda(s) \subseteq A$ can never hold anyway for infinite
$\lambda(s)$.

\begin{FIGURE}
\centering
\begin{tikzpicture}[node distance=1.3cm,>=stealth',bend angle=45,auto]
  \tikzstyle{place}=[circle,thick,draw=blue!75,fill=blue!20,minimum size=6mm]
  \tikzstyle{red place}=[place,draw=red!75,fill=red!20]
  \tikzstyle{transition}=[rectangle,thick,draw=black!75,
  			  fill=black!20,minimum size=4mm]
  \tikzstyle{every label}=[red]
  
  \begin{scope}
    \node [place] (w1) {$\Sigma$};
    \node [place] (e1) [below left of=w1] {$\{b,c\}$}
      edge [pre]  node[swap] {a}                 (w1);      
    \node [place] (e2) [below right of=w1] {$\emptyset$}
      edge [pre]  node[swap] {c}                 (w1);      
    \node [place] (e3) [below of=e1] {$\Sigma$}
      edge [pre]  node[swap] {b}                 (e1);      
  \end{scope}
    
\end{tikzpicture}
\caption{Example model.}\label{figure:elSmall}
\end{FIGURE}



We continue with concrete examples.  The model in Figure
\ref{figure:elSmall} satisfies all the following formulae, amongst
others.
\[
\begin{array}{lclclclcl}
\MAY{a} &\qquad&
\MAY{a} \MAY{b} &\qquad&
\MAY{a} \fBang \{b,c\} &\qquad&
\MAY{a} \fBang \{b,c,d\} &\qquad&
\MAY{c} \\[1mm]
\MAY{c} \fBang \emptyset &&
\MAY{c} \fBang \{a\} &&
\MAY{c} \fBang \{a,b\} &&
\MAY{a} \land \MAY{c} &&
\MAY{a} (\MAY{b} \land \fBang \{b,c\})
\end{array}
\]

\NI Here we assume, as we do with all subsequent figures, that the top
state is the start state.  The same model does not satisfy any of the
following formulae.
\[
\MAY{b} \qquad
\fBang \{a\} \qquad
\fBang \{a, c\} \qquad
\MAY{a} \fBang \{b\} \qquad
\MAY{a} \MAY{c} \qquad
\MAY{a} \MAY{b} \fBang \{c\} 
\]

\NI Figure \ref{threemodels} shows various models of $\MAY{a} \MAY{b}$
and Figure \ref{more models} shows one model that does, and one that
does not, satisfy the formula $\fBang \{a,b\}$.  Both models validate
$!\{a, b, c\}$.

\Cathoristic{} does not have the operators $\neg, \lor, $ or
$\IMPLIES$.  This has the following two significant consequences.
First, every satisfiable formula has a unique (up to isomorphism)
simplest model.  In Figure \ref{threemodels}, the left model is the
unique simplest model satisfying $\MAY{a} \MAY{b}$.  We will clarify
below that model simplicity is closely related to the process
theoretic concept of similarity, and use the existence of unique
simplest models in our quadratic-time decision procedure.

\begin{FIGURE}
\centering
\begin{tikzpicture}[node distance=1.3cm,>=stealth',bend angle=45,auto]
  \tikzstyle{place}=[circle,thick,draw=blue!75,fill=blue!20,minimum size=6mm]
  \tikzstyle{red place}=[place,draw=red!75,fill=red!20]
  \tikzstyle{transition}=[rectangle,thick,draw=black!75,
  			  fill=black!20,minimum size=4mm]
  \tikzstyle{every label}=[red]
  \begin{scope}[xshift=0cm]
    \node [place] (w1) {$\Sigma$};
    \node [place] (e1) [below of=w1] {$\Sigma$}
      edge [pre]  node[swap] {a}                 (w1);
    \node [place] (e2) [below of=e1] {$\Sigma$}
      edge [pre]  node[swap] {b}                 (e1);
  \end{scope}   
  \begin{scope}[xshift=4cm]
    \node [place] (w1) {$\Sigma$};
    \node [place] (e1) [below of=w1] {$\{a,b,c\}$}
      edge [pre]  node[swap] {a}                 (w1);
    \node [place] (e2) [below of=e1] {$\Sigma$}
      edge [pre]  node[swap] {b}                 (e1);
  \end{scope}   
  \begin{scope}[xshift=8cm]
    \node [place] (w1) {$\{a\}$};
    \node [place] (e1) [below of=w1] {$\{b\}$}
      edge [pre]  node[swap] {a}                 (w1);
    \node [place] (e2) [below of=e1] {$\emptyset$}
      edge [pre]  node[swap] {b}                 (e1);
  \end{scope}   
\end{tikzpicture}
\caption{Three models of $\langle a \rangle \langle b \rangle \top$}\label{figure:elAndBang:models}
\label{threemodels}
\end{FIGURE}


\begin{FIGURE}
\centering
\begin{tikzpicture}[node distance=1.3cm,>=stealth',bend angle=45,auto]
  \tikzstyle{place}=[circle,thick,draw=blue!75,fill=blue!20,minimum size=6mm]
  \tikzstyle{red place}=[place,draw=red!75,fill=red!20]
  \tikzstyle{transition}=[rectangle,thick,draw=black!75,
  			  fill=black!20,minimum size=4mm]
  \tikzstyle{every label}=[red]
  \begin{scope}[xshift=0cm]
    \node [place] (w1) {$\{a\}$};
    \node [place] (e1) [below of=w1] {$\Sigma$}
      edge [pre]  node {a}                 (w1);
  \end{scope}   
  \begin{scope}[xshift=4cm]
    \node [place] (w1) {$\{a, b, c\}$};
    \node [place] (e1) [below of=w1] {$\{a\}$}
      edge [pre]  node[swap] {c}                 (w1);
  \end{scope}   
\end{tikzpicture}
\caption{The model on the left validates $!\{a, b\}$
while the model on the right does not.}\label{figure:elAndBang:moreMdels}
\label{more models}
\end{FIGURE}


Secondly, \cathoristic\  is different from other logics in that there is an
asymmetry between tautologies and contradictories: logics with
conventional negation have an infinite number of non-trivial
tautologies, as well as an infinite number of contradictories.  In
contrast, because \cathoristic{} has no negation or disjunction
operator, it is expressively limited in the tautologies it can
express: $\top$ and conjunctions of $\top$ are its sole tautologies. On the
other hand, the tantum operator enables an infinite number of
contradictories to be expressed.  For example:
\[
   \MAY{a} \;\land\; \fBang \emptyset \qquad
   \MAY{a} \;\land\; \fBang \{b\} \qquad
   \MAY{a} \;\land\; \fBang \{b, c\} \qquad
   \MAY{b} \;\land\; \fBang \emptyset \qquad
\]


%% \begin{definition} Let $\Gamma$ be an arbitrary set of formulae. We
%% say \emph{$\Gamma$ semantically implies $\phi$}, written $\Gamma
%% \models \phi$, provided for all cathoristic models $\MMM$ if it is the
%% case that $\MMM \models \Gamma$ implies $\MMM \models \phi$.
%% \richard{We do not define $\models \Gamma$ for a set $\Gamma$, only
%% for individual formulae - do we want to change this definition to
%% be of the form $\phi \models \psi$?}  \end{definition}

\NI Next, we present the semantic consequence relation.
\begin{definition} 
 We say the formula $\phi$ \emph{semantically implies} $\psi$, written $\phi
   \models \psi$, provided for all cathoristic models $\MMM$ if $\MMM \models \phi$ then also $\MMM \models \psi$.
   We sometimes write $\models \phi$ as a shorthand for $\top \models \phi$.
\end{definition}

\NI \Cathoristic{} shares with other (multi)-modal logics the following
implications:
\begin{eqnarray*}
\MAY{a} \MAY{b} \models \MAY{a} 
 \qquad\qquad
\MAY{a} (\MAY{b} \land \MAY{c}) \models \MAY{a} \MAY{b}
\end{eqnarray*}
As \cathoristic{} is restricted to deterministic models, it also
validates the following formula:
\begin{eqnarray*}
\MAY{a} \MAY{b} \land \MAY{a} \MAY{c}  \models \MAY{a} (\MAY{b} \land \MAY{c})
\end{eqnarray*}
\Cathoristic{} also validates all implications in which the set of constraints is relaxed from left to right. For example:
\begin{eqnarray*}
\fBang \{c\} \models\ \fBang \{a, b, c\} 
 \qquad\qquad
\fBang \emptyset \models\ \fBang \{a, b\} 
\end{eqnarray*}

