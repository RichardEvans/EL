
\section{Capturing inferences between atomic sentences}\label{naturalLanguageInference}

\NI \Cathoristic{} arose in parts as an attempt to answer the
question: what is the simplest logic that can capture inferences
between atomic sentences of natural language?  In this section, we
give examples of such inferences, and then show how \cathoristic{}
handles them.  We also compare our approach with attempts at
expressing the inferences in first-order logic.

\subsection{Intra-atomic inferences in natural language}

\NI Natural language admits many types of inference between atomic
sentences.  First, exclusion:
\begin{quote}
``Jack is male'' is incompatible with ``Jack is female''
\end{quote}
Second, entailment inferences from dyadic to monadic predicates:
\begin{quote}
``Jack loves Jill'' implies ``Jack loves''
\end{quote}
Third, adverbial inferences:
\begin{quote}
``Jack walks quickly'' implies ``Jack walks''
\end{quote}
Fourth, inferences from conjunctions of sentences to conjunctions of noun-phrases (and vice-versa):
\begin{quote}
``Jack loves Jill'' and ``Jack loves Joan'' together imply that ``Jack loves Jill and Joan''
\end{quote}
Fifth, inferences from conjunctions of sentences to conjunction of predicates\footnote{See \cite{sommers} p.282 for a spirited defence of predicate conjunction against Fregean regimentation.} (and vice-versa):
\begin{quote}
``Jack is bruised'' and ``Jack is humiliated'' together imply that ``Jack is bruised and humiliated''.
\end{quote}

\NI They all can be handled directly and naturally in \cathoristic{}, as we
shall now show.


\subsection{Intra-atomic inferences in \cathoristic{}}
%% We shall show that each of the following inferences can be naturally expressed in \cathoristic{}:
%% \begin{itemize}
%% \item
%% ``Jack is male'' is incompatible with ``Jack is female''
%% \item
%% ``Jack loves Jill'' implies ``Jack loves''\footnote{Although natural languages are full of examples of inferences from dyadic to monadic predicates, there are certain supposed counterexamples to the general rule that a dyadic predicate always implies a monadic one. For example, ``Jack explodes the device'' does not, on its most natural reading, imply that ``Jack explodes''. Our response to cases like this is to distinguish between two distinct monadic predicates $explodes_1$ and $explodes_2$:
%% \begin{itemize}
%% \item
%% $X explodes_1$ iff $X$ is an object that undergoes an explosion
%% \item
%% $X explodes_2$ iff $X$ is an agent that initiates an explosion
%% \end{itemize}
%% Now ``Jack explodes the device'' does imply that ``Jack $explodes_2$'' but does not imply that ``Jack $explodes_1$''. 
%% There is no deep problem here - just another case where natural language overloads the same word in different situation to have different meanings.}
%% \item
%% ``Jack walks quickly'' implies ``Jack walks''
%% \item
%% ``Jack loves Jill'' and ``Jack loves Joan'' together imply that ``Jack loves Jill and Joan''
%% \item
%% ``Jack is bruised'' and ``Jack is humiliated'' together imply that ``Jack is bruised and humiliated''.
%% \end{itemize}

Incompatibility such as that between ``Jack is male'' and ``Jack is
female'' is translated into \cathoristic{} as the pair of incompatible
sentences:
\begin{eqnarray*}
\MAY{jack} \MAY{sex} (\MAY{male} \land \fBang \{male\}) 
   \qquad\qquad
\MAY{jack} \MAY{sex} (\MAY{female} \land \fBang \{female\})
\end{eqnarray*}

\NI Entailment from dyadic to monadic predicates\footnote{Although natural languages are full of examples of inferences from dyadic to monadic predicates, there are certain supposed counterexamples to the general rule that a dyadic predicate always implies a monadic one. For example, ``Jack explodes the device'' does not, on its most natural reading, imply that ``Jack explodes''. Our response to cases like this is to distinguish between two distinct monadic predicates $explodes_1$ and $explodes_2$:
 \begin{itemize}
 \item
 $X explodes_1$ iff $X$ is an object that undergoes an explosion
 \item
 $X explodes_2$ iff $X$ is an agent that initiates an explosion
 \end{itemize}
 Now ``Jack explodes the device'' does imply that ``Jack $explodes_2$'' but does not imply that ``Jack $explodes_1$''. 
There is no deep problem here - just another case where natural language overloads the same word in different situation to have different meanings.}:
``Jack loves Jill'' is translated into \cathoristic{} as:
\begin{eqnarray*}
   \MAY{jack} \MAY{loves} \MAY{jill}
\end{eqnarray*}
The semantics of modalities ensures that this directly entails:
\begin{eqnarray*}
   \MAY{jack} \MAY{loves}
\end{eqnarray*}

\NI Similarly, \cathoristic{} supports inferences from triadic to dyadic
predicates:
\begin{quote}
  ``Jack passed the biscuit to Mary'' implies ``Jack passed the biscuit''
\end{quote}

\NI This can be expressed directly in \cathoristic{} as:
\[
   \MAY{jack} \MAY{passed} \MAY{biscuit} \MAY{to} (\MAY{mary} \land !\{mary\}) \models \MAY{jack} \MAY{passed} \MAY{biscuit}
\]

\NI Adverbial inferences is captured in \cathoristic{} as follows.
\begin{eqnarray*}
\MAY{jack} \MAY{walks} \MAY{quickly}
\end{eqnarray*}
entails:
\begin{eqnarray*}
\MAY{jack} \MAY{walks}
\end{eqnarray*}

\NI \Cathoristic{} directly supports inferences from conjunctions of
sentences to conjunctions of noun-phrases.  As our models are
deterministic, we have the general rule that $ \MAY{a} \MAY{b} \land
\MAY{a} \MAY{c} \models \MAY{a} (\MAY{b} \land \MAY{c})$ from which
it follows that
\begin{eqnarray*}
   \MAY{jack} \MAY{loves} \MAY{jill}
      \qquad\text{and}\qquad
   \MAY{jack} \MAY{loves} \MAY{joan}
\end{eqnarray*}
together imply
\begin{eqnarray*}
\MAY{jack} \MAY{loves} (\MAY{jill} \land \MAY{joan})
\end{eqnarray*}

\NI Using the same rule, we can infer that
\begin{eqnarray*}
   \MAY{jack} \MAY{bruised} \land \MAY{jack} \MAY{humiliated}
\end{eqnarray*}

\NI together imply
\begin{eqnarray*}
\MAY{jack} (\MAY{bruised} \land \MAY{humiliated})
\end{eqnarray*}
 
\subsection{Representing incompatible predicates in \fol}\label{incompatiblepredicatesinfol}

\NI How are incompatible predicates represented in 
\fol{}?  Brachman and Levesque \cite{brachman} introduce the
topic of incompatible predicates by remarking:
\begin{quote}
   We would consider it quite ``obvious'' in this domain that if it
   were asserted that $John$ were a $Man$, then we should answer
   ``no'' to the query $Woman(John)$.
\end{quote}

\NI They propose adding an extra axiom to express the incompatibility:
\[
   \forall x. ( Man(x) \Rightarrow \neg Woman(x) )
\]  
 
\NI This proposal imposes an burden on the knowledge-representer: an
extra axiom must be added for every pair of incompatible predicates.
This is burdensome for large sets of incompatible predicates.  For
example, suppose there are 50 football teams, and a person can only
support one team at a time.  We would need to add $C \cdot {50 \choose
  2}$\martin{Remind me what $C$ is for?} axioms, which is unwieldy.
\[
\begin{array}{l}
  \forall x.  \neg (SupportsArsenal(x) \land SupportsLiverpool(x))  \\
  \forall x.  \neg (SupportsArsenal(x) \land SupportsManUtd(x))  \\
  \forall x.  \neg (SupportsLiverpool(x) \land SupportsManUtd(x))  \\
  \qquad \qquad \qquad \vdots
\end{array}
\]

\NI Or, if we treat the football-teams as objects, and have a
two-place $Supports$ relation between people and teams, we could have:
\[
   \forall x, y, z. (Supports(x,y) \land y \neq z \Rightarrow \neg Supports(x,z))
\]   

\NI If we also assume that each football team is distinct from all the
others, this certainly captures the desired uniqueness condition.  But
it does so by using relatively complex logical machinery.

\subsection{Supporting inferences from dyadic to monadic predicates in \fol}
If we want to capture the inference from ``Jack loves Jill`` to ``Jack
loves'' in \fol{}, we can use a non-logical axiom:
\[
(\forall x, y) Loves(x,y) \Rightarrow Loves(x)
\]

\NI We would have to add an extra  axiom like this for every
$n$-place predicate.  This is cumbersome at best.  In \cathoristic{}, by
contrast, we do not need to introduce any non-logical machinery 
to capture these inferences because they all follow from the general
rule that $\MAY{a} \MAY{b} \models \MAY{a}$.

\subsection{Supporting adverbial inferences in \fol{}}

\NI How can we represent verbs in traditional \fol{} so as to
support adverbial inference?  Davidson \cite{davidson2} proposed that
every $n$-place action verb be analysed as an $n$+1-place predicate,
with an additional slot representing an event.  For example, he
analysed ``I flew my spaceship to the Morning Star'' as
\[
\exists x. ( Flew(I, MySpaceship, x) \land To(x, TheMorningStar))
\]
This implies 
\[
\exists x.  Flew(I, MySpaceship, x)
\]
This captures the inference from ``I flew my spaceship to the Morning Star'' to ``I flew my spaceship''.

Predicate Logic cannot support logical inferences between atomic sentences. 
If it is going to support inferences from adverbial sentences, it \emph{cannot} treat them as atomic and must instead \emph{reinterpret} them as logically complex propositions.
The cost of Davidson's proposal is that a seemingly simple sentence - such as ``Jones walks'' - turns out, on closer inspection, not to be atomic at all -  but to involve existential quantification:
\[
\exists x.  Walks(Jones, x)
\]

\NI Although Predicate Logic can handle some of these inferences, it
can only do so by reinterpreting the sentences as logically-complex
compound propositions.

\subsection{Summary}
We have looked at inference between atomic sentences. \Cathoristic{}
can handle them directly and naturally.  Traditional \fol{} has a
harder time: it can express incompatibility, but only by using
complex machinery - universal quantification, a conditional,
negation. \Cathoristic{}, by contrast, expresses the source of the
incompatibility \emph{directly} using the $!$ operator.  Again, \fol{}
\emph{can} express adverbial inferences, but at the cost of using
quantification over events.  When it comes to inferences from
conjunctions of sentences to conjunctions of noun-phrases (or
predicates), \fol{} has nothing to say because it has no way of
expressing conjunctions between noun-phrases (or predicates) \emph{at
  all}. In \fol{}, ``Jack is bruised and humiliated'' has to be
regimented into ``Jack is bruised and Jack is humiliated''.  
It is not entirely clear this is a step in the right direction.

However, \cathoristic{} cannot handle \emph{all} instances of
sub-propositional inferences.  
One example that \cathoristic\ cannot handle is where the
\emph{subjects} are conjoined, such as
\begin{quote}
``Jack loves Jill'' and ``Jim loves Jill'' together imply that ``Jack and Jim love Jill''
\end{quote}

\NI \Cathoristic{} has no way to conjoin noun-phrases in subject-position.
