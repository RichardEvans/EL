
\section{Capturing Inferences Between Atomic Sentences}
\ELFULL{} arose as an attempt to answer the question: what is the simplest logic that can capture inferences between atomic sentences of natural language? 
In this section, we first enumerate the sorts of inferences we are trying to capture.
Then we show how \ELABR{} can handle these inferences.
Finally, we compare our approach with the various attempts to handle these inferences in First Order Logic.

\subsection{Intra-Atomic Inferences in Natural Language}
Natural language admits many types of inference between atomic sentences.
First, exclusion:
\begin{quote}
``Jack is male'' is incompatible with ``Jack is female''
\end{quote}
Second, entailment inferences from dyadic to monadic predicates:
\begin{quote}
``Jack loves Jill'' implies ``Jack loves''
\end{quote}
Third, adverbial inferences:
\begin{quote}
``Jack walks quickly'' implies ``Jack walks''
\end{quote}
Fourth, inferences from conjunctions of sentences to conjunctions of noun-phrases (and vice-versa):
\begin{quote}
``Jack loves Jill'' and ``Jack loves Joan'' together imply that ``Jack loves Jill and Joan''
\end{quote}
Fifth, inferences from conjunctions of sentences to conjunction of predicates\footnote{See \cite{sommers} p.282 for a spirited defence of predicate conjunction against Fregean regimentation.} (and vice-versa):
\begin{quote}
``Jack is bruised'' and ``Jack is humiliated'' together imply that ``Jack is bruised and humiliated''.
\end{quote}

All these types of inference can be handled directly and naturally in \ELABR{}, as we shall now show.


\subsection{Intra-Atomic Inferences in \ELFULL{}}
We shall show that each of the following inferences can be naturally expressed in \ELABR{}:
\begin{itemize}
\item
``Jack is male'' is incompatible with ``Jack is female''
\item
``Jack loves Jill'' implies ``Jack loves''
\item
``Jack walks quickly'' implies ``Jack walks''
\item
``Jack loves Jill'' and ``Jack loves Joan'' together imply that ``Jack loves Jill and Joan''
\item
``Jack is bruised'' and ``Jack is humiliated'' together imply that ``Jack is bruised and humiliated''.
\end{itemize}
First, incompatibility. ``... is male'' and ``... is female'' are translated into \ELABR{} as the pair of incompatible sentences:
\begin{verbatim}
<male> /\ !{male}
<female> /\ !{female}
\end{verbatim}

Second, entailment from dyadic to monadic predicates:
``Jack loves Jill'' is translated into \ELABR{} as:
\begin{verbatim}
<jack> <loves> <jill>
\end{verbatim}
This directly entails:
\begin{verbatim}
<jack> <loves>
\end{verbatim}
Third, adverbial inferences: 
\begin{verbatim}
<jack> <walks> <quickly>
\end{verbatim}
entails:
\begin{verbatim}
<jack> <walks>
\end{verbatim}
Fourth, \ELABR{} directly supports inferences from conjunctions of sentences to conjunctions of noun-phrases.
From the general rule that
\[
\MAY{a} \MAY{b} \land \MAY{a} \MAY{c} \models \MAY{a} (\MAY{b} \land \MAY{c})
\]
it follows that 
\begin{verbatim}
<jack> <loves> <jill>
\end{verbatim}
and:
\begin{verbatim}
<jack> <loves> <joan>
\end{verbatim}
together imply
\begin{verbatim}
<jack> <loves> (<jill> /\ <joan>)
\end{verbatim}
Using the same rule, we can infer that
\begin{verbatim}
<jack> <bruised>
<jack> <humiliated>
\end{verbatim}
together imply
\begin{verbatim}
<jack> (<bruised> /\ <humiliated>)
\end{verbatim}
 
\subsection{Representing Incompatible Predicates in Predicate Logic}
How are incompatible predicates represented in traditional Predicate Logic?
Brachman and Levesque\cite{brachman} introduce the topic of incompatible predicates by remarking:
\begin{quote}
We would consider it quite ``obvious'' in this domain that if it were asserted that $John$ were a $Man$, then we should answer ``no'' to the query $Woman(John)$.
\end{quote}
They propose adding an extra axiom to express the incompatibility:
\[
(\forall x) Man(x) \rightarrow \neg Woman(x)
\]   
This proposal imposes an authoring burden on the knowledge-representer.
We would have to add an extra axiom for every pair of incompatible predicates.
This proposal becomes particularly burdensome when dealing with large sets of incompatible predicates. 
For example, suppose there are 50 football teams, and a person can only support one team at a time. 
We would need to add $C^{50}_2$ axioms:
The incompatibility axioms start to get large and unwieldy:
\begin{eqnarray}
(\forall x) \neg (SupportsArsenal(x) \land SupportsLiverpool(x)) \nonumber \\
(\forall x) \neg (SupportsArsenal(x) \land SupportsManUtd(x)) \nonumber \\
(\forall x) \neg (SupportsLiverpool(x) \land SupportsManUtd(x)) \nonumber \\
... \nonumber
\end{eqnarray}   
Or, if we treat the football-teams as objects, and have a two-place $Supports$ relation between people and teams, we could have:
\[
(\forall x,y,z) \; \; Supports(x,y) \land y \neq z \rightarrow \neg Supports(x,z)
\]   
Together with the Unique Names Assumption (which lets us assume that each football team is distinct from all the others), this certainly captures the desired uniqueness condition.
But it does so by using relatively complex logical machinery.

\subsection{Supporting Inferences from Dyadic to Monadic Predicates in Predicate Logic}
If we want to capture the inference from ``Jack loves Jill`` to ``Jack loves'' in predicate logic, we have to add a non-logical axiom:
\[
(\forall x, y) Loves(x,y) \rightarrow Loves(x)
\]
We would have to add an extra non-logical axiom like this for every n-place predicate.
This is cumbersome at best. 
In \ELABR{}, by contrast, we do not need to introduce any non-logical machinery at all to capture these inferences because they all follow from the general rule that $\MAY{a} \MAY{b} \models \MAY{a}$.

\subsection{Supporting Adverbial Inferences in Predicate Logic}
How can we represent verbs in traditional predicate logic so as to support adverbial inference?
Davidson \cite{davidson2} proposed that every $n$-place action verb be analysed as an $n+1$-place predicate, with an additional slot representing an event.
For example, he  analysed ``I flew my spaceship to the Morning Star'' as 
\[
(\exists x) Flew(I, MySpaceship, x) \land To(x, TheMorningStar)
\]
This implies 
\[
(\exists x) Flew(I, MySpaceship, x)
\]
This captures the inference from ``I flew my spaceship to the Morning Star'' to ``I flew my spaceship''.

Predicate Logic cannot support logical inferences between elementary propositions. 
If it is going to support inferences from adverbial sentences, it \emph{cannot} treat them as elementary propositions and must instead \emph{reinterpret} them as logically complex propositions.
The cost of Davidson's proposal is that a seemingly simple sentence, such as ``Jones walks'', turns out on closer inspection not be elementary at all,  but to involve existential quantification:
\[
(\exists x) Walks(Jones, x)
\]
Although Predicate Logic can handle some of these inferences, it can only do so by reinterpreting the sentences as logically-complex compound propositions. 
It uses more complex machinery to achieve the same results that \ELFULL{} gets directly.

\subsection{Comparison}
We have been looking at five types of inference between atomic sentences:
\begin{itemize}
\item
Incompatibility
\item
Inferences from dyadic predicates to monadic predicates
\item
Adverbial inferences
\item
Inferences from conjunctions of sentences to conjunctions of noun-phrases
\item
Inferences from conjunctions of sentences to conjunctions of predicates
\end{itemize}
\ELFULL{} can handle all five types of inference directly and naturally.
Traditional predicate logic has a harder time.
Traditional predicate logic \emph{can} express incompatibility. But it can only do so by bringing in relatively complex machinery - a universal quantifier, a conditional, and a negation operator. \ELFULL{}, by contrast, expresses the source of the incompatibility \emph{directly} using the $!$ operator.
Again, predicate logic \emph{can} express adverbial inferences. But, again, it can only do so by introducing relatively complex machinery (quantification over events). 

But when it comes to inferences from conjunctions of sentences to conjunctions of noun-phrases (or predicates), predicate logic has nothing to say because it has no way of expressing conjunctions between noun-phrases (or predicates) \emph{at all}. In predicate logic, ``Jack is bruised and humiliated'' has to be regimented into ``Jack is bruised and Jack is humiliated''. 
This might seem, to anyone who has not already become desensitised to the differences between predicate logic and our ergonomic mother tongue, to be exactly the wrong way around.

\subsection{Expressive Limitations}
As we have seen, \ELFULL{} can handle inferences where predicates are conjoined:
\begin{quote}
``Jack is bruised'' and ``Jack is humiliated'' together imply that ``Jack is bruised and humiliated''.
\end{quote}
It can also handle inferences were predicate-objects are conjoined:
\begin{quote}
``Jack loves Jill'' and ``Jack loves Joan'' together imply that ``Jack loves Jill and Joan''
\end{quote}
But it cannot express related inferences where the \emph{subjects} are conjoined. For example:
\begin{quote}
``Jack loves Jill'' and ``Jim loves Jill'' together imply that ``Jack and Jim love Jill''
\end{quote}
\ELFULL{} has no way to conjoin noun-phrases in subject-position.
