\begin{abstract}
Eremic Logic is a simple modal logic for capturing inferences between
``atomic'' sentences.  It is the only Hennessy-Milner style logic
satisfying the Hennessy-Milner theorem that has a linear-time decision
procedure\martin{We have not proved this fact. I suspect that it's
  actually false!}.  It has been put to use in an industrial
application, as the representational core of a new logic programming
language. The resulting code is shorter, less error-prone, and more
efficient than the corresponding code in traditional predicate
logic.\martin{too subjective for an abstract in particular. Only state
  proven facts!}

\martin{here is an alternative abstract. What do you think?}

Eremic Logic is a modal logic with a novel operator expressing
\emph{exclusion} between sentences. Exclusion can be used to express
the concept of negation.  We present the syntax, proof rules, and
semantics of eremic logic, and establish key meta-theorems such as
completeness and compactness.  We also show that the syntactic notion
of elementary equivalence induced by the the formulae of eremic logic
coincides with the semantic notion of bisimilarity on models.  Then we
give a linear-time decision procedure for an important fragment of of
eremic logic. This fragment has been used in an industrial
application.  We argue that this fragment is a suitable query language
for knowledge representation and logic programming. We also further
elusidate eremic logic by translation to first-order logic as well as
Hennessy-Milner logic. \martin{do we go the other way round?} We
conclude with a discussion of using eremic logic to model features of
natural language, such as inferences between ``atomic'' sentences,
refining Brandom's incompatibility semantics.

\end{abstract}
