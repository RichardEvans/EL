\begin{abstract}
\ELFULL{} is a modal logic with a novel operator expressing
\emph{exclusion} between sentences.  Instead of the traditional
negation operator, \ELFULL{} uses a simple non-compositional exclusion
operator.  At a high level, the exclusion operator and the negation
operator play the \emph{same role}: they are both used to make
incompatible claims.

Because it does not have complexifying logical operators ($\neg$,
$\lor$, or $\Rightarrow$), \ELFULL{} has a \emph{linear-time decision
  procedure}.  Nevertheless, despite its simplicity, it is expressive
enough to satisfy the Hennessy-Milner Theorem and Brandom's
Incompatibility Semantics property.

Our interest in \ELFULL{} derives from our belief that it occupies a
sweet spot in logical space: it is expressive enough to satisfy key
theorems (such as Hennessy-Milner) but simple enough for validity to
be decidable in linear-time.

In the introductory section, we motivate the preference for exclusion
over negation, by appealing to the arguments of Robert Brandom.  Then
we present the syntax, semantics, and proof rules for \ELFULL{}, and
establish key meta-theorems such as completeness and compactness.  We
provide a decision procedure, and show that it is linear-time.  We
show that the syntactic notion of elementary equivalence induced by
the the formulae of \ELFULL{} coincides with the semantic notion of
mutual simulation on models.

\ELFULL{} was designed to express logical inferences between \emph{atomic sentences}. 
We show how \ELFULL{} can capture these inferences, and how it has been used as the representational core of a large industrial knowledge-based system.

\textbf{Martin's rephrasing.}

\NI \ELFULL{} is a multi-modal logic replacing negation with a novel
operator expressing the exclusion between sentences.  We present the
syntax and semantics of the logic including complete proof rules, and
establish a number of results such as a semantic characterisation of
elementary equivalence, the existence of a linear-time decision
procedure, and Brandom's Incompatibility Semantics property.  We
demonstrate the usefulness of the  logic with example applications
in the philosophy of language and in knowledge representation.

\end{abstract}
