\subsection{V1}

First-order logic is \emph{the} logic, the dominant formalism that all
other logics relate to in some form or other. However, for many
purposes first-order logic is unwieldy and better alternatives have
been proposed. Two examples are modal logic and many-sorted
first-order logic.  Both have a distinct advantage over first-order
logic in their respective domain of application.

\begin{itemize}

\item Modal logic leads to shorter formulae and proofs when 
reasoning 'locally' meaning that all quantifiers are
bound \cite{BlackburnP:modlog}, because (simplifying a great deal)
first-order statements like for example $\forall y. ( x R
y \IMPLIES \phi)$ can be abbreviated to $\diamond \phi$.

\item Many-sorted first-order logic leads to shorter 
formulae and proofs when reasoning about things in mutually exclusive
domains, because simplifying a great deal)
first-order statements like for example statements like
\[
     \forall x. \exists ab.( person(x) \AND name (a) \AND isCalled( x, a ) \AND age(b) \AND isAge(x,b))
\]
can be abbreviated to
\[
     \forall x. \exists ab.( isCalled( x, a ) \AND isAge(x,b))
\]
assuming the meta-language conventions that $x$ ranges over the
universe of people, $a$ over names and $b$ over ages.

\end{itemize}

\NI In short, such specialised logics \martin{dare we say DSLs = domain
specific logics?}  are more succinct.  

Based on intutions from the philosophy of language, in particular on
theories of negation, and from from knowlege representation/logic
programming, the present work explores what is gained when we combine
the two abbreviation mechanism above: modalities and



\subsection{V2}
Logic can be seen as a formalisation and simplification of key
structure found in natural language, in particular of formally valid
inferences such as that from ``all scientific texts are error free''
and ``the present work is scientific'' to ``the present work is error
free''. Aristotle, in particular his ``Prior Analytics'', is often
taken to be the begining of a scientific approach towards ..., the
most famous outcome of which might be the formalisation, nearly two
millenia later, of first-order logic by Frege.  The cornerstone of
first-order logic, and key advance over earlier formalisations of
propositional logic, is quantification over individuals which refines
the analysis of atomic propositions.  The key idea of propositional
logic, already is to combine, in an algebraic manner, atomic
propositions. The key algebraic combinators are, on one hand unary
negation, and on the other, binary operators such as conjunction,
disjunction and implication.

Let us consider some examples of formalisation: Humans are either male or female,
could become
\[
   \begin{array}{l}
   \forall x. (human(x) \IMPLIES gender(x, male) \OR gender(x, female)) \\
   \qquad\AND        \\
   \forall x.( \exists y.gender(y, x) \IMPLIES ( x = male \OR x = female )) \\
   \qquad\AND \\
   male \neq female
   \end{array}
\]
A similar analysis can conducted for the political statement ``you are
either with us, or with the terrorists.
\[
   \begin{array}{l}
   \forall x.(group(x) \IMPLIES ( supports(x, us) \OR supports(x, terrorists)) \\
   \qquad\AND\\
   \forall x.( \exists y.supports(y, x) \IMPLIES ( x = us \OR x = terrorists )) \\
   \qquad\AND\\
   ( us \neq terrorists)
   \end{array}
\]
Mutually exclusive alternatives do note have to be binary, for traffic
lights in most countries feature three colours, red, amber and green:
\[
   \begin{array}{l}
   \forall x.(colour(x) \IMPLIES ( colour(x, green) \OR colour(x, red) \OR colour(x, amber)) \\
   \qquad\AND\\
   \forall x.( \exists y.colour(y, x) \IMPLIES ( x = green \OR x = red \OR x = amber )) \\
   \qquad\AND\\
   ( green \neq red \AND green \neq amber \AND red \neq amber)
   \end{array}
\]

\NI We could easily present a near unlimited number of similar
examples because the choice of a limited number of exclusive opions is
an ubiquitous and important feature of the world, and that importance
is reflected in how frequently we discus such matters.

Because of the importance of this phenomenon, let us term it
``exclusion'' and reflect on the features of its formalisations in
first-order logic: exclusion is a derived concept and reduced to a
combination of universal quantification and propositional connectives
including negation. This is several interesting and related
consequences.

\begin{itemize}


\item Quantification is an expensive concept in the sense that it comes with
      many rules governing it's behaviour such as the distinction
      between free and bound variables with its
      ramifications\footnote{As of 2014 there is a substantial amount
      of ongoing research regarding good formalisations of handling
      free/bound variables, see e.g.~nominal approaches to
      logic \cite{PittsAM:newaas,PittsAM:nomsetnasics}. The problem
      was put in focus in recent years with the rise in intrest in the
      compuational cost of syntax manipluation in languages with
      binders. While these matters used to swept under the carpet of
      informality, the rise in mechanical verification has rendered
      this option problematic.} The costs of quantification have to be
      borne every time one uses exclusion, even though exclusion does
      not, prima facie, appear to have anything to do with the
      free/bound variable distinction.

\item Negation and quantification introduce issues into the formalisation
      that are not germane to exclusion, making reasoning about
      exclusion more cumbersome.

\end{itemize}

\NI The present paper propses a different approach. Instead seeing exclusion as
a derived concept, it sees exclusion as a first-class or atomic
concept that is directly backed into the logic. For this purpose a
logical connective

\[
   !A
\]
called \emph{just} $A$, or \emph{tantum} $A$ is used and its behaviour
axiomatised. Here $A$ is a finite set of alternatives that exhaust all
possibilities. For example:
\[
   !\{male, female\}
      \qquad
   !\{red, amber, green\}.
\]
In addition, we can state that something is the case, using a modal operator.
\[
   !\{male, female\} \AND male
\]
Any statement of a fact that exceed what tantum $A$ allows to exist is
false:
\[
   !\{red, amber, green\} \AND blue
\]
is false.

