\renewcommand{\labelitemi}{$-$}

\newcommand{\NI}{\noindent}

%% \theoremstyle{plain}
%%     \newtheorem{thm}{Theorem}
%%     \newtheorem{lemma}[thm]{Lemma}
%%     \newtheorem{prop}[thm]{Proposition}
%%     \newtheorem{cor}[thm]{Corollary}
%%     \newtheorem{fact}[thm]{Fact}
%%     \newtheorem{question}[thm]{Question}
%%     \newtheorem{conjecture}[thm]{Conjecture}
%%     \newtheorem{Lemma}[thm]{Lemma}
%%     \newtheorem{theorem}[thm]{Theorem}
%%     \newtheorem{Theorem}[thm]{Theorem}
%%     \newtheorem{Fact}[thm]{Fact}
%%     \newtheorem{NOTE}[thm]{Note}

\theoremstyle{definition}
\newtheorem{thm}{Theorem}
\newtheorem{prob}[thm]{Problem}
\newtheorem{definition}[thm]{Definition}
\newtheorem{note}[thm]{Note}
\newtheorem*{ex}{Example}
\newtheorem*{Examples}{Examples}
\newtheorem*{todo}{To do}

%% \newcommand{\QUOTATION}[1]{
%% \hfill
%% \hspace{74mm}
%% \begin{minipage}{70mm}\tiny #1\end{minipage}
%% }
%% \newcommand{\HANDLE}[3]{\PROGRAM{handle}_{#1}\ #2\ #3}
%% \newcommand{\OPTIMISE}[2]{\mathsf{opt}^{#1}(#2)}
\newcommand{\infer}[2]{\frac{\displaystyle{ #1 }}{\displaystyle{ #2 }}}
\newcommand{\ZEROPREMISERULE}[1]{\infer{-}{#1}}
\newcommand{\ONEPREMISERULE}[2]{\infer{#1}{#2}}
\newcommand{\TWOPREMISERULE}[3]{\infer{#1 \quad #2}{#3}}
\newcommand{\THREEPREMISERULE}[4]{\infer{#1 \quad #2 \quad #3}{#4}}
\newcommand{\FOURPREMISERULE}[5]{\infer{#1 \quad #2 \quad #3 \quad #4}{#5}}
\newcommand{\SIXPREMISERULE}[7]{\infer{#1 \quad #2 \quad #3 \quad #4 \quad #5 \quad #6}{#7}}
%% \newcommand{\RULENAME}[1]{\textsc{#1}}
%% \newcommand{\SMALLRULENAME}[1]{\textsc{\tiny #1}}
%% \newcommand{\SUBST}[2]{[#1/#2]}
%% \newenvironment{FIGURE}{\begin{figure}\rule{\linewidth}{.5pt}}{\rule{\linewidth}{.5pt}\end{figure}}
%% \newcommand{\FS}{\rightarrow}
%% \newcommand{\IMPLIES}{\supset}
%% \newcommand{\MAX}[2]{\text{\emph{max}}(#1, #2)}
%% \newcommand{\STRING}[1]{\langle #1 \rangle}
%% \newcommand{\HEAD}[2]{\mathsf{head}^{#1}(#2)}
\newcommand{\MAY}[1]{\langle #1 \rangle}
%% \newcommand{\MUST}[1]{[ #1 ]}
%% \newcommand{\VEC}[1]{\tilde{#1}}
%% \newcommand{\HW}[1]{\mathsf{hw}(#1)}
%% \newcommand{\MUTATION}[1]{\mathsf{mutation}(#1)}
%% \newcommand{\GF}[1]{\mathsf{gf}(#1)}
%% \newcommand{\HD}[1]{\PROGRAM{hd}(#1)}
%% \newcommand{\TL}[1]{\PROGRAM{tl}(#1)}
\newenvironment{GRAMMAR}{\[\begin{array}{lcl}}{\end{array}\]}
\newenvironment{RULES}{\[\begin{array}{c}}{\end{array}\]}
\newcommand{\VERTICAL}{\  \mid\hspace{-3.0pt}\mid \ }
%% \newcommand{\UNIT}{\mathsf{Unit}}
%% \newcommand{\IFTHEN}[2]{\mathsf{if}\; #1\;\mathsf{then}\; #2}
%% \newcommand{\IFTHENELSE}[3]{\IFTHEN{#1}{#2}\;\mathsf{else}\;#3}
%% \newcommand{\BOOL}{\mathbb{B}}
%% \newcommand{\INT}{\mathsf{Int}}
%% \newcommand{\PROGRAM}[1]{\mathsf{#1}}
%% \newcommand{\TRACE}[2]{\PROGRAM{trace}\; #1\;\PROGRAM{in}\; #2}
%% \newcommand{\SIZE}[1]{{|#1|}}
%% \newcommand{\EEE}[1]{\CAL{E}[#1]}
%% \newcommand{\PPP}{\CAL{P}}
%% \newcommand{\EXTYPE}[1]{\mathsf{Ex}(#1)}
%% \newcommand{\RED}{\rightarrow}
%% \newcommand{\LUB}{\vee}
\newcommand{\LTS}[1]{{\mathscr{L}(#1)}}
\newcommand{\DLTS}[1]{{\mathscr{DL}(#1)}}
%% \newcommand{\GLB}{\wedge}
%% \newcommand{\BIGLUB}{\bigvee}
%% \newcommand{\BIGGLB}{\bigwedge}
%% \newcommand{\TRUE}{\PROGRAM{true}}
%% \newcommand{\LIST}[1]{\mathsf{List}(#1)}
%% \usepackage{ifthen}
%% \usepackage{amssymb}
%% \usepackage{color}
%% \newboolean{showcomments}
%% \setboolean{showcomments}{true}
%% \ifthenelse{\boolean{showcomments}}
%%   {\newcommand{\mynote}[2]{
%%    \textcolor{red}{%
%%     \fbox{\bfseries\sffamily\scriptsize#1}
%%     {\small$\blacktriangleright$\textsf{\emph{#2}}$\blacktriangleleft$}
%%    }
%%   }
%%   }
%%   {\newcommand{\mynote}[2]{}
%%   }
%% \newcommand\martin[1]{\mynote{Martin}{#1}}
%% \newcommand{\MARGINPAR}[1]{\marginpar{{\small #1}}}
%% \newcommand{\EEE}[1]{\CAL{E}^{#1}[\cdot]}
%% \newcommand{\INST}[1]{\mathsf{inst}(#1)}
%% \newcommand{\INVERT}[2]{|\!|#1 : #2 |\!|}
%% \newcommand{\PROVES}[2]{#1 \vdash #2}
%% \newcommand{\MODALFORALL}[2]{\forall #1^{\mathsf{\square #2}}}
%% \newcommand{\UC}[2]{\uparrow\!(#1, #2)}
%% \newcommand{\UP}[1]{\mathsf{up}^{#1}}
%% \newcommand{\DOWN}[1]{\mathsf{down}^{#1}}
%% \newcommand{\LIFT}[1]{\mathsf{lift}_{#1}}
%% \newcommand{\APP}[1]{\mathsf{app}_{#1}}
%% \newcommand{\EVAL}{\mathsf{eval}}
%% \newcommand{\LIST}[1]{\mathsf{List}_{#1}}
%% \newcommand{\TL}[1]{\mathsf{tl}(#1)}
%% \newcommand{\HD}[1]{\mathsf{hd}(#1)}
%% \newcommand{\NIL}[1]{\mathsf{\epsilon}_{#1}}
%% \newcommand{\POWERSET}[1]{\frak{P}(#1)}
%% \newcommand{\MODAL}[1]{\square{#1}}
%% \newcommand{\EXT}[1]{\mathsf{Ext}{(#1)}}
%% \newcommand{\EXTQ}[1]{\mathsf{Ext_q}{(#1)}}
%% \newcommand{\CASE}[3]{\PROGRAM{case}\ #1\ \PROGRAM{of}\ \{\INJ{}{i}{#2}.#3\}_{i \in \{0, 1\}}}
%% \newcommand{\FULLCASE}[5]{\PROGRAM{case}\ #1\ \PROGRAM{of}\ \{\INJ{}{1}{#2}.#3\;\&\; \INJ{}{2}{#4}.#5\}}
%% \newcommand{\LISTCASE}[5]{\PROGRAM{case}\ #1\ \PROGRAM{of}\ \{\NIL{}.#2\;\&\; #3 :: #4.#5\}}
%% \newcommand{\STACKEDLISTCASE}[5]{\PROGRAM{case}\ #1\ \PROGRAM{of}\ \begin{cases}\NIL{}.#2 \\ #3 :: #4.#5\end{cases}}
%% \newcommand{\INJ}[3]{\PROGRAM{inj}^{#1}_{#2}(#3)}
%% \newcommand{\PCFD}{\textsc{Pcf$_{\text{\textsc{d}}}$}}
%% \newcommand{\PCF}{\textsc{Pcf}}
%% \newcommand{\PCFDP}{\textsc{Pcf$_{\text{\textsc{dp}}}$}}
%% \newcommand{\PCFTN}{\textsc{Pcf$_{\text{\textsc{tn}}}$}}
%% \newcommand{\FORMULATYPES}[2]{#1 \vdash {#2}}
%% \newcommand{\JUDGEMENTTYPES}[2]{\FORMULATYPES{#1}{#2}}
%% \newcommand{\FV}[1]{\mathsf{fv}(#1)}
%% \newcommand{\BV}[1]{\mathsf{bv}(#1)}
%% \newcommand{\QQ}[1]{\langle #1 \rangle}
%% \newcommand{\QQEVAL}[3]{#1 = \QQ{#2}\{#3\}}
%% \newcommand{\ONEEVAL}[4]{#1 \bullet #2=#3 \{#4\}}
%% \newcommand{\CSP}[1]{\% #1}
%% \newcommand{\FEC}[1]{\mathsf{fec}(#1)}
%% \newcommand{\SPLICE}[1]{\tilde{\, } #1}
%% \newcommand{\RUN}[1]{\mathtt{run}(#1)}
%% \newcommand{\LOGICRUN}[1]{\mathsf{run}(#1)}
%% \newcommand{\DEFEQ}{\stackrel{\text{\emph{def}}}{=}}
%% \newcommand{\ERROR}{\mathsf{error}}
%% \newcommand{\SCR}[1]{\mathscr{#1}}
\newcommand{\CAL}[1]{\mathcal{#1}}
%% \newcommand{\AAA}{\CAL{A}}
%% \newcommand{\BBB}{\CAL{B}}
%% \newcommand{\CCC}{\CAL{C}}
%% \newcommand{\DDD}{\CAL{D}}
%% \newcommand{\GGG}{\CAL{G}}
\newcommand{\MMM}{\frak{M}}
\newcommand{\LLL}{\CAL{L}}
%% \newcommand{\RRR}{\CAL{R}}
%% \newcommand{\QQQ}{\CAL{Q}}
%% \newcommand{\VVV}{\CAL{V}}
%% \newcommand{\TTT}{\CAL{T}}
%% \newcommand{\NAT}{\mathbb{N}}
%% \newcommand{\INT}{\mathsf{Int}}
%% \newcommand{\BOOL}{\mathsf{Bool}}
%% \newcommand{\PAIR}[2]{(#1, #2)}
%% \newcommand{\UNIT}{\mathsf{Unit}}
%% \newcommand{\IFTHEN}[2]{\mathtt{if}\; #1\;\mathtt{then}\; #2}
%% \newcommand{\IFTHENELSE}[3]{\IFTHEN{#1}{#2}\;\mathtt{else}\;#3}
%% \newcommand{\IFZEROTHENELSE}[3]{\IFZEROTHEN{#1}{#2}\;\mathtt{else}\;#3}
%% \newcommand{\IFZEROTHEN}[2]{\mathtt{if0}\; #1\;\mathtt{then}\; #2}
%% \newcommand{\STAGEDIFTHENELSE}[3]{\begin{array}{l}\mathtt{if}\; {#1}\;\mathtt{then} \\ \quad #2\\ \mathtt{else}\\ \quad #3 \end{array} }
%% \newcommand{\FN}[1]{\mathsf{fn}(#1)}
%% \newcommand{\FC}[1]{\mathsf{fc}(#1)}
%% \newcommand{\FPV}[1]{\mathsf{fpv}(#1)}
%% \newcommand{\BN}[1]{\mathsf{bn}(#1)}
%% \newcommand{\PVAR}[1]{\text{\textsc{#1}}}
%% \newcommand{\AST}[1]{\QQ{#1}}
%% \newcommand{\LET}[3]{\mathtt{let}\ #1 = #2\ \mathtt{in}\ #3}
%% \newcommand{\LETQQ}[3]{\LET{\QQ{#1}}{#2}{#3}}
%% \newcommand{\VLETQQ}[3]{\LETQQ{\VEC{#1}}{\VEC{#2}}{#3}}
%% \newcommand{\NEXTSTAGE}[1]{\mathsf{next}(#1)}
%% \newcommand{\PREVSTAGE}[1]{\mathsf{prev}(#1)}
%% \newcommand{\BOX}[1]{\mathsf{box}(#1)}
%% \newcommand{\CLOSURE}[1]{\mathsf{cl}(#1)}
%% \newcommand{\PCFCLOSURE}[1]{\mathsf{cl}^{\PCF}(#1)}
%% \newcommand{\UNBOX}[2]{\mathsf{unbox}_{#1}(#2)}
%% \newcommand{\DOM}[1]{\mathsf{dom}(#1)}
%% \newcommand{\ECDOM}[1]{\mathsf{ecdom}(#1)}
%% \newcommand{\COD}{\mathsf{cod}}
%% \newcommand{\TYPES}[3]{#1 \vdash #2 : #3}
%% \newcommand{\PCFTYPES}[3]{#1 \vdash^{\PCF} #2 : #3}
%% \newcommand{\EXPRESSIONTYPES}[3]{#1 \vdash #2 : #3}
%% \newcommand{\ECTYPES}[4]{#1 \vdash^{#2} #3 : #4}
%% \newcommand{\WELLFORMED}[2]{#1 \vdash #2}
%% \newcommand{\COMP}{\asymp}
%% 
%% \newcommand{\FW}[1]{\textsf{fw}_{#1}}
%% \newcommand{\VNU}[1]{(\nu \VEC{#1})}
%% \newcommand{\CONV}{\Downarrow}
%% \newcommand{\PCFCONV}{\Downarrow^{\PCF}}
%% \newcommand{\LEQ}{\sqsubseteq}
%% \newcommand{\EQ}{\cong}
%% \newcommand{\CONG}{\simeq}
%% \newcommand{\PCFCONG}{\CONG_{\PCF}}
%% \newcommand{\infer}[2]{\frac{\displaystyle{ #1 }}{\displaystyle{ #2 }}}
%% \newcommand{\ZEROPREMISERULE}[1]{\infer{}{#1}}
%% \newcommand{\ONEPREMISERULE}[2]{\infer{#1}{#2}}
%% \newcommand{\TWOPREMISERULE}[3]{\infer{#1 \quad #2}{#3}}
%% \newcommand{\THREEPREMISERULE}[4]{\infer{#1 \quad #2 \quad #3}{#4}}
%% \newcommand{\FOURPREMISERULE}[5]{\infer{#1 \quad #2 \quad #3 \quad #4}{#5}}
%% \newcommand{\FIVEPREMISERULE}[6]{\infer{#1 \quad #2 \quad #3 \quad #4 \quad #5}{#6}}
%% \newcommand{\SIXPREMISERULE}[7]{\infer{#1 \quad #2 \quad #3 \quad #4 \quad #5 \quad #6}{#7}}
%% \newcommand{\RULENAME}[1]{\textsc{#1}}
%% \newcommand{\LENGTH}[2]{\mathsf{length}(#1, #2)}
%% \newcommand{\SMALLRULENAME}[1]{\textsc{\tiny #1}}
%% \newcommand{\SUBST}[2]{[#1/#2]}
%% \newcommand{\VSUBST}[2]{\SUBST{\VEC{#1}}{\VEC{#2}}}
%% \newcommand{\MINUS}{{\mbox{\bf\small -}}}
\newcommand{\LOGIC}[1]{\mathsf{#1}}
%% \newcommand{\PROGRAM}[1]{\mathtt{#1}}
%% \newcommand{\REC}{\PROGRAM{rec}\;}
%% \newcommand{\SEMB}[1]{\lbrack\!\lbrack #1 \rbrack\!\rbrack}
%% \newcommand{\TRANSLATE}[1]{\ulcorner #1 \urcorner}
%% \newcommand{\MODELTRANSLATE}[1]{\ulcorner #1 \urcorner}
%% \newcommand{\TYPETRANSLATE}[1]{\ulcorner #1 \urcorner}
%% \newcommand{\EXPRESSIONTRANSLATE}[1]{\ulcorner #1 \urcorner}
%% \newcommand{\JUDGEMENTTRANSLATE}[1]{\ulcorner #1 \urcorner}
%% \newcommand{\ZEROPREMISERULENAMEDRIGHT}[2]{\ZEROPREMISERULE{#1}\,\SMALLRULENAME{#2}}
%% \newcommand{\ONEPREMISERULENAMEDRIGHT}[3]{\ONEPREMISERULE{#1}{#2}\,\SMALLRULENAME{#3}}
%% \newcommand{\TWOPREMISERULENAMEDRIGHT}[4]{\TWOPREMISERULE{#1}{#2}{#3}\,\SMALLRULENAME{#4}}
%% \newcommand{\THREEPREMISERULENAMEDRIGHT}[5]{\THREEPREMISERULE{#1}{#2}{#3}{#4}\,\SMALLRULENAME{#5}}
%% \newcommand{\FOURPREMISERULENAMEDRIGHT}[6]{\FOURPREMISERULE{#1}{#2}{#3}{#4}{#5}\,\SMALLRULENAME{#6}}
%% \newcommand{\ZEROPREMISERULENAMEDLEFT}[2]{\SMALLRULENAME{#2}\,\ZEROPREMISERULE{#1}}
%% \newcommand{\ONEPREMISERULENAMEDLEFT}[3]{\SMALLRULENAME{#3}\,\ONEPREMISERULE{#1}{#2}}
%% \newcommand{\TWOPREMISERULENAMEDLEFT}[4]{\SMALLRULENAME{#4}\,\TWOPREMISERULE{#1}{#2}{#3}}
%% \newcommand{\THREEPREMISERULENAMEDLEFT}[5]{\SMALLRULENAME{#5}\,\THREEPREMISERULE{#1}{#2}{#3}{#4}}
%% \newcommand{\FOURPREMISERULENAMEDLEFT}[6]{\SMALLRULENAME{#6}\,\FOURPREMISERULE{#1}{#2}{#3}{#4}{#5}}
%%  \newenvironment{FIGURE}{\begin{figure}\rule{\linewidth}{0.5pt}
%%  \vspace{3.2mm}
%%  }{\rule{\linewidth}{0.5pt}\end{figure}}
\newcommand{\NOVSPACEPARAGRAPH}[1]{\NI\textbf{\emph{#1}.}}
\newcommand{\PARAGRAPH}[1]{\vspace{2mm}\NOVSPACEPARAGRAPH{#1}}
%% \newcommand{\ANSWER}{\NOVSPACEPARAGRAPH{Answer}\ }
%% \newcommand{\SUBSECTION}[1]{\subsection{\textbf{\emph{#1}}}}
%% \newenvironment{DEFINITION}{\begin{definition}\rm}{\end{definition}}
%% \newtheorem{remark}{Remark}{\bfseries}{\rmfamily}
%% \newtheorem{definition}[bla]{Definition}{\bfseries}{\itshape}
%% \newtheorem{theorem}[bla]{Theorem}{\bfseries}{\rmfamily}
%% \newcommand{\TRUTH}{\LOGIC{T}}
\newcommand{\TRUE}{\LOGIC{\top}}
%% \newcommand{\FALSITY}{\LOGIC{F}}
%% \newcommand{\FALSE}{\LOGIC{f}}
%% \newcommand{\FS}{\rightarrow}
%% \newcommand{\IMPLIES}{\supset}
%% \newcommand{\STRING}[1]{\langle #1 \rangle}
%% \newcommand{\VEC}[1]{\tilde{#1}}
%% \newenvironment{GRAMMAR}{\[\begin{array}{rcl c rcl c rcl}}{\end{array}\]}
%% \newenvironment{RULES}{\[\begin{array}{c}}{\end{array}\]}
%% \newcommand{\VERTICAL}{\;  \mid\hspace{-3.0pt}\mid \; }
%% \newcommand{\OUT}[2]{\overline{#1} #2 }
%% \newcommand{\VOUT}[2]{\OUT{#1}{\VEC{#2}}}
%% \newcommand{\LOGICINJ}[3]{\LOGIC{inj}^{#1}_{#2}(#3)}
\newcommand{\AND}{\land}
%% \newcommand{\OR}{\lor}
%% \newcommand{\BIGAND}{\bigwedge}
%% \newcommand{\BIGOR}{\bigvee}
%% \newcommand{\OR}{\vee}
%% \newcommand{\IFF}{\equiv}
%% \newcommand{\BRANCHIN}[4]{#1[\&_{#2} (#3).#4]}
%% \newcommand{\VBRANCHIN}[4]{\BRANCHIN{#1}{\VEC{#2}}{\VEC{#3}}{#4}}
%% \newcommand{\BAREBRANCHOUT}[3]{\OL{#1}#2\langle#3\rangle}
%% \newcommand{\BBAREBRANCHOUT}[3]{\OL{#1}#2(#3)}
%% \newcommand{\VBAREBRANCHOUT}[3]{\BAREBRANCHOUT{#1}{#2}{\VEC{#3}}}
%% \newcommand{\VBBAREBRANCHOUT}[3]{\BBAREBRANCHOUT{#1}{#2}{\VEC{#3}}}
%% \newcommand{\BRANCHOUT}[3]{\overline{#1}\,\mathsf{inj}_{#2}\langle #3 \rangle}
%% \newcommand{\VBRANCHOUT}[3]{\BRANCHOUT{#1}{#2}{\VEC{#3}}}
%% \newcommand{\IN}[2]{#1(#2)}
%% \newcommand{\INs}[1]{#1}
%% \newcommand{\VIN}[2]{\IN{#1}{\VEC{#2}}}
%% \newcommand{\OUTs}[1]{\overline{#1}}
%% \newcommand{\BOUT}[2]{\overline{#1}(#2)}
%% \newcommand{\BBRANCHOUT}[3]{\overline{#1}\LOGIC{inj}_{#2}(#3)}
%% \newcommand{\VBBRANCHOUT}[3]{\BBRANCHOUT{#1}{#2}{\VEC{#3}}}
%% \newcommand{\VBOUT}[2]{\BOUT{#1}{\VEC{#2}}}
%% \newcommand{\OL}[1]{\overline{#1}}
%% \newcommand{\RIN}{!}
%% \newcommand{\BOT}{\bot}
%% \newcommand{\ROUT}{?}
%% \newcommand{\VAL}{\mathsf{val}}
%% \newcommand{\MODE}[1]{\LOGIC{md}(#1)}
%% \newcommand{\OMODE}{\text{\sc o}}
%% \newcommand{\IMODE}{\text{\sc i}}
%% \newcommand{\GREYBOX}[1]{\psboxit{box .9 setgray fill}{$#1$}}
%% \newcommand{\GREYBOXPARA}[1]{\psboxit{box .9 setgray fill}{#1}}
%% \newcommand{\GREYBOXPARA}[1]{\begin{boxitpara}{box .9 setgray fill}#1\end{boxitpara}}
\newcommand{\TRANS}[1]{\stackrel{#1}{\longrightarrow}}
%% \newcommand{\VTRANS}[1]{\TRANS{\VEC{#1}}}
%% \newcommand{\RED}{\rightarrow}
%% \newcommand{\NRED}{\rightarrow\hspace{-3.0mm}\rightarrow}
%% \newcommand{\PCFRED}{\rightarrow_{\PCF}}
%% \newcommand{\NPCFRED}{\rightarrow\hspace{-3.0mm}\rightarrow_{\PCF}}
%% \newcommand{\CONVERGES}{\Downarrow}
%% \newcommand{\DIV}{\Uparrow}
%% \newcommand{\CONG}{\cong}
%% \newcommand{\PRECONG}{\lesssim}
%% \newcommand{\EQUIV}{\sim_{\mathsf{fta}}}
%%  \newcommand{\nextLine}{\\[1mm] \hline \\[-3mm]}
%%  \newcommand{\LINESKELETON}[3]{\!\!\!#1#2\;\hfill\quad\text{{#3}}\!\!\!}
%%  \newcommand{\NONUMBERLINESKELETON}[2]{#1\;\hfill\quad\text{{#2}}}
%%  \newcommand{\LASTLINE}[3]{\LINESKELETON{#1}{#2}{#3}}
%%  \newcommand{\LINE}[3]{\LINESKELETON{#1}{#2}{#3}\nextLine}
%%  \newcommand{\NONUMBERLASTLINE}[2]{\NONUMBERLINESKELETON{#1}{#2}}
%%  \newcommand{\NONUMBERLINE}[2]{\NONUMBERLINESKELETON{#1}{#2}\nextLine}
%%  \newenvironment{DERIVATION}{\[\begin{array}{l}}{\end{array}\]}
%% \newenvironment{NDERIVATION}[1]{\setcounter{line}{#1}\[\begin{array}{ll}}{\end{array}\]}
%% \newcommand{\NLINESKELETON}[2]{\theline &\quad  #1\ \quad\hfill \text{\emph{#2}}\addtocounter{line}{1}}
%% \newcommand{\NLINE}[2]{\NLINESKELETON{#1}{#2}\nextLine}
%% \newcommand{\NLASTLINE}[2]{\NLINESKELETON{#1}{#2}}
%% \newcommand{\ONEASSERT}[3]{#1 :_{#2} #3}
%% \newcommand{\ASSERT}[4]{\{#1\}\; #2 :_{#3} \{#4\}}
%% \newcommand{\TCAPASSERT}[4]{[#1]\; #2 :_{#3} [#4]}
%% \newcommand{\CONST}[2]{#1\STRING{#2}}
%% \newcommand{\VCONST}[2]{\CONST{#1}{\VEC{#2}}}


