\section{Cathoristic logic}
operator.

\begin{definition} Let $\Sigma$ be a non-empty set of \emph{actions}.
Actions are ranged over by $a, a', a_1, b, ...$, and $A$ ranges over
finite subsets of $\Sigma$. The \emph{formulae} of cathoristic logic, ranged over by $\phi,
\psi, \xi ...$, are given by the
following grammar.

\begin{GRAMMAR}
  \phi 
     &\quad ::= \quad & 
  \TRUE 
     \VERTICAL 
  \phi \AND \psi
     \VERTICAL 
  \MAY{a}{\phi}
     \VERTICAL 
  !A 
\end{GRAMMAR}
\end{definition}

\begin{definition}\label{cathoristicTS}
A \emph{cathoristic transition system} is a triple $\LLL = (S,
\rightarrow, \lambda)$, where $(S, \rightarrow)$ is a deterministic
labelled transition system over $\Sigma$, and $\lambda$ is a function
from states to sets of actions (not necessarily finite), subject to
the following constraints:
\begin{itemize}

\item For all states $s \in S$ it is the case that $ \{a \ |\  \exists
  t \; s \xrightarrow{a} t\} \subseteq \lambda(s)$. We call this
  condition \emph{admissibility}.

\item For all states $s \in S$, $\lambda (s)$ is either finite or
  $\Sigma$. We call this condition \emph{well-sizedness}.

\end{itemize}
\end{definition}


\begin{definition}
A \emph{cathoristic model}, ranged over by $\MMM, \MMM', ...$, is a
pair $(\LLL, s)$, where $\LLL$ is a cathoristic transition system $(S,
\rightarrow, \lambda)$, and $s$ is a state from $S$. We call $s$ the
\emph{start state} of the model.  An cathoristic model 
 is a \emph{tree} if the underlying transition system is a tree
whose root is the start state.
\end{definition}

\NI Satisfaction of a formula is defined relative to a cathoristic model.

\begin{definition}\label{ELsatisfaction}
The \emph{satisfaction relation} $\MMM \models \phi$ is defined
inductively by the following clauses, where we assume that $\MMM =
(\LLL, s)$ and $\LLL = (S, \rightarrow, \lambda)$.
\[
\begin{array}{lclcl}
  \MMM & \models & \top   \\
  \MMM & \models & \phi \AND \psi &\ \mbox{ iff } \ & \MMM  \models \phi \mbox { and } \MMM \models \psi  \\
  \MMM & \models & \langle a \rangle \phi & \mbox{ iff } & \text{there is transition } s \xrightarrow{a} t \mbox { such that } (\LLL, t) \models \phi  \\
  \MMM & \models & ! A &\mbox{ iff } & \lambda(s) \subseteq A
\end{array}
\]
\end{definition}
