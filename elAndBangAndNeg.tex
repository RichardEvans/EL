\section{Eremic logic and negation}\label{ELAndNegation}

\subsection{The logic $EL[\land, !, \neg]$}

Our motivation for investigating eremic logic has been to give an
alternative account of negation, based on the concept of exclusion.
In this section we study how to recover negation in eremic logic. For
this purpose, we enrich eremic logic with negation as well as
disjunction, and show how negation can be explained in terms of plain
eremic logic with disjunction.

\begin{definition}
Given a set $\Sigma$ of symbols, the \emph{formulae of eremic logic
  with negation}, collectively denoted EL$[\land, !, \neg]$, are given
by the following grammar.

\begin{GRAMMAR}
  \phi 
     &\quad ::= \quad & 
   \top \fOr \bot \fOr \neg \phi \fOr \phi \land \psi \fOr \langle a \rangle \phi \fOr \fBang A 
\end{GRAMMAR}

\NI We can now define disjunction $\phi \lor \psi$ and implication
$\phi \IMPLIES \psi$ by de Morgan duality: $\phi \OR \psi$ is short
for $\neg (\neg \phi \land \neg \psi )$, and $\phi \IMPLIES \psi$  abbreviates
$\neg\phi \OR \psi$.
\end{definition}

The semantics of EL$[\land, !, \neg]$ is just as in EL$[\land, !]$
above, except for the obvious clauses for negation and disjunction:
\begin{eqnarray*}
\MMM \models \neg \phi &\quad\mbox{ iff }\quad& \MMM \nvDash \phi  
%\MMM \models \phi \lor \psi &\mbox{ iff }& \MMM \models \phi \text{ or }   \MMM \models \psi
\end{eqnarray*}

Negation is a core operation of classical logic, and its absence makes
eremic logic unusual. In order to understand eremic logic better, we
now investigate how negation can be seen as a definable abbreviation
in eremic logic with disjunction. The key idea is that 
\[
   \neg \MAY{a}{\phi}
\]
can be false in two ways: there is no $a$-labelled action at the
current state, or there is, but $\phi$ is false. Both arms of this
disjunction can be expressed in eremic logic:

\begin{itemize}

\item $!\Sigma \setminus \{a\}$.

\item $\MAY{a}{\neg \phi}$.

\end{itemize}

\NI Hence, we can see $\neg \MAY{a}{\phi}$ as a shorthand for 
\[
   !(\Sigma \setminus \{a\}) \OR \MAY{a}{\neg \phi}
\]

\NI Negation still occurs in this term, but prefixing a formula of
lower complexity.

This leaves the question of negating the tantum. That's easy: when
$\neg !A$, then clearly the current state can do an action $a \notin
A$. In other words
\[
   \BIGOR_{a \in \Sigma}\MAY{a}{\TRUE}
\]

\NI Note that both, the negation of the modality and the negation of
the tantum involve the set $\Sigma$ of actions. It is sometimes
advantageous to translate away negation with respect to a subset $S
\subseteq \Sigma$.  So far, we have defined negation with respect to
the whole (possibly infinite) set \Sigma. For technical reasons, we
generalise negation and define it with respect to a \emph{finite}
subset S \subseteq \Sigma. We use this finitely-restricted version of
negation in the decision procedure below.

\begin{definition}
The function $x \neg_{S}(\phi)$ removes negation from $\phi$
relative to $S \subseteq \Sigma$:

\begin{align*}
  \neg_{S}(\top) &\ =\  \bot  &
  \neg_S(\bot) &\ =\  \top  \\
  \neg_S(\phi \land \psi) &\ =\  \neg_S(\phi) \lor \neg_S(\psi)  &
  \neg_S(\phi \lor \psi) &\ =\  \neg_S(\phi) \land \neg_S(\psi)  \\
  \neg_S(\langle a \rangle \phi) &\ =\  \fBang(S-\{a\}) \lor \langle a \rangle \neg_S(\phi)  &
  \neg_S(\fBang A) &\ =\  \bigvee_{a \in S - A} \langle a \rangle \top
\end{align*}


\end{definition}

\NI Note that, if $S$ is infinite, then the last two clauses will
generate infinitary formulae.

\subsection{A decision procedure}

\NI We can use the fact that the $[\land, \fBang]$ fragment of SNEL
has a linear-time decision procedure to build an exponential-time
decision procedure for EL $[\land, \fBang, \neg]$:

Given a claim $\phi \models \psi$, let $S = \mathsf{symbols}(\phi) \cup \mathsf{symbols}(\psi) \cup \{\nu\}$ (where $\nu$ is a new symbol which does not occur in $\phi$ or $\psi$).
First, translate away all negations in $\phi$ using $\neg_S()$ as defined in Section 1.4.
Let the result be $\phi'$.
Second, reduce $\phi'$ to Disjunctive Normal Form by repeated application of the rewrite rules:
\begin{eqnarray*}
\phi \land (\psi \lor \xi) & \leadsto & (\phi \land \psi) \lor (\phi \land \xi)  \\
(\phi \lor \psi) \land \xi & \leadsto & (\phi \land \xi) \lor (\psi \land \xi) 
\end{eqnarray*}
Let the resulting disjuncts be $\phi_1, ..., \phi_n$. 
Now 
\[
\phi \models \psi \mbox{ iff } \forall \phi_{i=1}^n \phi_i \models \psi
\]
Now, to check whether each $\phi_i \models \psi$, we will construct a
model of $\psi_i$ (in linear time).  Now define the $S$ extensions of
an annotated model $\MMM$ as all models which extend the states of
$\MMM$ with extra transitions taken from $S$ which respect the state
labelling on $\MMM$.  Now $\phi_i \models \psi$ if and only if all
$S$-extensions of $\SIMPL{\phi_i}$ satisfy $\psi$.  So, to check
whether $\phi_i \models \psi$, we enumerate the $S$-extensions of
$\SIMPL{\phi_i}$ (there are a finite number of such extensions - the
exact number is exponential in the size of $\SIMPL{\phi_i}$) and check
for each one whether it satisfies $\psi$.

To fill out this sketch, We will describe the structure of the
annotated model, define the $\mu$ function, and show how to compute
the extensions of an annotated model relative to a set $S$ of symbols.


\subsection{Computing the Extensions of a Model}

Recall that, to check whether $\phi_i \models \psi$, we enumerate the $S$-extensions of $\SIMPL{\phi_i}$ (there are a finite number of such extensions - the exact number is exponential in the size of $\SIMPL{\phi_i}$) and check for each one whether it satisfies $\psi$.

\begin{definition}
Given an eremic transition system $(\mathcal{W},\rightarrow,\lambda)$,  and a set $S$ of symbols, then $(\mathcal{W'},\rightarrow',\lambda')$ is a {\bf $S$-extension} of $(\mathcal{W},\rightarrow,\lambda)$ if it is a valid LTS (recall Definition 1) and for all $(x,a,y) \in \rightarrow'$, either:
\begin{itemize} 
\item
$(x, a, y) \in \rightarrow$,  or;
\item
 $x \in \mathcal{W}$ and $a \in S$ and $y$ is a new state not appearing elsewhere in $\mathcal{W}$ or $\mathcal{W'}$.
\end{itemize}
\end{definition}
In other words, $\MMM'$ is an extension of an annotated model $\MMM$, if all its transitions are either from $\MMM$ or involve states of $\MMM$ transitioning via elements of $S$ to new states not appearing in $\MMM$ or $\MMM'$.

The number of extensions can quickly grow very large.
If the model $\MMM$ has $n$ states, then the number of possible extensions is:
\[
({2^{|S|}})^n
\] 
But recall that we are computing these extensions in order to verify $\psi$. So we can make a significant optimisation by restricting the height of each tree to $|\psi|$.

