\section{\Cathoristic{} and negation}\label{ELAndNegation}

\NI We have presented \cathoristic{} as a language that can express
incompatible claims without negation.  In this section, we briefly
consider \cathoristic{} enriched with negation.

\subsection{Syntax and semantics}

\begin{definition}
Given a set $\Sigma$ of actions, the \emph{formulae of \cathoristic{}
  with negation} are given by the following grammar.
\begin{GRAMMAR}
  \phi 
     &\quad ::= \quad & 
   ... \fOr \neg \phi
\end{GRAMMAR}

\NI We can now define disjunction $\phi \lor \psi$ and implication
$\phi \IMPLIES \psi$ by de Morgan duality: $\phi \OR \psi$ is short
for $\neg (\neg \phi \land \neg \psi )$, and $\phi \IMPLIES \psi$  abbreviates
$\neg\phi \OR \psi$.
\end{definition}

The semantics of \cathoristic\ with negation is just that of plain
\cathoristic\, except for the obvious clause for negation.
\begin{eqnarray*}
\MMM \models \neg \phi &\quad\mbox{ iff }\quad& \MMM \nvDash \phi  
%\MMM \models \phi \lor \psi &\mbox{ iff }& \MMM \models \phi \text{ or }   \MMM \models \psi
\end{eqnarray*}

\NI Negation is a core operation of classical logic, and its absence makes
\cathoristic{} unusual. In order to understand \cathoristic{} better, we
now investigate how negation can be seen as a definable abbreviation
in \cathoristic{} with disjunction. The key idea is to use the fact that
\[
   \neg \MAY{a}{\phi}
\]
can be false in two ways: either there is no $a$-labelled action at
the current state - or there is, but $\phi$ is false. Both arms of
this disjunction can be expressed in \cathoristic{}, the former as
$!\Sigma \setminus \{a\}$, the latter as $\MAY{a}{\neg \phi}$.
 Hence, we can see $\neg \MAY{a}{\phi}$ as a shorthand for 
\[
   !(\Sigma \setminus \{a\}) \OR \MAY{a}{\neg \phi}
\]

\NI Negation still occurs in this term, but prefixing a formula of
lower complexity.

This leaves the question of negating the tantum. That's easy: when
$\neg !A$, then clearly the current state can do an action $a \notin
A$. In other words
\[
   \BIGOR_{a \in \Sigma}\MAY{a}{\TRUE}
\]

\NI When $\Sigma$ is infinite, then so is the disjunction.

Note that both the negation of the modality and the negation of
the tantum involve the set $\Sigma$ of actions. 
So far, we have defined negation with respect to
the whole (possibly infinite) set $\Sigma$. For technical reasons, we
generalise negation and define it with respect to a \emph{finite}
subset $S \subseteq \Sigma$. We use this finitely-restricted version of
negation in the decision procedure below.

\begin{definition}
The function $\neg_{S}(\phi)$ removes negation from $\phi$
relative to a finite subset $S \subseteq \Sigma$:

\begin{align*}
  \neg_{S}(\top) &\ =\  \bot  &
  \neg_S(\bot) &\ =\  \top  \\
  \neg_S(\phi \land \psi) &\ =\  \neg_S(\phi) \lor \neg_S(\psi)  &
  \neg_S(\phi \lor \psi) &\ =\  \neg_S(\phi) \land \neg_S(\psi)  \\
  \neg_S(\langle a \rangle \phi) &\ =\  \fBang(S-\{a\}) \lor \langle a \rangle \neg_S(\phi)  &
  \neg_S(\fBang A) &\ =\  \bigvee_{a \in S - A} \langle a \rangle \top
\end{align*}


\end{definition}

\subsection{Decision procedure}

\NI We can use the fact that \cathoristic\ has a quadratic-time
decision procedure to build an super-polynomial time decision procedure for
\cathoristic\ with negation.  Given $\phi \models \psi$, let $S =
\ACTIONS{\phi} \cup \ACTIONS{\psi} \cup \{a\}$.  - where $a$ is a
fresh action.  The function $\ACTIONS{\cdot}$ returns all actions
occurring in a formula, e.g. $\ACTIONS{\MAY{a}{\phi}} = \{a\} \cup
\ACTIONS{\phi}$ and $\ACTIONS{!A} = A$.,

\begin{enumerate}

\item Inductively translate away all negations in $\phi$ using
  $\neg_S(\phi)$ as defined above.  Let the result be $\phi'$.

\item Reduce $\phi'$ to disjunctive normal form by repeated
  application of the rewrite rules:
  \[
    \phi \land (\psi \lor \xi)  \ \leadsto \ (\phi \land \psi) \lor (\phi \land \xi)  
       \qquad
    (\phi \lor \psi) \land \xi  \ \leadsto \ (\phi \land \xi) \lor (\psi \land \xi) 
  \]

\item Let the resulting disjuncts be $\phi_1, ..., \phi_n$. 
Now note that
\[
\phi \models \psi \quad\text{ iff }\quad \phi_i \models \psi \text{ for all } i = 1, ..., n,
\]
and for each disjunct $\phi_i$ do the following.
\begin{itemize}

\item Notice that $\phi_i \models \psi$ if and only if all
  $S$-extensions (defined below) of $\SIMPL{\phi_i}$ satisfy $\psi$.
  So, to check whether $\phi_i \models \psi$, we enumerate the
  $S$-extensions of $\SIMPL{\phi_i}$ (there are a finite number of
  such extensions - the exact number is exponential in the size of
  $\SIMPL{\phi_i}$) and check for each such $S$-extension $\MMM$
  whether $\MMM\models \psi$, using the algorithm of Section
  \ref{decisionprocedure}.

\end{itemize}
\end{enumerate}

\NI  Here is the definition of $S$-extension.

\begin{definition}
Given an cathoristic transition system $\LLL =
(\mathcal{W},\rightarrow,\lambda)$, and a set $S$ of actions, then
$(\mathcal{W'},\rightarrow',\lambda')$ is a \emph{$S$-extension} of
$\LLL$ if it is a valid cathoristic transition system (recall
Definition \ref{cathoristicTS}) and for all $(x,a,y) \in
\rightarrow'$, either:
\begin{itemize} 

\item $(x, a, y) \in \rightarrow$,  or;

\item $x \in \mathcal{W}$, $a \in S$, $a \in \lambda(x)$, and $y$ is a new state not
  appearing elsewhere in $\mathcal{W}$ or $\mathcal{W'}$.

\end{itemize}
\end{definition}
The state-labelling $\lambda'$ is:
\begin{eqnarray*}
\lambda'(x) & = & \lambda(x) \text{ if } x \in \mathcal{W} \\
\lambda'(x) & = & \Sigma \text{ if } x \notin \mathcal{W} \\
\end{eqnarray*}

\NI In other words, $\MMM'$ is an extension of an annotated model
$\MMM$, if all its transitions are either from $\MMM$ or involve
states of $\MMM$ transitioning via elements of $S$ to new states not
appearing in $\MMM$ or $\MMM'$.  The number of extensions grows
quickly.  If the model $\MMM$ has $n$ states, then the number of
possible extensions is:
\[
   ({2^{|S|}})^n
\] 

\NI But recall that we are computing these extensions in order to
verify $\psi$. So we can make a significant optimisation by
restricting the height of each tree to $|\psi|$. We state, without proof, that this optimisation preserves correctness.
A Haskell implementation of the decision procedure is available
\cite{HaskellImplementation}.
