\section{Quantified \cathoristic{}}\label{quantifiedEL}

\NI So far, we have presented \cathoristic\ as a propositional modal
logic.  This section sketches quantified \cathoristic, primarily to
demonstrate that this extension works smoothly. 

\begin{definition} 
Let $\Sigma$ be a non-empty set of actions, ranged over by $a, a',
...$ as before.  Given a set $\VVV$ of \emph{variables}, with
$x, x', y, y', ...$ ranging over $\VVV$, the \emph{terms},
ranged over by $t, t', ...$ and formulae of quantified \cathoristic{}
are given by the following grammar:

\begin{GRAMMAR}
  t
     &\quad ::= \quad & 
  x
     \VERTICAL 
  a
  \\[1mm]
  \phi 
     &\quad ::= \quad & 
  \TRUE 
     \VERTICAL 
  \phi \AND \psi
     \VERTICAL 
  \MAY{t}{\phi}
     \VERTICAL 
  \fBang A 
     \VERTICAL 
  \exists x . {\phi}
     \VERTICAL 
  \forall x . {\phi}
\end{GRAMMAR}

\NI Now $A$ ranges over finite subsets of terms. The \emph{free
  variables} of a $\phi$, denoted $\FV{\phi}$ is given as expected,
e.g.~$\FV{\MAY{t}{\phi}} = \FV{t} \cup \FV{\phi}$ and $\FV{!A} =
\bigcup_{t \in A}\FV{t}$ where $\FV{a} = \emptyset$ and $\FV{x} =
\{x\}$.
\end{definition}

\begin{definition}
The semantics of quantified \cathoristic{} is constructed along
conventional lines. An \emph{environment} is a map $\sigma : \VVV
\rightarrow \Sigma$ with finite domain.  We write $\sigma, x : a$ for
the environment that is just like $\sigma$, except it also maps $x$ to
$a$, implicitly assuming that $x$ is not in $\sigma$'s domain.  The
\emph{denotation} $\SEMB{t}_{\sigma}$ of a term $t$ under an
environment $\sigma$ is given as follows:
\[
   \SEMB{a}_{\sigma} = a
      \qquad\qquad
   \SEMB{x}_{\sigma} = \sigma(x)
\]
where we assume that $\FV{t}$ is a subset of the domain of $\sigma$.

The \emph{satisfaction
  relation} $\MMM \models_{\sigma} \phi$ is defined whenever
$\FV{\phi}$ is a subset of $\sigma$'s domain. It is given by the
following clauses, where we assume that $\MMM = (\LLL, s)$ and $\LLL =
(S, \rightarrow, \lambda)$.

\[
\begin{array}{lclcl}
  \MMM & \models_{\sigma} & \top   \\
  \MMM & \models_{\sigma} & \phi \AND \psi &\ \mbox{ iff } \ & \MMM  \models_{\sigma} \phi \mbox { and } \MMM \models_{\sigma} \psi  \\
  \MMM & \models_{\sigma} & \langle t \rangle \phi & \mbox{ iff } & \text{there is transition } s \TRANS{\SEMB{t}_{\sigma}} s' \mbox { such that } (\LLL, s') \models_{\sigma} \phi  \\
  \MMM & \models_{\sigma} & \fBang A &\mbox{ iff } & \lambda(s) \subseteq \{\SEMB{t}\ |\ t \in A\} \\
  \MMM & \models_{\sigma} & \forall x.\phi &\mbox{ iff } & \text{for all} \ a \in \Sigma\ \text{we have}\ \MMM \models_{\sigma, x : a} \phi \\
  \MMM & \models_{\sigma} & \exists x.\phi &\mbox{ iff } & \text{there exists} \ a \in \Sigma \ \text{such that}\  \MMM \models_{\sigma, x : a} \phi
\end{array}
\]


\end{definition}

\NI In quantified \cathoristic{}, we can say that there is exactly one
king of France, and he is bald, as:
\[
   \exists x . (\MAY{king} \MAY{france} ! \{x\} \land \MAY{x} \MAY{bald})
\]

\NI Expressing this in \fol{} is more cumbersome:
\[
   \exists x. ( king(france, x) \land bald(x) \land \forall y. ( king(france, y) \rightarrow y = x ))
\]

\NI The \fol{} version uses an extra universal quantifier, and also
requires the identity relation with concomitant axioms.

To say that every person has exactly one sex, which is either male or
female, we can write in quantified \cathoristic{}:
\[
   \forall x . 
      ( \MAY{x} \MAY{person} \rightarrow \MAY{x} \MAY{sex} !\{male, female\} 
      \land 
      \exists y . \MAY{x} \MAY{sex} (\MAY{y} \land
   \fBang \{y\}) )
\]

\NI This is more elegant than the equivalent in \fol{}:
\[
   \forall x. ( person(x) \rightarrow \exists y .
   \left(
      \begin{array}{l}
        sex(x,y) \\
        \quad\land\\
        (y = male \; \lor \; y = female)\\ 
        \quad\land\\
        \forall z . sex(x,z) \rightarrow    y = z 
      \end{array}
   \right))
\]

\NI To say that every traffic light is coloured either green, amber or
red, we can write in quantified \cathoristic{}:
\[
   \forall x. (\MAY{x} \MAY{light} \rightarrow \MAY{x} \MAY{colour}
   !\{green, amber, red\} \land \exists y . \MAY{x} \MAY{colour}
   (\MAY{y} \land !\{y\}))
\]

\NI Again, this is less verbose than the equivalent in
\fol{}:
\[
   \forall x. ( light(x) \rightarrow \exists y .
   \left(
      \begin{array}{l}
        colour(x,y) \\
        \quad\land\\
        (y = green \; \lor \; y = amber \; \lor \; y = red) \\
        \quad\land \\
        \forall z . colour(x,z) \rightarrow y = z
      \end{array}
   \right)  )
\]

