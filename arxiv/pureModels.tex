\section{Alternative semantics for \cathoristic{}}\label{pureModels}

\NI We use state-labelled transition systems as models for
\cathoristic{}. The purpose of the labels on states is to express
constraints, if any, on outgoing actions. This concern is reflected in
the semantics of $!A$.
\[
\begin{array}{lclcl}
  ((S, \rightarrow, \lambda), s) & \models & !A  &\mbox{\quad iff\quad } & \lambda(s) \subseteq A
\end{array}
\]

\NI There is an alternative, and in some sense even simpler approach
to giving semantics to $!A$ which does not require state-labelling: we
simply check if all actions of all outgoing transitions at the current
state are in $A$.  As the semantics of other formula requires state-labelling in its satisfaction condition, this means we can use plain
labelled transition systems (together with a current state) as
models. This gives rise to a subtly different theory that we now
explore, albeit not in depth.

\subsection{Pure cathoristic models}

\begin{definition}\label{pureModelsDef}
By a \emph{pure cathoristic model}, ranged over by $\PPP, \PPP', ...$,
we mean a pair $(\LLL, s)$ where $\LLL = (S, \rightarrow)$ is a
deterministic labelled transition system and $s \in S$ a state.
\end{definition}

\NI Adapting the satisfaction relation to pure cathoristic models is
straightforward.

\begin{definition}
Using pure cathoristic models, the  \emph{satisfaction relation} is defined 
inductively by the following clauses, where we assume that $\MMM =
(\LLL, s)$ and $\LLL = (S, \rightarrow)$.

\[
\begin{array}{lclcl}
  \MMM & \models & \top   \\
  \MMM & \models & \phi \AND \psi &\ \mbox{ iff } \ & \MMM  \models \phi \mbox { and } \MMM \models \psi  \\
  \MMM & \models & \langle a \rangle \phi & \mbox{ iff } & \text{there is a } s \xrightarrow{a} t \mbox { such that } (\LLL, t) \models \phi  \\
  \MMM & \models & A &\mbox{ iff } & \{a\ |\ \exists t.s \TRANS{a} t \} \subseteq A
\end{array}
\]
\end{definition}

\NI Note that all but the last clause are unchanged from Definition
\ref{ELsatisfaction}.

In this interpretation, $!A$ restricts the out-degree of the current
state $s$, i.e.~it constraints the 'width' of the graph.  It is easy
to see that all rules in Figure \ref{figure:elAndBangRules} are sound
with respect to the new semantics.  The key advantage pure cathoristic
models have is their simplicity: they are unadorned labelled
transition systems, the key model of concurrency theory
\cite{SassoneV:modcontac}. The connection with concurrency theory is
even stronger than that, because, as we show below (Theorem
\ref{hennessymilnertheorem}), the elementary equivalence on (finitely
branching) pure cathoristic models is bisimilarity, one of the more
widely used notions of process equivalence. This characterisation even
holds if we remove the determinacy restriction in Definition \ref{pureModelsDef}.

\subsection{Relationship between pure and cathoristic models}

The obvious way of converting an cathoristic model into a pure cathoristic model
is by forgetting about the state-labelling:
\[
   ((S, \rightarrow, \lambda), s ) \qquad\mapsto\qquad ((S, \rightarrow), s ) 
\]
Let this function be $\FORGET{\cdot}$. For going the other way, we
have two obvious choices:

\begin{itemize}

\item $((S, \rightarrow), s ) \mapsto ((S, \rightarrow, \lambda), s )$
  where $\lambda(t) = \Sigma$ for all states $t$. Call this map $\MAX{\cdot}$.

\item $((S, \rightarrow), s ) \mapsto ((S, \rightarrow, \lambda), s )$
  where $\lambda(t) = \{a \ |\ \exists t'. t \TRANS{a} t'\}$ for all
  states $t$. Call this map $\MIN{\cdot}$.

\end{itemize}

\begin{lemma}\label{modelRelationships}
Let $\MMM$ be an cathoristic model, and $\PPP$ a pure cathoristic model.
\begin{enumerate}

\item\label{modelRelationships:1}  $\MMM \models \phi$ implies
  $\FORGET{\MMM} \models \phi$. The reverse implication does not hold.

\item\label{modelRelationships:2}  $ \MAX{\PPP} \models \phi$ implies
  $\PPP \models \phi$. The reverse implication does not hold.

\item\label{modelRelationships:3} $\MIN{\PPP} \models \phi$ if and only if
  $\PPP \models \phi$. 

\end{enumerate}
\end{lemma}

\begin{proof}
The implication in (\ref{modelRelationships:1}) is immediate by
induction on $\phi$. A counterexample for the reverse implication is
given by the formula $\phi = !\{a\}$ and the cathoristic model $\MMM = ( \{s,
t\}, s \TRANS{a} t, \lambda), s)$ where $\lambda (s) = \{a, b, c\}$:
clearly $\FORGET{\MMM} \models \phi$, but $\MMM \not\models
\phi$.

The implication in (\ref{modelRelationships:2}) is immediate by
induction on $\phi$. To construct a counterexample for the reverse
implication, assume that $\Sigma$ is a strict superset of $\{a\}$
$a$. The formula $\phi = !\{a\}$ and the pure cathoristic model $\PPP = (
\{s, t\}, s \TRANS{a} t ), s)$ satisfy $\PPP \models \phi$, but clearly
$\MAX{\PPP} \not\models \phi$.

Finally, (\ref{modelRelationships:3}) is also straightforward by
induction on $\phi$.

\end{proof}

\subsection{Non-determinism and cathoristic models}

Both, cathoristic models and pure cathoristic models must be deterministic. That
is important for the incompatibility semantics. However, formally, the
definition of satisfaction makes sense for non-deterministic models as
well, pure or otherwise. Such models are important in the theory of
concurrent processes. Many of the theorems of the precious section
either hold directly, or with small modifications for
non-deterministic models. The rules of inference in Figure
\ref{figure:elAndBangRules} are sound except for
    [\RULENAME{Determinism}] which cannot hold in properly
    non-deterministic models. With this omission, they are also
    complete.  Elementary equivalence on non-deterministic cathoristic
    models also coincides with mutual simulation, while elementary
    equivalence on non-deterministic pure cathoristic models is
    bisimilarity. The proofs of both facts follow those of Theorems
    \ref{theorem:completeLattice} and \ref{hennessymilnertheorem},
    respectively. Compactness by translation can be shown following
    the proof in Section \ref{compactness}, except that the constraint
    $\phi_{det}$ is unnecessary.

We have experimented with a version of \cathoristic{} in which the models are
\emph{non-deterministic} labelled-transition systems.  Although
non-determinism makes some of the constructions simpler,
non-deterministic \cathoristic{} is unable to express incompatibility properly.
Consider, for example, the claim that Jack is married\footnote{We assume, in this discussion, that $married$ is a many-to-one predicate. We assume that polygamy is one person \emph{attempting} to marry two people (but failing to marry the second).} to Jill
In standard deterministic \cathoristic{} this would be rendered as:
\begin{eqnarray*}
  \MAY{jack} \MAY{married} (\MAY{jill} \land \fBang \{jill\})
\end{eqnarray*}
There are three levels at which this claim can be denied.
First, we can claim that Jack is married to someone else - Joan, say:
\begin{eqnarray*}
   \MAY{jack} \MAY{married} (\MAY{joan} \land \fBang \{joan\})
\end{eqnarray*}
Second, we can claim that Jack is unmarried (specifically, that being unmarried is Jack's only property):
\begin{eqnarray*}
  \MAY{jack} !\{unmarried\}
\end{eqnarray*}
Third, we can claim that Jack does not exist at all. Bob and Jill, for example, are the only people in our domain:
\begin{eqnarray*}
  \fBang \{bob, jill\}
\end{eqnarray*}

Now we can assert the same sentences in non-deterministic \cathoristic{}, but they
are \emph{no longer incompatible with our original sentence}.  In
non-deterministic \cathoristic{}, the following sentences are compatible (as long
as there are two separate transitions labelled with $married$, or two
separate transitions labelled with $jack$):
\begin{eqnarray*}
  \MAY{jack} \MAY{married} (\MAY{jill} \land \fBang \{jill\}) 
      \qquad
  \MAY{jack} \MAY{married} (\MAY{joan} \land \fBang \{joan\})
\end{eqnarray*}

\NI Similarly, the following sentences are fully compatible as long as
there are two separate transitions labelled with $jack$:
\begin{eqnarray*}
  \MAY{jack} \MAY{married}
     \qquad
  \MAY{jack} \fBang \{unmarried\}
\end{eqnarray*}

\NI Relatedly, non-deterministic \cathoristic{} does not satisfy Brandom's
incompatibility semantics property:
\[
   \phi \models \psi \; \mbox{ iff } \; \mathcal{I}(\psi) \subseteq \mathcal{I}(\phi)
\]

\NI To take a simple counter-example, $\MAY{a}\MAY{b}$ implies $\MAY{a}$,
but not conversely.  But in non-deterministic \cathoristic{}, the set of sentences
incompatible with $\MAY{a}\MAY{b}$ is identical with the set of
sentences incompatible with $\MAY{a}$.

\subsection{Semantic characterisation of elementary equivalence}

In Section \ref{elementaryEquivalence} we presented a semantic
analysis of elementary equivalence, culminating in Theorem
\ref{theorem:completeLattice} which showed that elementary equivalence
coincides with $\MODELEQ$, the relation of mutual simulation of
models. We shall now carry out a similar analysis for pure cathoristic
models, and show that elementary equivalence coincides with
bisimilarity, an important concept in process theory and modal logics
\cite{SangiorgiD:intbisac}. Bisimilarity is strictly finer on
non-deterministic transition systems than $\MODELEQ$, and more
sensitive to branching structure.  In the rest of this section, we
allow non-deterministic pure models, because the characterisation is
more interesting that in the deterministic case.

\begin{definition}
A pure cathoristic model $(\LLL, s)$ is finitely branching if its
underlying transition system $\LLL$ is finitely branching.
\end{definition}

\begin{definition}
A binary relation $\RRR$ is a \emph{bisimulation} between pure cathoristic
models $\PPP_i = (\LLL_i), s_i)$ for $i = 1, 2$ provided (1) $\RRR$ is
a bisimulation between $\LLL_1$ and $\LLL_2$, and (2) $(s_1, s_2) \in
\RRR$. We say $\PPP_1$ and $\PPP_2$ are \emph{bisimilar}, written
$\PPP_1 \BISIM \PPP_2$ if there is a bisimulation between $\PPP_1$ and
$\PPP_2$.
\end{definition}

\begin{definition}
The \emph{theory} of $\PPP$, written $\THEORY{\PPP}$, is the set
$\{\phi\ |\ \PPP \models \phi\}$.
\end{definition}

\begin{theorem}
\label{hennessymilnertheorem}
Let $\PPP$ and $\PPP'$ be two finitely branching pure cathoristic
models. Then: $\PPP \BISIM \PPP'$ if and only if $\THEORY{\PPP} =
\THEORY{\PPP'}$. 
\end{theorem}

\begin{proof}
\NI Let $\PPP = (\LLL, w)$ and $\PPP' = (\LLL', w')$ be finitely
branching, where $\LLL = (W, \rightarrow)$ and $(W', \rightarrow')$.
We first show the left to right direction, so assume that $\PPP \BISIM
\PPP'$.

The proof is by induction on formulae.  The only case which differs
from the standard Hennessy-Milner theorem is the case for $!A$, so
this is the only case we shall consider.  Assume $w \BISIM w'$ and $w
\models !A$. We need to show $w' \models !A$.
From the semantic clause for $!$, $w \models !A$ implies $\lambda(w)
\subseteq A$.  If $w \BISIM w'$, then $\lambda(w) = \lambda'(w')$.
Therefore $\lambda'(w') \subseteq A$, and hence $w' \models !A$.

The proof for the other direction is more involved.
For states $x \in W$ and $x' \in W$, we write 
\[
   x \equiv x'
      \qquad\text{iff}\qquad
   \THEORY{(\LLL, x)} = \THEORY{(\LLL', x')}.
\]

We define the bisimilarity relation:
\[
   Z = \{(x,x') \in \mathcal{W} \times \mathcal{W}' \fOr x \equiv x' \}
\]
To prove $w \BISIM w'$, we need to show:
\begin{itemize}

\item $(w,w') \in Z$. This is immediate from the definition of Z.

\item The relation $Z$ respects the transition-restrictions: if
  $(x,x') \in Z$ then $\lambda(x) = \lambda'(x')$

\item The forth condition: if $(x,x') \in Z$ and $x \xrightarrow{a}
  y$, then there exists a $y'$ such that $x' \xrightarrow{a} y'$ and $(y, y') \in Z$.

\item The back condition: if $(x,x') \in Z$ and $x' \xrightarrow{a}
  y'$, then there exists a $y$ such that $x \xrightarrow{a} y$ and $(y, y') \in Z$.

\end{itemize}
To show that $(x,x') \in Z$ implies $\lambda(x) = \lambda'(x')$, we
will argue by contraposition.  Assume $\lambda(x) \neq \lambda'(x')$.
Then either $\lambda'(x') \nsubseteq \lambda(x)$ or $\lambda(x)
\nsubseteq \lambda'(x')$.  If $\lambda'(x') \nsubseteq \lambda(x)$,
then $x' \nvDash \fBang \lambda(x)$.  But $x \models \fBang
\lambda(x)$, so $x$ and $x'$ satisfy different sets of propositions
and are not equivalent.  Similarly, if $\lambda(x) \nsubseteq
\lambda'(x')$ then $x \nvDash \fBang \lambda'(x')$.  But $x' \models
\fBang \lambda'(x')$, so again $x$ and $x'$ satisfy different sets of
propositions and are not equivalent.

We will show the forth condition in detail. The back condition is very
similar.  To show the forth condition, assume that $x \xrightarrow{a}
y$ and that $(x,x') \in Z$ (i.e. $x \equiv x'$).  We need to show that
$\exists y'$ such that $x' \xrightarrow{a} y'$ and $(y,y') \in Z$
(i.e. $y \equiv y'$).

Consider the set of $y'_i$ such that $x' \xrightarrow{a} y'_i$. Since
$x \xrightarrow{a} y$, $x \models \langle a \rangle \top$, and as $x
\equiv x'$, $x' \models \langle a \rangle \top$, so we know this set
is non-empty.  Further, since $(\mathcal{W}', \rightarrow')$ is
finitely-branching, there is only a finite set of such $y'_i$, so we
can list them $y'_1, ..., y'_n$, where $n >= 1$.

Now, in the Hennessy-Milner theorem for Hennessy-Milner logic, the proof proceeds as
follows: assume, for reductio, that of the $y'_1, ..., y'_n$, there is
no $y'_i$ such that $y \equiv y'_i$.  Then, by the definition of
$\equiv$, there must be formulae $\phi_1, ..., \phi_n$ such that for
all $i$ in $1$ to $n$:
\[
y'_i \models \phi_i \mbox{ and } y \nvDash \phi_i
\]
Now consider the formula:
\[
[a] (\phi_1 \lor ... \lor \phi_n)
\]
As each $y'_i \models \phi_i$, $x' \models [a] (\phi_1 \lor ... \lor \phi_n)$, but $x$ does not satisfy this formula, as each $\phi_i$ is not satisfied at $y$.
Since there is a formula which $x$ and $x'$ do not agree on, $x$ and $x'$ are not equivalent, contradicting our initial assumption.

But this proof cannot be used in \cathoristic{} because it relies on a formula $[a] (\phi_1 \lor ... \lor \phi_n)$ which cannot be expressed in \cathoristic{}: 
\Cathoristic{} does not include the box operator or disjunction, so this formula is ruled out on two accounts.
But we can massage it into a form which is more amenable to \cathoristic{}'s expressive resources:
\begin{eqnarray*}
[a] (\phi_1 \lor ... \lor \phi_n) & = & \neg \langle a \rangle \neg (\phi_1 \lor ... \lor \phi_n)  \\
	& = & \neg \langle a \rangle (\neg \phi_1\AND ... \AND \neg \phi_n) 
\end{eqnarray*}
Further, if the original formula $[a] (\phi_1 \lor ... \lor \phi_n)$ is true in $x'$ but not in $x$, then its negation will be true in $x$ but not in $x'$. 
So we have the following formula, true in $x$ but not in $x'$:
\[
 \langle a \rangle (\neg \phi_1\AND ... \AND \neg \phi_n)
 \]
The reason for massaging the formula in this way is so we can express it in \cathoristic{} (which does not have the box operator or disjunction).
At this moment, the revised formula is \emph{still} outside \cathoristic{} because it uses negation. 
But we are almost there: the remaining negation is in innermost scope, and innermost scope negation can be simulated in \cathoristic{} by the $!$ operator. 

We are assuming, for reductio, that of the $y'_1, ..., y'_n$, there is no $y'_i$ such that $y \equiv y'_i$.
But in \cathoristic{} without negation, we cannot assume that each $y'_i$ has a formula $\phi_i$ which is satisfied by $y'_i$ but not by $y$ - it might instead be the other way round: $\phi_i$ may be satisfied by $y$ but not by $y'_i$. So, without loss of generality, assume that $y'_1, ..., y'_m$ fail to satisfy formulae $\phi_1, ..., \phi_m$ which $y$ does satisfy, and that $y'_{m+1}, ..., y'_n$ satisfy formulae $\phi_{m+1}, ..., \phi_n$ which $y$ does not:
\begin{eqnarray*}
y \models \phi_i \mbox{ and } y'_i \nvDash \phi_i & & i = 1 \mbox{ to } m  \\
y \nvDash \phi_j \mbox{ and } y'_j \models \phi_j & & j = m+1 \mbox{ to } n 
\end{eqnarray*}
The formula we will use to distinguish between $x$ and $x'$ is:
\[
 \langle a \rangle ( \bigwedge_{i=1}^m \phi_i \; \AND \; \bigwedge_{j=m+1}^n \mathsf{neg}(y, \phi_j))
 \]
 Here, $\mathsf{neg}$ is a meta-language function that, given a state y and a formula $\phi_j$, returns a formula that is true in $y$ but incompatible with $\phi_j$. We will show that, since $y \nvDash \phi_j$, it is always possible to construct $ \mathsf{neg}(y, \phi_j)$ using the $!$ operator.

Consider the possible forms of $\phi_j$:
\begin{itemize}
\item
$\top$: this case cannot occur since all models satisfy $\top$.
\item
$\phi_1 \AND \phi_2$: we know $y'_j \models \phi_1 \AND \phi_2$ and $y \nvDash \phi_1 \AND \phi_2$. There are three possibilities:
\begin{enumerate}
\item
$y \nvDash \phi_1$ and $y \models \phi_2$. In this case, $\mathsf{neg}(y, \phi_1 \AND \phi_2) = \mathsf{neg}(y, \phi_1) \AND \phi_2$.
\item
$y \models \phi_1$ and $y \nvDash \phi_2$. In this case, $\mathsf{neg}(y, \phi_1 \AND \phi_2) = \phi_1 \AND \mathsf{neg}(y, \phi_2)$.
\item
$y \nvDash \phi_1$ and $y \nvDash \phi_2$. In this case, $\mathsf{neg}(y, \phi_1 \AND \phi_2) =  \mathsf{neg}(y, \phi_1) \AND \mathsf{neg}(y, \phi_2)$.
\end{enumerate}
\item
$!A$: if $y \nvDash !A \mbox{ and } y'_j \models !A$, then there is an action $a \in \Sigma-A$ such that $y \xrightarrow{a} z$ for some $z$ but there is no such $z$ such that $y'_j \xrightarrow{a} z$. In this case, let $\mathsf{neg}(y, \phi_j) = \langle a \rangle \top$.
\item
$\langle a \rangle \phi$. There are two possibilities:
\begin{enumerate}
\item
$y \models \langle a \rangle \top$. In this case, $\mathsf{neg}(y, \langle a \rangle \phi) =  \bigwedge\limits_{y \xrightarrow{a} z}  \langle a \rangle \mathsf{neg}(z, \phi)$.
\item
$y \nvDash \langle a \rangle \top$. In this case, $\mathsf{neg}(y, \langle a \rangle \phi) = \fBang \{ b \fOr \exists z. y \xrightarrow{b} z\}$. This set of $b$s is finite since we are assuming the transition system  is finitely-branching.
\end{enumerate}
\end{itemize}

\end{proof}

\begin{figure}[h]
\centering
\begin{tikzpicture}[node distance=1.3cm,>=stealth',bend angle=45,auto]
  \tikzstyle{place}=[circle,thick,draw=blue!75,fill=blue!20,minimum size=6mm]
  \tikzstyle{red place}=[place,draw=red!75,fill=red!20]
  \tikzstyle{transition}=[rectangle,thick,draw=black!75,
  			  fill=black!20,minimum size=4mm]
  \tikzstyle{every label}=[red]

  \begin{scope}[xshift=0cm]
    \node [place] (t) {$y$};
    \node [place] (a) [below left of=t] {$z_1$}
      edge [pre]  node[swap] {a}                 (t);
    \node [place] (a2) [below right of=t] {$z_2$}
      edge [pre]  node[swap] {a}                 (t);
    \node [place] (b) [below of=a] {$w_1$}
      edge [pre]  node[swap] {b}                 (a);
    \node [place] (c) [below of=a2] {$w_2$}
      edge [pre]  node[swap] {c}                 (a2);
  \end{scope}  
  \begin{scope}[xshift=6cm]
    \node [place] (t) {$y'_j$};
    \node [place] (a) [below of=t] {$z'$}
      edge [pre]  node[swap] {a}                 (t);
    \node [place] (b) [below left of=a] {$w'_1$}
      edge [pre]  node[swap] {b}                 (a);
    \node [place] (c) [below right of=a] {$w'_2$}
      edge [pre]  node[swap] {c}                 (a);
  \end{scope}  
\end{tikzpicture}
\caption{{\small Worked example of $\mathsf{neg}$. Note that the transition
  system on the left is non-deterministic.}}\label{figure:example:neg}
\end{figure}



\NI We continue with a worked example of $\mathsf{neg}$.  Consider
  $y$ and $y'_j$ as in Figure \ref{figure:example:neg}.  One formula
  that is true in $y'_j$ but not in $y$ is

\[
   \langle a \rangle (\langle b \rangle \top \AND \langle c \rangle \top)
\]

\NI Now:

\begin{eqnarray*}
\lefteqn{\mathsf{neg}(y, \langle a \rangle (\langle b \rangle \top \AND \langle c \rangle \top))}\qquad \qquad \qquad  \\
& = & \bigwedge\limits_{y \xrightarrow{a} z} \langle a \rangle \mathsf{neg}(z, \langle b \rangle \top \AND \langle c \rangle \top)  \\
& = & \langle a \rangle \mathsf{neg}(z_1, \langle b \rangle \top \AND \langle c \rangle \top) \AND \langle a \rangle\mathsf{neg}(z_2, \langle b \rangle \top \AND \langle c \rangle \top)  \\
& = & \langle a \rangle (\langle b \rangle \top \AND \mathsf{neg}(z_1, \langle c \rangle \top)) \AND \langle a \rangle\mathsf{neg}(z_2, \langle b \rangle \top \AND \langle c \rangle \top)  \\
& = & \langle a \rangle (\langle b \rangle \top \AND \mathsf{neg}(z_1, \langle c \rangle \top)) \AND \langle a \rangle(\mathsf{neg}(z_2, \langle b \rangle \top) \AND \langle c \rangle \top)  \\
& = & \langle a \rangle (\langle b \rangle \top \AND \fBang \{b\}) \AND \langle a \rangle(\mathsf{neg}(z_2, \langle b \rangle \top) \AND \langle c \rangle \top)  \\
& = & \langle a \rangle (\langle b \rangle \top \AND \fBang \{b\}) \AND \langle a \rangle(\fBang \{c\} \AND \langle c \rangle \top) 
\end{eqnarray*}

\NI The resulting formula is true in $y$ but not in $y'_j$.


