\subsection{A translation from first-order logic to first-order eremic logic}

We now sketch how to translate from first-order logic to eremic
logic. Given that the latter is not nearly as expressive as the
former, we have to restrict our attention to a fragment of first-order
logic. For simplicity, we do not seek maximal generality. Instead we
look at a simple example that presentes the key ideas behind the
translation \emph{in nuce}.

Assume given a relational first-order signature. The relational
symbols are split into three parts: $R_1$ of unary relation symbols
ranged over by $u, ...$, and two sets of binary relation symbols
$R_2^{inj}$, ranged over by $i, ...$ and $R_2$, ranged over by $r,
...$. The meaning of the superscript in $R_2^{inj}$ will become clear
later.  We let $t, t', ...$ range over terms in this signature, which
can only be constants or variables. We look at the following
restricted set of formulae, called \emph{nice}.
\begin{GRAMMAR}
  \phi
     & \quad::=\quad &
  u(t) 
     \VERTICAL 
  i(t, t') 
     \VERTICAL 
  r(t, t') 
     \VERTICAL 
  \phi \AND \psi 
     \VERTICAL 
  \forall x.\phi 
     \VERTICAL 
  \exists x.\phi
\end{GRAMMAR}

\NI Note the absence of negation, disjunction, implication and
equality. We translate this fragment into first-order eremic logic as
follows.
\begin{itemize}

\item $\SEMB{u(t)} = \MAY{t}{u}$.

\item $\SEMB{i(t, t')} = \MAY{t}{\MAY{i}{(\MAY{t'}{} \AND !\{t'\})}}$.

\item $\SEMB{r(t, t')} = \MAY{t}{\MAY{i}{\MAY{t'}{}}}$.

\item $\SEMB{\phi \AND \psi} = \SEMB{\phi} \AND \SEMB{\psi}$.

\item $\SEMB{\forall x.\phi} = \forall x.\SEMB{\phi}$.

\item $\SEMB{\exists x.\phi} = \exists x.\SEMB{\phi}$.

\end{itemize}

\NI A first-order model $\CAL{M}$ for the signature above is called
\emph{nice} provided the interpretation $r^{\CAL{M}}$ of all relations
$i \in R_2^{inj}$ is injective, i.e.~whenever $(x, y), (x, z) \in
i^{\CAL{M}}$, then $y = z$. We now translate nice models $\CAL{M}$
with universe $U$ to eremic models $\SEMB{\CAL{M}}$.

The eremic transition system $\LLL = (S, \rightarrow, \lambda)$ is given by
the following data.
\begin{itemize}

\item The \emph{actions} are given as $\Sigma = U \cup R_1 \cup
  R_2^{inj} \cup R_2$

\item A \emph{home} state $h \in S$, labelled $\Sigma$.

\item For each $u \in R_1$ and all $x \in u^{\CAL{M}}$ a fresh state
  $s \in S$ and transitions
\[
   h \TRANS{x} s \TRANS{u} h,
\]
where $s$ is labelled $\Sigma$.

\item For each $i \in R_2^{inj}$ and all $(x, y)  \in
i^{\CAL{M}}$ two fresh states $s, s' \in S$ and transitions
\[
   h \TRANS{x} s \TRANS{i} s' \TRANS{y} h,
\]
where $s$ is labelled $\Sigma$ and $s'$ is labelled $\{y\}$.

\item For each $r \in R_2$ and all $(x, y)  \in                                                                         
r^{\CAL{M}}$ two fresh states $s, s' \in S$ and transitions
\[
   h \TRANS{x} s \TRANS{r} s' \TRANS{y} h,
\]
where $s$ and $s'$ are both labelled $\Sigma$.

\end{itemize}

\NI Now $\SEMB{\CAL{M}} = (\LLL, h)$.

\begin{theorem}[Conjecture]
Let $\phi$ be a nice formula, $\CAL{M}$ be a nice model and $\sigma$
an environment. Then:
\[
   \CAL{M} \models_{\sigma} \phi
      \qquad\text{iff}\qquad
   \SEMB{\CAL{M}} \models_{\sigma} \SEMB{\phi}
\]
\end{theorem}

Note that the encoding can easily be generalised to $n$-ary relations
for all $n$, and also to finite-width relations.  What does the latter
mean? Let $r$ be an $n$-ary relation symbol in the ambient first-order
signature. Let $\CAL{M}$ be a model. The interpretation $r^{\CAL{M}}$
of $r$ in $\CAL{M}$ is called an $(i, k)$-relation, where $0 < i \leq
n$, and $k$ is arbitrary, provided for the cardinality of the set
\[
   \{ y | (x_1, ..., x_{i-1}, y, x_{i+1}, ..., x_{n}) \in r^{\CAL{M}} \}
\]
does not exceed $k$. We say $r^{\CAL{M}}$ is width-restricted if for each $i$
there is $k$ such that $r^{\CAL{M}}$ is a $(i, k)$-relation.

