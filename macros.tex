%% \usepackage{ifthen}
%% \usepackage{amssymb}
%% \newboolean{showcomments}
%% \setboolean{showcomments}{true}
%% \ifthenelse{\boolean{showcomments}}
%%   {\newcommand{\mynote}[2]{
%%     \fbox{\bfseries\sffamily\scriptsize#1}
%%     {\small$\blacktriangleright$\textsf{\emph{#2}}$\blacktriangleleft$}
%%    }
%%   }
%%   {\newcommand{\mynote}[2]{}
%%   }
%% \newcommand\martin[1]{\mynote{Martin}{#1}}
%% \newcommand\richard[1]{\mynote{Richard}{#1}}

\newtheorem{convention}{Convention}
\renewenvironment{proof}{\paragraph{Proof:}}{\hfill\qed}


\def \Cathoristic {Cathoristic logic}
\def \cathoristic {cathoristic logic}
\def\FOL{First-order logic}
\def\fol{first-order logic}
\newcommand{\NOVSPACEPARAGRAPH}[1]{\NI\textbf{#1.}}
\newcommand{\PARAGRAPH}[1]{\NOVSPACEPARAGRAPH{#1}}
\newcommand{\NI}{\noindent}
\newcommand{\LEQ}{\sqsubseteq}
\newcommand{\MODELLEQ}{\preceq}
\newcommand{\MODELEQ}{\simeq}
\newcommand{\SIM}{\preceq_{sim}}
\newcommand{\BISIM}{\sim}
\newcommand{\BIGLUB}{\bigsqcup}
\newcommand{\BIGGLB}{\bigsqcap}
\newenvironment{FIGURE}{\begin{figure}[h]\rule{\linewidth}{0.5pt}
 %\vspace{3.2mm}
}{\rule{\linewidth}{0.5pt}\end{figure}}
\newenvironment{SIDEWAYSFIGURE}{\begin{sidewaysfigure}[ht]\rule{\linewidth}{0.5pt}
 %\vspace{3.2mm}
}{\rule{\linewidth}{0.5pt}\end{sidewaysfigure}}
\newenvironment{RULES}{\[\begin{array}{c}}{\end{array}\]}
\newenvironment{GRAMMAR}{\[\begin{array}{lcl}}{\end{array}\]}
\newcommand{\VERTICAL}{\  \mid\hspace{-3.0pt}\mid \ }
\newcommand{\infer}[2]{\frac{\displaystyle{ #1 }}{\displaystyle{ #2 }}}
\newcommand{\ZEROPREMISERULE}[1]{\infer{-}{#1}}
\newcommand{\ONEPREMISERULE}[2]{\infer{#1}{#2}}
\newcommand{\TWOPREMISERULE}[3]{\infer{#1 \quad #2}{#3}}
\newcommand{\THREEPREMISERULE}[4]{\infer{#1 \quad #2 \quad #3}{#4}}
\newcommand{\FOURPREMISERULE}[5]{\infer{#1 \quad #2 \quad #3 \quad #4}{#5}}
\newcommand{\SIXPREMISERULE}[7]{\infer{#1 \quad #2 \quad #3 \quad #4 \quad #5 \quad #6}{#7}}
\newcommand{\RULENAME}[1]{\textsc{#1}}
\newcommand{\SMALLRULENAME}[1]{\textsc{\small #1}}
\newcommand{\ZEROPREMISERULENAMEDRIGHT}[2]{\ZEROPREMISERULE{#1}\,\SMALLRULENAME{#2}}
\newcommand{\ONEPREMISERULENAMEDRIGHT}[3]{\ONEPREMISERULE{#1}{#2}\,\SMALLRULENAME{#3}}
\newcommand{\TWOPREMISERULENAMEDRIGHT}[4]{\TWOPREMISERULE{#1}{#2}{#3}\,\SMALLRULENAME{#4}}
\newcommand{\THREEPREMISERULENAMEDRIGHT}[5]{\THREEPREMISERULE{#1}{#2}{#3}{#4}\,\SMALLRULENAME{#5}}
\newcommand{\FOURPREMISERULENAMEDRIGHT}[6]{\FOURPREMISERULE{#1}{#2}{#3}{#4}{#5}\,\SMALLRULENAME{#6}}
\newcommand{\ZEROPREMISERULENAMEDLEFT}[2]{\SMALLRULENAME{#2}\,\ZEROPREMISERULE{#1}}
\newcommand{\ONEPREMISERULENAMEDLEFT}[3]{\SMALLRULENAME{#3}\,\ONEPREMISERULE{#1}{#2}}
\newcommand{\TWOPREMISERULENAMEDLEFT}[4]{\SMALLRULENAME{#4}\,\TWOPREMISERULE{#1}{#2}{#3}}
\newcommand{\THREEPREMISERULENAMEDLEFT}[5]{\SMALLRULENAME{#5}\,\THREEPREMISERULE{#1}{#2}{#3}{#4}}
\newcommand{\FOURPREMISERULENAMEDLEFT}[6]{\SMALLRULENAME{#6}\,\FOURPREMISERULE{#1}{#2}{#3}{#4}{#5}}

\newcommand{\FV}[1]{\mathsf{fv}(#1)}
\newcommand{\THEORY}[1]{\mathsf{Th}(#1)}
\newcommand{\FORGET}[1]{\mathsf{forget}(#1)}
\newcommand{\ACTIONS}[1]{\mathsf{actions}(#1)}
\newcommand{\MAX}[1]{\mathsf{max}(#1)}
\newcommand{\MIN}[1]{\mathsf{min}(#1)}
\newcommand{\SIMPL}[1]{\mathsf{simpl}(#1)}
\newcommand{\CHAR}[1]{\mathsf{char}(#1)}
\newcommand{\MAY}[2]{\langle #1 \rangle #2}
\newcommand{\MUST}[2]{[ #1 ] #2}
\newcommand{\NEG}[1]{\mathsf{neg}(#1)}
\newcommand{\AND}{\land}
\newcommand{\BIGAND}{\bigwedge}
\newcommand{\BIGOR}{\bigvee}
\newcommand{\OR}{\lor}
\newcommand{\INC}[1]{\mathsf{Inc}(#1)}
\newcommand{\PVAR}[1]{\text{\textsc{#1}}}
\newcommand{\CAL}[1]{\mathcal{#1}}
\newcommand{\FRAK}[1]{\mathfrak{#1}}
\newcommand{\SEMB}[1]{\lbrack\!\lbrack #1 \rbrack\!\rbrack}
\newcommand{\SEMBTWO}[1]{\langle\!\langle #1 \rangle\!\rangle}
\newcommand{\STATE}{\mathsf{State}}
\newcommand{\SYMBOL}{\mathsf{Symbol}}
\newcommand{\DEFEQ}{\stackrel{\text{\emph{def}}}{=}}
\newcommand{\TRUE}{\top}
\newcommand{\FALSE}{\bot}
\newcommand{\LOGIC}[1]{\mathsf{#1}}
\newcommand{\MMM}{\FRAK{M}}
\newcommand{\NNN}{\FRAK{N}}
\newcommand{\PPP}{\FRAK{P}}
\newcommand{\SSS}{\CAL{S}}
\newcommand{\VVV}{\CAL{V}}
\newcommand{\IMPLIES}{\rightarrow}
\newcommand{\RED}{\rightarrow}
\newcommand{\RESTRICT}[2]{\mathsf{Restrict}_{#1}(#2)}
\newcommand{\ALLOWED}[3]{\mathsf{Allow}^{#1}(#2, #3)}
\newcommand{\RESTRICTTWO}[3]{\mathsf{Restrict}^{#1}(#2, #3)}
\newcommand{\ARROW}[3]{\mathsf{Arrow}_{#1}(#2, #3)}
\newcommand{\ARROWTWO}[4]{\mathsf{Arrow}^{#1}(#2, #3, #4)}
\newcommand{\LLL}{\mathcal{L}}
\newcommand{\RRR}{\mathcal{R}}
\newcommand{\TRANS}[1]{\stackrel{#1}{\longrightarrow}}

\newcommand{\QUOTATION}[1]{
\hfill
\hspace{58mm}
\begin{minipage}{60mm}\tiny #1\end{minipage}
}

\lstnewenvironment{code}
    {\lstset{}%
      \csname lst@SetFirstLabel\endcsname}
    {\csname lst@SaveFirstLabel\endcsname}
    \lstset{
      basicstyle=\small\ttfamily,
      flexiblecolumns=false,
      basewidth={0.5em,0.45em},
      literate={+}{{$+$}}1 {/}{{$/$}}1 {*}{{$*$}}1 {=}{{$=$}}1
               {>}{{$>$}}1 {<}{{$<$}}1 {\\}{{$\lambda$}}1
               {\\\\}{{\char`\\\char`\\}}1
               {->}{{$\rightarrow$}}2 {>=}{{$\geq$}}2 {<-}{{$\leftarrow$}}2
               {<=}{{$\leq$}}2 {=>}{{$\Rightarrow$}}2 
               {\ .}{{$\bigcirc$}}2 {\ .\ }{{$\bigcirc$}}2
               {>>}{{>>}}2 {>>=}{{>>=}}2
               {|}{{$\mid$}}1               
    }


\newcommand{\CASE}[1]{\underline{Case #1.}}
\newcommand{\SUBCASE}[1]{\underline{Subcase #1.}}
%\newtheorem{mycase}{Case}
%\newtheorem{subcase}{Case}
%\numberwithin{subcase}{mycase}


% Dot
\def\fDot {\ast}
% Bang
\def\fBang {!}
% Or
\def\fOr {\ | \ }

% Turnstiles with subscripts
\def\judgeX {\sststile{\mathrm{X}}{}}
\def\judgeY {\sststile{\mathrm{Y}}{}}
%\def\judge {\sststile{\mathrm{}}{}}

\newcommand{\judge}{\vdash}

\EnableBpAbbreviations
