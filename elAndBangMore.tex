\section{Further properties of the EL$[\AND, !]$ fragment}

In Section \ref{coreEL} we presented the core fragment of eremic logic
and established key properties like completeness of the proof rules
and compactness of satisfaction relation. This section investigates
the core fragment further: first we give a linear-time decision
pocedure. Then we give a logical characterisation of bisimilarity,
following ideas from the theory of process calculi
\cite{HennessyM:alglawfndac}.  The linear-time decision complexity is
an indication that eremic logic can be useful as a query language in
knowledge representation. Section \ref{kr} investigates this matter in
more detail. The logical characterisation of a semantical equivalence
on the other hand is an indication that our syntactic and semantical
constructs cohere with each other.

\subsection{Decision Procedure}

\NI Since EL has no connectives for disjunction or implication, its
decision procedure is straightforward and efficient.  Although there
are an infinite number of models which satisfy any expression, the
satisfying models form a lattice with a least upper bound.  The $\mu$
function defined above gives us the minimal model satisfying an
expression.  Using this least upper bound, we can calculate entailment
by checking a \emph{single model}.  To decide whether $p \models q$,
we use the following theorem:

\begin{theorem}\label{theorem:decision}
  Forall models $m$, the following are equivalent:
  \begin{enumerate}
    \item\label{theorem:decision:1} $m \models p \Rightarrow m \models q$.
    \item\label{theorem:decision:2} $\SIMPL{p} \models q$.
  \end{enumerate}
  \martin{Richard, is this rephrasing correct?}
\end{theorem}

\begin{proof}
We first show the implication from (\ref{theorem:decision:1}) to
(\ref{theorem:decision:2}), and from (\ref{theorem:decision:2}) to
(\ref{theorem:decision:1}).  \setcounter{mycase}{0}
\begin{mycase}
$\forall m \; m \models p \Rightarrow m \models q \; \text{implies} \; \SIMPL{p} \models q$
\end{mycase}

\NI Given that $\SIMPL{p} \models p$, we substitute $\SIMPL{p}$ for $m$ in
$\forall m \; m \models p \Rightarrow m \models q$, to infer $\SIMPL{p}
\models q$.

\begin{mycase}
$\SIMPL{p} \models q \; \text{implies} \; \forall m \; m \models p \Rightarrow m \models q$
\end{mycase}

\NI Assume $m \models p$. We need to show $m \models q$.  Now if $m
\models p$ then $m \LEQ \SIMPL{p}$ (by definition of $\mu$).  Further, if
$n \models x$ and $m \LEQ n$ then $m \models x$. So, substituting $q$
for $x$ and $\SIMPL{p}$ for $n$, it follows that $m \models q$.  \qed
\end{proof}

\NI Given this theorem, the decision procedure is straightforward: to
test if $p \models q$, we construct $\SIMPL{p}$, and then inspect whether
$\SIMPL{p} \models q$.  Construction of $\SIMPL{p}$ is linear in the size of
$p$, and computing whether a model satisfies $q$ is linear in the size
of $q$, so computing whether $p \models q$ is $O(|p|+|q|)$.

\martin{Can we say more about the decision procedure? Example? Talk
  about Haskell code?}

\subsection{Hennessy-Milner Theorem for EL}

Bisimulations are an important concept in process theory and modal
logics \cite{SangiorgiD:intbisac}. Bisimulations give a notion of
equality of labelled transition systems. An interesting facett of
bisimulation is it's co-inductive rather than inductive nature, which
is helpful when working with infinite transition systems.

Elementary equivalence is a natural notion of equality of models that
can be defined for any logic: two models are \emph{elementarily
  equivalent} exactly when they make the same formulae true. In eremic
logic, this definition looks as follows.

\begin{definition}
Two states $w$ and $w'$ are {\bf equivalent}, written $w \equiv w'$,
iff $\{\phi \; | \; w \models \phi\} = \{\phi \; | \; w' \models
\phi\}$.
\end{definition}

\NI A natural question now is how bisimilarity and equivalence
relate. We now show that the two notions of equivalence coincide in
eremic logic.

\begin{theorem}
If two models are finitely-branching (if the set $\{y \fOr \exists s
. x \xrightarrow{s} y\}$ is finite for all states $x$), $w \BISIM w'
\mbox{ iff } w \equiv w' $.\martin{discuss the case of non-finitely-branching}
\end{theorem}
First, left to right.
\begin{case}
If $w \BISIM w'$, then $w \equiv w'$.
\end{case}
\begin{proof}
Proof is by induction on formulae.  The only case which differs from
the standard Hennessy-Milner Theorem is the case for $!A$, so this is
the only case we shall consider.  Assume $w \BISIM w'$ and $w \models
!A$. We need to show $w' \models !A$.

From the semantic clause for $!$,  $w \models !A$ implies $\lambda(w) \subseteq A$.
If $w \BISIM w'$, then $\lambda(w) = \lambda'(w')$.
Therefore $\lambda'(w') \subseteq A$, and hence $w' \models !A$.
\qed
\end{proof}

\begin{case}
Given two finitely-branching models $(\mathcal{W}, \rightarrow,
\lambda)$ and $(\mathcal{W}', \rightarrow', \lambda')$, with $w \in
\mathcal{W}$ and $w' \in \mathcal{W}'$, if $w \equiv w'$, then $w
\BISIM w'$.
\end{case}
\begin{proof}
We define the bisimilarity relation:
\[
Z = \{(x,x') \in \mathcal{W} \times \mathcal{W}' \fOr x \equiv x' \}
\]
To prove $w \BISIM w'$, we need to show:
\begin{itemize}
\item
$(w,w') \in Z$. This is immediate from the definition of Z.
\item
The relation $Z$ respects the transition-restrictions: if $(x,x') \in Z$ then $\lambda(x) = \lambda'(x')$
\item
The forth condition: if $(x,x') \in Z$ and $x \xrightarrow{s} y$, then there exists a $y'$ such that $x' \xrightarrow{s} y'$
\item
The back condition: if $(x,x') \in Z$ and $x' \xrightarrow{s} y'$, then there exists a $y$ such that $x \xrightarrow{s} y$
\end{itemize}
To show that $(x,x') \in Z$ implies $\lambda(x) = \lambda'(x')$, we will argue by contraposition.
Assume $\lambda(x) \neq \lambda'(x')$.
Then either $\lambda'(x') \nsubseteq  \lambda(x)$ or $\lambda(x) \nsubseteq  \lambda'(x')$.
If $\lambda'(x') \nsubseteq  \lambda(x)$, then $x' \nvDash \fBang \lambda(x)$.
But $x \models \fBang \lambda(x)$, so $x$ and $x'$ satisfy different sets of propositions and are not equivalent.
Similarly, if $\lambda(x) \nsubseteq  \lambda'(x')$ then $x \nvDash \fBang \lambda'(x')$.
But $x' \models \fBang \lambda'(x')$, so again $x$ and $x'$ satisfy different sets of propositions and are not equivalent.

I will show the forth condition in detail. The back condition is very similar.
To show the forth condition, assume that  $x \xrightarrow{s} y$ and that $(x,x') \in Z$ (i.e. $x \equiv x'$).
We need to show that $\exists y'$ such that $x' \xrightarrow{s} y'$ and $(y,y') \in Z$ (i.e. $y \equiv y'$).

Consider the set of $y'_i$ such that $x' \xrightarrow{s} y'_i$. Since $x \xrightarrow{s} y$, $x \models \langle s \rangle \top$, and as $x \equiv x'$,  $x' \models \langle s \rangle \top$, so we know this set is non-empty.
Further, since $(\mathcal{W}', \rightarrow')$ is finitely-branching, there is only a finite set of such $y'_i$, so we can list them $y'_1, ..., y'_n$,  where $n >= 1$.

Now, in the Hennessy-Milner theorem for HML, the proof proceeds as follows:
assume, for reductio, that of the $y'_1, ..., y'_n$, there is no $y'_i$ such that $y \equiv y'_i$.
Then, by the definition of $\equiv$, there must be formulae $\phi_1, ..., \phi_n$ such that for all $i$ in $1$ to $n$:
\[
y'_i \models \phi_i \mbox{ and } y \nvDash \phi_i
\]
Now consider the formula:
\[
[s] (\phi_1 \lor ... \lor \phi_n)
\]
As each $y'_i \models \phi_i$, $x' \models [s] (\phi_1 \lor ... \lor \phi_n)$, but $x$ does not satisfy this formula, as each $\phi_i$ is not satisfied at $y$.
Since there is a formula which $x$ and $x'$ do not agree on, $x$ and $x'$ are not equivalent, contradicting our initial assumption.

But this proof cannot be used in EL because it relies on a formula $[s] (\phi_1 \lor ... \lor \phi_n)$ which cannot be expressed in EL. 
EL does not include the box operator or disjunction, so this formula is ruled out on two accounts.
But we can massage it into a form which is more amenable to EL's expressive resources:
\begin{eqnarray*}
[s] (\phi_1 \lor ... \lor \phi_n) & = & \neg \langle s \rangle \neg (\phi_1 \lor ... \lor \phi_n)  \\
	& = & \neg \langle s \rangle (\neg \phi_1\AND ... \AND \neg \phi_n) 
\end{eqnarray*}
Further, if the original formula $[s] (\phi_1 \lor ... \lor \phi_n)$ is true in $x'$ but not in $x$, then its negation will be true in $x$ but not in $x'$. 
So we have the following formula, true in $x$ but not in $x'$:
\[
 \langle s \rangle (\neg \phi_1\AND ... \AND \neg \phi_n)
 \]
The reason for massaging the formula in this way is so we can express it in EL (which does not have the box operator or disjunction).
At this moment, the revised formula is \emph{still} outside EL because it uses negation. 
But we are almost there: the remaining negation is in innermost scope, and innermost scope negation can be simulated in EL by the $!$ operator. 

We are assuming, for reductio, that of the $y'_1, ..., y'_n$, there is no $y'_i$ such that $y \equiv y'_i$.
But in EL without negation, we cannot assume that each $y'_i$ has a formula $\phi_i$ which is satisfied by $y'_i$ but not by $y$ - it might instead be the other way round: $\phi_i$ may be satisfied by $y$ but not by $y'_i$. So, without loss of generality, assume that $y'_1, ..., y'_m$ fail to satisfy formulae $\phi_1, ..., \phi_m$ which $y$ does satisfy, and that $y'_{m+1}, ..., y'_n$ satisfy formulae $\phi_{m+1}, ..., \phi_n$ which $y$ does not:
\begin{eqnarray*}
y \models \phi_i \mbox{ and } y'_i \nvDash \phi_i & & i = 1 \mbox{ to } m  \\
y \nvDash \phi_j \mbox{ and } y'_j \models \phi_j & & j = m+1 \mbox{ to } n 
\end{eqnarray*}
The formula we will use to distinguish between $x$ and $x'$ is:
\[
 \langle s \rangle ( \bigwedge_{i=1}^m \phi_i \; \AND \; \bigwedge_{j=m+1}^n \mathsf{neg}(y, \phi_j))
 \]
 Here, $\mathsf{neg}$ is a meta-language function that, given a state y and a formula $\phi_j$, returns a formula that is true in $y$ but incompatible with $\phi_j$. I will show that, since $y \nvDash \phi_j$, it is always possible to construct $ \mathsf{neg}(y, \phi_j)$ using the $!$ operator.

Consider the possible forms of $\phi_j$:
\begin{itemize}
\item
$\top$: this case cannot occur since all models satisfy $\top$.
\item
$\phi_1 \AND \phi_2$: we know $y'_j \models \phi_1 \AND \phi_2$ and $y \nvDash \phi_1 \AND \phi_2$. There are three possibilities:
\begin{enumerate}
\item
$y \nvDash \phi_1$ and $y \models \phi_2$. In this case, $\mathsf{neg}(y, \phi_1 \AND \phi_2) = \mathsf{neg}(y, \phi_1) \AND \phi_2$.
\item
$y \models \phi_1$ and $y \nvDash \phi_2$. In this case, $\mathsf{neg}(y, \phi_1 \AND \phi_2) = \phi_1 \AND \mathsf{neg}(y, \phi_2)$.
\item
$y \nvDash \phi_1$ and $y \nvDash \phi_2$. In this case, $\mathsf{neg}(y, \phi_1 \AND \phi_2) =  \mathsf{neg}(y, \phi_1) \AND \mathsf{neg}(y, \phi_2)$.
\end{enumerate}
\item
$!A$: if $y \nvDash !A \mbox{ and } y'_j \models !A$, then there is a symbol $a \in \mathcal{S}-A$ such that $y \xrightarrow{a} z$ for some $z$ but there is no such $z$ such that $y'_j \xrightarrow{a} z$. In this case, let $\mathsf{neg}(y, \phi_j) = \langle a \rangle \top$.
\item
$\langle a \rangle \phi$. There are two possibilities:
\begin{enumerate}
\item
$y \models \langle a \rangle \top$. In this case, $\mathsf{neg}(y, \langle a \rangle \phi) =  \bigwedge\limits_{y \xrightarrow{a} z}  \langle a \rangle \mathsf{neg}(z, \phi)$.
\item
$y \nvDash \langle a \rangle \top$. In this case, $\mathsf{neg}(y, \langle a \rangle \phi) = \fBang \{ b \fOr \exists z. y \xrightarrow{b} z\}$. This set of $b$s is finite since we are assuming the LTS is finitely-branching.
\end{enumerate}
\end{itemize}
\qed
\end{proof}

\begin{figure}[h]
\centering
\begin{tikzpicture}[node distance=1.3cm,>=stealth',bend angle=45,auto]
  \tikzstyle{place}=[circle,thick,draw=blue!75,fill=blue!20,minimum size=6mm]
  \tikzstyle{red place}=[place,draw=red!75,fill=red!20]
  \tikzstyle{transition}=[rectangle,thick,draw=black!75,
  			  fill=black!20,minimum size=4mm]
  \tikzstyle{every label}=[red]

  \begin{scope}[xshift=0cm]
    \node [place] (t) {$y$};
    \node [place] (a) [below left of=t] {$z_1$}
      edge [pre]  node[swap] {a}                 (t);
    \node [place] (a2) [below right of=t] {$z_2$}
      edge [pre]  node[swap] {a}                 (t);
    \node [place] (b) [below of=a] {$w_1$}
      edge [pre]  node[swap] {b}                 (a);
    \node [place] (c) [below of=a2] {$w_2$}
      edge [pre]  node[swap] {c}                 (a2);
  \end{scope}  
  \begin{scope}[xshift=6cm]
    \node [place] (t) {$y'_j$};
    \node [place] (a) [below of=t] {$z'$}
      edge [pre]  node[swap] {a}                 (t);
    \node [place] (b) [below left of=a] {$w'_1$}
      edge [pre]  node[swap] {b}                 (a);
    \node [place] (c) [below right of=a] {$w'_2$}
      edge [pre]  node[swap] {c}                 (a);
  \end{scope}  
\end{tikzpicture}
\caption{{\small Worked Example of $\mathsf{neg}$. Note that the transition
  system on the left is non-deterministic.}}\label{figure:example:neg}
\end{figure}



\NI We continue with a worked example of $\mathsf{neg}$.  Consider
  $y$ and $y'_j$ as in Figure \ref{figure:example:neg}.  One formula
  that is true in $y'_j$ but not in $y$ is

\[
   \langle a \rangle (\langle b \rangle \top \AND \langle c \rangle \top)
\]

\NI Now:

\begin{eqnarray*}
\lefteqn{\mathsf{neg}(y, \langle a \rangle (\langle b \rangle \top \AND \langle c \rangle \top))}\qquad \qquad \qquad  \\
& = & \bigwedge\limits_{y \xrightarrow{a} z} \langle a \rangle \mathsf{neg}(z, \langle b \rangle \top \AND \langle c \rangle \top)  \\
& = & \langle a \rangle \mathsf{neg}(z_1, \langle b \rangle \top \AND \langle c \rangle \top) \AND \langle a \rangle\mathsf{neg}(z_2, \langle b \rangle \top \AND \langle c \rangle \top)  \\
& = & \langle a \rangle (\langle b \rangle \top \AND \mathsf{neg}(z_1, \langle c \rangle \top)) \AND \langle a \rangle\mathsf{neg}(z_2, \langle b \rangle \top \AND \langle c \rangle \top)  \\
& = & \langle a \rangle (\langle b \rangle \top \AND \mathsf{neg}(z_1, \langle c \rangle \top)) \AND \langle a \rangle(\mathsf{neg}(z_2, \langle b \rangle \top) \AND \langle c \rangle \top)  \\
& = & \langle a \rangle (\langle b \rangle \top \AND \fBang \{b\}) \AND \langle a \rangle(\mathsf{neg}(z_2, \langle b \rangle \top) \AND \langle c \rangle \top)  \\
& = & \langle a \rangle (\langle b \rangle \top \AND \fBang \{b\}) \AND \langle a \rangle(\fBang \{c\} \AND \langle c \rangle \top) 
\end{eqnarray*}

\NI The resulting formula is true in $y$ but not in $y'_j$.


\subsection{Comparing EL with Hennessy-Milner Logic}

Section \ref{standardTranslation} investigated the relationship
between eremic and first-order logic. Now we compare eremic logic with
a logic that is much closer in spirit: Hennessy-Milner logic
\cite{HennessyM:alglawfndac}, a modal logic designed to reason about
process calculi. Indeed the present shape of eremic logic owes to
Hennessy-Milner logic. We now contrast both by translation from eremic
logic into Hennessy-Milner logic \martin{why not the other way?}. The
this will reveal the more clearly than the translation into
first-order logic the where eremic logic is novel.

Given a set $\mathcal{S}$ of symbols, with $s$ ranging over
$\mathcal{S}$, the \emph{formulae} of Hennessy-Milner logic are given
by the following grammar:
\begin{GRAMMAR}
  \phi 
     &\quad ::= \quad & 
  \top \fOr \phi_1 \AND \phi_2 \fOr \langle s \rangle \phi \fOr \neg \phi 
\end{GRAMMAR}

\NI Note that many presentations use infinite conjunction. However,
that's not required in the present context.

 \emph{Models} of Hennessy-Milner logic are the pairs $(W,
\rightarrow, w)$ where $(S, \rightarrow)$ is a labelled transition
system over $S$, and $w \in W$.  The \emph{satisfaction relation} $(l,
w) \models \phi$ is given by the following inductive clauses.

\[
\begin{array}{lclcl}
  (l,w) & \models & \top  \\
  (l,w) & \models & \phi_1 \AND \phi_2 &\  \mbox{ iff }\  & (l,w)  \models \phi_1 \mbox { and } (l,w) \models \phi_2  \\
  (l,w) & \models & \langle s \rangle \phi & \mbox{ iff } & \mbox{ there is a } w \xrightarrow{s} w' \mbox { such that } (l,w') \models \phi  \\
  (l,w) & \models & \neg \phi & \mbox{ iff } & (l,w)  \nvDash \phi 
\end{array}
\]

\NI We can now see the two differences between eremic logic, and
Hennessy-Milner logic, one syntacticm, the other semantic.

\begin{itemize}

\item Syntactically, EL has the transition-restriction operator ($!$)
  instead of logical negation ($\neg$).

\item Semantically, an EL model has an additional state-labelling,
  describing the allowable transitions from each vertex.

\end{itemize}

\NI We translate formulae of EL into HML using the function $t$:\martin{maybe use $\SEMB{X}$ instead?}

\begin{eqnarray*}
  t(\top) & \ = \ & \top  \\
  t(\phi_1 \AND \phi_2) & \ = \ & t(\phi_1) \AND t(\phi_2)  \\
  t(\langle s \rangle \phi) & \ = \ & \langle s \rangle t(\phi)  \\
  t(! A) & \ = \ & \bigwedge_{s \in S - A} \neg \langle s \rangle \top 
\end{eqnarray*}

\NI If $\mathcal{S}$ is an infinite set, then the translation of a $!$
formula will be an infinitary formula.  If $\mathcal{S}$ is finite,
then the size\martin{Define the size of a formula. Maybe when formulae
  are defined.} of the HML formula will be of the order of $n *
|\mathcal{S}|$ larger than the original EL formula (where $n$ is the
number of $!$ operators occurring in the EL formula).

\martin{Say something more interesting about this translation!}

We can now see why in Hennessy-Milner logic without negation $p \equiv
p'$ does \emph{not} entail $p \BISIM p'$

\begin{GRAMMAR}
  \phi 
    &\quad ::= \quad&
  \top \fOr \phi_1 \AND \phi_2  \fOr \langle a \rangle \phi
\end{GRAMMAR}

\begin{figure}[h]
\centering
\begin{tikzpicture}[node distance=1.3cm,>=stealth',bend angle=45,auto]
  \tikzstyle{place}=[circle,thick,draw=blue!75,fill=blue!20,minimum size=6mm]
  \tikzstyle{red place}=[place,draw=red!75,fill=red!20]
  \tikzstyle{transition}=[rectangle,thick,draw=black!75,
  			  fill=black!20,minimum size=4mm]
  \tikzstyle{every label}=[red]

  \begin{scope}[xshift=0cm]
    \node [place] (w1) {$x$};
    \node [place] (e1) [below left of=w1] {$y_1$}
      edge [pre]  node[swap] {a}                 (w1);
    \node [place] (c) [below of=e1] {$z_1$}
      edge [pre]  node[swap] {b}                 (e1);
    \node [place] (e2) [below right of=w1] {$y_2$}
      edge [pre]  node[swap] {a}                 (w1);
    \node [place] (c) [below of=e2] {$z_2$}
      edge [pre]  node[swap] {b}                 (e2);
    \node [place] (c) [below right of=e2] {$z_3$}
      edge [pre]  node[swap] {c}                 (e2);
  \end{scope}  
  \begin{scope}[xshift=6cm]
    \node [place] (t) {$x'$};
    \node [place] (a) [below of=t] {$y'$}
      edge [pre]  node[swap] {a}                 (t);
    \node [place] (b) [below left of=a] {$z'_1$}
      edge [pre]  node[swap] {b}                 (a);
    \node [place] (c) [below right of=a] {$z'_2$}
      edge [pre]  node[swap] {c}                 (a);
  \end{scope}  
\end{tikzpicture}
\caption{Without negation, $p \equiv p'$ does not entail $p \sim p'$}\label{figure:hml}
\end{figure}


\NI Consider Figure \ref{figure:hml}.  Here, $x \equiv x'$ - they both
satisfy the sub formulae of $\langle a \rangle (\langle b \rangle \top
\AND \langle c \rangle \top)$.  But there is no bisimulation between
$x$ and $x'$ ($y_1$ cannot be matched with $y'$ because $y'$ has an
additional outgoing $c$-transition).

If we remove negation, HML is insufficiently expressive to distinguish
between these two non-bisimilar models.  When we restore negation, we
can distinguish between these models via the formula:

\[
   \langle a \rangle \neg \langle c \rangle \top
\]

\NI Similarly, EL is able to distinguish between these models via the
formula:

\[
  \langle a \rangle ! \{b\}
\]

\NI EL, also, has the expressive capacity to distinguish between
non-bisimilar models. If the LTSs are finitely-branching, it can
simulate negation (in innermost scope) via the width restriction
operator !
