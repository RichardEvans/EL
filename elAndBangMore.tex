\section{Further properties of \ELFULL{}}

In Section \ref{coreEL} we presented \ELABR{}
and established the key properties of completeness and compactness.
This section investigates
the logic further: first we give a linear-time decision
procedure. Then we give a logical characterisation of bisimilarity,
following ideas from the theory of process calculi
\cite{HennessyM:alglawfndac}.  The linear-time decision complexity is
an indication that \ELABR{} can be useful as a query language in
knowledge representation. Section \ref{kr} investigates this matter in
more detail. The logical characterisation of a semantical equivalence
on the other hand is an indication that our syntactic and semantical
constructs cohere with each other.

\subsection{Decision Procedure}

\NI Since \ELABR{} has no connectives for disjunction or implication, its
decision procedure is straightforward and efficient.  Although there
are an infinite number of models which satisfy any expression, the
satisfying models form a lattice with a least upper bound.  The $\SIMPL$
function defined above gives us the minimal model satisfying an
expression.  Using this least upper bound, we can calculate entailment
by checking a \emph{single model}.  To decide whether $p \models q$,
we use the following theorem:

\begin{theorem}\label{theorem:decision}
  The following are equivalent:
  \begin{enumerate}
    \item\label{theorem:decision:1} For all models $\MMM$,  $\MMM \models \phi \Rightarrow \MMM \models \psi$.
    \item\label{theorem:decision:2} $\SIMPL{\phi} \models \psi$.
  \end{enumerate}
\end{theorem}

\begin{proof}
We first show the implication from (\ref{theorem:decision:1}) to
(\ref{theorem:decision:2}), and from (\ref{theorem:decision:2}) to
(\ref{theorem:decision:1}).  \setcounter{mycase}{0}
\begin{mycase}
If $\forall \MMM \; \MMM \models \phi \Rightarrow \MMM \models \psi \; \text{then} \; \SIMPL{\phi} \models \psi$
\end{mycase}

\NI Given that $\SIMPL{\phi} \models \phi$, we substitute $\SIMPL{\phi}$ for $\MMM $ in
$\forall \MMM \; \MMM \models \phi \Rightarrow \MMM \models \psi$, to infer $\SIMPL{\phi}
\models \psi$.

\begin{mycase}
If $\SIMPL{\phi} \models \psi \; \text{then} \; \forall \MMM \; \MMM \models \phi \Rightarrow \MMM \models \psi$
\end{mycase}

\NI Assume $\MMM \models \phi$. We need to show $\MMM \models \psi$.  Now if $\MMM
\models \phi$ then $\MMM \MODELLEQ \SIMPL{\phi}$ (by definition of $\SIMPL$).  Further, if
$\MMM' \models \xi $ and $\MMM \MODELLEQ \MMM'$ then $\MMM \models \xi $. So, substituting $\psi$
for $\xi $ and $\SIMPL{\phi}$ for $\MMM'$, it follows that $\MMM \models \psi$.  
\end{proof}

\NI Given this theorem, the decision procedure is straightforward: to
test if $\phi \models \psi$, we construct $\SIMPL{\phi}$, and then inspect whether
$\SIMPL{\phi} \models \psi$.  Construction of $\SIMPL{\phi}$ is linear in the size of
$\phi$, and computing whether a model satisfies $\psi$ is linear in the size
of $\psi$, so computing whether $\phi \models \psi$ is $O(|\phi|+|\psi|)$.

\martin{Can we say more about the decision procedure? Example? Talk
  about Haskell code?}

