\section{Inference Rules}\label{elAndBangMore}

\begin{FIGURE}
\begin{RULES}

  \ZEROPREMISERULENAMEDRIGHT
  {
    \phi \judge \phi
  }{Identity}
    \quad
  \ZEROPREMISERULENAMEDRIGHT
  {
    \phi \judge \top
  }{$\top$-Right}
    \quad
  \ZEROPREMISERULENAMEDRIGHT
  {
    \bot \judge \phi
  }{$\bot$-Left}
    \quad
  \TWOPREMISERULENAMEDRIGHT
  {
    \phi \judge \psi
  }
  {
    \psi \judge \xi
  }
  {
    \phi \judge \xi
  }{Transitivity}
    \\\\
  \ONEPREMISERULENAMEDRIGHT
  {
    \phi \judge \psi
  }
  {
    \phi \AND \xi \judge \psi
  }{$\AND$-Left 1}
     \quad
  \ONEPREMISERULENAMEDRIGHT
  {
    \phi \judge \psi
  }
  {
    \xi \AND \phi  \judge \psi
  }{$\AND$-Left 2}
     \quad
  \TWOPREMISERULENAMEDRIGHT
  {
    \phi \judge \psi
  }
  {
    \phi \judge \xi
  }
  {
    \phi \judge \psi \AND \xi
  }{$\AND$-Right}
     \\\\
     \ONEPREMISERULENAMEDRIGHT
     {
       a \notin A
     }
     {
       !A \AND \MAY{a}{\phi} \judge \bot
     }{$\bot$-Right 1}
        \quad
     \ZEROPREMISERULENAMEDRIGHT
     {
       \MAY{a}{\bot} \judge \bot
     }{$\bot$-Right 2}
        \quad
     \TWOPREMISERULENAMEDRIGHT
     {
       \phi \AND \, !A \judge \psi
     }
     {
       A' \subseteq A
     }
     {
       \phi \AND\, !A' \judge \psi
     }{!-Left}
     \\\\
     \TWOPREMISERULENAMEDRIGHT
     {
       \phi \judge !A
     }
     {
       A \subseteq A'
     }
     {
       \phi \judge!A'
     }{!-Right 1}
     \quad
     \TWOPREMISERULENAMEDRIGHT
     {
       \phi \judge !A
     }
     {
       \phi \judge !B
     }
     {
       \phi \judge !(A \cap B)
     }{!-Right 2}
     \quad
     \ONEPREMISERULENAMEDRIGHT
     {
       \phi \judge \psi
     }
     {
       \MAY{a}{\phi} \judge \MAY{a}{\psi}
     }{Transition Normal}
\end{RULES}
\caption{Proof rules.}\label{figure:elAndBangRules}
\end{FIGURE}


\NI We now present the inference rules for \cathoristic{}. There are no
axioms.

\begin{definition} Judgements are of the following form.
\[
  \phi \judge \psi.
\]
We also write $\judge \phi$ as a shorthand for $\top \judge
\phi$. Figure \ref{figure:elAndBangRules}
presents all proof rules.
\end{definition}

\NI Note that $\phi$ and $\psi$ are single formulae, not sequents.  By
using single formulae, we can avoid structural inference rules.  The
proof rules can be grouped in two parts: standard rules and rules
unique to \cathoristic{}.  Standard rules are [\RULENAME{Id}],
[\RULENAME{$\top$-Right}], [\RULENAME{$\bot$-Left}],
[\RULENAME{Trans}], [\RULENAME{$\AND$-Left 1}],
[\RULENAME{$\AND$-Left 2}] and [\RULENAME{$\AND$-Right}]. They hardly need
explanation as they are variants of familiar rules for propositional
logic, see e.g.~\cite{TroelstraAS:basprot,vanDalenD:logstr}.  We now
explain the rules that give \cathoristic{} its distinctive properties.

The rule [\RULENAME{$\bot$-Right 1}]  captures the core
exclusion property of the tantum !: for example if $A = \{male, female\}$
then $\MAY{orange}{\phi}$ is incompatible with $!A$. Thus $!A \AND
\MAY{orange}{\phi}$ must be false.

The rule [\RULENAME{$\bot$-Right 2}] expresses that falsity is `global'
  and cannot be surpressed by the modalities. For example
  $\MAY{orange}{\bot}$ is false, simply because $\bot$ is already
  false.

[\RULENAME{Normal}] enables us to prefix an inference with a
may-modality.  This rule can also be stated in the the following more
general form:
\[
   \ONEPREMISERULENAMEDRIGHT
   {
     \phi_1\AND ...\AND \phi_n \judge \psi
   }
   {
     \MAY{a}{\phi_1}\AND ...\AND \MAY{a}{\phi_n} \judge \MAY{a}{\psi}
   }{Normal-Multi}
\]

\NI But it is not necessary because [\RULENAME{Normal-Multi}] is
derivable from [\RULENAME{Normal}] as we show in the examples below.

\subsection{Example inferences}

\NI We prove that we can use $\phi \AND \psi \judge \xi$ to derive
$\MAY{a} \phi \land \MAY{a} \psi \judge \MAY{a} \xi$:

\begin{center}
  \AxiomC{$\phi \AND \psi  \judge  \xi$}
  \LeftLabel{\RULENAME{Normal}}
  \UnaryInfC{$\MAY{a} (\phi \AND \psi)  \judge  \MAY{a} \xi$}
  \AxiomC{$\MAY{a} \phi \land \MAY{a} \psi \judge \MAY{a} \phi \land \MAY{a} \psi$}
  \LeftLabel{\RULENAME{Det}}
  \UnaryInfC{$\MAY{a} \phi \land \MAY{a} \psi \judge \MAY{a} (\phi \land \psi)$}
  \LeftLabel{\RULENAME{Trans}}
  \BinaryInfC{$\MAY{a} \phi \land \MAY{a} \psi \judge \MAY{a} \xi$}
  \DisplayProof
\end{center}


\NI Figure \ref{figure:elAndBangMore:bigDerivations} demonstrates how
to infer $\MAY{a}!\{b,c\} \land \MAY{a}!\{c,d\} \judge \MAY{a}!\{c\}$
and $\MAY{a} !\{b\} \land \MAY{a} \MAY{c} \top \judge \MAY{d} \top$.

\subsection{\RULENAME{!-Left} and \RULENAME{!-Right}}

The rules [\RULENAME{!-Right 1}, \RULENAME{!-Right 2}] jointly express
how the subset relation $\subseteq$ on sets of actions relates to
provability. Why  don't we need a corresponding rule \RULENAME{!-Left} for
strengthening $!$ on the left hand side?
\[
   \TWOPREMISERULENAMEDRIGHT
     {
       \phi \AND \, !A \judge \psi
     }
     {
       A' \subseteq A
     }
     {
       \phi \AND\, !A' \judge \psi
     }{!-Left}
\]
The reason is that [\RULENAME{!-Left}] can be derived as follows.
\begin{center}
  \AxiomC{$\phi \AND\, !A'  \judge  \phi \AND\, !A'$}
  \AxiomC{$A' \subseteq A$}
  \LeftLabel{\small\RULENAME{!-Right 1}}
  \BinaryInfC{$\phi \AND\, !A'  \judge  \phi \AND\, !A$}
  \AxiomC{$\phi \AND\, !A  \judge  \psi$}
  \LeftLabel{\small\RULENAME{Trans}}
  \BinaryInfC{$\phi \AND\, !A'  \judge  \psi$}
  \DisplayProof
\end{center}

\NI Readers familiar with object-oriented programming will recognise
[\RULENAME{!-Left}] as contra-variant and [\RULENAME{!-Right 1}]
as covariant subtyping. Honda \cite{HondaK:thetypftpc} develops a full
theory of subtyping based on similar ideas.  All three rules embody
the intuition that whenever $A \subseteq A'$ then asserting that $!A'$
is as strong as, or a stronger statement than
$!A$. [\RULENAME{!-Left}] simply states that we can always strengthen
our premise, while [\RULENAME{!-right 1}] allows us to weaken the
conclusion.

\subsection{Characteristic formulae}

In order to prove completeness, below, we need the notion of a
\emph{characteristic formula} of a model.  The function
$\SIMPL{\cdot}$ takes a formula as argument and returns
the least upper bound of the satisfying models. Characteristic formulae go the other way: given a model
$\MMM$, $\CHAR{\MMM}$ is the logically weakest formula that describes that model. 
\clearpage
\begin{SIDEWAYSFIGURE}

\begin{center}
  \AxiomC{$!\{b,c\} \judge !\{b,c\}$}
  \LeftLabel{\RULENAME{$\land$ Left 1}}
  \UnaryInfC{$!\{b,c\} \land !\{c,d\} \judge !\{b,c\}$}
  \AxiomC{$!\{c,d\} \judge !\{c,d\}$}
  \LeftLabel{\RULENAME{$\land$ Left 2}}
  \UnaryInfC{$!\{b,c\} \land !\{c,d\} \judge !\{c,d\}$}
  \LeftLabel{\RULENAME{! Right 2}}
  \BinaryInfC{$!\{b,c\} \land !\{c,d\} \judge !\{c\}$}
  \LeftLabel{\RULENAME{Normal}}
  \UnaryInfC{$\MAY{a} (!\{b,c\} \land !\{c,d\}) \judge \MAY{a} !\{c\}$}
  \AxiomC{$\MAY{a} !\{b,c\} \land \MAY{a} !\{c,d\} \judge \MAY{a} !\{b,c\} \land \MAY{a} !\{c,d\}$}
  \LeftLabel{\RULENAME{Det}}
  \UnaryInfC{$\MAY{a} !\{b,c\} \land \MAY{a} !\{c,d\} \judge \MAY{a} (!\{b,c\} \land !\{c,d\})$}
  \LeftLabel{\RULENAME{Trans}}
  \BinaryInfC{$\MAY{a} !\{b,c\} \land \MAY{a} !\{c,d\} \judge \MAY{a}!\{c\}$}
  \DisplayProof
\end{center}
\vspace{7mm}
\begin{center}
  \AxiomC{$!\{b\} \land \MAY{c}\top \judge \bot$}
  \LeftLabel{\RULENAME{Normal}}
  \UnaryInfC{$\MAY{a}(!\{b\} \land \MAY{c}\top) \judge \MAY{a} \bot$}
  \AxiomC{$\MAY{a}!\{b\} \land \MAY{a} \MAY{c}\top \judge \MAY{a}!\{b\} \land \MAY{a} \MAY{c}\top$}
  \LeftLabel{\RULENAME{Det}}
  \UnaryInfC{$\MAY{a}!\{b\} \land \MAY{a} \MAY{c}\top \judge \MAY{a}(!\{b\} \land \MAY{c}\top)$}
  \LeftLabel{\RULENAME{Trans}}
  \BinaryInfC{$\MAY{a}!\{b\} \land \MAY{a} \MAY{c}\top  \judge \MAY{a} \bot$}  
  \AxiomC{$\MAY{a}\bot \judge \bot$}
  \LeftLabel{\RULENAME{Trans}}
  \BinaryInfC{$\MAY{a}!\{b\} \land \MAY{a} \MAY{c}\top  \judge \bot$}  
  \AxiomC{$\bot \judge \MAY{d} \top$}
  \LeftLabel{\RULENAME{Trans}}
  \BinaryInfC{$\MAY{a}!\{b\} \land \MAY{a} \MAY{c}\top  \judge  \MAY{d} \top$}    
  \DisplayProof
\end{center}

\caption{Derivations of $\MAY{a}!\{b,c\} \land \MAY{a}!\{c,d\} \judge \MAY{a}!\{c\}$ (top) and
 $\MAY{a} !\{b\} \land \MAY{a} \MAY{c} \top \judge \MAY{d} \top$ (bottom).}\label{figure:elAndBangMore:bigDerivations}
\end{SIDEWAYSFIGURE}

\clearpage

\begin{definition}
Let $\MMM$ be an cathoristic model that is a tree.
\begin{eqnarray*}
  \CHAR{\bot} &\ =\ & \langle a \rangle \top \AND ! \emptyset  \mbox{ for some fixed action }a \in \Sigma  \\
  \CHAR{\MMM, w} & = & \mathsf{bang}(\MMM,w) \AND \bigwedge_{w \xrightarrow{a} w'} \langle a \rangle \CHAR{\MMM, w'}  
\end{eqnarray*}

\end{definition}

\NI Note that $\bot$ requires a particular action $a \in \Sigma$. This
is why we required, in Section \ref{elsyntax}, that $\Sigma$ is
non-empty.

The functions $\mathsf{bang}(\cdot)$ on models are given by the
following clauses.

\begin{eqnarray*}
  \mathsf{bang}((S,\rightarrow,\lambda),w) 
     & \ = \ & 
  \begin{cases}
    \top & \mbox{ if } \lambda(w) = \Sigma  \\
    ! \; \lambda(w) & \mbox{ otherwise }  
  \end{cases} \\
\end{eqnarray*}

\NI Note that $\CHAR{\MMM}$ is finite if $\MMM$ contains no cycles and
if $\lambda(x)$ is either $\Sigma$ or finite for all states $x$.  We
state without proof that $\SIMPL{\cdot}$ and $\CHAR{\cdot}$ are
inverses of each other (for tree models $\MMM$) in that:

\begin{itemize}

\item $\SIMPL{\CHAR{\MMM}} \ \MODELEQ \  \MMM$. 

\item $\models \CHAR{\SIMPL{\phi}}$ iff $\models\phi$.

\end{itemize}


\subsection{Soundness and completeness}

\begin{theorem}\label{theorem:elAndBang:soundComplete}
The rules in Figure \ref{figure:elAndBangRules} are sound and complete:
\begin{enumerate}

\item\label{theorem:elAndBang:sound} (Soundness) $\phi \judge \psi$ implies $\phi \models \psi$.

\item\label{theorem:elAndBang:complete} (Completeness) $\phi \models \psi$ implies $\phi \judge \psi$.

\end{enumerate}
\end{theorem}

\NI Soundness is immediate from the definitions. 
 To prove completeness  we will show that $\phi
\models \psi$ implies there is a derivation of $\phi \judge \psi$.  Our proof
will make use of two key facts (proved in Sections \ref{prooflemma4} and \ref{prooflemma5} below):

%% \begin{definition}
%% Let $\Gamma$ be an arbitrary set of eremic formulae. We write $\MMM
%% \models \Gamma$ provided $\MMM \models \phi$ for all $\phi \in
%% \Gamma$.  We write $\Gamma \models \phi$ if for all eremic models
%% $\MMM$ $\MMM \models \Gamma$ implies $\MMM \models \phi$.  We write
%% $\phi_1, ..., \phi_n \judge \psi$ whenever $\BIGAND_{i}\phi_i \judge
%% \psi$.  For arbitrary sets $\Gamma$ of formulae, we write $\Gamma
%% \judge \psi$ provided there are $\phi_1, ..., \phi_n \in \Gamma$ such
%% that $\phi_1, ..., \phi_n \judge \psi$.
%% \end{definition}

%% \begin{corollary}
%% The rules in Figure \ref{figure:elAndBangRules} are sound and complete:
%% \begin{enumerate}

%% \item(Soundness) $\Gamma \judge \psi$ implies $\Gamma \models \psi$.

%% \item (Completeness) $\Gamma \models \psi$ implies $\Gamma \judge \psi$. \martin{This is not yet proven.}

%% \end{enumerate}
%% \end{corollary}

%% \begin{proof}
%% For soundness assume that $\MMM$ is a model with $\MMM \models
%% \Gamma$.  We know that $\Gamma \judge \psi$ means that we can find a
%% finite number of formulae $\phi_i \in \Gamma$ with $\AND_i \phi \judge
%% \psi$.  By soundness then also $\AND_i \phi \models \psi$, hence
%% clearly $\Gamma \models \psi$.


\begin{lemma}\label{lemma:completeness:4}
If $\MMM \models \phi$ then $\CHAR{\MMM} \judge \phi$.
\end{lemma}

\begin{lemma}\label{lemma:completeness:5}
For all formulae $\phi$, we can derive $\phi \judge \CHAR{\SIMPL{\phi}}$.
\end{lemma}

\NI Lemma \ref{lemma:completeness:4} states that, if $\phi$ is
satisfied by a model, then there is a proof that the characteristic
formula describing that model entails $\phi$.  Lemma
\ref{lemma:completeness:5} says that $\SIMPL{\phi}$ is the simplest
model satisfying $\phi$, and $\CHAR{\MMM}$ is the simplest formula
describing $m$, so $\CHAR{\SIMPL{\phi}}$ is a simplified form of
$\phi$. This lemma states that \cathoristic\ has the inferential capacity to
transform any proposition into its simplified form.

With these two lemmas in hand, the proof of completeness is
straightforward.  Assume $\phi \models \psi$.  Then all models which satisfy
$\phi$ also satisfy $\psi$.  In particular, $\SIMPL{\phi} \models \psi$.  Then
$\CHAR{\SIMPL{\phi}} \judge \psi$ by Lemma \ref{lemma:completeness:4}.  But we
also have, by Lemma \ref{lemma:completeness:5}, $\phi \judge
\CHAR{\SIMPL{\phi}} $.  So by transitivity, we have $\phi \judge \psi$.  

\subsection{Proofs of Lemmas \ref{lemma:completeness:4}, \ref{lemma:completeness:5} and \ref{final_completeness_lemma}}

\subsubsection{Proof of Lemma \ref{lemma:completeness:4}}
\label{prooflemma4}
If $\MMM\models \phi$ then $\CHAR{\MMM} \judge \phi$.

\NI We proceed by induction on $\phi$.

%\setcounter{mycase}{0}

\CASE{$\phi$ is $\top$} Then we can prove $ \CHAR{\MMM} \judge \phi$
immediately using axiom [\RULENAME{$\top$ Right}.

\CASE{$\phi$ is $\psi \AND \psi'$} By the induction hypothesis, $
\CHAR{\MMM} \judge \psi$ and $ \CHAR{\MMM} \judge \psi'$.  The proof
of $ \CHAR{\MMM} \judge \psi \AND \psi'$ follows immediately using
[\RULENAME{$\AND$ Right}.

\CASE{$\phi$ is $\langle a \rangle \psi$}
If $\MMM \models \langle a \rangle \psi$, then either $\MMM = \bot$ or $\MMM$ is a  model of the form $(\CAL{L},w)$.

\SUBCASE{$\MMM = \bot$} In this case, $ \CHAR{\MMM} = \CHAR{\bot} =
\bot$. (Recall, that we are overloading $\bot$ to mean both the model
at the bottom of our lattice and a formula (such as $\langle s \rangle
\top \AND !\emptyset$) which is always false).  In this case, $ \CHAR{\bot}
\judge \langle a \rangle \psi$ using [\RULENAME{$\bot$ Left}.

\SUBCASE{$m$ is a model of the form $(\CAL{L},w)$} Given $\MMM \models
\langle a \rangle \psi$, and that $\MMM$ is a model of the form
$(\CAL{L},w)$, we know that:
\[
(\CAL{L},w) \models \langle a \rangle \psi
\]
From the satisfaction clause for $\langle a \rangle$, it follows that:
\[
\exists w' \mbox{ such that } w \xrightarrow{a} w' \mbox { and } (\CAL{L},w') \models \psi
\]
By the induction hypothesis:
\[
 \CHAR{(\CAL{L},w')} \judge \psi
\]
Now by [\RULENAME{Normal}]:
\[
\langle a \rangle  \CHAR{(\CAL{L},w')} \judge \langle a \rangle \psi
\]
Using repeated application of [\RULENAME{$\AND$ Left}], we can show:
\[
 \CHAR{(\CAL{L},w)} \judge \langle a \rangle  \CHAR{(\CAL{L},w')}
\]
Finally, using [\RULENAME{Trans}], we derive:
\[
 \CHAR{(\CAL{L},w)} \judge  \langle a \rangle \psi
\]

\CASE{$\phi$ is $\fBang \psi$} If $(\CAL{L},w) \models \fBang A$,
then $\lambda(w) \subseteq A$.  Then $ \CHAR{(\CAL{L},w)} = ! \;
\lambda(w) \AND \phi$.  Now we can prove $! \; \lambda(w) \AND \phi
\judge \fBang A$ using [\RULENAME{$!$ Right 1}] and repeated applications of
       [\RULENAME{$\AND$ Left}].

\subsubsection{Proof of Lemma \ref{lemma:completeness:5}}
\label{prooflemma5}

Now we prove Lemma \ref{lemma:completeness:5}: 
For all formulae $\phi$, we can derive $\phi \judge \CHAR{\SIMPL{\phi}}$.

\begin{proof}
Induction on $\phi$.

%\setcounter{mycase}{0}

\CASE{$\phi$ is $\top$} Then we can prove $\top \judge \top$ using
either [\RULENAME{$\top$ Right}] or [\RULENAME{Id}].

\CASE{$\phi$ is $\psi \AND \psi'$} By the induction hypothesis,
$\psi \judge \CHAR{\SIMPL{\psi}}$ and $\psi' \judge
\CHAR{\SIMPL{\psi'}}$.  Using [\RULENAME{$\AND$ Left}] and [\RULENAME{$\AND$
  Right}], we can show:
\[
\psi \AND \psi' \judge  \CHAR{\SIMPL{\psi}} \AND  \CHAR{\SIMPL{\psi'}}
\]


In order to continue the proof, we need the following lemma, proven
in the next subsection.

\begin{lemma}
\label{final_completeness_lemma}
For all cathoristic  models $\MMM$ and $\MMM_2$ that are trees, $ \CHAR{\MMM} \AND
\CHAR{\MMM_2} \judge \CHAR{\MMM \sqcap \MMM_2}$.
\end{lemma}

\NI From Lemma \ref{final_completeness_lemma} (substituting $\SIMPL{\psi}$ for $\MMM$ and $\SIMPL{\psi'}$ for $\MMM_2$, and noting that $\SIMPL{}$ always produces acyclic models), it follows that:
\[
 \CHAR{\SIMPL{\psi}} \AND  \CHAR{\SIMPL{\psi'}} \judge  \CHAR{\SIMPL{\psi \AND \psi'}}
\]
Our desired result follows using [\RULENAME{Trans}].

\CASE{$\phi$ is $\langle a \rangle \psi$} By the induction
hypothesis, $\psi \judge \CHAR{\SIMPL{\psi}}$.  Now there are two
sub-cases to consider, depending on whether or not $
\CHAR{\SIMPL{\psi}} = \bot$.  

\SUBCASE{$ \CHAR{\SIMPL{\psi}} = \bot$} In this case, $
\CHAR{\SIMPL{\langle a \rangle \psi}}$ also equals $\bot$.  By the
induction hypothesis:
\[
\psi \judge \bot
\]
By [\RULENAME{Normal}]:
\[
\langle a \rangle \psi \judge \langle a \rangle \bot
\]
By [\RULENAME{$\bot$ Right 2}]:
\[
\langle a \rangle \bot \judge \bot
\]
The desired proof that:
\[
\langle a \rangle \psi \judge \bot
\]
follows by [\RULENAME{Trans}].

\SUBCASE{$ \CHAR{\SIMPL{\psi}} \neq \bot$}
By the induction hypothesis, $\psi \judge  \CHAR{\SIMPL{\psi}}$.
So, by [\RULENAME{Normal}]:
\[
\langle a \rangle \psi \judge \langle a \rangle  \CHAR{\SIMPL{\psi}}
\]
The desired conclusion follows from noting that:
\[
 \langle a \rangle  \CHAR{\SIMPL{\psi}} =  \CHAR{\SIMPL{\langle a \rangle \psi}}
 \]

 \CASE{$\phi$ is $\fBang A$} If $\phi$ is $\fBang A$, then $
 \CHAR{\SIMPL{\phi}}$ is $\fBang A \AND \top$.  We can prove $\fBang A
 \judge \fBang A \AND \top$ using [\RULENAME{$\AND$ Right}], [\RULENAME{$\top$
   Right}] and [\RULENAME{Id}].
\end{proof}

\subsubsection{Proof of Lemma \ref{final_completeness_lemma}}
\label{prooflemma6}

We can now finish the proof of Lemma \ref{lemma:completeness:5} by
giving the missing proof of Lemma \ref{final_completeness_lemma}.

\begin{proof}
There are two cases to consider, depending on whether or not $(\MMM
\sqcap \MMM_2) = \bot$.

%\setcounter{mycase}{0}

\CASE{$(\MMM \sqcap \MMM_2) = \bot$}
If $(\MMM \sqcap \MMM_2) = \bot$, there are three possibilities:
\begin{itemize}
\item
$\MMM = \bot$
\item
$\MMM_2 = \bot$
\item
Neither $\MMM$ nor $\MMM_2$ are $\bot$, but together they are incompatible. 
\end{itemize}
If either $\MMM$ or $\MMM_2$ is $\bot$, then the proof is a simple application of [\RULENAME{Id}] followed by [\RULENAME{$\AND$ Left}].

Next, let us consider the case where neither $\MMM$ nor $\MMM_2$ are $\bot$, but together they are incompatible.
Let $\MMM = (\mathcal{L}, w_1)$ and $\MMM' = (\mathcal{L}', w'_1)$.
If $\MMM \sqcap \MMM_2 = \bot$, then there is a finite sequence of actions $a_1, ..., a_{n-1}$ such that both $\MMM$ and $\MMM'$ satisfy $\MAY{a_1} ... \MAY{a_{n-1}}\top$, but they disagree about the state-labelling on the final state of this chain. In other words, there is a $b$-transition from the final state in $\MMM$ which is ruled-out by the $\lambda'$ state-labelling in $\MMM'$. So there is a set of states $w_1, ..., w'_1, ...$ and a finite set $X$ of actions such that:
\begin{eqnarray*}
w_1 \xrightarrow{a_1} w_2 \xrightarrow{a_2} ... \xrightarrow{a_{n-1}} w_n \\
w_1' \xrightarrow{a_1} w'_2 \xrightarrow{a_2} ... \xrightarrow{a_{n-1}} w'_n \\
w_n \xrightarrow{b} w_{n+1} \\
\lambda'(w'_n) = X \text{ with } b \notin X
\end{eqnarray*}
Now it is easy to show, using [\RULENAME{$\AND$ Left}], that
\begin{eqnarray*}
\CHAR{\mathsf{\MMM}} \judge \MAY{a_1} ... \MAY{a_{n-1}} \MAY{b} \top \\
\CHAR{\mathsf{\MMM'}} \judge \MAY{a_1} ... \MAY{a_{n-1}} \fBang X
\end{eqnarray*}
Now using [\RULENAME{$\AND$ Left}] and [\RULENAME{$\AND$ Right}]:
\[
\CHAR{\mathsf{\MMM}} \land \CHAR{\mathsf{\MMM}'} \judge  \MAY{a_1} ... \MAY{a_{n-1}} \MAY{b} \top \land  \judge \MAY{a_1} ... \MAY{a_{n-1}} \fBang X
\]
Now using [\RULENAME{Det}]:
\[
\CHAR{\mathsf{\MMM}} \land \CHAR{\mathsf{\MMM}'} \judge  \MAY{a_1} ... \MAY{a_{n-1}} (\MAY{b} \top \land \fBang X)
\]
Now, using [\RULENAME{$\bot$ Right 1}]:
\[
\MAY{b} \top \land \fBang X \judge \bot
\]
Using $n-1$ applications of  [\RULENAME{$\bot$ Right 2}]:
\[
\MAY{a_1} ... \MAY{a_{n-1}} (\MAY{b} \top \land \fBang X) \judge \bot
\]
Finally, using [\RULENAME{Trans}], we derive:
\[
\CHAR{\mathsf{\MMM}} \land \CHAR{\mathsf{\MMM}'} \judge \bot
\]
\CASE{$(\MMM \sqcap \MMM_2) \neq \bot$} From the construction of
$\mathsf{merge}$, if $\MMM$ and $\MMM'$ are acyclic, then $\MMM \sqcap
\MMM'$ is also acyclic.  If $\MMM \sqcap \MMM'$ is acyclic, then
$\CHAR{\MMM \sqcap \MMM'}$ is equivalent to a set $\Gamma$ of
sentences of one of two forms:
\begin{eqnarray*}
\MAY{a_1} ... \MAY{a_n} \top \\
\MAY{a_1} ... \MAY{a_n} ! X
\end{eqnarray*}

\begin{figure}[H]
\centering
\begin{tikzpicture}[node distance=1.3cm,>=stealth',bend angle=45,auto]
  \tikzstyle{place}=[circle,thick,draw=blue!75,fill=blue!20,minimum size=6mm]
  \tikzstyle{red place}=[place,draw=red!75,fill=red!20]
  \tikzstyle{transition}=[rectangle,thick,draw=black!75,
  			  fill=black!20,minimum size=4mm]
  \tikzstyle{every label}=[red]
    
  \begin{scope}
    \node [place] (w1) {$\Sigma$};
    \node [place] (e1) [below left of=w1] {$\{c,d\}$}
      edge [pre]  node[swap] {a}                 (w1);      
    \node [place] (e2) [below right of=w1] {$\Sigma$}
      edge [pre]  node[swap] {b}                 (w1);      
    \node [place] (e3) [below of=e1] {$\Sigma$}
      edge [pre]  node[swap] {c}                 (e1);      
  \end{scope}
    
\end{tikzpicture}
\caption{Example of $\sqcap$}
\label{setofpaths}
\end{figure}
For example, if $\MMM \sqcap \MMM'$ is as in Figure \ref{setofpaths}, then 
\[
\CHAR{\MMM \sqcap \MMM'} = \MAY{a}(\fBang \{c,d\} \land \MAY{c} \top) \land \MAY{b} \top
\]
This is equivalent to the set $\Gamma$ of sentences:
\begin{eqnarray*}
\MAY{a}\MAY{c} \top \\
\MAY{b} \top \\
\MAY{a}\fBang\{c,d\}
\end{eqnarray*}
Now using [\RULENAME{$\AND$ Right}] and [\RULENAME{Det}] we can show that
\[
\bigwedge_{\phi \in \Gamma} \phi \judge \CHAR{\MMM \sqcap \MMM'}
\]
We know that for all $\phi \in \Gamma$
\[
\MMM \sqcap \MMM' \models \phi
\]
We just need to show that:
\[
\CHAR{\MMM} \land \CHAR{\MMM'} \judge \phi
\]
Take any $\phi \in \Gamma$ of the form $\MAY{a_1} ... \MAY{a_n} ! X$ for some finite $X \subseteq \Sigma$. (The case where $\phi$ is of the form $\MAY{a_1} ... \MAY{a_n} \top$ is very similar, but simpler).
If $\MMM \sqcap \MMM' \models \MAY{a_1} ... \MAY{a_n} ! X$ then either:
\begin{enumerate}
\item
$\MMM \models \MAY{a_1} ... \MAY{a_n} ! X$ but $\MMM' \nvDash \MAY{a_1} ... \MAY{a_n} \top$
\item
$\MMM' \models \MAY{a_1} ... \MAY{a_n} ! X$ but $\MMM \nvDash \MAY{a_1} ... \MAY{a_n} \top$
\item
$\MMM \models \MAY{a_1} ... \MAY{a_n} ! X_1$ and $\MMM' \models \MAY{a_1} ... \MAY{a_n} ! X_2$ and $X_1 \cap X_2 \subseteq X$
\end{enumerate}
In the first two cases, showing $\CHAR{\MMM} \land \CHAR{\MMM'} \judge \phi$ is just a matter of repeated application of   [\RULENAME{$\AND$ Left}] and [\RULENAME{$\AND$ Right}].
In the third case, let $\MMM = (\mathcal{L}, w_1)$ and $\MMM' = (\mathcal{L}', w'_1)$.
If $\MMM \models \MAY{a_1} ... \MAY{a_n} ! X_1$ and $\MMM' \models \MAY{a_1} ... \MAY{a_n} ! X_2$ then there exists sequences $w_1, ..., w_{n+1}$ and $w'_1, ..., w'_{n+1}$ of states such that
\begin{eqnarray*}
w_1 \xrightarrow{a_1} ... \xrightarrow{a_n} w_{n+1} \\
w'_1 \xrightarrow{a_1} ... \xrightarrow{a_n} w'_{n+1} \\
\lambda(w_{n+1}) \subseteq X_1 \\
\lambda'(w'_{n+1}) \subseteq X_2 \\
\end{eqnarray*}
Now from the definition of $\CHAR{}$:
\begin{eqnarray*}
\CHAR{(\mathcal{L}, w_{n_1})} \judge \fBang X_1 \\
\CHAR{(\mathcal{L}', w'_{n_1})} \judge \fBang X_2
\end{eqnarray*}
Now using [\RULENAME{\fBang Right 2}]:
\[
\CHAR{(\mathcal{L}, w_{n_1})} \land \CHAR{(\mathcal{L}', w'_{n_1})} \judge \fBang (X_1 \cap X_2)
\]
Using [\RULENAME{\fBang Right 1}]:
\[
\CHAR{(\mathcal{L}, w_{n_1})} \land \CHAR{(\mathcal{L}', w'_{n_1})} \judge \fBang X
\]
Using $n$ applications of [\RULENAME{Normal}]:
\[
\MAY{a_1} ... \MAY{a_n} (\CHAR{(\mathcal{L}, w_{n_1})} \land \CHAR{(\mathcal{L}', w'_{n_1})}) \judge \MAY{a_1} ... \MAY{a_n} \fBang X
\]
Finally, using $n$ applications of [\RULENAME{Det}]:
\[
\CHAR{ (\mathcal{L}, w_1)} \land \CHAR{ (\mathcal{L}', w'_1)} \judge \MAY{a_1} ... \MAY{a_n} (\CHAR{(\mathcal{L}, w_{n_1})} \land \CHAR{(\mathcal{L}', w'_{n_1})})
\]
So, by [\RULENAME{Trans]}
\[
\CHAR{\MMM} \land \CHAR{\MMM'} \judge \MAY{a_1} ... \MAY{a_n} \fBang X
\]
\end{proof}
