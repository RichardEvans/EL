\section{Core eremic logic: the EL$[\AND, !]$ fragment}

\subsection{Semantic Constructions}
The following semantic constructions are used throughout the proofs that follow.
\subsection{Simulations and Bisimulations}

A relation $Z \subseteq \mathcal{W} \times \mathcal{W}'$ is a (one-way) {\bf simulation} from  $(\mathcal{W}, \rightarrow, \lambda)$ to $(\mathcal{W}', \rightarrow', \lambda')$ iff for all $(x,x') \in Z$ and $s \in S$:
\begin{enumerate}
\item
If $x \xrightarrow{s} y$ then there exists a $y' \in \mathcal{W}'$ such that $(y,y') \in Z$ and $x' \xrightarrow{s} y'$
\item
$\lambda'(x') \subseteq \lambda(x)$
\end{enumerate}
A relation $Z \subseteq \mathcal{W} \times \mathcal{W}'$ is a {\bf bisimulation} between  $(\mathcal{W}, \rightarrow, \lambda)$ and $(\mathcal{W}', \rightarrow', \lambda')$ iff for all $(x,x') \in Z$ and $s \in S$:
\begin{enumerate}
\item
If $x \xrightarrow{s} y$ then there exists a $y' \in \mathcal{W}'$ such that $(y,y') \in Z$ and $x' \xrightarrow{s} y'$
\item
If $x' \xrightarrow{s} y'$ then there exists a $y \in \mathcal{W}$ such that $(y,y') \in Z$ and $x \xrightarrow{s} y$
\item
$\lambda(x) = \lambda'(x')$
\end{enumerate}

\subsection{A Partial Ordering on Pointed Models}
We use the notion of simulation to define a partial ordering $\leq$ on pointed models:
\begin{definition}
$(l,w) \leq (l',w')$ if there is a simulation $Z$ from $l'$ to $l$ with $(w',w) \in Z$
\end{definition}
Intuitively, $m \leq n$ if $m$ can match all the transitions of $n$ while respecting the transitions-restrictions.

To make our models into a lattice, we add a bottom element $\bot$ and stipulate that $\bot \leq m$ for all pointed models $m$.
The topmost element in the lattice is the pointed model $( (\{w\}, \{\}, \{w \mapsto \mathcal{S}\}), w)$ (for some state $w$): this is the model with no transitions and no transition restrictions.

\subsection{Defining $\mu$ - the Simplest Pointed Model Satisfying a Formula}
We define a function $\mu$ which, given a formula $\phi$, produces the simplest\footnote{``Simplest'' as in the least upper bound, according to $\leq$, defined in formulae of simulation.} model which satisfies $\phi$:
\begin{eqnarray}
\mu (\top) & = & ( (\{v\}, \{\}, \{v \mapsto \mathcal{S}\}), v) \nonumber \\
\mu (\fBang A) & = & ( (\{v\}, \{\}, \{v \mapsto A\}), v) \nonumber \\
\mu (\phi_1 \AND \phi_2) & = & \mu(\phi_1) \sqcap \mu(\phi_2) \nonumber \\
\mu (\langle a \rangle \phi) & = & ( (\mathcal{W} \cup \{w'\}, \rightarrow \cup (w' \xrightarrow{a} w), \lambda \cup \{w' \mapsto \mathcal{S}\}]), w') \nonumber \\
		& & \mbox{where }\mu(\phi) = ( (\mathcal{W}, \rightarrow, \lambda), w) \nonumber \\
		& & \mbox{and }w' \mbox{ is a new state not appearing in }\mathcal{W} \nonumber
\end{eqnarray}
The only complex case is the clause for $\mu (\phi_1 \AND \phi_2)$, which uses the $\sqcap$ function, defined as\footnote{We assume that the sets of states in the two pointed models are disjoint.}:

\begin{eqnarray*}
  \bot \sqcap X 
     & = & 
  \bot \nonumber 
     \\
  X \sqcap \bot 
     & = & 
  \bot \nonumber 
     \\
  m \sqcap n 
     & = & 
  \begin{cases}
    \mathsf{merge}(m, n) & \text{if}\ \mathsf{consistent}(m, n) \\
    \bot & \text{else}
  \end{cases}
\end{eqnarray*}

\noindent The $\mathsf{consistent}$ predicate is true of pointed models $m$ and $n$ if the out-transitions on $m$'s root node respect the labelling on $n$'s root node, and the out-transitions on $n$'s root node respect the labelling on $m$'s root node. In other words:
\begin{eqnarray}
\mathsf{consistent}(m, n) & \mbox{ iff } & \mathsf{out}(m) \subseteq \mathsf{restriction}(n) \mbox{ and} \nonumber \\
& & \mathsf{out}(n) \subseteq \mathsf{restriction}(m) \nonumber
\end{eqnarray}
where:
\begin{eqnarray}
\mathsf{out}(((\mathcal{W},\rightarrow,\lambda),w)) & = & \{ s \fOr \exists x . w \xrightarrow{s} x \} \nonumber \\
\mathsf{restriction}(((\mathcal{W},\rightarrow,\lambda),w)) & = & \lambda(w) \nonumber
\end{eqnarray}
Now the $\mathsf{merge}$ function fuses two pointed models together:
\[
\mathsf{merge}( ( (\mathcal{W}, \rightarrow, \lambda), w),  ( (\mathcal{W}', \rightarrow', \lambda'), w')) = ((\mathcal{W} \cup \mathcal{W}', \rightarrow \cup \rightarrow'_2, \lambda_2 \cup \lambda'_2), w)
\]
where:
\begin{eqnarray}
\rightarrow'_2 & = & \rightarrow' \mbox{ with } w' \mbox{ replaced by } w \nonumber \\
\lambda_2 & = & \lambda \mbox{ with } w \mapsto \lambda(w) \cap \lambda'(w') \nonumber \\
\lambda'_2 & = & \lambda' \mbox{ with } w' \mbox{ removed } \nonumber
\end{eqnarray}
It is easy to show that $\mu$ satisfies the following propositions:
\begin{eqnarray}
\mu(\phi) & \models & \phi \nonumber \\
\mbox{if }n \models \phi \mbox{ and } m \leq n & \mbox{ then } & m \models \phi \nonumber
\end{eqnarray}

\subsection{Defining $\theta$ - a Formula that Characterises a Model}
The inverse function $\theta$ produces a formula that characterises a given pointed model:
\begin{eqnarray}
\theta(\bot) & = & \langle a \rangle \top \AND ! \{ \} \mbox{ for some symbol }a \nonumber \\
\theta(l, w) & = & \mathsf{bang}(l,w) \AND \bigwedge_{(s,w') \in \mathsf{trans}(l,w)} \langle s \rangle \theta(l, w') \nonumber 
\end{eqnarray}
Here:
\begin{eqnarray}
\mathsf{bang}((\mathcal{W},\rightarrow,\lambda),w) & = & \top \mbox{ if } \lambda(w) = \mathcal{S} \nonumber \\
\mathsf{bang}((\mathcal{W},\rightarrow,\lambda),w) & = & ! \; \lambda(w) \mbox{ otherwise } \nonumber \\
\mathsf{trans}((\mathcal{W},\rightarrow, \lambda),w) & = & \{(s,w') | w \xrightarrow{s} w' \} \nonumber
\end{eqnarray}
Note that $\theta(m)$ is finite if $m$ contains no cycles and if $\lambda(x)$ is either $\mathcal{S}$ or finite for all states $x$.
Note also that $\mu$ and $\theta$ are inverses of each other in that:
\begin{eqnarray}
\mu(\theta(m)) & = & m \nonumber \\
\theta(\mu(p)) & \mbox{ iff } & p \nonumber
\end{eqnarray}

\subsection{Decision Procedure}
Since EL has no connectives for disjunction or implication, its decision procedure is straightforward and efficient. 
Although there are an infinite number of models which satisfy any expression, the satisfying models form a lattice with a least upper bound. 
The $\mu$ function defined above gives us the minimal model satisfying an expression.
Using this least upper bound, we can calculate entailment by checking a \emph{single model}.
To decide whether $p \models q$,  we use the following theorem:
\begin{theorem} $\forall m \; m \models p \Rightarrow m \models q \; \text{iff} \; \mu(p) \models q $
\end{theorem}
\begin{proof}
We first show left to right, and then show right to left.
\setcounter{mycase}{0}
\begin{mycase}
$\forall m \; m \models p \Rightarrow m \models q \; \text{implies} \; \mu(p) \models q$
\end{mycase}
Given that $\mu(p) \models p$, we substitute $\mu(p)$ for $m$ in $\forall m \; m \models p \Rightarrow m \models q$, to infer $\mu(p) \models q$.
\begin{mycase}
$\mu(p) \models q \; \text{implies} \; \forall m \; m \models p \Rightarrow m \models q$
\end{mycase}
Assume $m \models p$. We need to show $m \models q$.
Now if $m \models p$ then $m \leq \mu(p)$ (by definition of $\mu$).
Further, if $n \models x$ and $m \leq n$ then $m \models x$. So, substituting $q$ for $x$ and $\mu(p)$ for $n$, it follows that $m \models q$. 
\qed
\end{proof}
Given this theorem, the decision procedure is straightforward: to test if $p \models q$, we construct $\mu(p)$, and then inspect whether $\mu(p) \models q$.
Construction of $\mu(p)$ is linear in the size of $p$, and computing whether a model satisfies $q$ is linear in the size of $q$, so computing whether $p \models q$ is $O(|p|+|q|)$.

\begin{FIGURE}
\begin{RULES}

  \ZEROPREMISERULENAMEDRIGHT
  {
    \phi \judge \phi
  }{Identity}
    \quad
  \ZEROPREMISERULENAMEDRIGHT
  {
    \phi \judge \top
  }{$\top$-Right}
    \quad
  \ZEROPREMISERULENAMEDRIGHT
  {
    \bot \judge \phi
  }{$\bot$-Left}
    \quad
  \TWOPREMISERULENAMEDRIGHT
  {
    \phi \judge \psi
  }
  {
    \psi \judge \xi
  }
  {
    \phi \judge \xi
  }{Transitivity}
    \\\\
  \ONEPREMISERULENAMEDRIGHT
  {
    \phi \judge \psi
  }
  {
    \phi \AND \xi \judge \psi
  }{$\AND$-Left 1}
     \quad
  \ONEPREMISERULENAMEDRIGHT
  {
    \phi \judge \psi
  }
  {
    \xi \AND \phi  \judge \psi
  }{$\AND$-Left 2}
     \quad
  \TWOPREMISERULENAMEDRIGHT
  {
    \phi \judge \psi
  }
  {
    \phi \judge \xi
  }
  {
    \phi \judge \psi \AND \xi
  }{$\AND$-Right}
     \\\\
     \ONEPREMISERULENAMEDRIGHT
     {
       a \notin A
     }
     {
       !A \AND \MAY{a}{\phi} \judge \bot
     }{$\bot$-Right 1}
        \quad
     \ZEROPREMISERULENAMEDRIGHT
     {
       \MAY{a}{\bot} \judge \bot
     }{$\bot$-Right 2}
        \quad
     \TWOPREMISERULENAMEDRIGHT
     {
       \phi \AND \, !A \judge \psi
     }
     {
       A' \subseteq A
     }
     {
       \phi \AND\, !A' \judge \psi
     }{!-Left}
     \\\\
     \TWOPREMISERULENAMEDRIGHT
     {
       \phi \judge !A
     }
     {
       A \subseteq A'
     }
     {
       \phi \judge!A'
     }{!-Right 1}
     \quad
     \TWOPREMISERULENAMEDRIGHT
     {
       \phi \judge !A
     }
     {
       \phi \judge !B
     }
     {
       \phi \judge !(A \cap B)
     }{!-Right 2}
     \quad
     \ONEPREMISERULENAMEDRIGHT
     {
       \phi \judge \psi
     }
     {
       \MAY{a}{\phi} \judge \MAY{a}{\psi}
     }{Transition Normal}
\end{RULES}
\caption{Proof rules.}\label{figure:elAndBangRules}
\end{FIGURE}



\subsection{Incompatibility Semantics}

Define the set of formulae\footnote{Brandom \cite{brandom} defines incompatibility slightly differently: he defines the set of \emph{sets} of formulae which are incompatible with a \emph{set} of formulae. 
But in EL, if a set of formulae is incompatible, then there is an incompatible subset of that set with exactly two members.
So we can work with the simpler definition in the text above.}
 incompatible with $p$ as:
\[
\mathcal{I}(p) = \{ q \; | \; \mu(p) \sqcap \mu(q) = \bot \}
\]
EL satisfies Robert Brandom's \textbf{incompatibility semantics}  property:
\[
p \models q \; \mbox{ iff } \; \mathcal{I}(q) \subseteq \mathcal{I}(p)
\]
Before proving this, I want to say something about why satisfying this incompatibility semantics property is important.
Not all logics satisfy this property. 
Brandom has shown that First Order Logic and S5 satisfy the incompatibility semantics property, but it is an open question which other logics satisfy it.
HML satisfies it, but HML without negation does not.
EL is the \emph{simplest logic we have found} that satisfies the property.

To prove this, we need to first define a related incompatibility function on pointed models.
$\mathcal{J}(m)$ is the set of models that are incompatible with $m$:
\[
\mathcal{J}(m) = \{ n \; | \; m \sqcap n = \bot \}
\]
We shall make use of three lemmas:
\begin{lemma}
$\mbox{if }p \models q \mbox{ then } \mu(p) \leq \mu(q)$
\end{lemma}
\begin{lemma}
$\mbox{if }m \leq n \mbox{ then } x \sqcap m \leq x \sqcap n$
\end{lemma}
\begin{lemma}
$\mbox{if }\mathcal{I}(q) \subseteq \mathcal{I}(p) \mbox{ then } \mathcal{J}(\mu(q)) \subseteq \mathcal{J}(\mu(p))$
\end{lemma}

\begin{theorem}
$p \models q \; \mbox{ iff } \; \mathcal{I}(q) \subseteq \mathcal{I}(p)$
\end{theorem}

\begin{proof}

Left to right: Assume $p \models q$ and $r \in \mathcal{I}(q)$.
By Lemma 1, $\mu(p) \leq \mu(q)$.
From $r \in \mathcal{I}(q)$, $\mu(r) \sqcap \mu(q) = \bot$.
By Lemma 2, $\mu(r) \sqcap \mu(p) \leq \mu(r) \sqcap \mu(q)$ (substituting $\mu(r)$ for $x$, $\mu(p)$ for $m$, and $\mu(q)$ for $n$).
But if $\mu(r) \sqcap \mu(q) = \bot$, and $\mu(r) \sqcap \mu(p) \leq \mu(r) \sqcap \mu(q)$, then $\mu(r) \sqcap \mu(p) = \bot$ also, because the only element that is $\leq \bot$ is $\bot$ itself.
But if $\mu(r) \sqcap \mu(p) = \bot$, then $r \in \mathcal{I}(p)$.
\qed

Right to left: assume, for reductio, that $m \models p$ and $m \nvDash q$. we will show that $\mathcal{I}(q) \nsubseteq \mathcal{I}(p)$. 
Assume $m \models p \mbox{ and } m \nvDash q$. We will construct another model $n$ such that $n \in \mathcal{J}(\mu(q))$ but $n \notin \mathcal{J}(\mu(p))$.
This will entail, via Lemma 3, that $\mathcal{I}(q) \nsubseteq \mathcal{I}(p)$.

If $m \nvDash q$, then there is a formula $q'$ that does not contain $\AND$ such that $q \models q'$ and $m \nvDash q'$. $q'$ must be either of the form (i) $\langle a_1 \rangle ... \langle a_n \rangle \top$ (for $n > 0$) or (ii) of the form $\langle a_1 \rangle ... \langle a_n \rangle \; !\{A\}$ where $A \subseteq \mathcal{S} \mbox{ and } n >= 0$.

In case (i), there must be an $i$ between $0$ and $n$ such that $m \models \langle a_1 \rangle ... \langle a_i \rangle \top$ but $m \nvDash  \langle a_1 \rangle ... \langle a_{i+1} \rangle \top$. We need to construct another model $n$ such that $n \sqcap \mu(q) = \bot$, but $n \sqcap \mu(p) \neq \bot$. Letting $m = ((\mathcal{W},\rightarrow,\lambda),w)$, then $m \models \langle a_1 \rangle ... \langle a_i \rangle \top$ implies that there is at least one sequence of states of the form $w, w_1, ..., w_i$ such that $w \xrightarrow{a_1} w_1 \rightarrow ... \xrightarrow{a_i} w_i$. 
Now let $n$ be just like $m$ but with additional transition-restrictions on each $w_i$ that it not include $a_{i+1}$. 
In other words, $\lambda_n(w_i) = \lambda_m(w_i)  - \{a_{i+1}\}$ for all $w_i$ in sequences of the form $w \xrightarrow{a_1} w_1 \rightarrow ... \xrightarrow{a_i} w_i$. Now $n \sqcap \mu(q) = \bot$ because of the additional transition restriction we added to $n$, which rules out $\langle a_1 \rangle ... \langle a_{i+1} \rangle \top$, and a-forteriori $q$. But $n \sqcap \mu(p) \neq \bot$, because $m \models p$ and $n \leq m$ together imply $n \models p$. So $n$ is indeed the model we were looking for, that is incompatible with $\mu(q)$ while being compatible with $\mu(p)$.

In case (ii), $m \models \langle a_1 \rangle ... \langle a_n \rangle \top$ but $m \nvDash \langle a_1 \rangle ... \langle a_n \rangle !A$ for some $A \subset \mathcal{S}$. We need to produce a model $n$ that is incompatible with $\mu(q)$ but not with $\mu(p)$. Given that $m \models \langle a_1 \rangle ... \langle a_n \rangle \top$, there is a sequence of states $w, w_1, ..., w_n$ such that $w \xrightarrow{a_1} w_1 \rightarrow ... \xrightarrow{a_i} w_n$. Let $n$ be the model just like $n$ except it has an additional transition from each such $w_n$ with a symbol $s \notin A$. Clearly, $n \sqcap \mu(q') = \bot$ because of the additional $s$-transition, and given that $q \models q'$, it follows that $n \sqcap \mu(q) = \bot$. Also, $n \sqcap \mu(p) \neq \bot$, because $n \leq m$ and $m \models p$.


\end{proof}

\subsection{Hennessy-Milner Theorem for EL}

\begin{definition}
Two states $w$ and $w'$, in $(\mathcal{W}, \rightarrow, \lambda)$ and $(\mathcal{W}', \rightarrow', \lambda')$ respectively, are {\bf bisimilar}, written $w \sim w'$, iff there is a bisimulation $Z \subseteq  \mathcal{W} \times \mathcal{W}'$ with $(w,w') \in Z$.
\end{definition}

\begin{definition}
Two states $w$ and $w'$ are {\bf equivalent}, written $w \equiv w'$, iff $\{\phi \; | \; w \models \phi\} = \{\phi \; | \; w' \models \phi\}$.
\end{definition}
\begin{theorem}
If two models are finitely-branching (if the set $\{y \fOr \exists s . x \xrightarrow{s} y\}$ is finite for all states $x$), $w \sim w' \mbox{ iff } w \equiv w' $.
\end{theorem}
First, left to right.
\begin{case}
If $w \sim w'$, then $w \equiv w'$.
\end{case}
\begin{proof}
Proof is by induction on formulae.
The only case which differs from the standard Hennessy-Milner Theorem is the case for $!A$, so this is the only case we shall consider.
Assume $w \sim w'$ and $w \models !A$. We need to show $w' \models !A$.

From the semantic clause for $!$,  $w \models !A$ implies $\lambda(w) \subseteq A$.
If $w \sim w'$, then $\lambda(w) = \lambda'(w')$.
Therefore $\lambda'(w') \subseteq A$, and hence $w' \models !A$.
\qed
\end{proof}

\begin{case}
Given two finitely-branching models $(\mathcal{W}, \rightarrow, \lambda)$ and $(\mathcal{W}', \rightarrow', \lambda')$, with $w \in \mathcal{W}$ and $w' \in \mathcal{W}'$,
if $w \equiv w'$, then $w \sim w'$.
\end{case}
\begin{proof}
We define the bisimilarity relation:
\[
Z = \{(x,x') \in \mathcal{W} \times \mathcal{W}' \fOr x \equiv x' \}
\]
To prove $w \sim w'$, we need to show:
\begin{itemize}
\item
$(w,w') \in Z$. This is immediate from the definition of Z.
\item
The relation $Z$ respects the transition-restrictions: if $(x,x') \in Z$ then $\lambda(x) = \lambda'(x')$
\item
The forth condition: if $(x,x') \in Z$ and $x \xrightarrow{s} y$, then there exists a $y'$ such that $x' \xrightarrow{s} y'$
\item
The back condition: if $(x,x') \in Z$ and $x' \xrightarrow{s} y'$, then there exists a $y$ such that $x \xrightarrow{s} y$
\end{itemize}
To show that $(x,x') \in Z$ implies $\lambda(x) = \lambda'(x')$, we will argue by contraposition.
Assume $\lambda(x) \neq \lambda'(x')$.
Then either $\lambda'(x') \nsubseteq  \lambda(x)$ or $\lambda(x) \nsubseteq  \lambda'(x')$.
If $\lambda'(x') \nsubseteq  \lambda(x)$, then $x' \nvDash \fBang \lambda(x)$.
But $x \models \fBang \lambda(x)$, so $x$ and $x'$ satisfy different sets of propositions and are not equivalent.
Similarly, if $\lambda(x) \nsubseteq  \lambda'(x')$ then $x \nvDash \fBang \lambda'(x')$.
But $x' \models \fBang \lambda'(x')$, so again $x$ and $x'$ satisfy different sets of propositions and are not equivalent.

I will show the forth condition in detail. The back condition is very similar.
To show the forth condition, assume that  $x \xrightarrow{s} y$ and that $(x,x') \in Z$ (i.e. $x \equiv x'$).
We need to show that $\exists y'$ such that $x' \xrightarrow{s} y'$ and $(y,y') \in Z$ (i.e. $y \equiv y'$).

Consider the set of $y'_i$ such that $x' \xrightarrow{s} y'_i$. Since $x \xrightarrow{s} y$, $x \models \langle s \rangle \top$, and as $x \equiv x'$,  $x' \models \langle s \rangle \top$, so we know this set is non-empty.
Further, since $(\mathcal{W}', \rightarrow')$ is finitely-branching, there is only a finite set of such $y'_i$, so we can list them $y'_1, ..., y'_n$,  where $n >= 1$.

Now, in the Hennessy-Milner theorem for HML, the proof proceeds as follows:
assume, for reductio, that of the $y'_1, ..., y'_n$, there is no $y'_i$ such that $y \equiv y'_i$.
Then, by the definition of $\equiv$, there must be formulae $\phi_1, ..., \phi_n$ such that for all $i$ in $1$ to $n$:
\[
y'_i \models \phi_i \mbox{ and } y \nvDash \phi_i
\]
Now consider the formula:
\[
[s] (\phi_1 \lor ... \lor \phi_n)
\]
As each $y'_i \models \phi_i$, $x' \models [s] (\phi_1 \lor ... \lor \phi_n)$, but $x$ does not satisfy this formula, as each $\phi_i$ is not satisfied at $y$.
Since there is a formula which $x$ and $x'$ do not agree on, $x$ and $x'$ are not equivalent, contradicting our initial assumption.

But this proof cannot be used in EL because it relies on a formula $[s] (\phi_1 \lor ... \lor \phi_n)$ which cannot be expressed in EL. 
EL does not include the box operator or disjunction, so this formula is ruled out on two accounts.
But we can massage it into a form which is more amenable to EL's expressive resources:
\begin{eqnarray}
[s] (\phi_1 \lor ... \lor \phi_n) & = & \neg \langle s \rangle \neg (\phi_1 \lor ... \lor \phi_n) \nonumber \\
	& = & \neg \langle s \rangle (\neg \phi_1\AND ... \AND \neg \phi_n) \nonumber
\end{eqnarray}
Further, if the original formula $[s] (\phi_1 \lor ... \lor \phi_n)$ is true in $x'$ but not in $x$, then its negation will be true in $x$ but not in $x'$. 
So we have the following formula, true in $x$ but not in $x'$:
\[
 \langle s \rangle (\neg \phi_1\AND ... \AND \neg \phi_n)
 \]
The reason for massaging the formula in this way is so we can express it in EL (which does not have the box operator or disjunction).
At this moment, the revised formula is \emph{still} outside EL because it uses negation. 
But we are almost there: the remaining negation is in innermost scope, and innermost scope negation can be simulated in EL by the $!$ operator. 

We are assuming, for reductio, that of the $y'_1, ..., y'_n$, there is no $y'_i$ such that $y \equiv y'_i$.
But in EL without negation, we cannot assume that each $y'_i$ has a formula $\phi_i$ which is satisfied by $y'_i$ but not by $y$ - it might instead be the other way round: $\phi_i$ may be satisfied by $y$ but not by $y'_i$. So, without loss of generality, assume that $y'_1, ..., y'_m$ fail to satisfy formulae $\phi_1, ..., \phi_m$ which $y$ does satisfy, and that $y'_{m+1}, ..., y'_n$ satisfy formulae $\phi_{m+1}, ..., \phi_n$ which $y$ does not:
\begin{eqnarray}
y \models \phi_i \mbox{ and } y'_i \nvDash \phi_i & & i = 1 \mbox{ to } m \nonumber \\
y \nvDash \phi_j \mbox{ and } y'_j \models \phi_j & & j = m+1 \mbox{ to } n \nonumber
\end{eqnarray}
The formula we will use to distinguish between $x$ and $x'$ is:
\[
 \langle s \rangle ( \bigwedge_{i=1}^m \phi_i \; \AND \; \bigwedge_{j=m+1}^n \mathsf{neg}(y, \phi_j))
 \]
 Here, $\mathsf{neg}$ is a meta-language function that, given a state y and a formula $\phi_j$, returns a formula that is true in $y$ but incompatible with $\phi_j$. I will show that, since $y \nvDash \phi_j$, it is always possible to construct $ \mathsf{neg}(y, \phi_j)$ using the $!$ operator.

Consider the possible forms of $\phi_j$:
\begin{itemize}
\item
$\top$: this case cannot occur since all models satisfy $\top$.
\item
$\phi_1 \AND \phi_2$: we know $y'_j \models \phi_1 \AND \phi_2$ and $y \nvDash \phi_1 \AND \phi_2$. There are three possibilities:
\begin{enumerate}
\item
$y \nvDash \phi_1$ and $y \models \phi_2$. In this case, $\mathsf{neg}(y, \phi_1 \AND \phi_2) = \mathsf{neg}(y, \phi_1) \AND \phi_2$.
\item
$y \models \phi_1$ and $y \nvDash \phi_2$. In this case, $\mathsf{neg}(y, \phi_1 \AND \phi_2) = \phi_1 \AND \mathsf{neg}(y, \phi_2)$.
\item
$y \nvDash \phi_1$ and $y \nvDash \phi_2$. In this case, $\mathsf{neg}(y, \phi_1 \AND \phi_2) =  \mathsf{neg}(y, \phi_1) \AND \mathsf{neg}(y, \phi_2)$.
\end{enumerate}
\item
$!A$: if $y \nvDash !A \mbox{ and } y'_j \models !A$, then there is a symbol $a \in \mathcal{S}-A$ such that $y \xrightarrow{a} z$ for some $z$ but there is no such $z$ such that $y'_j \xrightarrow{a} z$. In this case, let $\mathsf{neg}(y, \phi_j) = \langle a \rangle \top$.
\item
$\langle a \rangle \phi$. There are two possibilities:
\begin{enumerate}
\item
$y \models \langle a \rangle \top$. In this case, $\mathsf{neg}(y, \langle a \rangle \phi) =  \bigwedge\limits_{y \xrightarrow{a} z}  \langle a \rangle \mathsf{neg}(z, \phi)$.
\item
$y \nvDash \langle a \rangle \top$. In this case, $\mathsf{neg}(y, \langle a \rangle \phi) = \fBang \{ b \fOr \exists z. y \xrightarrow{b} z\}$. This set of $b$s is finite since we are assuming the LTS is finitely-branching.
\end{enumerate}
\end{itemize}
\qed
\end{proof}

\subsection{Worked example of $\mathsf{neg}$}

\begin{figure}[h]
\centering
\begin{tikzpicture}[node distance=1.3cm,>=stealth',bend angle=45,auto]
  \tikzstyle{place}=[circle,thick,draw=blue!75,fill=blue!20,minimum size=6mm]
  \tikzstyle{red place}=[place,draw=red!75,fill=red!20]
  \tikzstyle{transition}=[rectangle,thick,draw=black!75,
  			  fill=black!20,minimum size=4mm]
  \tikzstyle{every label}=[red]

  \begin{scope}[xshift=0cm]
    \node [place] (t) {$y$};
    \node [place] (a) [below left of=t] {$z_1$}
      edge [pre]  node[swap] {a}                 (t);
    \node [place] (a2) [below right of=t] {$z_2$}
      edge [pre]  node[swap] {a}                 (t);
    \node [place] (b) [below of=a] {$w_1$}
      edge [pre]  node[swap] {b}                 (a);
    \node [place] (c) [below of=a2] {$w_2$}
      edge [pre]  node[swap] {c}                 (a2);
  \end{scope}  
  \begin{scope}[xshift=6cm]
    \node [place] (t) {$y'_j$};
    \node [place] (a) [below of=t] {$z'$}
      edge [pre]  node[swap] {a}                 (t);
    \node [place] (b) [below left of=a] {$w'_1$}
      edge [pre]  node[swap] {b}                 (a);
    \node [place] (c) [below right of=a] {$w'_2$}
      edge [pre]  node[swap] {c}                 (a);
  \end{scope}  
\end{tikzpicture}
\caption{{\small Worked Example of $\mathsf{neg}$. Note that the transition
  system on the left is non-deterministic.}}\label{figure:example:neg}
\end{figure}



Consider $y$ and $y'_j$ as in Figure \ref{figure:example:neg}.
One formula that is true in $y'_j$ but not in $y$ is
\[
\langle a \rangle (\langle b \rangle \top \AND \langle c \rangle \top)
\]
Now:
\begin{eqnarray*}
\lefteqn{\mathsf{neg}(y, \langle a \rangle (\langle b \rangle \top \AND \langle c \rangle \top))}\qquad \qquad \qquad  \\
& = & \bigwedge\limits_{y \xrightarrow{a} z} \langle a \rangle \mathsf{neg}(z, \langle b \rangle \top \AND \langle c \rangle \top)  \\
& = & \langle a \rangle \mathsf{neg}(z_1, \langle b \rangle \top \AND \langle c \rangle \top) \AND \langle a \rangle\mathsf{neg}(z_2, \langle b \rangle \top \AND \langle c \rangle \top)  \\
& = & \langle a \rangle (\langle b \rangle \top \AND \mathsf{neg}(z_1, \langle c \rangle \top)) \AND \langle a \rangle\mathsf{neg}(z_2, \langle b \rangle \top \AND \langle c \rangle \top)  \\
& = & \langle a \rangle (\langle b \rangle \top \AND \mathsf{neg}(z_1, \langle c \rangle \top)) \AND \langle a \rangle(\mathsf{neg}(z_2, \langle b \rangle \top) \AND \langle c \rangle \top)  \\
& = & \langle a \rangle (\langle b \rangle \top \AND \fBang \{b\}) \AND \langle a \rangle(\mathsf{neg}(z_2, \langle b \rangle \top) \AND \langle c \rangle \top)  \\
& = & \langle a \rangle (\langle b \rangle \top \AND \fBang \{b\}) \AND \langle a \rangle(\fBang \{c\} \AND \langle c \rangle \top) 
\end{eqnarray*}

\NI The resulting formula is true in $y$ but not in $y'_j$.



\subsection{Discussion}
Consider the fragment of HML without negation:

\begin{GRAMMAR}
  \phi 
    &\quad ::= \quad&
  \top \fOr \phi_1 \AND \phi_2  \fOr \langle a \rangle \phi
\end{GRAMMAR}

\NI Without negation, $p \equiv p'$ does \emph{not} entail $p \sim
p'$, as Figure \ref{figure:hml} shows.

\begin{figure}[h]
\centering
\begin{tikzpicture}[node distance=1.3cm,>=stealth',bend angle=45,auto]
  \tikzstyle{place}=[circle,thick,draw=blue!75,fill=blue!20,minimum size=6mm]
  \tikzstyle{red place}=[place,draw=red!75,fill=red!20]
  \tikzstyle{transition}=[rectangle,thick,draw=black!75,
  			  fill=black!20,minimum size=4mm]
  \tikzstyle{every label}=[red]

  \begin{scope}[xshift=0cm]
    \node [place] (w1) {$x$};
    \node [place] (e1) [below left of=w1] {$y_1$}
      edge [pre]  node[swap] {a}                 (w1);
    \node [place] (c) [below of=e1] {$z_1$}
      edge [pre]  node[swap] {b}                 (e1);
    \node [place] (e2) [below right of=w1] {$y_2$}
      edge [pre]  node[swap] {a}                 (w1);
    \node [place] (c) [below of=e2] {$z_2$}
      edge [pre]  node[swap] {b}                 (e2);
    \node [place] (c) [below right of=e2] {$z_3$}
      edge [pre]  node[swap] {c}                 (e2);
  \end{scope}  
  \begin{scope}[xshift=6cm]
    \node [place] (t) {$x'$};
    \node [place] (a) [below of=t] {$y'$}
      edge [pre]  node[swap] {a}                 (t);
    \node [place] (b) [below left of=a] {$z'_1$}
      edge [pre]  node[swap] {b}                 (a);
    \node [place] (c) [below right of=a] {$z'_2$}
      edge [pre]  node[swap] {c}                 (a);
  \end{scope}  
\end{tikzpicture}
\caption{Without negation, $p \equiv p'$ does not entail $p \sim p'$}\label{figure:hml}
\end{figure}


Here, $x \equiv x'$ - they both satisfy the sub formulae of $\langle a \rangle (\langle b \rangle \top \AND \langle c \rangle \top)$.
But there is no bisimulation between $x$ and $x'$ ($y_1$ cannot be matched with $y'$ because $y'$ has an additional outgoing $c$-transition). 

If we remove negation, HML  is insufficiently expressive to distinguish between these two non-bisimilar models.
When we restore negation, we can distinguish between these models via the formula:
\[
\langle a \rangle \neg \langle c \rangle \top
\]
Similarly, EL is able to distinguish between these models via the formula:
\[
\langle a \rangle ! \{b\}
\]
EL, also, has the expressive capacity to distinguish between non-bisimilar models. If the LTSs are finitely-branching, it can simulate negation (in innermost scope) via the width restriction operator !

\subsection{Comparing EL with Hennessy-Milner Logic}

Given a set $\mathcal{S}$ of symbols, with $s$ ranging over $\mathcal{S}$, the formulae of Hennessy-Milner Logic (HML) are given by:
\begin{GRAMMAR}
  \phi 
     &\quad ::= \quad & 
  \top \fOr \phi_1 \AND \phi_2 \fOr \langle s \rangle \phi \fOr \neg \phi 
\end{GRAMMAR}
\subsection{Semantics}
A {\bf Labelled Transition System} (LTS) is a pair $(\mathcal{W}, \rightarrow)$, where $\mathcal{W}$ is a set of states, and $\rightarrow$ is a set of transition relations: $\rightarrow \; \subseteq \; \mathcal{W} \times S \times \mathcal{W}$.
We write $x \xrightarrow{s} y$ to abbreviate $(x,s,y) \in \rightarrow$.

A {\bf pointed LTS} is a pair $(l, w)$, where $l$ is a LTS $(\mathcal{W}, \rightarrow)$ and $w$ is a distinguished state $w \in \mathcal{W}$.

A formula of HML is evaluated in a pointed LTS $(l, w)$:
\begin{eqnarray}
(l,w) & \models & \top \nonumber \\
(l,w) & \models & \phi_1 \AND \phi_2 \mbox{ iff } (l,w)  \models \phi_1 \mbox { and } (l,w) \models \phi_2 \nonumber \\
(l,w) & \models & \langle s \rangle \phi \mbox{ iff there is a } w \xrightarrow{s} w' \mbox { such that } (l,w') \models \phi \nonumber \\
(l,w) & \models & \neg \phi \mbox{ iff } (l,w)  \nvDash \phi \nonumber
\end{eqnarray}


\subsection{Translating EL into HML}
The only differences between EL and HML are:
\begin{itemize}
\item
Syntactically, EL has the transition-restriction operator ($!$) instead of logical negation ($\neg$)
\item
Semantically, an EL model has an additional node-labelling, describing the allowable transitions from each vertex
\end{itemize}
We can translate formulae of EL into HML using the function $t$:
\begin{eqnarray}
t(\top) & = & \top \nonumber \\
t(\phi_1 \AND \phi_2) & = & t(\phi_1) \AND t(\phi_2) \nonumber \\
t(\langle s \rangle \phi) & = & \langle s \rangle t(\phi) \nonumber \\
t(! A) & = & \bigwedge_{s \in S - A} \neg \langle s \rangle \top \nonumber
\end{eqnarray}
If $\mathcal{S}$ is an infinite set, then the translation of a $!$
formula will be an infinitary formula.  If $\mathcal{S}$ is finite,
then the size of the HML formula will be of the order of $n *
|\mathcal{S}|$ larger than the original EL formula (where $n$ is the
number of $!$ operators occurring in the EL formula).

