\section{Conclusion}\label{conclusion}

In this paper, we have introduced \cathoristic{}, and established key
meta-logical properties such as completeness and compactness. However,
many questions are left open. 

\subsection{Excluded middle}

A better understanding of \cathoristic{}'s relationship with first-order
logic might also lead to insight into excluded middle in \cathoristic{}.
The logical law of excluded middle states that either a proposition or
its negation must be true. In \cathoristic{}
\[
\models \phi \lor \neg_S(\phi)
\]

\NI does not hold in general. (The negation operator $\neg_{S}(\cdot)$
was defined in Section \ref{ELAndNegation}) For example, let $\phi$ be
$\langle a \rangle \top$ and $S = \Sigma = \{a, b\}$.  Then
\[
   \phi \lor \neg_{S} \phi 
       \quad=\quad 
   \langle a \rangle \top \; \lor \; ! \{b\} \; \lor \; \langle a \rangle \bot
\]

\NI Now this will not in general be valid - it will be false for
example in the model $((\{x\}, \emptyset, \{(x, \Sigma)\}), x)$, the
model having just the start state (labelled $\Sigma$) and no transitions.
Restricting $S$ to be a proper subset of $\Sigma = \{a, b\}$ is also not
enough. For example with $S = \{a\}$ we have
\[
   \MAY{a}{\TRUE} \lor \neg_{S}(\MAY{a}{\TRUE})
      \quad=\quad
   \MAY{a}{\TRUE}\ \lor\ !\emptyset \lor \MAY{a}{\FALSE}
\]
This formula cannot hold in any cathoristic model which contains a
$b$-labelled transition, but no $a$-transition from the start state.

Is it possible to identify classes of models that nevertheless verify
excluded middle? The answer to this question appears to depend 
on the chosen notion of semantic model.

\subsection{Understanding the expressive strength of \cathoristic{}}

\subsubsection{Comparing \cathoristic{} and Hennessy-Milner logic}

Section \ref{standardTranslation} investigated the relationship
between \cathoristic{} and first-order logic. Now we compare
\cathoristic{} with a logic that is much closer in spirit:
Hennessy-Milner logic \cite{HennessyM:alglawfndac}, a modal logic
designed to reason about process calculi. Indeed the present shape of
\cathoristic{} owes much to Hennessy-Milner logic. We contrast both by
translation from the former into the latter.  This will reveal more
clearly than the translation into first-order logic the novelty of
\cathoristic{}.

\begin{definition}
Assume a set $\Sigma$ of symbols, with $s$ ranging over
$\Sigma$, the \emph{formulae} of Hennessy-Milner logic are given
by the following grammar:
\begin{GRAMMAR}
  \phi 
     &\quad ::= \quad & 
  \top \fOr \BIGAND_{i \in I} \phi_i  \fOr \langle s \rangle \phi \fOr \neg \phi 
\end{GRAMMAR}
\end{definition}

\NI The index set $I$ in the conjunction can be infinite, and needs to
be so for applications in process theory.

\begin{definition}
 \emph{Models} of Hennessy-Milner logic are simply pairs $(\LLL, s)$
 where $\LLL = (S, \rightarrow)$ is a labelled transition system over
 $\Sigma$, and $s \in S$.  The \emph{satisfaction relation} $(\LLL, s)
 \models \phi$ is given by the following inductive clauses.

\[
\begin{array}{lclcl}
  (\LLL, s) 
     & \models & 
  \top  \\
  (\LLL, s) 
     & \models & 
  \BIGAND_{i \in I} \phi_i  &\  \mbox{ iff }\  & \mbox { for all $i \in I$ }: (\LLL, s) \models \phi_i  \\
  (\LLL, s) 
     & \models & 
  \langle a \rangle \phi & \mbox{ iff } & \mbox{ there is a } s \xrightarrow{a} s' \mbox { such that } (l,s') \models \phi  \\
  (\LLL, s) 
     & \models & 
  \neg \phi & \mbox{ iff } & (\LLL, s)  \nvDash \phi 
\end{array}
\]
\end{definition}

\NI We can now see the two differences between \cathoristic{}, and
Hennessy-Milner logic, one syntactic, the other semantic.

\begin{itemize}

\item Syntactically, \cathoristic{} has the tantum operator ($!$) instead of
  logical negation ($\neg$).

\item Semantically,  cathoristic models are deterministic,
  while typcially models of Hennessy-Milner logic are
  non-deterministic, although the semantics makes perfect sense for
  deterministic transistion systems too. 
  Moreover models of Hennessy-Milner logic lack state labels.

\end{itemize}

\begin{definition}
 We translate formulae of \cathoristic{} into Hennessy-Milner logic
 using the function $\SEMB{\cdot}$:

\begin{eqnarray*}
  \SEMB{\top} & \ = \ & \top  \\
  \SEMB{\phi_1 \AND \phi_2} & \ = \ & \SEMB{\phi_1} \AND \SEMB{\phi_2}  \\
  \SEMB{\langle a \rangle \phi} & \ = \ & \langle a \rangle \SEMB{\phi}  \\
  \SEMB{! A} & \ = \ & \bigwedge_{a \in \Sigma \setminus A}\!\!\!\! \neg \langle a \rangle \top 
\end{eqnarray*}

\end{definition}

\NI If $\Sigma$ is an infinite set, then the translation of a $!$
formula will be an infinitary formula.  If $\Sigma$ is finite, then
the size of the Hennessy-Milner logic formula will be of the order of $n \cdot | \Sigma |$
larger than the original cathoristic formula, where $n$ is the number of
$!$ operators occurring in the cathoristic formula). In both logics we
use the number of logical operators as a measure of size.

We can also translate cathoristic models by forgetting state-labelling:
\[
   \SEMB{((S, \rightarrow, \lambda), s)} 
      =
   ((S, \rightarrow), s)
\]

\NI We continue with an obvious consequence of the translation.

\begin{theorem}
Let $\MMM$ be a (deterministic or non-deterministic) cathoristic
  model. Then $\MMM \models \phi$ implies $\SEMB{\MMM} \models
  \SEMB{\phi}$.
\end{theorem}
\begin{proof}
Straightforward by induction on $\phi$.
\end{proof}

\NI However, note that the following natural extension is \emph{not} true
under the translation above:
\[
  \text{If } \phi \models \psi \text{ then } \SEMB{\phi} \models \SEMB{\psi}
\]
To see this, consider an entailment which relies on determinism, such as
\[
\MAY{a} \MAY{b} \land \MAY{a} \MAY{c} \models \MAY{a} (\MAY{b} \land \MAY{c})
\]

\NI The first entailment is valid in \cathoristic\ because of the
restriction to deterministic models, but not in Hennessy-Milner
logic, where it is invalidated by any model with two outgoing $a$
transitions, one of which satisfies $\MAY{b}$ and one of which
satisfies $\MAY{c}$.

We can restore the desired connection between cathoristic implication and
Hennessy-Milner logic implication in two ways. First we can restrict
our attention to deterministic models of Hennessy-Milner logic.  The
second solution is to add a determinism constraint to our
translation. Given a set $\Gamma$ of cathoristic formulae, closed under
sub formulae, that contains actions from the set $A \subseteq \Sigma$,
let the determinism constraint for $\Gamma$ be:
\[
\bigwedge_{a \in A, \phi \in \Gamma, \psi \in \Gamma} \neg \; (\MAY{a} \phi \land \MAY{a} \psi \land \neg \MAY{a} (\phi \land \psi) )
\]
If we add this sentence as part of our translation $\SEMB{\cdot}$, we
do get the desired result that
\[
\text{If } \phi \models \psi \text{ then } \SEMB{\phi} \models \SEMB{\psi}
\]


\subsubsection{Comparing \cathoristic{} with Hennessy-Milner logic and propositional logic}

Consider the following six languages:

\begin{FIGURE}
\begin{tikzpicture}[node distance=3.3cm,>=stealth',bend angle=45,auto]
  \tikzstyle{place}=[circle,thick,draw=blue!75,fill=blue!20,minimum size=6mm]
  \tikzstyle{red place}=[place,draw=red!75,fill=red!20]
  \tikzstyle{transition}=[rectangle,thick,draw=black!75,
  			  fill=black!20,minimum size=4mm]
  \tikzstyle{every label}=[red]
  \begin{scope}  
    \node [place] (w1) {PL[$\land$]};
    \node [place] (e1) [below of=w1] {PL [$\land, \neg$] };
  \end{scope}
  \begin{scope}[xshift=4cm]
    \node [place] (w1) {HML[$\land$]};
    \node [place] (e1) [below of=w1] {HML [$\land, \neg$] };
  \end{scope} 
  \begin{scope}[xshift=8cm]
    \node [place] (w1) {EL[$\land, !$]};
    \node [place] (e1) [below of=w1] {EL [$\land, !, \neg$] };
  \end{scope}
  \draw (2,0) node {$\subseteq $};
  \draw (6,0) node {$\subseteq$};
  \draw (2,-3) node {$\subseteq $};
  \draw (6,-3) node {$\subseteq$};
  \draw (0,-1.5) node {$\subseteq $};
  \draw (4,-1.5) node {$\subseteq $};
  \draw (8,-1.5) node {$\subseteq $};
\end{tikzpicture}
\caption{Relationships between logics established in this paper. Here $L_1 \subseteq L_2$
means that the logic $L_2$ is a conservative extension of $L_1$. NB: we need to
explain the concept of conservative extension.}\label{figure:relationships}
\end{FIGURE}


\begin{center}
\begin{tabular}{ l | r }
Language & Description \\
\hline
PL[$\land$] & Propositional logic without negation \\
Hennessy-Milner logic[$\land$] & Hennessy-Milner logic without negation \\
CL[$\land, !$] & \Cathoristic{} \\
PL [$\land, \neg$] & Full propositional logic \\
HML [$\land, \neg$] & Full Hennessy-Milner logic \\
CL [$\land, !, \neg$] & \Cathoristic{} with negation\\
\end{tabular}
\end{center}


\NI The top three languages are simple. In each case: there is no
facility for expressing disjunction, every formula that is satisfiable
has a simplest satisfying model, and there is a simple quadratic-time
decision procedure But there are two ways in which CL[$\land, !$] is
more expressive.  Firstly, CL[$\land, !$], unlike HML[$\land$], is expressive enough to be able to distinguish
between any two models that are not bisimilar, cf.~Theorem
\ref{hennessymilnertheorem}.  The second way in which
CL[$\land, !$] is significantly more expressive than both PL[$\land$]
and HML[$\land$] is in its ability to express incompatibility.  No two
formulae of PL[$\land$] or HML[$\land$] are incompatible\footnote{The notion of incompatibility applies to all logics: two formulae are incompatible if there is no model which satisfies both.} with each
other.  But many
pairs of formulae of CL[$\land, !$] are incompatible.  (For example:
$\langle a \rangle \top$ and $! \emptyset$).  Because CL[$\land, !$] is
expressive enough to be able to make incompatible claims, it satisfies
Brandom's incompatibility semantics constraint.
CL[$\land, !$] is the only logic we are aware of with a
quadratic-time decision procedure that is expressive enough to respect
this constraint. 

The bottom three language can all be decided in exponential time.  We
claim that Hennessy-Milner logic is more expressive than PL, and CL[$\land, !, \neg$] is more expressive than full Hennessy-Milner logic.
To see that full Hennessy-Milner logic is more expressive than full
propositional logic, fix a propositional logic with the nullary
operator $\top$ plus an infinite number of propositional atoms
$P_{(i,j)}$, indexed by $i$ and $j$.  Now translate each formula of
Hennessy-Milner logic via the rules:
\begin{align*}
  \SEMB{\top}  & =  \top  &
  \SEMB{\phi \land \psi} & =  \SEMB{\phi} \land \SEMB{\psi}  \\
  \SEMB{\neg \phi} & =  \neg \SEMB{\phi}   &
  \SEMB{\langle a_i \rangle \phi_j} & =  P_{(i,j)} 
\end{align*}

\NI We claim Hennessy-Milner logic is more expressive because there
are formulae $\phi$ and $\psi$ of Hennessy-Milner logic such that
\[
\phi \models_{\text{{HML}}} \psi \mbox{ but } \SEMB{\phi} \nvDash_{\text{PL}} \SEMB{\psi}
\]
For example, let $\phi = \langle a \rangle \langle b \rangle \top$ and
$\psi = \langle a \rangle \top$.  Clearly, $\phi \models_{\text{HML}}
\psi$. But $\SEMB{\phi} = P_{(i,j)}$ and $\SEMB{\psi} = P_{(i',j')}$
for some $i,j,i',j'$, and there are no entailments in propositional
logic between arbitrary propositional atoms.

We close by stating that CL[$\land, !, \neg$] is more expressive than
full Hennessy-Milner logic: the formula $\fBang A$ of
\cathoristic\ can be translated into Hennessy-Milner logic as:
\[
\bigwedge_{a \in \Sigma - A} \neg \langle a \rangle \top
\]
But if $\Sigma$ is infinite, then this is an infinitary disjunction.
\Cathoristic{} can express the same proposition in a finite sentence.

