\section{Conclusion}\label{conclusion}


\QUOTATION{As long as a branch of science offers an abundance of
  problems, so long it is alive; a lack of problems foreshadows
  extinction or the cessation of independent development. \textsc{D. Hilbert.}}

\subsection{Open problems} In the spirit of Hilbert's quote, I'm including this
section to encourage us to think about interesting open problems,
ideally of varying degrees of difficulty. What I recommend to avoid
are vague problems, there are enough of those already. I'm looking for
clearcut questions, i.e. we know when we've succeeded in solving them.
A good question would ideally come with a plausible story why solving
the problem is important.  Here are some examples to get the ball
rolling.

\begin{enumerate}

\item Is there a reasonable Curry-Howard correspondence for eremic
  logic?

\item\label{conclusion:openProblems:2}  Develop a theory of SAT solving based on eremic rather than
  propositional logic.

\item Once (\ref{conclusion:openProblems:2}) works, develop efficient
  solvers for eremic satisfiability.

\item\label{conclusion:openProblems:4} Develop machine-oriented proof-rules that relate to eremic logic
  in the way that unification/resolution relate to first-order logic.

\item Once (\ref{conclusion:openProblems:4}) works, develop a
  programming language that relates to eremic logic in the same way
  Prolog relates to classical logic.

\item Develop and axiomatise an eremic $\mu$-calulus, following
  Kozen's modal-$\mu$-calculus.

\item Can we give a good characterisation of the first-order fragment
  identified by the standard translation of eremic logic?

\end{enumerate}

