igure:elAndBang:models.tex
figure:elSmall.tex1;2c\section{Conclusion}\label{conclusion}

We have defined \cathoristic{}, a multi-modal logic based on
exclusion as a logical primitive and investigated its key properties.
We close with an overview of related work and a discussion of 
open problems.


%% The work described here was inspired by Sellars' and Brandom's claim
%% that material incompatibility is conceptually prior to logical
%% negation.
%% \Cathoristic{} is a logic without negation in which you can, nevertheless, make incompatible claims.\martin{We can drop this
%% paragraph and start straightaway with ``related work''!}

\subsection{Related work} 


\subsubsection{Brandom's incompatibility semantics}
\martin{I think this whole subsection can be dropped. Instead we say something like:
``we have already discussed the importance of Brandom's work in the introduction'' ... ALso we don't
need to introduce this new maths here. We have truely mentioned Brandom and incompatibility often enoguh!}
\richard{I hear you, Martin, but the symbolism here is needed for the next two subsections where we discuss Peregrin and Turbanti. I have gone out of my way to get the formalism consistent between the three, and need to introduce the formalism somewhere. I hear your general complaint that we have talked about this enough - but this might be partly because we are getting near the end of editing the paper, and are starting to get sick of seeing the same stuff over and over. The \emph{reader} may not find the repetition so insufferable. I would much rather he felt a bit of repetition than was confronted with some formalism that he did not understand because it had not been defined properly.}


\NI In \cite{brandom}, Chapter 5, Appendix I, Brandom developed a new
type of semantics, incompatibility semantics, that takes material
incompatibility - rather than truth-assignment - as the semantically
primitive notion.

Incompatibility semantics applies to any language, $\mathcal{L}$,
given as a set of sentences.  Given a predicate $\mathsf{Inc}(X)$
which is true of sets $X \subseteq \mathcal{L}$ that are incompatible,
he defines an incompatibility function $\mathcal{I}$ from subsets of
$\mathcal{L}$ to sets of subsets of $\mathcal{L}$:
\[
X \in \mathcal{I}(Y) \text{ iff } \mathsf{Inc}(X \cup Y)
\]
We assume that $\mathcal{I}$ satisfies the
monotonicity requirement (Brandom calls it ``Persistence''):
\[
   \text{If } X \in \mathcal{I}(Y) \text{ and } X \subseteq X' \text{ then } X' \in \mathcal{I}(Y)
\]

\NI Now Brandom defines entailment in terms of the incompatiblity
function. Given a set $X \subseteq \mathcal{L}$ and an individual
sentence $\phi \in \mathcal{L}$:

\[
   X \models \phi\quad \text{ iff }\quad \mathcal{I}(\{\phi\}) \subseteq \mathcal{I}(X)
\]

\NI Now, given material incompatibility (as captured by the
$\mathcal{I}$ function) and entailment, he introduces logical negation
as a \emph{derived} concept via the rule:

\[
   \{\neg \phi\} \in \mathcal{I}(X)\quad \text{ iff }\quad X \models \phi
\]

\NI Brandom goes on to show that the $\neg$ operator, as defined, satisfies
the laws of classical negation.  He also introduces a modal operator,
again defined in terms of material incompatibility, and shows that
this operator satisfies the laws of $S5$.

\Cathoristic{} was inspired by Brandom's vision that material
incompatibility is conceptually prior to logical negation: in other
words, it is possible for a community of language users to make incompatible claims, even if that
language has no explicit logical operators such as negation.  The
language users of this simple language may go on to introduce logical
operators, in order to make certain inferential properties explicit -
but this is an optional further development.  The language before that
addition was already in order as it is.

The approach taken in this paper takes Brandom's original insight in a
different direction.  While Brandom defines an unusual (non
truth-conditional) semantics that applies to any language, we have
defined an unusual logic with a standard (truth-conditional) semantics, and then shown that this logic satisfies the Brandomian connection between incompatibility and entailment.

\subsubsection{Peregrin on negation in incompatibility semantics}\label{peregrin}

Peregrin \cite{peregrin} investigates the  structural
rules that any logic must satisfy if it is to connect incompatibility
($\mathsf{Inc}$) and entailment ($\models$) via the Brandomian
incompatibility semantics constraint:
\[
X \models \phi \quad\text{ iff }\quad \mathcal{I}(\{\phi\}) \subseteq \mathcal{I}(X)
\]

\NI The general structural rules are:
\begin{eqnarray*}
  (\bot) & & \text{if } \mathsf{Inc}(X) \text{ and } X \subseteq Y \text{ then } \mathsf{Inc}(Y) \\
  (\models 1) & & \phi, X \models \phi \\
  (\models 2) & & \text{if }X, \phi \models \psi \text{ and } Y \models \phi \text{ then } X, Y \models \psi \\
  (\bot \models 2) & & \text{if } X \models \phi \text{ for all } \phi, \text{ then } \mathsf{Inc}(X) \\
  (\models \bot 2) & & \text{if } \mathsf{Inc}(Y \cup \{\phi\}) \text{ implies } \mathsf{Inc}(Y \cup X) \text{ for all } Y, \text{ then } X \models \phi
\end{eqnarray*}

\NI Peregrin shows that if a logic satisfied the above laws, then
incompatibility and entailment are mutually interdefinable, and the
logic satisfies the Brandomian incompatibility semantics constraint.

Next, Peregrin gives a pair of laws for defining negation in terms
of $\mathsf{Inc}$ and $\models$\footnote{The converse of $(\neg 2)$
  follows from $(\neg 1)$ and the general structural laws above.}:
\begin{eqnarray*}
  (\neg 1) & & \mathsf{Inc}(\{\phi, \neg \phi\}) \\
  (\neg 2) & & \text{if } \mathsf{Inc}(X, \phi) \text{ then } X \models \neg \phi
\end{eqnarray*}

\NI These laws characterise intuitionistic negation as the
\emph{minimal incompatible}\footnote{$\psi$ is the minimal
  incompatible of $\phi$ iff for all $\xi$, if $\mathsf{Inc}(\{\phi\}
  \cup \{\xi\})$ then $\xi \models \psi$.}.  
  Now, in \cite{brandom},
Brandom defines negation slightly differently. He uses the rule:
\begin{eqnarray*}
  (\neg B) & &\mathsf{Inc}(X, \neg \phi) \text{ iff } X \models \phi
\end{eqnarray*}
Using this stronger rule, we can infer the classical law of
double-negation: $\neg \neg \phi \models \phi$.  Peregrin establishes
that Brandom's rule for negation entail $(\neg 1)$ and $(\neg 2)$
above, but not conversely: Brandom's rule is stronger than Peregrin's
minimal laws $(\neg 1)$ and $(\neg 2)$.\martin{Nothing in our paper is
  about the relationship between negation and incompatibility, so no
  need to intro that stuff here.}

Peregrin concludes that the Brandomian constraint between
incompatibility and entailment is satisfied by many different logics.
Brandom and Aker happened to choose a particular rule for negation
that led to classical logic, but the general connection between
incompatibility and entailment is satisfied by many different logics,
including intuitionistic logic.  This paper supports Peregrin's
conclusion: we have shown that \cathoristic{} also satisfies the
Brandomian constraint.

\martin{I think the subsectin can be much shorter, like so: Peregrin
  \cite{peregrin} investigates the structural rules that any logic
  must satisfy if it is to connect incompatibility ($\mathsf{Inc}$)
  and entailment  as suggested by Brandom.  Peregrin concludes that  many
  logics do this, including intuitionistic logic.  This paper
  supports Peregrin's conclusion since  \cathoristic{}
  also satisfies ...}
  \richard{I feel these sections are important. They show that, pace Brandom, there are many different logics that satisfy the incompatibility semantics constraint. Another reason I don't want to prune them down too much is that these guys, Peregrin and Turbanti, are likely to be two of our early reviewers, and I want to pay them the respect of having read their papers carefully.}

\subsubsection{Peregrin and Turbanti on defining necessity in incompatibility semantics}\label{peregrin}

\martin{As with the previous section, I think this section is too long and connects only loosely
with our paper. I think some of it might fit better into the open question section.}
\richard{Again, I don't want to cut these down just yet. If we are going to show this paper to Peregrin and Turbanti, I believe they will be more receptive if we are more respectful of their work. }

In \cite{brandom}, Brandom gives a rule for defining necessity in terms of incompatibility and entailment:
\[
X \in \mathcal{I}(\{\Box \phi\}) \text{ iff } \mathsf{Inc}(X) \lor \exists Y . \; Y \notin \mathcal{I}(X) \land Y \nvDash \phi
\]
In other words, $X$ is incompatible with $\Box \phi$ if $X$ is compatible with something that does not entail $\phi$.

The trouble is, as Peregrin and Turbanti point out, if $\phi$ is not tautological, then \emph{every set} $X \subseteq \mathcal{L}$ is incompatible with $\Box \phi$.
To show this, take any set $X \subseteq \mathcal{L}$. 
If $\mathsf{Inc}(X)$, then $X \in \mathcal{I}(\Box \phi)$ by definition.
If, on the other hand, $\neg \mathsf{Inc}(X)$, then let $Y = \emptyset$.
Now $\neg \mathsf{Inc}(X \cup Y)$ as $Y = \emptyset$, and $Y \nvDash \phi$ as $\phi$ is not tautological.
Hence $X \in \mathcal{I}(\Box \phi)$ for all $X \subseteq \mathcal{L}$. 
Brandom's rule, then, is only capable of specifying a very specific form of necessity: logical necessity.

In \cite{peregrine} and \cite{turbanti}, Peregrin and Turbanti describe alternative ways of defining necessity.
These alternative rule sets can be used to characterise modal logics other than S5.
For example, Peregrin defines the accessibility relation between worlds in terms of a \emph{compossibility relation}, and then argues that the S4 axiom of transitivity fails because compossibility is not transitive.

We draw two conclusions from this work.
The first is, once again, that a commitment to connecting incompatibility and entailment via the Brandomian constraint:
\[
X \models \phi \text{ iff } \mathcal{I}(\{\phi\}) \subseteq \mathcal{I}(X)
\]
does not commit us to any particular logical system. 
There are a variety of logics that can satisfy this constraint.
Second, questions about the structure of the accessibility relation in Kripke semantics - questions that can seem hopelessly abstract and difficult to answer - can be re-cast in terms of concrete questions about the incompatibility relation.
Incompatibility semantics can shed light on possible-world semantics \cite{turbanti}. 

\subsubsection{Linear logic}

Linear logic \cite{GirardJY:linlog,GirardJY:protyp} is a refinement of
first-order logic and was introduced by J.-Y.~Girard with the aim of
bringing the symmetries of classical logic to constructive
logic. Linear logic has been fruitful in a variety of fields, in
particular in the study of typing systems, where the concept of
linearity puts type-based resource handling on a sound logical basis.

Linear logic splits conjunction into two: additive and multiplicative
conjunction The former, additive conjunction $A \& B$, is especially
interesting in the context of \cathoristic{}. It can be interpreted
\cite{AbramskyS:comintoll} as, in the terminology of process calculus,
an external choice operation. External, because the choice is offered
to the environment.  This interpretation has been influential in the
study of types for process calculus,
e.g.~\cite{HondaK:unitypsfsifLONG,TakeuchiK:intbaslaits,HondaK:lanpriatdfscbp}.
Implicitly, additive conjunction gives an explicit upper bound on how
many different options the environment can choose from. For example in
$A \& B \& C$ we have three options (assuming that none of $A, B, C$
can be decomposed into further additive conjunctions).  With this in
mind, and simplifying a great deal, a key difference between $!A$ and
additive conjunction $A \& B$ is that the individual actions in $!A$
have no continuation, while they do with $A \& B$: $!\{l, r\}$ says
that at this point the only available actions are $l$ and $r$. What
happens at later states is not constraind by $!A$.  In contrast, $A \&
B$ says not only that at this point the only available options are $A$
and $B$, but also that if we choose $A$, then $A$ holds 'for ever',
and likewise for choosing $B$. To be sure, the alternatives in $A \&
B$ may themselves contain further additive conjunctions, and in this
way express how exclusion changes 'over time'.

In summary, \cathoristic{} and linear logic offer  operators that restrict
the available options. How are they related? Linear logic has an
explicit linear negation $(\cdot)^{\bot}$ which, unlike classical
negation, is constructive. In constrast, \cathoristic{} defines a restricted
form negation from $!A$. Can these two perspectives be frutifully
reconciled?

\subsubsection{Process calculus}

Process calculi are models of concurrent computation.  They are based
on the idea of message passing between actors running in parallel.
Labelled transition systems are often used as models for process
calculi, and many concepts used in the development of \cathoristic{},
for example bisimulations and Hennessy-Milner logic, originated in
process theory (although some, such as bisimulation, evolved
independently in other contexts).

Process calculi typically feature a construct called sum, that is an
explicit description of mutually exclusive option:
\[
     \sum_{i \in I} P_i
\]
That is a process that can internally choose, or be chosen to evolve
into the process $P_i$ for each $i$. Once the choice is made, all
other options disappear.  Sums also relate closely to linear logic's
additive conjunction. Is this conceptual proximity a coincidence or
indicative of deeper common structure?


\subsubsection{Failures/divergences in process calculi}

Our cathoristic models are close to a form of the failures/divergences
models that has been used in the denotational semantics of process
calculi, primarily Hoare's CSP \cite{HoareC:comseq,RoscoeAW:theapoc}
and its descendants.  In this model, the denotation of a process $P$
is given as a pair $(traces(P), fail(P))$ where $traces(P) = \{
\sigma\ |\ (\sigma, X) \in \SEMB{P} \}$ are $P$'s traces, and
$fail(P)$ it's failures.  A \emph{failure} is a pair $(\sigma, R)$
where $\sigma \in \Sigma^*$ and $R \subseteq \Sigma$. The intended
interpretation is that a process $P$ has failure $(\sigma, R)$
provided that $\sigma$ is a trace of $P$ and after $P$ executes all
the actions in $\sigma$ it refuses to do any action given in $R$. The
denotation $\SEMB{P}$ of $P$ in a failures/divergences model is the
the set of all of $P$'s failures. The set $ traces(P)$ is
prefix-closed, hence gives rise of a deterministic labelled transition
system: States are given by the set $traces(P)$ of all traces.
Transitions are of the form $\sigma \TRANS{a} \sigma.a$, where
$\sigma.a$ is the string extending $\sigma$ with the action $a$.  The
start state is the empty string.  We can decorate all states as
follows: $ \lambda (\sigma) = \Sigma \setminus R $ provided that
$(\sigma, R) \in \SEMB{P}$.  Whenever the set $\Sigma$ of symbols is
finite, we obtain an cathoristic model this way.

While the failures/divergences semantics of CSP are somewhat more
complicated due to the possibility of diverging programs, the close
connection between cathoristic models and the denotations of CSP processes,
as well as the syntatic similarity with Hennessy-Milner logic suggest
that it might be fruitful to investigate how \cathoristic{} can be used
as a program logic for process calculi.

\subsubsection{Linguistics}

Linguists have also investigated how mutually exclusive alternatives
are expressed, often in the context of antonymy
\cite{OKeeffeA:rouhanocl,AronoffM:hanlin,AllanK:conencos}, but, to the
best of our knowledge have not proposed formal theories of linguistic
exclusion.

\subsection{Open problems} 

In this paper, we have introduced \cathoristic{}, and established key
meta-logical properties such as completeness and compactness. However,
many questions are left open. 

\subsubsection{Excluded middle}

A better understanding of \cathoristic{}'s relationship with first-order
logic might also lead to insight into excluded middle in \cathoristic{}.
The logical law of excluded middle states that either a proposition or
its negation must be true. In \cathoristic{}
\[
\models \phi \lor \neg_S(\phi)
\]

\NI does not hold in general. (The negation operator $\neg_{S}(\cdot)$
was defined in Section \ref{ELAndNegation}) For example, let $\phi$ be
$\langle a \rangle \top$ and $S = \Sigma = \{a, b\}$.  Then
\[
   \phi \lor \neg_{S} \phi 
       \quad=\quad 
   \langle a \rangle \top \; \lor \; ! \{b\} \; \lor \; \langle a \rangle \bot
\]

\NI Now this will not in general be valid - it will be false for
example in the model $((\{x\}, \emptyset, \{(x, \Sigma)\}), x)$, the
model having just the start state (labelled $\Sigma$) and no transitions.
Restricting $S$ to be a proper subset of $\Sigma = \{a, b\}$ is also not
enough. For example with $S = \{a\}$ we have
\[
   \MAY{a}{\TRUE} \lor \neg_{S}(\MAY{a}{\TRUE})
      \quad=\quad
   \MAY{a}{\TRUE}\ \lor\ !\emptyset \lor \MAY{a}{\FALSE}
\]
This formula cannot hold in any cathoristic model which contains a
$b$-labelled transition, but no $a$-transition from the start state.

Is it possible to identify classes of models that nevertheless verify
excluded middle? The answer to this question appears to depend 
on the chosen notion of semantic model.

\subsubsection{Understanding the expressive strength of \cathoristic{}}

Consider the following six languages:

\begin{FIGURE}
\begin{tikzpicture}[node distance=3.3cm,>=stealth',bend angle=45,auto]
  \tikzstyle{place}=[circle,thick,draw=blue!75,fill=blue!20,minimum size=6mm]
  \tikzstyle{red place}=[place,draw=red!75,fill=red!20]
  \tikzstyle{transition}=[rectangle,thick,draw=black!75,
  			  fill=black!20,minimum size=4mm]
  \tikzstyle{every label}=[red]
  \begin{scope}  
    \node [place] (w1) {PL[$\land$]};
    \node [place] (e1) [below of=w1] {PL [$\land, \neg$] };
  \end{scope}
  \begin{scope}[xshift=4cm]
    \node [place] (w1) {HML[$\land$]};
    \node [place] (e1) [below of=w1] {HML [$\land, \neg$] };
  \end{scope} 
  \begin{scope}[xshift=8cm]
    \node [place] (w1) {EL[$\land, !$]};
    \node [place] (e1) [below of=w1] {EL [$\land, !, \neg$] };
  \end{scope}
  \draw (2,0) node {$\subseteq $};
  \draw (6,0) node {$\subseteq$};
  \draw (2,-3) node {$\subseteq $};
  \draw (6,-3) node {$\subseteq$};
  \draw (0,-1.5) node {$\subseteq $};
  \draw (4,-1.5) node {$\subseteq $};
  \draw (8,-1.5) node {$\subseteq $};
\end{tikzpicture}
\caption{Relationships between logics established in this paper. Here $L_1 \subseteq L_2$
means that the logic $L_2$ is a conservative extension of $L_1$. NB: we need to
explain the concept of conservative extension.}\label{figure:relationships}
\end{FIGURE}


\begin{center}
\begin{tabular}{ l | r }
Language & Description \\
\hline
PL[$\land$] & Propositional logic without negation \\
Hennessy-Milner logic[$\land$] & Hennessy-Milner logic without negation \\
CL[$\land, !$] & \Cathoristic{} \\
PL [$\land, \neg$] & Full propositional logic \\
HML [$\land, \neg$] & Full Hennessy-Milner logic \\
CL [$\land, !, \neg$] & \Cathoristic{} with negation\\
\end{tabular}
\end{center}


\NI The top three languages are simple. In each case: there is no
facility for expressing disjunction, every formula that is satisfiable
has a simplest satisfying model, and there is a simple quadratic-time
decision procedure But there are two ways in which CL[$\land, !$] is
more expressive.  Firstly, CL[$\land, !$], unlike HML[$\land$], is expressive enough to be able to distinguish
between any two models that are not bisimilar, cf.~Theorem
\ref{hennessymilnertheorem}.  The second way in which
CL[$\land, !$] is significantly more expressive than both PL[$\land$]
and HML[$\land$] is in its ability to express incompatibility.  No two
formulae of PL[$\land$] or HML[$\land$] are incompatible with each
other\martin{Have we defined incompatibility of PL/HML?}\richard{No, but it is just the same as before: a set of sentences is incompatible if there is no model that satisfies them all.}.  But many
pairs of formulae of CL[$\land, !$] are incompatible.  (For example:
$\langle a \rangle \top$ and $! \emptyset$).  Because CL[$\land, !$] is
expressive enough to be able to make incompatible claims, it satisfies
Brandom's incompatibility semantics constraint.
CL[$\land, !$] is the only logic we are aware of with a
quadratic-time decision procedure that is expressive enough to respect
this constraint. 

The bottom three language can all be decided in exponential time.  We
claim that Hennessy-Milner logic is more expressive than PL, and CL
[$\land, !, \neg$] is more expressive than full Hennessy-Milner logic.
To see that full Hennessy-Milner logic is more expressive than full
propositional logic, fix a propositional logic with the nullary
operator $\top$ plus an infinite number of propositional atoms
$P_{(i,j)}$, indexed by $i$ and $j$.  Now translate each formula of
Hennessy-Milner logic via the rules:
\begin{align*}
  \SEMB{\top}  & =  \top  &
  \SEMB{\phi \land \psi} & =  \SEMB{\phi} \land \SEMB{\psi}  \\
  \SEMB{\neg \phi} & =  \neg \SEMB{\phi}   &
  \SEMB{\langle a_i \rangle \phi_j} & =  P_{(i,j)} 
\end{align*}

\NI We claim Hennessy-Milner logic is more expressive because there
are formulae $\phi$ and $\psi$ of Hennessy-Milner logic such that
\[
\phi \models_{\text{{HML}}} \psi \mbox{ but } \SEMB{\phi} \nvDash_{\text{PL}} \SEMB{\psi}
\]
For example, let $\phi = \langle a \rangle \langle b \rangle \top$ and
$\psi = \langle a \rangle \top$.  Clearly, $\phi \models_{\text{HML}}
\psi$. But $\SEMB{\phi} = P_{(i,j)}$ and $\SEMB{\psi} = P_{(i',j')}$
for some $i,j,i',j'$, and there are no entailments in propositional
logic between arbitrary propositional atoms.

We close with a hint that may indicate \richard{These are weasel words. Can we not be a bit bolder and say ``We claim that''} how CL [$\land, !, \neg$] is
more expressive than full Hennessy-Milner logic: the formula $\fBang
A$ of CL can be translated into Hennessy-Milner logic as:
\[
\bigwedge_{a \in \Sigma - A} \neg \langle a \rangle \top
\]
But if $\Sigma$ is infinite, then this is an infinitary disjunction.
\Cathoristic{} can express the same proposition in a finite sentence.



