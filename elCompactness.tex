\section{Compactness}
\label{compactness}

Here, we prove a compactness theorem via a translation into
first-order logic. 

\subsection{The standard translation from  \ELABR{} into 
            first-order logic}\label{standardTranslation}

We now present a translation from EL to first-order logic. We have two
key purposes in mind:

\begin{itemize}

\item To facilitate the comparison between EL and conventional
  first-order logic, to pin down precisely where EL and first-order
  logic differ and where they don't.

\item To enable technology transfer between EL and other logics. We
  are inspired here in particular by the standard translation of modal
  logic into first-order logic \cite{BlackburnP:modlog} which has
  allows the transfer of interesting results such as compactness to
  modal logic, but has also given rise to many interesting concepts in
  first-order logic.  Historically, fragments of first-order logic
  that were studied were defined by quantifier hierarchies. Modal
  logics picks out very different fragments. For example fragments
  closed under bisimulation, guarded fragments, fragments with
  restricted numbers of variables.

\end{itemize}

\begin{FIGURE}
\begin{center}
\includegraphics[width=8cm]{embedding.pdf}
\end{center}
\caption{The standard translation of eremic logic identifies a fragment of
  first-order logic.\textbf{Is it worth keeping this picture?}}\label{figure:embedding}
\end{FIGURE}



\NI We will translate EL into a restricted fragment of first-order
logic (cf.~Figure \ref{figure:embedding}). The first-order signature
$\SSS$ has a nullary predicate $\top$, a family of unary predicates
$\RESTRICT{A}{\cdot}$, one for each finite subset $A \subseteq
\Sigma$, and a family of binary predicates $\ARROW{a}{x}{y}$, one for
each action $a \in \Sigma$.  The intended interpretation is as
follows.

\begin{itemize}

\item The universe is composed of states.

\item The predicate $\top$, which is true everywhere.

\item For each finite $A \subseteq \Sigma$ and each state $s$,  $\RESTRICT{A}{s}$
is true if 
  $\lambda(x) \subseteq A$.

\item A set of two-place predicates $\ARROW{a}{x}{y}$, one for each $a
  \in \Sigma$, where $x$ and $y$ range over states. $\ARROW{a}{x}{y}$
  is true if $x \xrightarrow{a} y$.


\end{itemize}

\NI Note that if $\Sigma$ is infinite, then $\RESTRICT{A}{\cdot}$ and
$\ARROW{a}{\cdot}{\cdot}$ are infinite families of relations.


 Choose two fixed variables $x, y$, let $a$ range over actions in
$\Sigma$, and $A$ over finite subsets of $\Sigma$. Then the restricted
fragment of FOL that is the target of our translation is given by the
following grammar, where $w, z$ range over $x, y$.

\begin{GRAMMAR}
  \phi 
     &\quad ::= \quad&
  \top \fOr \ARROW{a}{w}{z}\fOr \RESTRICT{A}{z} \fOr \phi \AND \psi \fOr \exists x. \phi 
\end{GRAMMAR}

\NI Notice that this fragment of FOL has no negation, disjunction,
implication, or universal quantification. 

The translations $\SEMB{\phi}_x$ and $\SEMB{\phi}_y$ of an EL formula
$\phi$ are given relative to a state, denoted by either $x$ or $y$.

\[
\begin{array}{rclcrcl}
  \SEMB{\top}_x & \ = \ & \top  
     &\quad& 
  \SEMB{\top}_y & \ = \ & \top 
     \\
  \SEMB{\phi \AND \psi}_x & = & \SEMB{\phi}_x \AND \SEMB{\psi}_x  
     && 
  \SEMB{\phi \AND \psi}_y & = & \SEMB{\phi}_y \AND \SEMB{\psi}_y  
     \\
  \SEMB{\langle a \rangle \phi}_x & = & \exists y.(\ARROW{a}{x}{y} \AND \SEMB{\phi}_y)  
     &&
  \SEMB{\langle a \rangle \phi}_y & = & \exists x.(\ARROW{a}{y}{x} \AND \SEMB{\phi}_x)  
     \\
  \SEMB{\fBang A}_x & = & \RESTRICT{A}{x}
     &&
  \SEMB{\fBang A}_y & = & \RESTRICT{A}{y}
\end{array}
\]

\NI The translations on the left and right are identical except for
switching $x$ and $y$. We continue with some example translations.

\[
   \SEMB{\langle a \rangle \top \AND \fBang \{a\}}_x 
      = 
   \exists y.(\ARROW{a}{x}{y} \AND \top ) \AND \RESTRICT{\{a\}}{x}
\]

\martin{Shall we discuss 1 or 2 more? Maybe one that illustrates the
variable switching?}

\NI We now establish the correctness of the encoding. The key issue
issue is that not every first-order model of our first-order signature
corresponds to an eremic model. The problem is that eremic models have
constraints that are not enforced by our signature, e.g.~$s \TRANS{a}
t$ implies that $a \in \lambda(s)$. Such models are 'junk' from the
point of view of \ELABR{}. 

We deal with this problem following ideas from modal logic
\cite{BlackburnP:modlog}: we add a translation $\SEMB{\LLL}$ for
eremic transition systems, and then prove the following theorem.

\begin{theorem}[correspondence theorem]\label{correspondence:theorem:1}
Let $\phi$ be an \ELABR{} formula and $\MMM = (\LLL, s)$ an eremic
model.
\[
   \MMM \models \phi \quad  \text{iff} \quad \SEMB{\LLL} \models_{x \mapsto s} \SEMB{\phi}_x.
\]
And likewise for $\SEMB{\phi}_y$.
\end{theorem}

\NI Before proving the theorem, we define $\SEMB{\LLL}$, which, given
an eremic transition system $\LLL$, produces a corresponding
first-order model for the signature $\SSS$ introduced above. The
translation is simple.

\begin{definition}
Let $\LLL = (S, \rightarrow, \lambda)$ be an eremic transition
system. Clearly $\LLL$ gives rise to an $\SSS$-model $\SEMB{\LLL}$ as
follows.
\begin{itemize}

\item The universe is the set $S$ of states.

\item The relation symbols are interpreted as follows.

  \begin{itemize}

    \item $\top^{\SEMB{\LLL}}$ always holds.

    \item $\mathsf{Restrict}_{A}^{\SEMB{\LLL}} = \{ s \in S\ |\ \lambda(s) \subseteq A\}$.

    \item $\mathsf{Arrow^{\SEMB{\LLL}}}_{a} = \{(s, t) \in S \times S\ |\ s \TRANS{a} t\}$.

  \end{itemize}
\end{itemize}
\end{definition}

\NI We are now ready to prove Theorem \ref{correspondence:theorem:1}.
\begin{proof}
By induction on the structure of $\phi$. The cases $\top$ and $\phi_1
\AND \phi_2$ are straightfoward.  The case $\MAY{a}\psi$ is handeled
as follows.
\begin{eqnarray*}
  \lefteqn{
  \SEMB{\LLL} \models_{x \mapsto s} \SEMB{\MAY{a}\psi}_x}\hspace{5mm} 
     \\
     &\quad \text{iff}\quad &
  \SEMB{\LLL} \models_{x \mapsto s} \exists y.(\ARROW{a}{x}{y} \AND \SEMB{\psi}_y) 
     \\
     &\text{iff}&
  \text{exists}\ t \in S. \SEMB{\LLL} \models_{x \mapsto s, y \mapsto t} \ARROW{a}{x}{y} \AND \SEMB{\psi}_y
     \\
     &\text{iff}&
  \text{exists}\ t \in S. \SEMB{\LLL} \models_{x \mapsto s, y \mapsto t} \ARROW{a}{x}{y} \ \text{and}\ \SEMB{\LLL} \models_{x \mapsto s, y \mapsto t}  \SEMB{\psi}_y
     \\
     &\text{iff}&
  \text{exists}\ t \in S. s \TRANS{a} t \ \text{and}\ \SEMB{\LLL} \models_{x \mapsto s, y \mapsto t}  \SEMB{\psi}_y
     \\
     &\text{iff}&
  \text{exists}\ t \in S. s \TRANS{a} t \ \text{and}\ \SEMB{\LLL} \models_{y \mapsto t}  \SEMB{\psi}_y \qquad (\text{as $x$ is not free in $\psi$})
     \\
     &\text{iff}&
  \text{exists}\ t \in S. s \TRANS{a} t \ \text{and}\ \MMM \models \psi
     \\
     &\text{iff}&
  \MMM \models \MAY{a}\psi  
\end{eqnarray*}

\NI Finally, if $\phi$ is $!A$ the derivation comes straight from the
definitions.
\begin{eqnarray*}
  \SEMB{\LLL} \models_{x \mapsto s} \SEMB{!A}_x
    &\quad \text{iff}\quad &
  \SEMB{\LLL} \models_{x \mapsto s} \RESTRICT{A}{x}
     \\
     &\text{iff}&
  \lambda(s) \subseteq A
     \\
     &\text{iff}&
  \MMM \models\ !A.
\end{eqnarray*}

\end{proof}

\subsection{Compactness}\label{compactnessProof}

\NI First-order logic has compactness: a set $S$ of sentences has a
model exactly when every finite subset of $S$ does \cite[Chapter
  4.3]{EndertonHB:matinttl}. What about \ELABR{}? 

We can prove compactness of modal logics using the standard
translation from modal to first-order logic \cite{BlackburnP:modlog}:
we start from a set of modal formula such that each finite subset has
a model. We translate the modal formulae and models to first-order
logic, getting a set of first-order formulae such that each finite
subset has a first-order model. By compactness of first-order logic we
obtain a first-order model of the translated modal formulae. Then we
translate that first-order model back to modal logic, obtaining a
model for the original modal formulae, as required.

Unfortunately we cannot do the same with the translation from \ELABR{}
to first-order logic presented in the previous section. The problem
are the first-order models termed 'junk' above: they do not correspond
to eremic transition systems.  For example the constraint $s
\TRANS{a} t$ implies $a \in \lambda(s)$ might be violated. After all,
merely having signature $\SSS$ is not strong enough a constraint. The
target language of the translation from the previous section is not
expressive enough to have formulae that can guarantee such
constraints.  As we have no reason to belive that the first-order
model whose existence is guarnateed by compactness isn't 'junk', we
cannot use this translation.

We solve this problem with a second translation from \ELABR{} to
first-order logic, this time into a more expressive fragment were we
can constrain first-order models enough to ensure that they always can
be translated back to \ELFULL{}.

The second embedding translates \ELFULL{} to two-sorted first-order
logic. The many-sorted first-order signature $\SSS'$ is given as follows.
\begin{itemize}

\item $\SSS'$ has two sorts, states and actions. 

\item The action constants are given by $\Sigma$. There
are no state constants. 

\item $\SSS'$ has a nullary predicate $\top$.

\item A binary predicate $\ALLOWED{}{\cdot}{\cdot}$. The intended
  meaning of $\ALLOWED{}{x}{a}$ is that at the state denoted by $x$ we
  are allowed to do the action $a$.

\item A ternary predicate $\ARROWTWO{}{\cdot}{\cdot}{\cdot}$ where
  $\ARROWTWO{}{x}{a}{y}$ means that there is a transition from the
  state denoted by $x$ to the state denoted by $y$, and that
  transition is labelled $a$.

\end{itemize}

\NI So $\SSS'$ is a relational signature, i.e.~has no function
symbols.  The intended interpretation should should be
clear.\martin{Phrase this better.}

With the target logic in place, we can now present a second encoding
$\SEMBTWO{\phi}_x$ of \ELABR{} formulae.

\begin{eqnarray*}
  \SEMBTWO{\top}_x & \ = \ & \top
     \\
  \SEMBTWO{\phi \AND \psi}_x & = & \SEMBTWO{\phi}_x \AND \SEMBTWO{\psi}_x
     \\
  \SEMBTWO{\langle a \rangle \phi}_x & = & \exists^{st} y.(\ARROWTWO{}{x}{a}{y} \AND \SEMBTWO{\phi}_y)
     \\
  \SEMBTWO{\fBang A}_x & = & \forall^{act} a.(\ALLOWED{}{x}{a} \IMPLIES a \in A) 
\end{eqnarray*}

\NI Here we use $\exists^{st}$ to indicate that the quantifier ranges
of the sort of states, and $\forall^{act}$ for a quantifier ranging
over actions. The expression $a \in A$ is a shorthand for the
first-order formula
\[
   a = a_1 \OR a = a_2 \OR \cdots \OR a = a_n
\]
assuming that $A = \{a_1, ..., a_n\}$. Since by definition, $A$ is always a finite
set, this is well-defined.

Note that the translation above could be restricted to a two-variable
fragment. Moreover, the standard reduction from many-sorted to
one-sorted first-order logic, does not increase the number of
variables used (although predicates are added, one per sort)
\cite{EndertonHB:matinttl}. For simplicity we will not consider this
matter further.

Before we can state and prove a correspondence theorem for
$\SEMBTWO{\phi}_x$ along the lines of Theorem
\ref{correspondence:theorem:1}, we must also translate eremic
  transition systems $\SEMBTWO{\LLL}$.

\begin{definition}
Let $\LLL = (S, \rightarrow, \lambda)$ be an eremic transition
system. Clearly $\LLL$ gives rise to an $\SSS'$-model $\SEMBTWO{\LLL}$
as follows.
\begin{itemize}

\item For each constant $a \in \Sigma$, $a^{\SEMBTWO{\LLL}}$ is $a$ itself.

\item The sort of states is interpreted by the set $S$.

\item The sort of actions is interpreted by the set $\Sigma$.

\item The relation symbols are interpreted as follows.

  \begin{itemize}

    \item $\top^{\SEMBTWO{\LLL}}$ always holds.

    \item $\ALLOWED{\SEMBTWO{\LLL}}{s}{a}$ holds whenever $a \in \lambda(s)$.

    \item $\ARROWTWO{\SEMBTWO{\LLL}}{s}{a}{t}$ holds whenver $s \TRANS{a} t$.

  \end{itemize}
\end{itemize}
\end{definition}


\begin{theorem}[correspondence theorem]\label{correspondence:theorem:2}
Let $\phi$ be an \ELABR{} formula and $\MMM = (\LLL, s)$ an eremic
model.
\[
   \MMM \models \phi \quad  \text{iff} \quad \SEMBTWO{\LLL} \models_{x \mapsto s} \SEMBTWO{\phi}_x.
\]
\end{theorem}
\begin{proof}
The proof proceeds by induction on the structure of $\phi$ and is
similar to that of Theorem \ref{correspondence:theorem:2}.
The case for the may modality proceeds as follows.  

\begin{alignat*}{2}
  \MMM \models \MAY{a}\phi
     &\quad\text{iff}\quad 
  \text{exists state $t$ with }\ s \TRANS{a} t\ \text{and}\ (\LLL, t) \models \phi \\
     &\quad\text{iff}\quad
  \text{exists state $t$ with }\ s \TRANS{a} t\ \text{and}\ \SEMBTWO{\LLL} \models_{y \mapsto t} \SEMBTWO{\phi}_y &\qquad& \text{by (IH)}\\
     &\quad\text{iff}\quad
  \SEMBTWO{\LLL} \models_{x \mapsto s} \exists^{st} y.(\ARROWTWO{}{x}{a}{y} \AND \SEMBTWO{\phi}_y) \\
     &\quad\text{iff}\quad
   \SEMBTWO{\LLL} \models_{x \mapsto s} \SEMBTWO{\MAY{a}{\phi}}_x
\end{alignat*}
\martin{the 2nd to last inference may need elaboration!}

Finally $!A$.
\begin{alignat*}{2}
  \MMM \models !A
     &\quad\text{iff}\quad
  \lambda (s) \subseteq A \\
      &\quad\text{iff}\quad
  \text{for all }\ a \in \Sigma. a \in A \\
     &\quad\text{iff}\quad
  \SEMBTWO{\LLL} \models_{x \mapsto s} \forall^{act} a. (\ALLOWED{}{x}{a} \IMPLIES a \in A) \\
     &\quad\text{iff}\quad
  \SEMBTWO{\LLL} \models_{x \mapsto s} \SEMBTWO{!A}_x
\end{alignat*}
\martin{again, some steps need elaboration? Also should this go in the appendix?}

\end{proof}

\NI We now use the translation $\SEMBTWO{\phi}_x$ to show that \ELABR{} must also have compactness. The key steps in the proof are
simple, following standard techniques from modal logic
\cite{BlackburnP:modlog}:

\begin{enumerate}

\item Choose a set $T$ of \ELABR{} formulae, such that each finite
  subset $T'$ of $T$ has an eremic model $(\LLL, s)$.

\item Using the translations gives a set $\SEMBTWO{T} =
  \{\SEMBTWO{\phi}\ |\ \phi \in T\}$ of first-order formulae such that
  each finite subset has a first-order model $\SEMBTWO{\LLL}$.

\item By compactness of first-order logic, we can find a first-order
  model $\CAL{M}$ of $\SEMBTWO{T}$.

\item\label{compactness:step:4} Convert $\CAL{M}$ into an eremic transition system
  $\CAL{M}^{\sharp}$ such that $(\CAL{M}^{\sharp}, s) \models T$.

\end{enumerate}

\NI The problematic step is (\ref{compactness:step:4}), for how would
we know that the model $\CAL{M}$ can be converted back to an eremic
transition system? Why should $\CAL{M}$ exhibit admissibility or
well-sizedness? Why would the transition relation be deterministic?
The mere fact that $\CAL{M}$ is a first-order model of signature
$\SSS'$ is not strong enough to guarantee these properties.  We deal
with this in two ways. To ensure admissibility, we define a formula
that guarantees that models satisfying the formula are admissible.
\begin{eqnarray*}
   \phi_{admis} 
      & \ =\ &
   \forall^{st} s.\forall^{act} a.\forall^{st} t.( \ARROWTWO{}{s}{a}{t} \IMPLIES \ALLOWED{}{s}{a}) \\
   \phi_{det} 
      & \ =\ &
   \forall^{st} s.\forall^{act} a.\forall^{st} t.\forall^{st} t'.
   ((\ARROWTWO{}{s}{a}{t}  \AND \ARROWTWO{}{s}{a}{t'} ) \IMPLIES t = t' )   
\end{eqnarray*}

\NI Clearly, whenever a model satisfies $\phi_{admis}$, then it must
be admissible, and whenever it satisfies $\phi_{det}$ is is a
deterministic transition system.

\begin{lemma}\label{compactness:lemma:23399}
If $\LLL$ is an eremic transistion system then $\SEMBTWO{\LLL} \models
\phi_{admis} \AND \phi_{det}$.
\end{lemma}

\begin{proof}
Straightforward from the definitions.

\end{proof}

\NI We can now add, without changing satisfiability, $\phi_{admis}
\AND \phi_{det}$ to any set of first-order formulae that has a model
that is the translation of an eremic model.

%% We deal with the absence of well-sizedness by 

%% First we add a formula $\phi_a$ for each action $a \in \Sigma$.
%% \begin{eqnarray*}
%%    \phi_{a} 
%%       & = &
%%    \exists^{act} c. a = c 
%% \end{eqnarray*}

%% \begin{lemma}\label{compactness:lemma:666}
%% If $\LLL$ is an eremic transistion system then $\SEMBTWO{\LLL} \models
%% \phi_{a}$ for all $a \in \Sigma$.
%% \end{lemma}

%% \begin{proof}
%% Straightforward from the definitions.
%% 
%% \end{proof}

\begin{definition}
Let $\LLL = (S, \rightarrow, \lambda)$ be an eremic transition system
and $X$ a set, condidered to contain actions. The \emph{restriction of
  $\LLL$ to $X$}, written $\LLL \setminus X$ is the eremic model $(S,
\rightarrow', \lambda')$ where $\rightarrow' = \{(s, a, t) \in
\rightarrow \ |\ a \notin X\}$, and for all states $s$ we set:
\[
   \lambda'(s) 
        =
   \begin{cases}
       \lambda(s) \setminus  X & \text{whenever}\ \lambda(s) \neq \Sigma \\
       \Sigma & \text{otherwise}
   \end{cases}
\]

\end{definition}

\begin{lemma}\label{compactness:lemma:1717}
Let $\phi$ be an \ELABR{} formula and $X$ be a set such that no action
occuring in $\phi$ is in $X$. Then:
\[
   (\LLL, s) \models \phi
      \quad\text{iff}\quad
   (\LLL \setminus X, s) \models \phi.
\]
\end{lemma}
\begin{proof}
By straightforward induction on the structure of $\phi$, using the
fact that by assumption $X$ only contains actions not occuring in
$\phi$.  
\end{proof}

\begin{definition}
Let $\CAL{M}$ be a first-order model for the signature $\SSS'$.
We construct an eremic transition system
$\CAL{M}^{\sharp} = (S, \rightarrow, \lambda)$.
\begin{itemize}

\item The actions $\Sigma$ are given by the $\CAL{M}$ interpretation of actions.

\item The states $S$ are given by the $\CAL{M}$ interpretation of states.

\item The reduction relation $s \TRANS{a} t$ holds exactly when
  $\ARROWTWO{\CAL{M}}{s}{a}{t}$.

\item The function $\lambda$ is given by the following clause:
  \[
     \lambda(s) 
        =
     \begin{cases} 
       X & \text{whenever}\ X = \{a \ |\ \ALLOWED{\CAL{M}}{s}{a} \}\ \text{ is finite} \\
       \Sigma & \text{otherwise}
     \end{cases}
  \]

\end{itemize}

\end{definition}

\begin{lemma}
If $\CAL{M}$ be a first-order model for $\SSS'$ such that $\CAL{M}
\models \phi_{admis} \AND \phi_{det}$.  Then $\CAL{M}^{\sharp}$ is an
eremic transition system with actions $\Sigma$.
\end{lemma}
\begin{proof}
Immediate from the definitions.

\end{proof}

\begin{theorem}[correspondence theorem]\label{correspondence:theorem:223}
Let $\CAL{M}$ be a first-order model for the signature $\SSS'$ such
that $\CAL{M} \models \phi_{admis} \AND \phi_{det}$.  Then we have for
all \ELABR{} formulae $\phi$ with actions from $\Sigma$:
\[
   \CAL{M} \models_{x \mapsto s} \SEMBTWO{\phi}_x 
        \quad  \text{iff} \quad 
   (\CAL{M}^{\sharp} \setminus X, s) \models \phi.
\]
\end{theorem}
Here $X$ is the set of all elements in the universe of $\CAL{M}$ interpreting
actions that are not in $\Sigma$.
\begin{proof}
The proof proceeds by induction on the structure of $\phi$. \textbf{To do.}
\end{proof}

\begin{definition}
Let $T$ be a set of \ELABR{} formulae, and $\MMM$ an eremic model.  We
write $\MMM \models T$ provided $\MMM \models \phi$ for all $\phi \in
T$.  We say $T$ is \emph{satisfiable} provided $\MMM \models T$.
\end{definition}

\begin{theorem}[Compactness of \ELFULL{}]
A set $T$ of \ELABR{} formulae is satisfiable iff each finite subset of
$T$ is satisfiable.
\end{theorem}
\begin{proof}
For the non-trivial direction, let $T$ be a set of \ELABR{} formulae
such that any finite subset has an eremic model. Define 
\[
  \SEMBTWO{T} 
     \ =\ 
  \{\SEMBTWO{\phi}\ |\ \phi \in T\} 
     \qquad\qquad
  T^*
     \ =\ 
  \SEMBTWO{T} \cup \{\phi_{admis} \AND  \phi_{det}\}
\]
which both are sets of first-order formulae. Clearly each finite subset $T'$ of 
$T^*$ has a first-order model. Why? First consider the subset $T'_{EL}$ of $T'$
which is given as follows.
\[
   T'_{EL} \ =\ \{ \phi \in T\ |\ \SEMBTWO{\phi} \in T' \}
\]
Since $T'_{EL}$ is finite, by assumption there is an eremic model 
\[
   (\LLL, s) \models T'_{EL}
\]
which means we can apply Theorem \ref{correspondence:theorem:223} to get
\[
   \SEMBTWO{\LLL} \models_{x \mapsto s} \SEMBTWO{T'_{EL}},
\]
By construction $T' \setminus \SEMBTWO{T'_{EL}} \subseteq
\{\phi_{admis} \AND \phi_{det}\}$, so all we have to
show for $T'$ to have a model is that
\[
    \SEMBTWO{\LLL} \models_{x \mapsto s} \{\phi_{admis}\} \cup \{ \phi_a\ |\ a \in \Sigma\},
\]
but that is a direct consequence of Lemma
\ref{compactness:lemma:23399}.  That
means each finite subset of $T^*$ has a model and by appealing to
compactness of first-order many-sorted logic (which is an immediate
consequence of compactenss of one-sorted first-order logic
\cite{EndertonHB:matinttl}), we know there must be a first-order model
$\CAL{M}$ of $T^*$, i.e.
\[
   \CAL{M} \models T^*.
\]
Since $\CAL{M} \models \phi_{admis} \AND \phi_{det}$ we can apply
Theorem \ref{correspondence:theorem:223} that also
\[
   (\CAL{M}^{\sharp} \setminus X, s) \models T
\]
where $X$ is the set of all actions in $\CAL{M}^{\sharp}$ that are not
in $\Sigma$. Hence $T$ is satisfiable.
 
\end{proof}





