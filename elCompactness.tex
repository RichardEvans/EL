\section{Compactness and the standard translation to first-order logic }
\label{compactness}

This section studies two embeddings of \cathoristic{} into first-order
logic. The second embedding is used to prove that \cathoristic{} satisfies compactness.

\subsection{Translating from  cathoristic to
            first-order logic}\label{standardTranslation}

We study how a logic embeds into other logics in parts because it
casts a new light on the logic that is the target of the embedding.  A
good example is the standard translation of modal into first-order
logic.  The translation produces various fragments: the finite
variable fragments, the fragment closed under bisimulation, 
guarded fragments.  These fragments have been investigated deeply, and
found to have unusual properties not shared by the whole of \fol.
Translations also enable us to push techniques, constructions and
results between logics.

%\begin{FIGURE}
\begin{center}
\includegraphics[width=8cm]{embedding.pdf}
\end{center}
\caption{The standard translation of eremic logic identifies a fragment of
  first-order logic.\textbf{Is it worth keeping this picture?}}\label{figure:embedding}
\end{FIGURE}



%In this section, we translate \cathoristic{} into a restricted fragment of
%first-order logic.

\begin{definition}
The first-order signature $\SSS$ has a nullary predicate $\top$, a
family of unary predicates $\RESTRICT{A}{\cdot}$, one for each finite
subset $A \subseteq \Sigma$, and a family of binary predicates
$\ARROW{a}{x}{y}$, one for each action $a \in \Sigma$. 

\end{definition}

\NI The intended interpretation is as follows.

\begin{itemize}

\item The universe is composed of states.

\item The predicate $\top$ is true everywhere.

\item For each finite $A \subseteq \Sigma$ and each state $s$,  $\RESTRICT{A}{s}$
is true if 
  $\lambda(x) \subseteq A$.

\item A set of two-place predicates $\ARROW{a}{x}{y}$, one for each $a
  \in \Sigma$, where $x$ and $y$ range over states. $\ARROW{a}{x}{y}$
  is true if $x \xrightarrow{a} y$.


\end{itemize}

\NI If $\Sigma$ is infinite, then $\RESTRICT{A}{\cdot}$ and
$\ARROW{a}{\cdot}{\cdot}$ are infinite families of relations.

\begin{definition}
 Choose two fixed variables $x, y$, let $a$ range over actions in
$\Sigma$, and $A$ over finite subsets of $\Sigma$. Then the restricted
fragment of first-order logic that is the target of our translation is given by the
following grammar, where $w, z$ range over $x, y$.

\begin{GRAMMAR}
  \phi 
     &\quad ::= \quad&
  \top \fOr \ARROW{a}{w}{z}\fOr \RESTRICT{A}{z} \fOr \phi \AND \psi \fOr \exists x. \phi 
\end{GRAMMAR}

\end{definition}

\NI This fragment has no negation, disjunction, implication, or
universal quantification.

\begin{definition}
The translations $\SEMB{\phi}_x$ and $\SEMB{\phi}_y$ of cathoristic formula 
$\phi$ are given relative to a state, denoted by either $x$ or $y$.

\[
\begin{array}{rclcrcl}
  \SEMB{\top}_x & \ = \ & \top  
     &\quad& 
  \SEMB{\top}_y & \ = \ & \top 
     \\
  \SEMB{\phi \AND \psi}_x & = & \SEMB{\phi}_x \AND \SEMB{\psi}_x  
     && 
  \SEMB{\phi \AND \psi}_y & = & \SEMB{\phi}_y \AND \SEMB{\psi}_y  
     \\
  \SEMB{\langle a \rangle \phi}_x & = & \exists y.(\ARROW{a}{x}{y} \AND \SEMB{\phi}_y)  
     &&
  \SEMB{\langle a \rangle \phi}_y & = & \exists x.(\ARROW{a}{y}{x} \AND \SEMB{\phi}_x)  
     \\
  \SEMB{\fBang A}_x & = & \RESTRICT{A}{x}
     &&
  \SEMB{\fBang A}_y & = & \RESTRICT{A}{y}
\end{array}
\]

\end{definition}

\NI The translations on the left and right are identical, except for
switching $x$ and $y$. Here is an example translation.
\[
   \SEMB{\langle a \rangle \top \AND \fBang \{a\}}_x 
      = 
   \exists y.(\ARROW{a}{x}{y} \AND \top ) \AND \RESTRICT{\{a\}}{x}
\]

\NI We now establish the correctness of the encoding. The key issue is
that not every first-order model of our first-order signature
corresponds to a cathoristic model because determinism, well-sizedness and
admissibility are not enforced by our signature alone. In other words,
models may contain `junk'.  We deal with this problem following ideas
from modal logic \cite{BlackburnP:modlog}: we add a translation
$\SEMB{\LLL}$ for cathoristic transition systems, and then prove the
following theorem.

\begin{theorem}[correspondence theorem]\label{correspondence:theorem:1}
Let $\phi$ be a \cathoristic{} formula and $\MMM = (\LLL, s)$ a cathoristic
model.
\[
   \MMM \models \phi \quad  \text{iff} \quad \SEMB{\LLL} \models_{x \mapsto s} \SEMB{\phi}_x.
\]
And likewise for $\SEMB{\phi}_y$.
\end{theorem}

\NI The definition of $\SEMB{\LLL}$ is simple.

\begin{definition}
Let $\LLL = (S, \rightarrow, \lambda)$ be a cathoristic transition
system. Clearly $\LLL$ gives rise to an $\SSS$-model $\SEMB{\LLL}$ as
follows.
\begin{itemize}

\item The universe is the set $S$ of states.

\item The relation symbols are interpreted as follows.

  \begin{itemize}

    \item $\top^{\SEMB{\LLL}}$ always holds.

    \item $\mathsf{Restrict}_{A}^{\SEMB{\LLL}} = \{ s \in S\ |\ \lambda(s) \subseteq A\}$.

    \item $\mathsf{Arrow^{\SEMB{\LLL}}}_{a} = \{(s, t) \in S \times S\ |\ s \TRANS{a} t\}$.

  \end{itemize}
\end{itemize}
\end{definition}

\NI We are now ready to prove Theorem \ref{correspondence:theorem:1}.
\begin{proof}
By induction on the structure of $\phi$. The cases $\top$ and $\phi_1
\AND \phi_2$ are straightforward.  The case $\MAY{a}\psi$ is handled
as follows.
\begin{eqnarray*}
  \lefteqn{
  \SEMB{\LLL} \models_{x \mapsto s} \SEMB{\MAY{a}\psi}_x}\hspace{5mm} 
     \\
     &\quad \text{iff}\quad &
  \SEMB{\LLL} \models_{x \mapsto s} \exists y.(\ARROW{a}{x}{y} \AND \SEMB{\psi}_y) 
     \\
     &\text{iff}&
  \text{exists}\ t \in S. \SEMB{\LLL} \models_{x \mapsto s, y \mapsto t} \ARROW{a}{x}{y} \AND \SEMB{\psi}_y
     \\
     &\text{iff}&
  \text{exists}\ t \in S. \SEMB{\LLL} \models_{x \mapsto s, y \mapsto t} \ARROW{a}{x}{y} \ \text{and}\ \SEMB{\LLL} \models_{x \mapsto s, y \mapsto t}  \SEMB{\psi}_y
     \\
     &\text{iff}&
  \text{exists}\ t \in S. s \TRANS{a} t \ \text{and}\ \SEMB{\LLL} \models_{x \mapsto s, y \mapsto t}  \SEMB{\psi}_y
     \\
     &\text{iff}&
  \text{exists}\ t \in S. s \TRANS{a} t \ \text{and}\ \SEMB{\LLL} \models_{y \mapsto t}  \SEMB{\psi}_y \qquad (\text{as $x$ is not free in $\psi$})
     \\
     &\text{iff}&
  \text{exists}\ t \in S. s \TRANS{a} t \ \text{and}\ \MMM \models \psi
     \\
     &\text{iff}&
  \MMM \models \MAY{a}\psi  
\end{eqnarray*}

\NI Finally, if $\phi$ is $!A$ the derivation comes straight from the
definitions.
\begin{eqnarray*}
  \SEMB{\LLL} \models_{x \mapsto s} \SEMB{!A}_x
    &\quad \text{iff}\quad &
  \SEMB{\LLL} \models_{x \mapsto s} \RESTRICT{A}{x}
     \\
     &\text{iff}&
  \lambda(s) \subseteq A
     \\
     &\text{iff}&
  \MMM \models\ !A.
\end{eqnarray*}
\end{proof}

\subsection{Compactness by translation}\label{compactnessProof}

\NI First-order logic satisfies \emph{compactness}: a set $S$ of sentences has a
model exactly when every finite subset of $S$ does. What about
\cathoristic{}?

We can prove compactness of modal logics using the standard
translation from modal to first-order logic \cite{BlackburnP:modlog}:
we start from a set of modal formula such that each finite subset has
a model. We translate the modal formulae and models to first-order
logic, getting a set of first-order formulae such that each finite
subset has a first-order model. By compactness of first-order logic, we
obtain a first-order model of the translated modal formulae. Then we
translate that first-order model back to modal logic, obtaining a
model for the original modal formulae, as required. The last step
proceeds without a hitch because the modal and the first-order notions
of model are identical, save for details of presentation.

Unfortunately we cannot do the same with the translation from
\cathoristic{} to first-order logic presented in the previous
section. The problem are the first-order models termed `junk' above.
The target language of the translation is not expressive enough to
have formulae that can guarantee such constraints.  As we have no
reason to believe that the first-order model whose existence is
guaranteed by compactness isn't `junk', we cannot prove compactness
with the translation.  We solve this problem with a second translation, this time
into a more expressive first-order fragment where we can constrain
first-order models easily using formulae. The fragment we use now
lives in two-sorted first-order logic (which can easily be reduced to
first-order logic \cite{EndertonHB:matinttl}).

\begin{definition}
The two-sorted first-order signature $\SSS'$ is given as follows.
\begin{itemize}

\item $\SSS'$ has two sorts, states and actions. 

\item The action constants are given by $\Sigma$. There
are no state constants. 

\item $\SSS'$ has a nullary predicate $\top$.

\item A binary predicate $\ALLOWED{}{\cdot}{\cdot}$. The intended
  meaning of $\ALLOWED{}{x}{a}$ is that at the state denoted by $x$ we
  are allowed to do the action $a$.

\item A ternary predicate $\ARROWTWO{}{\cdot}{\cdot}{\cdot}$ where
  $\ARROWTWO{}{x}{a}{y}$ means that there is a transition from the
  state denoted by $x$ to the state denoted by $y$, and that
  transition is labelled $a$.

\end{itemize}
\end{definition}

\NI So $\SSS'$ is a relational signature, i.e.~has no function
symbols. 

\begin{definition}
The encoding $\SEMBTWO{\phi}_x$ of \cathoristic{} formulae is given by the following clauses.
\begin{eqnarray*}
  \SEMBTWO{\top}_x & \ = \ & \top
     \\
  \SEMBTWO{\phi \AND \psi}_x & = & \SEMBTWO{\phi}_x \AND \SEMBTWO{\psi}_x
     \\
  \SEMBTWO{\langle a \rangle \phi}_x & = & \exists^{st} y.(\ARROWTWO{}{x}{a}{y} \AND \SEMBTWO{\phi}_y)
     \\
  \SEMBTWO{\fBang A}_x & = & \forall^{act} a.(\ALLOWED{}{x}{a} \IMPLIES a \in A) 
\end{eqnarray*}

\end{definition}

\NI Here we use $\exists^{st}$ to indicate that this existential
quantifier ranges over the sort of states, and $\forall^{act}$ for the
universal quantifier ranging over actions. The expression $a \in A$ is
a shorthand for the first-order formula
\[
   a = a_1 \OR a = a_2 \OR \cdots \OR a = a_n
\]
assuming that $A = \{a_1, ..., a_n\}$. Since by definition, $A$ is
always a finite set, this is well-defined.  The translation could be
restricted to a two-variable fragment. Moreover, the standard
reduction from many-sorted to one-sorted first-order logic does not
increase the number of variables used (although predicates are added,
one per sort). We will not consider this matter further here.  

We also translate cathoristic transition systems $\SEMBTWO{\LLL}$.

\begin{definition}
Let $\LLL = (S, \rightarrow, \lambda)$ be a cathoristic transition
system. $\LLL$ gives rise to an $\SSS'$-model $\SEMBTWO{\LLL}$
as follows.
\begin{itemize}

\item The sort of states is interpreted by the set $S$.

\item The sort of actions is interpreted by the set $\Sigma$.

\item For each constant $a \in \Sigma$, $a^{\SEMBTWO{\LLL}}$ is $a$ itself.

\item The relation symbols are interpreted as follows.

  \begin{itemize}

    \item $\top^{\SEMBTWO{\LLL}}$ always holds.

    \item $\ALLOWED{\SEMBTWO{\LLL}}{s}{a}$ holds whenever $a \in \lambda(s)$.

    \item $\ARROWTWO{\SEMBTWO{\LLL}}{s}{a}{t}$ holds whenever $s \TRANS{a} t$.

  \end{itemize}
\end{itemize}
\end{definition}


\begin{theorem}[correspondence theorem]\label{correspondence:theorem:2}
Let $\phi$ be a \cathoristic{} formula and $\MMM = (\LLL, s)$ a cathoristic
model.
\[
   \MMM \models \phi \quad  \text{iff} \quad \SEMBTWO{\LLL} \models_{x \mapsto s} \SEMBTWO{\phi}_x.
\]
\end{theorem}
\begin{proof}
The proof proceeds by induction on the structure of $\phi$ and is
similar to that of Theorem \ref{correspondence:theorem:2}.
The case for the may modality proceeds as follows.  

\begin{alignat*}{2}
  \MMM \models \MAY{a}\phi
     &\quad\text{iff}\quad 
  \text{exists state $t$ with }\ s \TRANS{a} t\ \text{and}\ (\LLL, t) \models \phi \\
     &\quad\text{iff}\quad
  \text{exists state $t$ with }\ s \TRANS{a} t\ \text{and}\ \SEMBTWO{\LLL} \models_{y \mapsto t} \SEMBTWO{\phi}_y &\qquad& \text{by (IH)}\\
     &\quad\text{iff}\quad
  \SEMBTWO{\LLL} \models_{x \mapsto s} \exists^{st} y.(\ARROWTWO{}{x}{a}{y} \AND \SEMBTWO{\phi}_y) \\
     &\quad\text{iff}\quad
   \SEMBTWO{\LLL} \models_{x \mapsto s} \SEMBTWO{\MAY{a}{\phi}}_x
\end{alignat*}

Finally $!A$.
\begin{alignat*}{2}
  \MMM \models !A
     &\quad\text{iff}\quad
  \lambda (s) \subseteq A \\
      &\quad\text{iff}\quad
  \text{for all }\ a \in \Sigma. a \in A \\
     &\quad\text{iff}\quad
  \SEMBTWO{\LLL} \models_{x \mapsto s} \forall^{act} a. (\ALLOWED{}{x}{a} \IMPLIES a \in A) \\
     &\quad\text{iff}\quad
  \SEMBTWO{\LLL} \models_{x \mapsto s} \SEMBTWO{!A}_x
\end{alignat*}
\end{proof}

\NI We use the following steps in our compactness proof.

\begin{enumerate}

\item Choose a set $\Gamma$ of \cathoristic{} formulae such that each finite
  subset $\Gamma'$ of $\Gamma$ has a cathoristic model $(\LLL, s)$.

\item The translation gives a set $\SEMBTWO{\Gamma} =
  \{\SEMBTWO{\phi}\ |\ \phi \in \Gamma\}$ of first-order formulae such that
  each finite subset has a first-order model $\SEMBTWO{\LLL}$.

\item By compactness of (two-sorted) first-order logic, we can find a
  first-order model $\CAL{M}$ of $\SEMBTWO{\Gamma}$.

\item\label{compactness:step:4} Convert $\CAL{M}$ into a cathoristic transition system
  $\CAL{M}^{\sharp}$ such that $(\CAL{M}^{\sharp}, s) \models \Gamma$.

\end{enumerate}

\NI The problematic step is (\ref{compactness:step:4}) - for how would
we know that the first-order model $\CAL{M}$ can be converted back to
a cathoristic transition system? What if it contains `junk' in the
sense described above?  
We solve this by adding formulae to 
$\SEMBTWO{\Gamma}$ that preserve finite satisfiability but force the
first-order models to be convertible to cathoristic models.
 To ensure admissibility we use this formula.
\begin{eqnarray*}
   \phi_{admis} 
      & \ =\ &
   \forall^{st} s.\forall^{act} a.\forall^{st} t.( \ARROWTWO{}{s}{a}{t} \IMPLIES \ALLOWED{}{s}{a}) 
\end{eqnarray*}

\NI The formula $\phi_{det}$ ensures model determinism.
\begin{eqnarray*}
   \phi_{det} 
      & \ =\ &
   \forall^{st} s.\forall^{act} a.\forall^{st} t.\forall^{st} t'.
   ((\ARROWTWO{}{s}{a}{t}  \AND \ARROWTWO{}{s}{a}{t'} ) \IMPLIES t = t' )   
\end{eqnarray*}

\begin{lemma}\label{compactness:lemma:23399}
If $\LLL$ is a cathoristic transition system then $\SEMBTWO{\LLL} \models
\phi_{admis} \AND \phi_{det}$.
\end{lemma}

\begin{proof}
Straightforward from the definitions.
\end{proof}

We can now add, without changing satisfiability, $\phi_{admis}
\AND \phi_{det}$ to any set of first-order formulae that has a model
that is the translation of a cathoristic model.

We also need to deal with well-sizedness in first-order models,
because nothing discussed so far prevents models whose state labels are
infinite sets without being $\Sigma$.  Moreover, a model may interpret
the set of actions with a proper superset of $\Sigma$.  This also
prevents conversion to cathoristic models. We solve these problems by
simply removing all actions that are not in $\Sigma$ and all
transitions involving such actions.  We map all
infinite state labels to $\Sigma$. It is easy to see that this does not
change satisfiability of (translations of) cathoristic formulae.

%% First we add a formula $\phi_a$ for each action $a \in \Sigma$.
%% \begin{eqnarray*}
%%    \phi_{a} 
%%       & = &
%%    \exists^{act} c. a = c 
%% \end{eqnarray*}

%% \begin{lemma}\label{compactness:lemma:666}
%% If $\LLL$ is a cathoristic transistion system then $\SEMBTWO{\LLL} \models
%% \phi_{a}$ for all $a \in \Sigma$.
%% \end{lemma}

%% \begin{proof}
%% Straightforward from the definitions.
%% 
%% \end{proof}

\begin{definition}
Let $\LLL = (S, \rightarrow, \lambda)$ be a cathoristic transition system
and $X$ a set, containing actions. The \emph{restriction of
  $\LLL$ to $X$}, written $\LLL \setminus X$ is the cathoristic model $(S,
\rightarrow', \lambda')$ where $\rightarrow' = \{(s, a, t) \in
\rightarrow \ |\ a \notin X\}$, and for all states $s$ we set:
\[
   \lambda'(s) 
        =
   \begin{cases}
       \lambda(s) \setminus  X & \text{whenever}\ \lambda(s) \neq \Sigma \\
       \Sigma & \text{otherwise}
   \end{cases}
\]

\end{definition}

\begin{lemma}\label{compactness:lemma:1717}
Let $\phi$ be a \cathoristic{} formula and $X$ be a set such that no action
occurring in $\phi$ is in $X$. Then:
\[
   (\LLL, s) \models \phi
      \quad\text{iff}\quad
   (\LLL \setminus X, s) \models \phi.
\]
\end{lemma}
\begin{proof}
By straightforward induction on the structure of $\phi$, using the
fact that by assumption $X$ only contains actions not occurring in
$\phi$.  
\end{proof}

\begin{definition}
Let $\CAL{M}$ be a first-order model for the signature $\SSS'$.
We construct a cathoristic transition system
$\CAL{M}^{\sharp} = (S, \rightarrow, \lambda)$.
\begin{itemize}

\item The actions $\Sigma$ are given by the $\CAL{M}$ interpretation of actions.

\item The states $S$ are given by the $\CAL{M}$ interpretation of states.

\item The reduction relation $s \TRANS{a} t$ holds exactly when
  $\ARROWTWO{\CAL{M}}{s}{a}{t}$.

\item The function $\lambda$ is given by the following clause:
  \[
     \lambda(s) 
        =
     \begin{cases} 
       X & \text{whenever}\ X = \{a \ |\ \ALLOWED{\CAL{M}}{s}{a} \}\ \text{ is finite} \\
       \Sigma & \text{otherwise}
     \end{cases}
  \]

\end{itemize}

\end{definition}

\begin{lemma}
If $\CAL{M}$ be a first-order model for $\SSS'$ such that $\CAL{M}
\models \phi_{admis} \AND \phi_{det}$.  Then $\CAL{M}^{\sharp}$ is an
cathoristic transition system with actions $\Sigma$.
\end{lemma}
\begin{proof}
Immediate from the definitions.
\end{proof}

\begin{theorem}[correspondence theorem]\label{correspondence:theorem:223}
Let $\CAL{M}$ be a first-order model for the signature $\SSS'$ such
that $\CAL{M} \models \phi_{admis} \AND \phi_{det}$.  Then we have for
all \cathoristic{} formulae $\phi$ with actions from $\Sigma$:
\[
   \CAL{M} \models_{x \mapsto s} \SEMBTWO{\phi}_x 
        \quad  \text{iff} \quad 
   (\CAL{M}^{\sharp} \setminus X, s) \models \phi.
\]
\end{theorem}
Here $X$ is the set of all elements in the universe of $\CAL{M}$ interpreting
actions that are not in $\Sigma$.
\begin{proof}
The proof proceeds by induction on the structure of $\phi$. 
\end{proof}

\begin{definition}
Let $\Gamma$ be a set of cathoristic formulae, and $\MMM$ a cathoristic model.  We
write $\MMM \models T$ provided $\MMM \models \phi$ for all $\phi \in
T$.  We say $\Gamma$ is \emph{satisfiable} provided $\MMM \models T$.
\end{definition}

\begin{theorem}[Compactness of \cathoristic{}]
A set $\Gamma$ of \cathoristic{} formulae is satisfiable iff each finite subset of
$\Gamma$ is satisfiable.
\end{theorem}
\begin{proof}
For the non-trivial direction, let $\Gamma$ be a set of \cathoristic{} formulae
such that any finite subset has a cathoristic model. Define 
\[
  \SEMBTWO{\Gamma} 
     \ =\ 
  \{\SEMBTWO{\phi}\ |\ \phi \in \Gamma\} 
     \qquad\qquad
  \Gamma^*
     \ =\ 
  \SEMBTWO{\Gamma} \cup \{\phi_{admis} \AND  \phi_{det}\}
\]
which both are sets of first-order formulae. Clearly each finite subset $\Gamma'$ of 
$\Gamma^*$ has a first-order model. Why? First consider the subset $\Gamma'_{CL}$ of $\Gamma'$
which is given as follows.
\[
   \Gamma'_{CL} \ =\ \{ \phi \in \Gamma\ |\ \SEMBTWO{\phi} \in \Gamma' \}
\]
Since $\Gamma'_{CL}$ is finite, by assumption there is a cathoristic model 
\[
   (\LLL, s) \models \Gamma'_{CL}
\]
which means we can apply Theorem \ref{correspondence:theorem:223} to get
\[
   \SEMBTWO{\LLL} \models_{x \mapsto s} \SEMBTWO{\Gamma'_{CL}},
\]
By construction $\Gamma' \setminus \SEMBTWO{\Gamma'_{CL}} \subseteq
\{\phi_{admis} \AND \phi_{det}\}$, so all we have to
show for $\Gamma'$ to have a model is that
\[
    \SEMBTWO{\LLL} \models_{x \mapsto s} \{\phi_{admis}\} \cup \{ \phi_a\ |\ a \in \Sigma\},
\]
but that is a direct consequence of Lemma
\ref{compactness:lemma:23399}.  That
means each finite subset of $\Gamma^*$ has a model and by appealing to
compactness of first-order many-sorted logic (which is an immediate
consequence of compactness of one-sorted first-order logic
\cite{EndertonHB:matinttl}), we know there must be a first-order model
$\CAL{M}$ of $\Gamma^*$, i.e.
\[
   \CAL{M} \models \Gamma^*.
\]
Since $\CAL{M} \models \phi_{admis} \AND \phi_{det}$ we can apply
Theorem \ref{correspondence:theorem:223} that also
\[
   (\CAL{M}^{\sharp} \setminus X, s) \models \Gamma
\]
where $X$ is the set of all actions in $\CAL{M}^{\sharp}$ that are not
in $\Sigma$. Hence $\Gamma$ is satisfiable. 
\end{proof}





