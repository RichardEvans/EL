\begin{FIGURE}
\begin{tikzpicture}[node distance=3.3cm,>=stealth',bend angle=45,auto]
  \tikzstyle{place}=[circle,thick,draw=blue!75,fill=blue!20,minimum size=6mm]
  \tikzstyle{red place}=[place,draw=red!75,fill=red!20]
  \tikzstyle{transition}=[rectangle,thick,draw=black!75,
  			  fill=black!20,minimum size=4mm]
  \tikzstyle{every label}=[red]
  \begin{scope}  
    \node [place] (w1) {PL[$\land$]};
    \node [place] (e1) [below of=w1] {PL [$\land, \neg$] };
  \end{scope}
  \begin{scope}[xshift=4cm]
    \node [place] (w1) {HML[$\land$]};
    \node [place] (e1) [below of=w1] {HML [$\land, \neg$] };
  \end{scope} 
  \begin{scope}[xshift=8cm]
    \node [place] (w1) {EL[$\land, !$]};
    \node [place] (e1) [below of=w1] {EL [$\land, !, \neg$] };
  \end{scope}
  \draw (2,0) node {$\subseteq $};
  \draw (6,0) node {$\subseteq$};
  \draw (2,-3) node {$\subseteq $};
  \draw (6,-3) node {$\subseteq$};
  \draw (0,-1.5) node {$\subseteq $};
  \draw (4,-1.5) node {$\subseteq $};
  \draw (8,-1.5) node {$\subseteq $};
\end{tikzpicture}
\caption{Relationships between logics established in this paper. Here $L_1 \subseteq L_2$
means that the logic $L_2$ is a conservative extension of $L_1$. NB: we need to
explain the concept of conservative extension.}\label{figure:relationships}
\end{FIGURE}
